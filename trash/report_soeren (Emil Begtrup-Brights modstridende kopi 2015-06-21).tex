%!TEX TS-program = pdflatex
%!TEX TS-program = pdflatex

%%% -------------- Preamble ----------------- %%%

%!TEX root = ./report.tex
%-*- coding: utf-8 -*-
%!TeX encoding = UTF-8


%%%%%%%%%%%%%%%%%%%%%%%%%%%%%%%%%%%%%
% ----------- Preamble ------------ %
%%%%%%%%%%%%%%%%%%%%%%%%%%%%%%%%%%%%%



% ----------- dokumentclass ------------ %

\documentclass[11pt,a4paper,report,danish,oneside]{memoir} % for a short document
%brug article til kortere opgaver, hvor en af effekterne er at chapters og sections har samme status


% ----------- Font og sprog ------------ %

\usepackage[utf8]{inputenc} % set input encoding to utf8
\usepackage[T1]{fontenc}
\usepackage{kpfonts}
\usepackage[danish]{babel} %dansk sprog
\renewcommand{\danishhyphenmins}{22} % gør ordopdeling bedre men den er lort i forvejen så hvor slemt kan det være hvis det her er det bedste?! Prøv at eksperimentér med det der tal, 22, se hvad der sker.


% \usepackage[avantgarde]{quotchap}   % Pænere kapitler

\usepackage[scaled]{helvet}
\renewcommand\familydefault{\sfdefault} 
% \usepackage[T1]{fontenc}


% ----------- Sidelayout------------ %
% Set up the paper to be as close as possible to both A4 & letter:
\setulmarginsandblock{3cm}{3cm}{*} % 50pt upper margins
\setlrmarginsandblock{3cm}{3cm}{*} % golden ratio again for left/right margins
\checkandfixthelayout 
% Ovenstående er fra the memoir manual

\pagestyle{ruled} % try also: empty , plain , headings , ruled , Ruled , companion

%\SingleSpacing
\OnehalfSpacing
%\DoubleSpacing
%\setstretch{1.1}
\setcounter{secnumdepth}{2}
\setcounter{tocdepth}{2}

%indholdsfortegnelse
\renewcommand\cftappendixname{\appendixname~} %gør at der står "bilag A" i stedet for bare "A"

% fjerner sidenr fra \part-sider
\makeatletter
\renewcommand\part{%
  \if@openright
    \cleardoublepage
  \else
    \clearpage
  \fi
  \thispagestyle{empty}%
  \if@twocolumn
    \onecolumn
    \@tempswatrue
  \else
    \@tempswafalse
  \fi
  \null\vfil
  \secdef\@part\@spart}
\makeatother




% Ændre skrifttypen:
% \renewcommand{\familydefault}{\ttdefault} % typewriter font i hele documentet, men fucker med linjebrydningen/hypernation, ved ikke hvorfor.


% ----------- Diverse pakker - rækkefølgen er IKKE ligegyldig ------------ %

%\usepackage{longtable}  For at kunne lave tabeller fx i Stata der rækker over flere sider

\usepackage[table,xcdraw]{xcolor} % farver i tabeller
\usepackage{enumitem} %en ekstra slags liste 
\usepackage{float} % for at kunne placere en floater PRÆCIS her med \begin{figure}[H]
\usepackage{pdfpages} % understøtter inkluderingen af pdf-filer
\usepackage[plainpages=false,pdfpagelabels,pageanchor=false,hidelinks]{hyperref}   % aktive links
\usepackage{memhfixc}       % rettelser til hyperref - ved ikke hvad den gør
\usepackage{booktabs} %fancy tabeller
\usepackage{tabularx} % til statas tabout
\usepackage{multirow} % flere rækker i tabeller
\usepackage{graphicx} % indsæt billeder
\usepackage{rotating} %tillader figurer der står sidelæns
\usepackage{csquotes}
\usepackage{enumitem} %mindre mellemrum mellem items
\usepackage{dashrule} %stiblede linier i tabeller
%\usepackage{comment} % kunne sætte sectioner er ikke helt overbevidst om den her er brugbar, måske bare ifelse-løsningen
\usepackage{nameref} %giver mulighed for at refere til chapters etc med navn i stedet for bare en counter eller sidetal. 
% \usepackage{bytefield} % til at skabe datafigurer, se http://tex.stackexchange.com/questions/108257/implementing-a-way-to-show-data-structures-in-latex
\usepackage[normalem]{ulem} %striketrough
\usepackage{array} %centering 
\usepackage{wallpaper}
\usepackage{wrapfig} % wrap figure
\usepackage[font=small]{caption}

% Test pakker
\usepackage{appendix} %[toc,page]
\usepackage{capt-of} % til at sættte figurer og tabeller side-by-side
\usepackage[table,xcdraw]{xcolor} % farver til tabeller
\usepackage{courier} %sætter \texttt{} command til courier i stedet for den anden. courier er angiveligt bedre fordi den visuelt er mere forskellig fra standardfonten.
\usepackage{tcolorbox}
% \usepackage{siunitx}
% \sisetup{locale = DA}
\usepackage{multirow} 
\usepackage{geometry}
\usepackage{nicefrac}
\usepackage{flowchart}
\usetikzlibrary{arrows}
% \usepackage{pgfplotstable}


% ----------- Bibliografi ------------ %



%%bibliografi med biblatex
\usepackage[citestyle=authoryear-ibid, sorting=nty,backend=biber,bibencoding=UTF8,maxcitenames=2,defernumbers=true]{biblatex}
%% options til ovenstaaende:
%% maxcitenames=x styrer antal af forfattere før der står et.al/m.fl.
%% uniquelist=false/true styrer om den alligevel skal skrive flere navne end specifieret i maxcitenames i tvivlspørgsmål, dvs. hvis en forfatter optræder i flere referencer med andre forfattere.


%\usepackage[style=authoryear,backend=biber]{biblatex}
%bibtex-fil

%lave filtre så bibliografien kan deles op
% \defbibfilter{notonline}{
%   not type=Electronic, 
%   % and not type=manual
%   not keyword={DSTmanual}
% }

% \defbibfilter{DSTmanual}{%
% keyword=DSTmanual %keyword er et felt i en bib-entry
% }

% \defbibfilter{onlinekilder}{%
% type={Electronic}, 
% notkeyword={DSTmanual}
% }



\addbibresource{biblio/bibliografi.bib}



% ----------- Floaters ------------ %


%%% Ikke sikker på effekten af følgende. Måske slet, eksperimenter dig frem når du står i en situation hvor du kan se floaters opfører sig underligt.
%Alter some LaTeX defaults for better treatment of figures:
    % See p.105 of "TeX Unbound" for suggested values.
    % See pp. 199-200 of Lamport's "LaTeX" book for details.
    %   General parameters, for ALL pages:
    \renewcommand{\topfraction}{0.9}	% max fraction of floats at top
    \renewcommand{\bottomfraction}{0.8}	% max fraction of floats at bottom
    %   Parameters for TEXT pages (not float pages):
    \setcounter{topnumber}{2}
    \setcounter{bottomnumber}{2}
    \setcounter{totalnumber}{4}     % 2 may work better
    \setcounter{dbltopnumber}{2}    % for 2-column pages
    \renewcommand{\dbltopfraction}{0.9}	% fit big float above 2-col. text
    \renewcommand{\textfraction}{0.07}	% allow minimal text w. figs
    %   Parameters for FLOAT pages (not text pages):
    \renewcommand{\floatpagefraction}{0.7}	% require fuller float pages
	% N.B.: floatpagefraction MUST be less than topfraction !!
    \renewcommand{\dblfloatpagefraction}{0.7}	% require fuller float pages
	% remember to use [htp] or [htpb] for placement


% ----------- Egne kommandoer ------------ %

% \newcommand{\5.1}{5.1} (bruges ikke længere men for eksemplets skyld)

% \newcommand{\antalkat}{150} (bruges ikke længere men for eksemplets skyld)

% \fref og venner er defineret saaledes i memoir

% \renewcommand{\fref}[1]{\figurerefname~ef{#1}} 
% \renewcommand{\tref}[1]{\tablerefname~ef{#1}} 
% \renewcommand{\pref}[1]{\pagerefname~\pageref{#1}} 
% \renewcommand{\Pref}[1]{\partrefnameef{#1}} 
% \renewcommand{\Cref}[1]{\chapterrefnameef{#1}} 
% \renewcommand{\Sref}[1]{\sectionrefnameef{#1}} 

% og de navne som er anvendt er defineret som det her, dvs her kan man ændre om det er på dansk eller engelsk:

% \renewcommand*{\figurerefname}{Figur} 
% \renewcommand*{\tablerefname}{Tabel} 
\renewcommand*{\pagerefname}{side} 
% \renewcommand*{\partrefname}{Part~} 
% \renewcommand*{\chapterrefname}{Chapter~} 
% \renewcommand*{\sectionrefname}{\S} 

% \renewcommand*{\pref}{side} 


%Local Variables: 
%mode: latex
%TeX-master: "report"
%End: 







%%% -------------- Dokument ----------------- %%%

\begin{document} \selectlanguage{danish}


% %%% -------------- forside ----------------- %%%
% \begin{titlingpage}
% % KU wallpaper
% \ThisLRCornerWallPaper{1}{fig/forside/samf-farve.pdf}
% % start
% % newcommand{\HRule}{\rule{\linewidth}{0.5mm}} % Defines a new command for the horizontal lines, change thickness here
% \center % Center everything on the page
% %Heading sections
% \begin{minipage} {1\textwidth}
% \begin{flushleft} \large
% \textsc{Københavns Universitet} 
% \end{flushleft}
% \end{minipage}
% \vspace{4,5cm}
% %Title section
% \HRule \\[0.4cm]
% { \huge \bfseries \textsc{Veje til Beskæftigelse}}\\[0.4cm] % Title of your document
% { \large \bfseries \textsc{En kortlægning af arbejdsmarkedsparates beskæftigelsesmobilitet på det danske arbejdsmarked fra 1996 til 2009}\\[0.4cm] % Title of your document
% \HRule \\[1.5cm]
% \vspace{12cm}
% %Authour section
% \begin{minipage}{1\textwidth}
% \begin{flushleft} \large
% Emil Erik Pula Bellamy \textsc{Begtrup-Bright} 
% \newline Søren \textsc{Nielsen-Gravholt} 
% \end{flushleft}
% \end{minipage}
% \end{titlingpage}

% %%% -------------- kolofon ----------------- %%%
% \begin{titlingpage}
% \begin{minipage}{1\textwidth}
% \vspace{11cm}
% \begin{flushleft} \large
% Sociologisk Institut, Københavns Universitet
% \newline September 2015
% \end{flushleft}
% \vspace{0,5cm}
% \begin{flushleft} \large
% \emph{Forfattere:}\\
% Emil Erik Pula Bellamy Begtrup-Bright
% \newline Søren Nielsen-Gravholt
% \end{flushleft}
% \begin{flushleft} \large
% \emph{Vejleder:}\\
% Jens Arnholtz
% \end{flushleft}
% \vspace{0,5cm}
% \begin{flushleft} \large
% \emph{Antal tegn:} xxx
% \newline \emph{Antal tegn i fodnoter:} xxx
% \end{flushleft}
% \end{minipage}
% \end{titlingpage}


%%% -------------- Indholdsfortegnelse ----------------- %%%

\pagenumbering{roman} %til preface etc.

\tableofcontents*		% Indholdsfortegnelse the asterisk means that the contents itself isn't put into the ToC

\newpage


%\thispagestyle{empty}
 
%\listoffigures
%\newpage 
%\listoftables"C:\\texlive\\2011\\bin\\win32;$PATH"
 


\pagenumbering{arabic}
%måske ikke nødvendigt


%%% -------------- Afsnit ----------------- %%%

\chapterstyle{southall} %hvordan kapitler skal se ud.
%Nogle valgmuligheder, se flere i the memoir manual:
% - madsen thatcher verville southall ger companion

%!TEX root = ../report.tex

%%%%%%%%%%%%%%%%%%%%%%%%%%%%%%%%%%%%%%%%%%%%%%%%%%%%%%%%%%%
\chapter{Indledning \label{indledning}}
%%%%%%%%%%%%%%%%%%%%%%%%%%%%%%%%%%%%%%%%%%%%%%%%%%%%%%%%%%%

Dette speciale er en undersøgelse af hvorvidt arbejdsmarkedet er delt op i flere forskellige segmenter, og i så fald, hvordan en sociologisk analyse kan belyse disse segmenters mulige forskellige funktionsmåder.

Hvorfor er det interessant? Der meget lidt sociologisk teori, der beskæftiger sig med arbejdsmarkedets funktionsmåder på et mesoniveau, og samtidig er åben overfor en empirisk kortlægning. For det meste beskæftiger den arbejdsmarkedssociologiske teori (mig bekendt) enten med et kvalitativt orienteret mikroniveau, eller de helt brede penselstrøg på makroniveau.

Hvorfor er det vigtigt med undersøgelse af arbejdsmarkedet på et mesoniveau, når der laves undersøgelser på et mikro- og makroniveau? Viden om arbejdsmarkedet på et mikroniveau giver et billedet af et bestemt og snævert sted på arbejdsmarkedet og altså ikke viden hele arbejdsmarkedet. Viden om arbejdsmarkedet på et makroniveau giver et billede af abstrakte lovmæssigheder, der skal forklare en eller anden tendens i moderne samfund generelt. Dette er ikke en ny kritik (se GER og Merton find henvisninger \#todo), men noget, der tilsyneladende alligevel ofte glemmes. Med en undersøgelse på mesoniveau fås et billede af hvordan arbejdsmarkdet i et givet land fungerer lige nu.

Den teori, der bedst egner sig til en sådan undersøgelse, er (for mig at se) arbejdsmarkedssegmenteringsteorierne, med elementer fra moderne empirisk klasseforskning. Arbejdsmarkedssegmenteringsteorierne blev udviklet i 1970'erne, men er kun i begrænset omfang er blevet benyttet de sidste 20 år. Denne afhandling vil benytte denne teoriretning, med udgangspunkt i registerdata fra Danmarks Statistik i perioden fra 1996 til 2009.


%%%%%%%%%%%%%%%%%%%%%%%%%%%%%%%%%%%%%%%%%%%%%%
\section{Problemformulering}
%%%%%%%%%%%%%%%%%%%%%%%%%%%%%%%%%%%%%%%%%%%%%%
%
Med udgangspunkt i arbejdsmarkedssegmenteringsteorien, moderne klasseteori og empiri fra registerdata i Danmarks Statistik, har dette speciale følgende problemformulering:
%
\vspace{\baselineskip}
%
\begin{tcolorbox}[title=\textbf{Problemformulering}]
Findes der segmenter på det danske arbejdsmarked, og hvordan kan forskelle i sociale processer være med til at forklare sådanne forskelle i segmentstrukturen?
\end{tcolorbox}
%
\vspace{\baselineskip}
Det vil jeg undersøge med følgende forskningsspørgsmålspørgsmål:
\vspace{\baselineskip}
\begin{tcolorbox}[title=Forskningspørgsmål,
subtitle style={boxrule=0.4pt} ]
	\tcbsubtitle{1.} Er der en opdeling af arbejdsmarkedet for arbejdstagere i delmarkeder, hvor mobilitet indenfor delmarkederne er hyppig, og mellem delmarkederne sjælden?
	\tcbsubtitle{2.} Kan forskelle i de sociale processer vise, at der er tale om segmenter, og ikke blot delmarkeder?
	\tcbsubtitle{3.} Kan klasseteori belyse denne segmentering?
\end{tcolorbox}

Teorier med blik for segmenteringsprocesser på arbejdsmarkedet har en række forskellige indfaldsvinkler: det, der primært binder dem sammen, er ideen om at arbejdsmarkedet mere er karakteriseret ved \emph{opdeling} end af \emph{enshed}, hvilket ellers ville være det basale præmis i neoklassisk økonomisk teori. Dette speciale læner sig op af Thomas Bojes udformning af dette præmis \parencite[174]{Boje1986}.


%%%%%%%%%%%%%%%%%%%%%%%%%%%%%%%%%%%%%%%%%%%%%%
\subsection{Forskningsspørgsmål 1: Opdeling af arbejdsmarkedet i delmarkeder}
%%%%%%%%%%%%%%%%%%%%%%%%%%%%%%%%%%%%%%%%%%%%%%

For Thomas Boje er det første kriterie for arbejdsmarkedssegmentering, at arbejdsmarkedet er delt op i delmarkeder, med begrænset mobilitet mellem de enkelte delmarkeder. Det betyder, at der i mellem visse typer jobs forekommer hyppige skift, og andre jobtyper, hvor der sjældent, eller aldrig, observeres skifte fra det ene til det andet.

Delmarkeder i denne tradition har ofte tilknyttet en forståelse af samfundet som opdelt i forskellige sociale klasser, med forskellige strukturelle livsbetingelser for individerne i dem. Det er denne afhandlings formål at benytte en sådan tilgang, for at vurdere om den kan belyse arbejdsmarkedets struktur.

En klassebaseret forståelse har været genstand for sociologiske analyser siden sociologiens første store teoretikere, Weber, Marx og Durkheim%
%
\footnote{Durkheim er ikke bredt kendt som klasseteoretikere, men elementer i hans sociologi, samt især videreudviklinger af denne, har disse elementer i sig (citer Harrits \#todo)}%
%
. Weber var en af de af de første til at se på forholdet mellem klasse og mobilitet, og til at beskrive klasser ud fra de sociale mobilitetsmønstre, individerne var en del af. Han definerer klasse således: “\emph{A »social class« makes up the totality of those class situations within which individual and generational mobility is easy and typical.}” \parencite[302]{Weber1978}. Weberiansk orienterede sociologer såsom Goldthorpe (henvisning \#todo) og Oesch ( henvisning \#todo) har bibeholdt dette fokus på social mobilitet i nyere forskning. Spørgsmålet om både intra- og intergenerationel mobilitet på arbejdsmarkedet, som det observeres empirisk, har en væsentlig rolle at spille i klasseforskningen både indenfor den marxistiske, weberianske og durkheimianske tradition (henvisning: Wright, Goldthorpe, Grusky).

Den marxistiske tradition er optaget af klasser som et spørgsmål om udbytning, baseret på et socialt forhold til produktionsmidlerne. Nymarxistiske sociologer som Olin-Wright benytter også en stratifikation af lønmodtagerne, hvor positioner på arbejdmarkedet, forstået som arbejderklassens interne sammensætning, også har stor betydning for den faktiske fordeling af goder, og forskellige individers mulighed for opnåelse af disse goder. Arbejdsmarkedet er med andre ord også her delt op ud fra andre kriterier end blot arbejdsgiver-arbejder dikotomien. Den faktiske klassestruktur, udover den grundlæggende modsætning mellem arbejde og kapital, skal undersøges empirisk (henvisning til Wright \#todo). Den weberianske og durkheimianske tradition har ikke samme fokus på blah blah (skriv måske ud: tvivlsomt om afsnittet skal være sådan her? \#todo)

Af nyere sociologisk teori kan desuden nævnes Pierre Bourdieus sammensmeltning og nytænkning af marxistisk og weberiansk funderet teori, i hans forståelse af samfundet som et socialt rum, opdelt i felter, der opererer ud fra forskellige sociale logikker (\#henvisning). Anthony Giddens understreger nødvendigheden af empirisk funderede analyser, for at forstå klassestrukturen i et bestemt samfund, og de muligheder individet har, alt efter "hvor det kommer fra" \parencite[48,110]{Giddens1973}. (inkluder eventuelt Gruskys "mikroklasser")

Indenfor arbejdsmarkedssegmenteringsteorierne har man i en amerikansk optik beskæftiget sig med det såkaldte "dual labour market", det vil arbejdsmarkedet opdelt i to overordnede delmarkeder, hvor det primære indeholder faste stillinger med tryghed i ansættelse og favorable lønninger og gode arbejdsvilkår, hvorimod det sekundære delmarked består af midlertidige ansættelser til lavere lønninger og dårligere arbejdsvilkår \parencite{Piore1980}. Boje karakteriserer denne opdeling et amerikansk fænomen, da en række institutionelle forhold samt en produktionsstruktur, med få, store virksomheder i landet, har skabt denne struktur, og kan ikke generaliseres til lande med andre instutionelle forhold. Dette understreger behovet for en analyse af hvilke danske delmarkeder, der findes. Her finder Boje, at det danske samfund består af langt flere delmarkeder, blandt andet på grund af kollektive overenskomster, sikring af arbejdsløshedsunderstøttelse, samt den langt større rolle, små og mellemstore firmaer spiller, hvilket skaber en anden social dynamik \parencite[36]{Boje1985}

I Danmark er en nyere kortlægning af arbejdsmarkedets delmarkeder allerede påbegyndt, ligeledes med udgangspunkt i arbejdsmarkedssegmenteringsteorien, af Toubøl og Grau Larsen \parencite{Touboel2013}, ved brug af social netværksanalyse. Denne afhandling benytter sig af den af Toubøl og Grau Larsen nyligt udviklede metode til at finde kliker i et socialt netværk, til at finde segmenter i det danske arbejdsmarked, baseret på mobiliet.  

%%%%%%%%%%%%%%%%%%%%%%%%%%%%%%%%%%%%%%%%%%%%%%
\subsection{Forskningsspørgsmål 2: Forskellen på et delmarked og et segment er påvisningen af særegne (specifikke? partikulære) sociale processer indenfor delmarkedet}
%%%%%%%%%%%%%%%%%%%%%%%%%%%%%%%%%%%%%%%%%%%%%%

Det er imidlertidig ikke nok at påvise høj intern mobilitet for at kunne påstå at der er tale om et selvstændigt segment. Det skal være muligt at påvise forskelle i de sociale processer, som findes i delmarkedet i forhold til andre delmarkeder. Et delmarked, hvor forskellen i mobilitet primært skyldes faglige eller geografiske forskelle, men andre væsentlige sociale processer ellers er ens, kan ikke karakteriseres som et segment \parencite[41]{Boje1985}. Det er bare et delmarked, da allokeringen af arbejdskraft i væsentligt grad sker uden (determinerende) sociale stratifikationsmekanismer. Hvis eksempelvis løndannelse og kønsforskelle viser sig tydeligt mellem to ellers sammenlignelige delmarkeder, kan man begynde at tale om segmenter i Bojes forstand. Sammenlignelige skal her forstås som, at uddannelseslængde og 

Sociale processer er et noget abstrakt begreb for den praksis, hvori livet på arbejdsmarkedet udspiller sig for den enkelte lønmodtager. Det teoretiske indhold, samt naturligvis empiriske målbarhed, vil blive udpenslet efterfølgende, her skal blot nævnes to kortfattede perspektiver på sådanne processer.

Frank Parkin definerede i slut 60'erne begrebet \emph{social lukning} som en måde at forstå opdeling mellem sociale grupper, baseret på forhåndenværende distinktionskriterier i et samfund. Begrebet bruges her til at definere de processer, hvorpå forskellige faggrupper sørger for at beskytte egne privilegier, på en sådan vis at det ekskluderer andre, og hæmmer mobiliteten ud fra andre hensyn end (åbenlyst) faglig eller uddannelsesmæssige \parencite{Parkin1994}. 
% Mark  Granovetters benytter social netværksteoris metodik(?? \#todo) om svage og stærke sociale forbindelser for at forstå social lukning. Hans hovedargument er individernes mulighed mobilitet af nye kanaler, gennem deres såkaldte "svage forbindelser" - det er igennem de mennesker, man ikke kender så godt, at der er adgang til nye muligheder indenfor for eksempel arbejdslivet \parencite{Granovetter1973}. (\emph{de her to eksempler skal strammes op, I know - E }) Giver ikke rigtig så god mening her. 



%%%%%%%%%%%%%%%%%%%%%%%%%%%%%%%%%%%%%%%%%%%%%%
\subsection{Forskningsspørgsmål 3:  Et klasseteoretisk perspektiv på segmentering}
%%%%%%%%%%%%%%%%%%%%%%%%%%%%%%%%%%%%%%%%%%%%%%

Her adskiller min afhandling sig væsentligt fra Toubøl \& Grau Larsen, Nielsen-Gravholt og Boje, da dette speciale udover en beskrivelse af delmarkederne, har et klasseteoretisk perspektiv på segmentering og de sociale processer der definerer dem, der derefter undersøges empirisk. Formålet er at bevæge sig over i et klasseteoretisk perspektiv på segmentering, for at undersøge klassebegrebets anvendelig i at forstå disse sociale processer, som jeg ser dem komme til udtryk i min empiri. 

Forbindelsen til segmenteringsteori er ikke kontroversiel, og teoriretningen er bestemt ikke fremmed overfor teorier om sociale klasser, omend det ofte frames anderledes og i mere arbejdsmarkedsfokuserede termer. 
Boje er optaget af, at forskelle i sociale processer skaber ulighed på arbejdsmarkedet, og ses i ulige vilkår for forskellige (segmenterede) delmarkeder, hvorved forskelle i livsvilkår fører til øget seggregering. Dette er tæt beslægtet med tankegangen i klasseteori. Arbejdsmarkedet kan, som den marxistiske arbejdsmarkedssegmentteoretiker Richard Edwards bemærker, ses som et helt særgent marked, der tydeliggør styrkeholdende i produktionen og i den arbejdende befolkning som helhed {\parencite[177]{Edwards1979}. 

Moderne klasseteori, som den kommer til udtryk hos Daniel Oesch og John Goldthorpe, mener jeg har frugtbare teoretiske såvel som empiriske indsigter i, hvad man kan kalde den over-tid segmenterede arbejdsmarkedsstruktur. Deres klasseinddeling er næsten udelukkende baseret på position på arbejdsmarkedet. En årsag til Oesch og Goldthorpes anvendelig i min kortlægning af arbejdsmarkedet, er deres stringente - næsten ydmyge - fokus på \emph{økonomiske} klasse, fremfor det mere vidtløftige begreb \emph{social} klasse. Det vil jeg komme nærmere ind på senere, foreløbigt skal det bare konstateres, at det er yderst anvendeligt, når man som mig har fokus på arbejdsmarkedets struktur. Det betyder, at deres teori og empiri om differentieringer på arbejdsmarkedet er yderst anvendelige for mig.






%%%%%%%%%%%%%%%%%%%%%%%%%%%%%%%%%%%%%%%%%%%%%%
\section{Fremgangsmåde}
%%%%%%%%%%%%%%%%%%%%%%%%%%%%%%%%%%%%%%%%%%%%%%

I de to næste kapitler gennemgår jeg teori. Andet kapitel handler således om arbejdsmarkedssegmenteringsteori og tredje kapitel handler om klasseteori.

I de to efterfølgende kapitler gennemgår jeg metode. Fjerde kapitel handler således om social netværksanalyse og femte kapitel handler om registerdatamaterialet fra Danmarks Statistik.

De tre efterfølgende kapitler er analysen opdelt tre delanalyser, som passer overens med de tre forskningsspørgsmål. Sjette kapitel er en delanalyse af opdeling af arbejdsmarkedet i delmarkeder. Syvende kapitel er en delanalyse af sociale processer i delmarkeder og segmenter. Ottende kapitel er en delanalyse af det klasseteoretiske perspektiv på segmentering.

Niende kapitel er en diskussion af analysen på baggrund af teori og metode.

De tiende og afsluttende kapitel er konklusion.


Denne afhandling vil fokusere på, hvorledes sociale processer kan ses afvige fra hinanden i delmarkederne, på sådan vis at vi kan tale om segmenter. Jeg vil benytte social stratifikationsteori, til at forklare forskellen i sociale processer, som de kommer til udtryk på et empirisk niveau, gennem intern mobilitet i delmarkederne, indkomst, uddannelse og køn. 
% % -*- coding: utf-8 -*-
% !TeX encoding = UTF-8
% !TeX root = ../report.tex


\chapter{TEORI - BOURDIEU OG ARBEJDSLØSHEDSFORSKERNE} \label{baggrund}


% Som start på at skrive om metoden, kunne man tage udgangspunkt i marginaliseringsmodellen + det at vi inddrager beskæftigede, arbejdsløse og udenfor arbejdsstyrken.
% Her kunne vi for eksempel sige, at sociologer som Jahoda beskæftiger sig med marginaliserings- og ekskluderingsprocessen, mens økonomerne og policy-forskerne beskæftiger sig med inkluderingsprocessen.


%%%%%%%%%%%%%%%%%%%%%%%%%%%%%%%%%%%%%%%%%%%%%%%%%%%%%%%%%%%%%%%%%%%%%%%%%%%%%%%%%%%%%%%%%%%%%%%%
%%%%%%%%%%%%%%%%%%%%%%%%%%%%%%%%%%%%%%%%%%%%%%%%%%%%%%%%%%%%%%%%%%%%%%%%%%%%%%%%%%%%%%%%%%%%%%%%
%%%%%%%%%%%%%%%%%%%%%%%%%%%%%%%%%%%%%%%%%%%%%%%%%%%%%%%%%%%%%%%%%%%%%%%%%%%%%%%%%%%%%%%%%%%%%%%%


Vi ønsker at bruge Bourdieu i vores primære teoriapparat til at fortælle en historie om hvilke muligheder man har for at ernære sig, når man oplever arbejdsløshed, med særligt fokus på hvilken beskæftigelse man får efter en periode med ledighed. Vender man tilbage til arbejde i samme felt, eller bevæger man sig ind på et nyt? Derved vil vi diskutere, hvilken praksis der hænger sammen med hvilke felter, og hvad det siger om hvilke felter der ligger nær hinanden. Eller måske siger noget om hvor desperat man skal være, for at bevæge sig ud over det felt man er trænet ind i. Vi vil gerne diskutere hvad der strukturerer folk, der oplever arbejdsløsheds opfattelse af handlingsrum, som vi ser det komme til udtryk igennem deres praksis mellem forskellige typer af jobs efter perioder med ledighed.

Her inddrager vi centrale økonomiske og sociologiske teorier om arbejdsløshed samt hvordan den danske arbejdsløshedsmodel historisk har udviklet sig til det den er i dag, og hvordan den ser ud i dag. Relevante forskere inden for arbejdsmarkedsforskning historisk, sociologisk og økonomisk er eksempelvis Jesper Due, Jørgen Steen Madsen, Bent Jensen, Per H. Jensen, Aage Huulgaard og Hans-Carl Jørgensen. Relevante økonomiske teorier er fx søgeteorien, matching-teorien og insider-outsider-teorien udarbejdet af George Stigler, Peter Diamon, Dalte T. Mortensen, Christopher A. Pissarides, Assar Lindbeck, Dennis Snower, mv. Relevante sociologiske teorier og teoretikere kunne fx være Marie Jahoda  Marienthal-studie, Philip Eisenberg og Paul Lazarfeld, Donald Tiffany, James Cowan og Phyllis Tiffany, Peter Warr, Knut Halvorsen, Barney Glaser og Anselm Strauss, Catharina Juul Kristensen og Jørgen Elm Larsen.

Vi kan endnu ikke endnu sige i hvilken grad den nævnte teori blive anvendt. Vi vil kæde disse retninger sammen med vores primære inspirationskilde Bourdieu, eller bruge det som en baggrund for at forstå den eksisterende litteratur om arbejdsløshed til så at vise i hvilken tradition vi skriver os ind på.


%%%%%%%%%%%%%%%%%%%%%%%%%%%%%%%%%%%%%%%%%%%%%%%%%%%%%%%%%%%%%%%%%%%%%%%%%%%%%%%%%%%%%%%%%%%%%%%%
%%%%%%%%%%%%%%%%%%%%%%%%%%%%%%%%%%%%%%%%%%%%%%%%%%%%%%%%%%%%%%%%%%%%%%%%%%%%%%%%%%%%%%%%%%%%%%%%
%%%%%%%%%%%%%%%%%%%%%%%%%%%%%%%%%%%%%%%%%%%%%%%%%%%%%%%%%%%%%%%%%%%%%%%%%%%%%%%%%%%%%%%%%%%%%%%%


\section{Økonomiske teorier og policy \label{}}

Jørgen Goul Andersen identificerer et skifte inden for en bred strømning af økonomisk teori fra slutningen af 1970'erne, hvor velfærdsstaten blev anskuet som et middel at afbøde *markedsfejl* til i stigende grad at fokusere på *politikfejl* og *forvridninger* på markedet, hvilket kan karakteriseres som et skifte fra efterspørgsel til udbud og et skifte fra makro til mikro (Goul Andersen 2003: 19).

\subsection{Søgeteorien \label{}}
Søgeteorien blev grundlagt i 1960'erne og kendte teoretikere er John J. McCall, George Stigler, Peter Diamon, Dalte T. Mortensen og Christopher A. Pissarides og den basale søgemodel handler om arbejdsløse som søger beskæftigelse. Forestil en arbejdsløs som af og til får jobtilbud. Den arbejdsløse kender først lønnen på jobtilbuddet, når det modtages. Efter at have modtaget tilbuddet, skal den arbejdsløse beslutte sig for, om tilbuddet accepteres eller afslås. Dette gør den arbejdsløse på baggrund af en reservationsløn, som er fastsat af forventningerne til lønnen og viden om lønfordelingerne på arbejdsmarkedet. Hvis det modtagne tilbud er større end reservationslønnen, accepteres tilbuddet, og ellers afslås det (Rosholm 2009: 159f). Marginalisering kan i denne sammenhæng forklares med, at den arbejdsløses kvalifikationer bliver mindre værd efterhånden som ledighedsperioden bliver længere, at den arbejdsløse gradvist mister forbindelsen til gamle kolleger eller at den arbejdsløse dømmes på baggrund af sin langtidsledighed (Rosholm 2009: 160f, Goul Andersen 2003: 20).

\subsection{Matching-teorien \label{}}
Matching-teorien deler arbejdsmarkedet op i to typer agenter: arbejdsgivere og lønmodtagere og er kendt for teoretikere som Dale T. Mortensen, Christopher A. Pissarides Peter A. Diamond, Alvin E. Roth and Lloyd Shapley. Her leder lønmodtagerne efter job i de perioder, hvor de er arbejdsløse, mens arbejdsgivere slår stillinger op, så længe de vurderer, at det kan betale sig. Ledige lønmodtagere og job mødes på arbejdsmarkedet i en matching-proces. Når der skabes kontakt mellem en arbejdsgiver og lønmodtagere, opstår der en forhandling mellem arbejdsgiver og lønmodtager om fordelingen af overskuddet i en eventuel ansættelse. Arbejdsgiveren kan på den ene side presse lønmodtageren til at acceptere en løn, som ligger under arbejdskraftens marginalprodukt, fordi lønmodtageren ikke uden videre kan finde nyt job på anden vis end ved at vente på det næste jobtilbud. Lønmodtageren kalkulerer på baggrund af forskellen mellem den tilbudte løn og værdien af at være ledig (*outside option*). Matching-teorien kan bruges til at analyse den umiddelbare effekt af at ændre i ledighedsydelser som eksempelvis dagpenge eller kontakthjælp (Rosholm 2009: 162f).

\subsection{Insider-outsider-teorien \label{}}
Insider-outsider-terien er udviklet af Assar Lindbeck og Dennis Snower i 1980'erne og fokuserer på omkostningerne forbundet med ansættelser og afskedigelser af arbejdskraft. Omkostningerne kan være i forbindelse med søge- og optræning, afskedigelse, uproduktiv konkurrence mellem to grupper af lønmodtagere. Insidere og outsidere kan blandt andet defineres som beskæftige og arbejdsløse, fagforeningsmedlemmer og ikke-medlemmer, ansatte i gode jobs/dårlige jobs. Insidere vil forsøge at forhandle sig til så høje lønninger som muligt og afholde andre for at underbyde dem på markedet. Insidernes magt består blandt andet i, at arbejdsgiverne har omkostninger vil at afskedige dem og kan skabe omkostninger ved strejke, aktioner og mobning af nyansatte. Langtidsledige, i modsætning til beskæftige og korttidsledige, kan tilbyde sin arbejdskraft til reduceret løn og forsøge at overbevise en arbejdsgiver om at blive ansat så længere arbejdsgiverens omkostninger ved at ansætte en outsider ikke stiger gevinsten ved at gøre det (Rosholm 2009:164, Goul Andersen 2003:20).



%%%%%%%%%%%%%%%%%%%%%%%%%%%%%%%%%%%%%%%%%%%%%%%%%%%%%%%%%%%%%%%%%%%%%%%%%%%%%%%%%%%%%%%%%%%%%%%%
%%%%%%%%%%%%%%%%%%%%%%%%%%%%%%%%%%%%%%%%%%%%%%%%%%%%%%%%%%%%%%%%%%%%%%%%%%%%%%%%%%%%%%%%%%%%%%%%
%%%%%%%%%%%%%%%%%%%%%%%%%%%%%%%%%%%%%%%%%%%%%%%%%%%%%%%%%%%%%%%%%%%%%%%%%%%%%%%%%%%%%%%%%%%%%%%%


\section{Sociologiske teorier \label{}}

\subsection{Jahoda, Lazarsfeld og Zeizels Marienthal-studie (Nørup:23) \label{}}
Marienthal er et klassisk studie af de sociale konsekvenser af arbejdsløshed i et lille samfund gennemført af Marie Jahoda i samarbejde med Paul Lazarsfeld og Hans Zeizel (1971). Marienthal var et industriby som led af høj arbejdsløshed i 1920'erne, og studiet undersøger hvad der sker med arbejderne i den østrigske by Marienthal, når de oplever arbejdsløshed. Med Marienthal udvikler Jahoda deprivationsperspektivet, som er det mest udbredte perspektiv i de teoretiske diskussioner af arbejdsløshed og eksklusion fra arbejdsmarkedet (Creed og Macintyre:2001). Hovedargumentet er, at arbejdsløshed medfører social eksklusion og isolation, tab af struktur i hverdagen og selvtillid og en betydelig øget risiko for psykiske problemer (Jahoda, 1981, Jahoda m.fl. 1997). Marienthal baserer sig på en grundlæggende antagelse om arbejde deltagelse på arbejdsmarkedet opfylder både et psykologisk behov for individet og et økonomisk behov for indtægt. Jahoda opstiller ikke knækket vilje, resignation, fortvivlelse og apati som fire stadier eller reaktioner den arbejdsløse gennemgår (Jahoda 1979, Jahoda m.fl. 1997). Arbejdsdeltagelsen har ifølge Jahoda fem funktioner: tidsmæssig struktur i dagligdagen, sociale kontakter, deltagelse i kollektive formål, status og identitet og regelmæssig aktivitet (Jahoda, 1981, Jahoda m.fl. 1997).
Nørup kritiserer brugen af deprivationsperspektivet i dansk regi, fordi det danske samfund i dag er en moderne velfærdstat med relativt højtuddannet arbejdskraft, og Marienthal er en mindre industriby med lavt uddannet arbejdskraft i et 1930'ernes Østrig som ikke er i nærheden af et velfærdssamfund (Nørup2012:34).

\subsection{Eisenberg og Lazarsfeld (Nørup:24) \label{}}
I *The psychological effects of unemployment* (1938) konkluderer Philip Eisenberg og Paul Lazarfeld, at arbejdsløse gennemlever tre stadier: “First there is shock, which is followed by an active hunt for a job, during which the individual is still optimistic and unresigned; he still maintains an unbroken attitude. Second, when all efforts fail, the individual becomes pessimistic, anxious, and suffers active distress; this is the most crucial state of all. And third, the individual becomes fatalistic and adapts himself to his new state but with a narrower scope He now has a broken attitude.” (Eisenberg og Lazarsfeld 1938:378).
Studiet har vundet udbredelse inden for socialpsykologien (Boyd 2014, Wang og Greenwood 2014, Kahn 2013, Ezzy 1993, Ragland-Sulivan og Barglow 1981, Finley og Lee 1981, Hayes og Nutman 1981, Hill 1978, Briar 1977, Harison 1976). Men studiet er også blevet kritiseret på baggrund af en problematisk og modsætningsfuld metode (Fryer 1985, Ezzy 1993) og på baggrund af det empiriske fundament i særdeleshed vedrørende psykologiske faktorer som eksempelvis mentalt helbred og selværd (Fryer 1985, Hartley 2011, Shamir 1986). (Nørup kalder det for *Stadie Model*)

\subsection{Tiffany, Cowan og Tiffany (Nørup:25) \label{}}
Hovedargumentet i studiet *The unemployed: A social-psychological portrait*  af Donald Tiffany, James Cowan og Phyllis Tiffany (1970) er, at majoriteten af arbejdsløse og ekskluderede fra arbejdsmarkedet står uden for arbejdsmarkedet på grund af psykologiske problemer. Sammenhængen mellem arbejdsløshed og psykologiske problemer går derfor begge veje, hvilket betyder, at psykologiske kan være årsagen til arbejdsløshed på samme tid med, at arbejdsløshed i sig selv også medfører psykologiske problemer: ”They show avoidance behaviour patterns or what has been referred to as ”work inhibition” which implies that they are physically capable of work but prevented from work because of psychological disabilities” (Tiffany, Cowan and Tiffany, 1970). Ifølge Tiffany, Cowan og Tiffany er løsningen, at staten rehabiliterer disse arbejdsløse, så de kan komme tilbage på arbejdsmarkedet gennem træning eller terapi (Tiffany, Cowan and Tiffany: 1970). Douglas Ezzy peger på, at denne tilgang har ligheden mellem den historiske distinktion mellem *deserving poor*, som fysisk var ude af stand til at arbejde og fortjente støtte og *non-deserving poor*, som ikke arbejde selvom de var i fysisk stand til at arbejde (Ezzy 1993). Ezzy påpeger ligeledes på, at tilgangen har været mest toneangivende i perioder med højkonjunktur og relativ lav ledighed til sammenligning med perioder med lavkonjunktur og lav ledig (Ezzy 1993).
Perspektivet kritiseres for at have lighedstræk med den neoklassiske økonomiske betragtning af arbejdsløshed som frivilligt og derfor ved at skyde skylden på ofret (Miles:1987) (Nørup kalder det for *Rehabiliteringstilgangen*)

\subsection{Warr (Nørup:25) \label{}}
I *Work, Unemployment and Mental Health* opstiller Peter Warr ni faktorer i omgivelser som har betydning for det mentale helbred i forbindelse med arbejdsløshed: mulighed for kontrol, mulighed for at benytte erhvervede færdigheder, eksternt genererede mål, variation, klarhed i forhold til omgivelserne, penge og  indtjening, fysisk sikkerhed, mulighed for social kontakt og social position (Warr:1987). Individets mentale helbred afspejler det akkumulerede niveau af faktorerne, så det at miste et arbejde eller det at have et dårligt arbejde i omgivelserne afspejler individets mentale heldbred (Ezzy:1993). (Nørup kalder det for *vitaminmodellen*)

\subsection{Halvorsens teori om mestring (Nørup:28) \label{}}
Knut Halvorsen udvikler i sit forfatterskab en modpol til Jahoda, Lazarsfeld og Zeizel (1971), Eisenberg og Lazarsfeld (1938), Tiffany, Cowan og Tiffany (1970) og Warrs (1987) passive og ensartede individperspektiv ved at betragte arbejdsløse som forskelligartede og handlende. De arbejdsløse anskues derfor som aktive aktører, der kan påvirke og forandre deres situation i stedet for at være ofre for omstændighederne (Halvorsen 1994, 1999). Den arbejdsløse vil indgå i forskellige fysiske, psykiske og sociale aktiviteter for at afbøde effekter af arbejdsløshed og minimere stress og mental belastning (Halvorsen: 1999, Fryer og Fagan:1993, Fryer:1986, Fryer og Payne:1984, O’Brien:1985). Halvorsen skelner mellem den problemorienterede mestring, som udgøres af konkrete strategi med formål at fjerne belastningen af den marginaliserede position (fx jobsøgning) og den emotionsorienterede mestring, som handler om hvordan man ser verdenen (Halvorsen 1999)

Nørup kritiserer Halvorsen for i sin ontologiske individualisme at fokuserer på, hvordan individet mestrer bestemte livssituationer under bestemte rammer (graden af eksklusion forklares som et resultat af individuelle handlinger og ressourcer) frem for på hvordan de samfundsmæssige strukturer og sociale relationer påvirker eksklusionen (Nørup 2012:37).

\subsection{Glaser og Strauss’ teori om sociale passager \label{}}
Barney Glaser og Anselm Strauss definerer en status passage som et individs ”movement into a different part of a social structure, or loss or gain of privilige, influence, or power, and changed behaviour”. Ezzy beskriver anvendelsen af teorien på arbejdsløshed og exit fra arbejdsmarkedet, som processer frem for enten-eller tilstande. Hermed kan exit fra arbejdsmarkedet sammenlignes med andre status passage som eksempelvis skilsmisse, sygdom eller dødsfald i familien (Ezzy:1993). Van Gennep benytter separation (begravelse), transition (overgangsfase mellem to jobs) og integration (ægteskab) som kategorier for sociale passager (Ezzy:1993, Van Gennep 1977). Ezzy identificerer exit fra arbejdsmarkedet som tab af job som en afhændelespassage i modsætning til exit fra arbejdsmarkedet som indtræden i uddannelsessystemet som noget helt andet (Ezzy 1993). En afhændelsespassager fører ikke nødvendigvis sig selv til mentale problemer, mistrivsel eller eksklusion, da det afhænger af den enkeltes identitet og selvopfattelse i relation til andre og samfundet og en lang række andre faktorer samt om afhændelsespassagen efterfølges af en reintegrativ passage, hvor den enkelte får en ny status eksempelvis et nyt job (Ezzy 1993).

% Kommentarer
% Deprivation (Jahoda, Lazarsfeld og Zeizel 1971; Eisenberg og Lazarsfeld 1938; Tiffany, Cowan og Tiffany 1970; Warrs (987) er delvist brugbart i forståelsen af arbejdsløshed som havende en social og psykologisk påvirkvning på arbejdsløse. Halvorsen er delvis brugbar i forståelsen af at individerne bliver aktører som kan handle og har ressourcer. Social passage (Glaser og Strauss 1971)er relevant i vores definition af arbejdsløshed som midlertidigt.


%%%%%%%%%%%%%%%%%%%%%%%%%%%%%%%%%%%%%%%%%%%%%%%%%%%%%%%%%%%%%%%%%%%%%%%%%%%%%%%%%%%%%%%%%%%%%%%%
%%%%%%%%%%%%%%%%%%%%%%%%%%%%%%%%%%%%%%%%%%%%%%%%%%%%%%%%%%%%%%%%%%%%%%%%%%%%%%%%%%%%%%%%%%%%%%%%
%%%%%%%%%%%%%%%%%%%%%%%%%%%%%%%%%%%%%%%%%%%%%%%%%%%%%%%%%%%%%%%%%%%%%%%%%%%%%%%%%%%%%%%%%%%%%%%%


\section{Bourdieu - Arbejdsløse, på kanten af arbejdsmarkedet \label{}}


%%%%%%%%%%%%%%%%%%%%%%%%%%%%%%%%%%%%%%%%%%%%%%%%%%%%%%%%%%%%%%%%%%%%%%%%%%%%%%%%%%%%%%%%%%%%%%%%
%%%%%%%%%%%%%%%%%%%%%%%%%%%%%%%%%%%%%%%%%%%%%%%%%%%%%%%%%%%%%%%%%%%%%%%%%%%%%%%%%%%%%%%%%%%%%%%%
%%%%%%%%%%%%%%%%%%%%%%%%%%%%%%%%%%%%%%%%%%%%%%%%%%%%%%%%%%%%%%%%%%%%%%%%%%%%%%%%%%%%%%%%%%%%%%%%


\section{Har vi andre centrale teoretikere \label{}}

% Boltanski og Chiapello

%%%%%%%%%%%%%%%%%%%%%%%%%%%%%%%%%%%%%%%%%%%%%%%%%%%%%%%%%%%%%%%%%%%%%%%%%%%%%%%%%%%%%%%%%%%%%%%%
%%%%%%%%%%%%%%%%%%%%%%%%%%%%%%%%%%%%%%%%%%%%%%%%%%%%%%%%%%%%%%%%%%%%%%%%%%%%%%%%%%%%%%%%%%%%%%%%
%%%%%%%%%%%%%%%%%%%%%%%%%%%%%%%%%%%%%%%%%%%%%%%%%%%%%%%%%%%%%%%%%%%%%%%%%%%%%%%%%%%%%%%%%%%%%%%%




%Local Variables: 
%mode: latex
%TeX-master: "report"
%End:
% -*- coding: utf-8 -*-
% !TeX encoding = UTF-8
% !TeX root = ../report.tex


%% Noter %%%

% i afsnit om ledighed: start evt med figur om brutto/nettoledighed og akuledighed, og udvid med egne betragtninger om hvad På kanten af arbejdsmarkedet vil sige - det er  vores bidrag, en mere nuanceret forståelse om at være en del af arbejdsmarkedet er et kontinuum. Slut afsnittet med samme figur, nu med endnu en cirkel rundt om der viser vores overkategori.





\chapter{METODE OM ARBEJDSLØSHED} \label{metode}


%%%%%%%%%%%%%%%%%%%%%%%%%%%%%%%%%%%%%%%%%%%%%%%%%%%%%%%%%%%%%%%%%%%%%%%%%%%%%%%%%%%%%%%%%%%%%%%%%%%
%%%%%%%%%%%%%%%%%%%%%%%%%%%%%%%%%%%%%%%%%%%%%%%%%%%%%%%%%%%%%%%%%%%%%%%%%%%%%%%%%%%%%%%%%%%%%%%%%%%
%%%%%%%%%%%%%%%%%%%%%%%%%%%%%%%%%%%%%%%%%%%%%%%%%%%%%%%%%%%%%%%%%%%%%%%%%%%%%%%%%%%%%%%%%%%%%%%%%%%



\section{At skabe en kritisk masse af ledige \label{ledigskab}}

Kernen i vores empiriske arbejde er en fundamental skelnen mellem beskæftigelse og den mellemliggende periode mellem beskæftigelse. Eller med andre ord at “være ledig eller ej”. Selvom det er en nødvendig skelnen i vores empiri, behøver det i midlertidig ikke også at betyde, at vi i vores begrebsdannelse accepterer denne dikotomi som et lige så fundamentalt socialt fakta eller at det bliver et mål i sig selv at reducere den sociale virkelighed til et spørgsmål om at “være ledig eller ej”. Snarere tværtimod. Men for at kunne skabe et overblik over ledighedsmobilitet i tidsperioden, er det nødvendigt for senere at kunne åbne begrebet op igen. Vores gennemgang af vores empiriske ledighedsbegreb vil netop vise, at dikotomien er langt mere mudret end den efterfølgende reduktion til en binær modstilling lader ane. 


%%%%%%%%%%%%%%%%%%%%%%%%%%%%%%%%%%%%%%%%%%%%%%%%%%%%%%%%%%%%%%%%%%%%%%%%%%%%%%%%%%%%%%%%%%%%%%%%%%%
%%%%%%%%%%%%%%%%%%%%%%%%%%%%%%%%%%%%%%%%%%%%%%%%%%%%%%%%%%%%%%%%%%%%%%%%%%%%%%%%%%%%%%%%%%%%%%%%%%%
%%%%%%%%%%%%%%%%%%%%%%%%%%%%%%%%%%%%%%%%%%%%%%%%%%%%%%%%%%%%%%%%%%%%%%%%%%%%%%%%%%%%%%%%%%%%%%%%%%%



\section{Operationalisering af ledige i binær form \label{ledig_operationalisering}} 
% 
Relevante beskæftigelsesvariable
 \begin{itemize} [topsep=6pt,itemsep=-1ex]
   \item ARSTIL, NYARB og SOCSTIL
   \item HELTID\_DELTID
   \item BESKST og BESKST02 -> Kan bruges til beskæftigelse
   \item SOCIO og SOCIO02
 \end{itemize}
% 
Relevante ledighedsvariable
 \begin{itemize} [topsep=6pt,itemsep=-1ex]
   \item FORANST kombineret med TIMERPU (går kun til 2006)
   \item \textbf{LEDDEL og LEDFULD} -> kan bruges til mikro-ledige (går kun til 2007)
   \item SUMGRAD
 \end{itemize}
% 


DST har ikke overraskende en lang række variable, der forholder sig direkte eller indirekte til begrebet ledighed. Mange af disse forholder sig specifikt til forskellige aspekter af det at være ledig, såsom \texttt{DPTIMER}, der beskriver det antal timer, der er udbetalt dagpenge for, indenfor en uge. At aggregere disse variable til et samlet ledighedsbegreb ville være en enorm opgave, og eftersom dokumentationen for variablene varierer fra ganske informativ til obskur intern system-jargon. I stedet for har vi udvalgt variablen \texttt{SOCSTIL} og kombineret denne med variablen \texttt{SOCIO}\footnote{I 2002 ændres \texttt{SOCIO} til \texttt{SOCIO02}, som er en ny udgave med mindre ændringer. For overskuelighedens skyld benytter vi navnet \texttt{SOCIO} selvom om det ville være mere hensigtsmæssigt at benytte navnet \texttt{SOCIO/SOCIO02}.}, som begge er blevet aggregeret af DST på en sådan vis, at vi kan skabe et binært ledighedsbegreb ud fra dem.

Vores ledighedsbegreb fokuserer på beskæftigede som kommer midlertidigt ud af beskæftigelse for så at vende tilbage til beskæftigelse igen. I den sammenhæng anvender vi \texttt{SOCSTIL}, som angiver befolkningens tilknytning til arbejdsmarkedet ultimo november. Befolkningen opgøres i beskæftigede og arbejdsløse som udgør arbejdsstyrken samt den øvrige del af befolkningen som betegnes uden for arbejdsstyrken. Beskæftigelse i vores ledighedsbegreb er ensbetydende med \texttt{SOCSTIL}s betegnelse, hvilket udgør selvstændige, medarbejdende ægtefæller og lønmodtagere\footnote{Selvstændige, medarbejdende ægtefæller og lønmodtagere har henholdsvis \texttt{SOCSTIL}-værdierne 115-118, 120 og 130-135.}. Med hensyn til de midlertidigt uden beskæftigelse er vi interesserede i alle som vender tilbage i beskæftigelse uanset om de er en del af arbejdsstyrken eller ej\footnote{Vi anerkender, at arbejdsstyrken er den del af befolkningen hvis arbejdskraft er til rådighed for arbejdsmarkedet og som enten er i beskæftigelse eller er ledige. Vi mener dog, når vi netop kigger på ledighed over tid, at vi ikke kun behøves at forholde os til arbejdsstyrken, fordi selvom en person ikke registreres, at denne står til rådighed for arbejdsmarkedet, kan vi netop se, at denne person kan vende tilbage i beskæftigelse på et senere tidspunkt. Dette kan vi netop gør, fordi vi ser på de ledige over en længere periode og ikke opgør ledige på sammen måde som DST gør.}. Derfor inkluderer vi mere end blot de arbejdsløse, fordi de i \texttt{SOCSTIL}s betegnelse kun udgør nettoledige\footnote{Nettoledige og bruttoledige har henholdsvis \texttt{SOCSTIL}-værdierne 200 og 201.}, og ikke eksempelvis flere forskellige former for aktivering og kontanthjælp. DST's definition af at være arbejdsløs følger nemlig ILOs betingelser om at man skal være uden arbejde, stå til rådighed for arbejdsmarkedet og være aktivt arbejdssøgende \parencite{ILO1982}. Disse betingelser er lavet for at have en international sammenlignelig standard og som ikke nødvendigvis passer til overens med det arbejdsmarked, vi ønsker at beskrive.

Til at moderere vores ledighedsbegreb, trækker vi på Jørgen Elms Larsens perspektiver om  marginalisering i sammenhæng med inklusion og eksklusion. Larsen definerer eksklusion som en ufrivillig ikke-deltagelse gennem forskellige typer af udelukkelsesmekanismer og -processer, som det ligger uden for indvidets og gruppens muligheder at få kontrol over \parencite[237]{Larsen2009}. Larsen er kritisk over for Luhmanns binære form inklusion/eksklusion, som han mener ikke er særlig hensigtsmæssig i forhold til virkeligheden \parencite[?]{Larsen2009}. Derfor argumenter han for, at marginalisering kan anvendes som en midtergruppe mellem de to \parencite[130f]{Larsen2009}. Til at illustrere dette har vi, som det fremgår af tabel \ref{tab_marginaliseringsmodel}, udviklet en model\footnote{Modellen er også inspireret af lignende modeller benyttet af Lars Svedberg \parencite[44]{Svedberg1995} og Catharina Juul Kristensen \parencite[18]{Kristensen1999}.} til at beskrive hvad der er på spil, når man går fra at være beskæftiget til at være “midlertidigt” uden beskæftigelse og tilbage til beskæftigelse igen. Processen med at gå fra at være beskæftiget kaldes her for en proces mod marginalisering og processen med at gå tilbage til beskæftigelse igen kaldes for en proces mod inklusion. \emph{... Det har antagelser om arbejdsfællesskabet...}
%
\begin{table}[H] \centering
\caption{Model over marginalisering}
\label{tab_marginaliseringsmodel}
\begin{tabular}{@{} m{2,5cm} c m{4cm} c m{4cm} @{}} \toprule
\textbf{Inkluderet} & & \multicolumn{1}{c}{\textbf{Marginaliseret}} & & \textbf{Ekskluderet} \\ \midrule
  beskæftiget  & & “midlertidigt” uden beskæftigelse & & vender ikke tilbage i beskæftigelse \\  
\end{tabular} \end{table}
%
\begin{table}[H] \centering
\label{tab_marginaliseringsmodel}
\begin{tabular}{@{} m{5,9cm} m{5,9cm} @{}} 
  \textbf{Marginaliseringsproces} & \textbf{Eksklusionsproces} \\  
  --------------------------------------------> & --------------------------------------------> \\ 
\end{tabular} \end{table}
%
%
\begin{table}[H] \centering
\label{tab_marginaliseringsmodel}
\begin{tabular}{@{} m{12,3cm} @{}} 
  \textbf{Integrationsproces} \\  
  <--------------------------------------------------------------------------------------------- \\ \bottomrule
\end{tabular} \end{table}
%
På baggrund af denne model har vi udover de arbejdsløse valgt, at inkludere de personer som DST betegner “midlertidigt uden for arbejdsstyrken”\footnote{Som “midlertidigt uden for arbejdsstyrken” har vi valgt at inddrage beskæftiget uden løn (\texttt{317}), orlov fra ledighed (\texttt{318}), uddannelsesforanstaltning/vejledning  og  opkvalificering (\texttt{319}), særlig/aktivering (\texttt{320}), uoplyst aktivering (\texttt{321}), sygedagpenge (\texttt{323}), revalideringsydelse (\texttt{327}), integrationsuddannelse (\texttt{333}), ledighedsydelse (\texttt{334}), aktivering  iflg. kontanthj.statistikregister (\texttt{335}), mens vi har fravalgt delvis ledighed (\texttt{316}) og Barselsdagpenge (\texttt{322}).}, “pensionister” eller “tilbagetrukket fra arbejdsstyrken”\footnote{Som “pensionister” eller “tilbagetrukket fra arbejdsstyrken” har vi valgt at inddrage efterløn (\texttt{324}), overgangsydelse (\texttt{325}), tjenestemandspension (\texttt{328}), folkepensionist (\texttt{329}) og førtidspensionist (\texttt{331}), mens vi har fravalgt flexydelse (\texttt{315}).} hvis de kommer i beskæftigelse igen og til sidst de personer som DST kalder “andre uden for arbejdsstyrken”\footnote{Som “andre uden for arbejdsstyrken” har vi valgt at inddrage kontanthjælp (\texttt{326}) og introduktionsydelse (\texttt{332}), mens vi har fravalgt uddannelsessøgende (\texttt{310}), øvrige  uden for arbejdsstyrken (\texttt{330}) og barn eller ung (d.v.s. under 16  år) (\texttt{400}).} hvis også de kommer i beskæftigelse igen.

Det betyder, at vi har inddelt danskerne i kategorierne beskæftigede og ledige. For overskuelighedens skyld har vi i tabel \ref{tab_SOCSTIL} skelnet mellem arbejdsløse og de personer uden for arbejdsstyrken, som vi mener er relevante i vores model. Tabellen inkluderer alle danskere inden for de tre grupper, og det fremgår først fra afsnit \ref{spells_runs}, at det kun er de personer som går fra at være beskæftiget til at være beskæftiget efter en mellemliggende periode med ledighed eller uden beskæftigelse. Det som tabel \ref{tab_SOCSTIL} dog viser er udviklingen i beskæftigelse og arbejdsløshed i perioden 1996 til 2009\footnote{Arbejdsløshedstallene kan eksempelvis ses i sammenhæng med lignende opgørelser fra Arbejderbevægelsens Erhversråd\parencite{Bjoersted2012}, Dansk Arbejdsgiverforening \parencite{Bang-Petersen2012} og DST \parencite{DST2014a}.}.
% hvor er de tabeller der viser overlappet mellem SOCIO og SOCTIL som vi talte om? Jeg mener at huske de arbejdede videre på dem Søren? #spmtilsoeren
%
\begin{table}[H] \centering
\caption{\texttt{SOCSTIL} omkodet i perioden 1996 til 2009. Kilde: DST}
\label{tab_SOCSTIL}
\begin{tabular}{@{}lrrr@{}} \toprule
Årstal & \multicolumn{1}{c}{Beskæftigede} & \multicolumn{1}{c}{Arbejdsløse} & Uden for arbejdsstyrken \\ \midrule
1996  & 2.598.866 & 193.672 & 798.902 \\ 
1997  & 2.632.485 & 168.991 & 795.763 \\ 
1998  & 2.680.115 & 132.179 & 796.388 \\ 
1999  & 2.691.568 & 117.689 & 802.352 \\ 
2000  & 2.705.333 & 118.520 & 788.038 \\ 
2001  & 2.716.827 & 110.501 & 791.043 \\ 
2002  & 2.676.979 & 119.250 & 814.652 \\ 
2003  & 2.643.590 & 147.666 & 818.258 \\ 
2004  & 2.652.214 & 134.586 & 829.698 \\ 
2005  & 2.696.097 & 107.734 & 828.069 \\ 
2006  & 2.761.924 & 80.270  & 815.445 \\ 
2007  & 2.796.580 & 59.860  & 816.498 \\ 
2008  & 2.725.310 & 43.895  & 874.735 \\ 
2009  & 2.617.170 & 95.756  & 918.659 \\  \bottomrule
\end{tabular} \end{table}
% 

Vi har valgt at kombinere \texttt{SOCSTIL} og \texttt{SOCIO}, fordi de indeholder definitioner af ledighed, der ligger tæt op af hinanden, men fanger forskellige aspekter. \texttt{SOCSTIL} er, som tidligere nævnt, dannet som den primære tilknytning til arbejdsmarkedet bestemt ved først at identificere de forskellige bruttobestande (tilknytninger til arbejdsmarkedet), den enkelte person indgår i ultimo november. Hvis en person indgår i mere end en bruttobestand, bestemmes den primære tilknytning til arbejdsmarkedet ud fra et sæt prioriteringsregler. Prioriteringsreglerne er fastlagt, således at de i videst muligt omfang følger ILO-retningslinierne. ILO-retningslinierne foreskriver, at \textbf{beskæftigelse skal vægtes højere end ledighed} (henvisning). \texttt{SOCIO} er dannet ud fra oplysninger om væsentligste indkomstkilde for personen, og ud fra denne fastlægges det, hvilken socioøkonomisk status vedkommende har i det år. I modsætning til \texttt{SOCSTIL} vægter \texttt{SOCIO} \textbf{ledighed højere end beskæftigelse}\footnote{I dannelsen af \texttt{SOCIO} findes først de personer, hvis hovedindkomst er efterløn og overgangsydelse (værdi 323). Derefter findes personer, som har været ledige mindst halvdelen af året (værdi 2). For de resterende personer følger \texttt{SOCIO} variablens hovedopdeling i variablen \texttt{BESKST} (beskæftigelsesstatus)}. Vi har valgt at inddele \texttt{SOCIO} ud fra samme princip som \texttt{SOCSTIL}, det vil sige i beskæftigede og ledige\footnote{Beskæftigede omfatter således ligesom \texttt{SOCSTIL} selvstændige erhvervsdrivende, medarbejdende ægtefæller og lønmodtagere (\texttt{SOCIO}=11-13, 111-114, 131-135; \texttt{SOCIO02}=111-139). Arbejdsløse afgrænses i overensstemmelse med ILOs fastlagte betegnelser, hvor kriterierne er, at arbejdsløse skal  være uden arbejde, stå til rådighed for arbejdsmarkedet og være aktivt arbejdssøgende (\texttt{SOCIO}=2; \texttt{SOCIO02}=210-220). Personer uden for arbejdsstyrken alle de personer, som ikke opfylder betingelserne for at være i arbejdsstyrken, hvilket er personer under uddannelse, pensionister mv., førtids- og folkepensionister, efterlønsmodtagere mv., andre personer og børn (\texttt{SOCIO}=31-33, 321-323, 4; \texttt{SOCIO02}=310-420).}. Som det fremgår af tabel \ref{tab_SOCIO_SOCSTIL_sammenligning} kan vi se, at \texttt{SOCSTIL} og \texttt{SOCIO} fanger forskelige aspekter ved, at de i deres binære form rammer samme indeling i 68 \% af tilfældende, mens det i 32 \% af tilfældende rammer en forkert inddeling.
% % 
% \begin{table}[H] \centering
% \caption{Sammenligning af \texttt{SOCIO} og \texttt{SOCSTIL}. Kilde: DST}
% \label{tab_SOCIO_SOCSTIL_sammenligning}
% \begin{tabular}{@{}llll@{}} \toprule
%  & & SOCSTIL &  \\ \midrule
%  & & Beskæftiget & Ledig \\ 
%  SOCIO & Beskæftigelse & & \\ 
%  & Ledig & & \\  \bottomrule
% \end{tabular} \end{table}
% % 
Her er vores to primære kilder til at se på tilknytning til arbejdsmarkedet altså ikke enige om inddelingen. Det giver os fire mulige løsninger, rangeret efter hvor restriktivt et ledighedsbegreb man ønsker at benytte.
%
\begin{description} [topsep=6pt,itemsep=-1ex]
  \item[Restriktiv] Udvælg de ledige, der defineres som sådan af både \texttt{SOCSTIL} \emph{og} \texttt{socio/SOCIO02}.
  \item[Semirestriktiv] Benyt enten \texttt{SOCSTIL} eller \texttt{SOCIO}s inddeling af ledige
  \item[Semibred] Benyt enten den ene variables inddeling, og supplere missing-værdierne med den anden variabel. \emph{... fomuler bedre}
 \item[Bred] Benyt begge variables inddeling således at hvis den ene variabel siger en person er ledig, overruler det den anden variabels bestemmelse af at vedkommende ikke er det.
\end{description}
%
Det er meget svært hvis ikke umuligt at verificere gyldigheden af enten \texttt{SOCSTIL} eller \texttt{SOCIO} som værende \emph{den helt korrekte} betegnelse, i tilfælde af tvivlsspørgsmål. Da vi arbejder med en meget bred forståelse af ledighed, og er interesseret i alle som på en eller anden måde kan karakteriseres som uden for beskæftigelse som vender tilbage til beskæftigelse igen, vælger vi at benytte den fjerde mulighed, hvor informationer fra begge variable inddrages. Vi antager, at hvis én af de to variable inddeler en person i en kategori udenfor beskæftigelse, så er det sandsynligt, at det forholder sig sådan. Det kan være, at man dermed kommer til at kategorisere en person, der i løbet af et år primært er på arbejdsmarkedet, og kun sekundært har været i kontakt med overførselsindkomster, som en person udenfor arbejdsmarkedet. Vi vælger denne løsning for at kunne udtale os bredt om dem, der i en periode har haft en løs eller ingen tilknytning til arbejdsmarkedet. 

% DST versus den virkelige verden. Et godt eksempel på relationen mellem den virkelige verden og den måde, hvor på et virkeligt menneske ender med at blive tastet som en speciel person. Et eksempel her på er et problematik i forhold til dagpengesystemet med en person som kom i karambolage med sin a-kasse. Årsagen til at det er et relevant eksempel er jo netop, fordi at mange DST henter mange informationer fra A-kasse. Vi har at gøre med en person - et virkeligt eksempel fra 2015, som arbejder 20 timer om ugen som kommunikationsmedarbejder i Frode Laursen (lastbilselskab) samtidig med at personen er selvstændig og er ejer af et interessantskab. For at genoptjene retten til dagpenge skal personen arbejde 30 timer om ugen som lønmodtager og det at arbejde i og være ejer af et interessentskab opgør ikke for det. Dette medfører en lang række problemer for denne person registreres i en a-kasse og har konsekvenser for denne person arbejde og ret til dagpenge. Det har også konsekvenser for hvordan denne person indtastes i DST. Eksempelvis kan denne person både registreres som i beskæftigelse (lønmodtager SOCISTIL i deltid som arbejder med kommunikation DISCO inden for lastbilbranchen NACE), selvstændig (selvstændig SOCSTIL i deltid som arbejder med kommunikation DISCO inden for kommunikationsbranchen NACE) eller til sidst som uden for beskæftigelse (enten som dagpengemodtager, delvis ledig eller noget tredje SOCSTIL enten med eller uden DISCO og NACE). DST og A-kasserne kan nemlig ikke to eller flere  informationer og udvælger en som den primære funktion. Det kan være, fordi indkomst mv.



%%%%%%%%%%%%%%%%%%%%%%%%%%%%%%%%%%%%%%%%%%%%%%%%%%%%%%%%%%%%%%%%%%%%%%%%%%%%%%%%%%%%%%%%%%%%%%%%%%%
%%%%%%%%%%%%%%%%%%%%%%%%%%%%%%%%%%%%%%%%%%%%%%%%%%%%%%%%%%%%%%%%%%%%%%%%%%%%%%%%%%%%%%%%%%%%%%%%%%%
%%%%%%%%%%%%%%%%%%%%%%%%%%%%%%%%%%%%%%%%%%%%%%%%%%%%%%%%%%%%%%%%%%%%%%%%%%%%%%%%%%%%%%%%%%%%%%%%%%%


\section{Spells \& runs \label{ledig_spellsrun}} 

For at skabe en datastruktur der ville give at mulighed for at undersøge perioder med ledighed har vi stået over for en udfordring. I modsætning til Larsen og Toubøls anvendelse af MONECA i forbindelse med social mobilitet blandt alle jobskift, står vi med det særlige benspænd, at der kan gå kort eller lang tid mellem at personer i vores data får nyt arbejde. Vi kan derfor ikke tælle skift per år, men bliver nødt til at lave en struktur, der tillader os at kollapse ledighedsperioden dynamisk således, at vi kan se hvilket job man gik fra og til uanset længden på ledighedsperioden. For at gøre dette, har vi som tidligere beskrevet reduceret informationsmængden i DSTs aggregerede ledighedsvariable til en binær variabel. Ved at skabe en sådan klar stop/start-indikator på ledighedsperioder, i kombination med en paneldatastruktur, kan vi ved kodning ved hjælp af indekseringsprogrammering\footnote{Det vil sige: skabe nye variable og lave beregninger baseret på værdier relativt til en given observations \emph{placering} i data, fremfor givne \emph{karakteristika} ved observationer.} opnå en struktur der viser skift, uagtet længden af ledighedsperioderne\footnote{Længden af ledighedsperioderne er naturligvis af stor analytisk interesse, men benyttes først på et senere trin i analysen.}. Det betyder, at vi - før nogen form for sortering - har 5.860.440 mennesker observeret over 14 år svarende til 82.046.160 observationer. Tabel \ref{tab_spellrun} er et illustrativt eksempel på denne struktur. 
%
\begin{table}[H]
\centering
\caption{Eksempel på datastruktur. Kilde: DST}
\label{tab_spellrun}
\resizebox{\textwidth}{!}{%
\begin{tabular}{@{}clrrc@{}}
\toprule
ID nummer & \multicolumn{1}{c}{År} & \multicolumn{1}{c}{SOCSTIL / SOCIO} & \multicolumn{1}{c}{DISCO-beskæftigelseskategori}      & Ledig \\ \midrule
7384973       & 1996                   & Lønmodtagere på grundniveau         & Bager og konditorarbejde (eksklusiv industri)         & Nej   \\
7384973       & 1997                   & Revalideringsydelse                 & -                                                     & Ja    \\
7384973       & 1998                   & Revalideringsydelse                 & -                                                     & Ja    \\
7384973       & 1999                   & Revalideringsydelse                 & -                                                     & Ja    \\
7384973       & 2000                   & Revalideringsydelse                 & -                                                     & Ja    \\
7384973       & 2001                   & Lønmodtager på mellemniveau         & Pædagogisk arbejde                                    & Nej   \\
7384973       & 2002                   & Kontanthjælp                        & \textit{(Pædagogisk arbejde)}                         & Ja    \\
7384973       & 2003                   & Lønmodtager uden nærmere angivelse  & Pædagogisk arbejde                                    & Nej   \\
7384973       & 2004                   & Lønmodtagere på grundniveau         & Operatør- og fremstillingsarbejde i næring og nydelse & Nej   \\
7384973       & 2005                   & Lønmodtagere på grundniveau         & Operatør- og fremstillingsarbejde i næring og nydelse & Nej   \\
7384973       & 2006                   & Lønmodtagere på grundniveau         & Operatør- og fremstillingsarbejde i næring og nydelse & Nej   \\
7384973       & 2007                   & Lønmodtagere på grundniveau         & Operatør- og fremstillingsarbejde i næring og nydelse & Nej   \\
7384973       & 2008                   & Lønmodtagere på grundniveau         & Operatør- og fremstillingsarbejde i næring og nydelse & Nej   \\
7384973       & 2009                   & Lønmodtagere på grundniveau         & Operatør- og fremstillingsarbejde i næring og nydelse & Nej   \\ \bottomrule
\end{tabular} }
\end{table}
%
Vi har at gøre med et enkelt panel, det vil her sige den samme anonymiserede person gennem 14 år. Det ses, at vedkommende i 1996 arbejder med bageri- eller konditorrelateret arbejde. Vedkommende er kategoriseret som lønmodtager på grundniveau i vores aggregerede beskæftigelsesvariabel \texttt{SOCSTIL/SOCIO}, hvilket betyder at han i vores binære ledighedsvariabel har et negativt udfald. Det kan konstateres, at han i 1997 tildeles en revalideringsydelse, som han er på de næste fire år. I vores optik er han derfor i denne periode “ledig”. Revalideringsydelsens formål er, ifølge Bekendtgørelsen om aktiv socialpolitik, “(...) \emph{at en person med begrænsninger i arbejdsevnen, herunder personer, der er berettiget til ledighedsydelse og særlig ydelse, fastholdes eller kommer ind på arbejdsmarkedet, således at den pågældendes mulighed for at forsørge sig selv og sin familie forbedres.}” (\textcite{lov_revalidering}).

Efter fire år på denne ydelse bliver vedkommende ansat inden for pædagogisk arbejde.  Året efter ender han på kontanthjælp, men kommer tilbage til det pædagogiske arbejde i 2003. I 2004 skifter han til beskæftigelseskategorien \emph{Operatør- og fremstillingsarbejde i næring og nydelse}. Dette job forbliver han i frem til panelets sidste observation i 2009\footnote{Mens denne person modtog revalideringsydelse og kontanthjælp, ville han blive af blandet andet DST og Beskæftigelsesministeriet blive kategoriseret som uden for arbejdsstyrken, men netop, fordi han vender tilbage til beskæftigelse igen, kommer han med i vores analyseudvalg og karakteriseres som “ledig” i denne periode.}. Derfor vil denne person blive registreret med to skift i vores mobilitetstabel: ét skift fra \emph{Bager- og konditorarbejde} til \emph{Pædagogisk arbejde}, og et andet fra \emph{Pædagogisk arbejde} til \emph{pædagogisk arbejde}. Det efterfølgende skift til fremstillingsarbejde i næringsindustrien medtages ikke, da han ikke har en periode med (registreret) ledighed ind i mellem. Vi mister en central information om denne person, da denne tilbagevending til hårdere fysisk arbejde indenfor madfremstilling er et vigtigt skifte tilbage til den type job, som manden havde i 1996, i det bager- og konditorrelaterede arbejde. Det hører med til historien, og er grunden til vi... %blah blah blah argument for at lave sekvensanalyse / en eller anden form for livsbane analyse #todo.

Eksemplet tjener også til at illustrerer noget andet centralt. Det ses, at personen i 2002 var på kontanthjælp, og dog havde han en \texttt{DISCO}-værdi tilknyttet. Det skal forstås sådan, at en inddeling af et menneskes arbejdsliv, baseret på en årsinddeling, grundlæggende er en kunstig inddeling, der ikke kan indfange den kontinuitet, livet leves i. En sådan årsinddeling har ofte en vis berettigelse, eftersom det er grundlag for en lang række adminstrative inddelinger, med meget reelle sociale konsekvenser. Ikke desto mindre kan man sagtens være kontanthjælpsmodtager og have en en, to eller flere jobs i løbet af samme år, og det er en kompleksitet, vi indenfor det enkelte år er tvunget til at reducere til en samlet vurdering af, hvad vedkommende hovedsageligt lavede i løbet af året. Som beskrevet tidligere er dannelsen af \texttt{DISCO}-variablen en kompliceret proces, hvor den endelige beskæftigelsesværdi er sammensat ud fra mange forskellige kilder og kriterier. Informationen til dannelsen af \texttt{DISCOALLE\_INDK} er primært sket ud fra det arbejdssted, hvor de har fået størst lønindkomst gennem året. Der er ingen vurdering af hvor lang en ansættelse, der er tale om. En ledighedsvariabel baseret på hvorvidt man har eller ikke har et udfald i \texttt{DISCOALLE\_INDK}, ville derfor være ekstremt upålidelig, og ved at teste data igennem for konsistensen mellem \texttt{DISCOALLE\_INDK} og \texttt{SOCIO/SOCSTIL} var den præget af uacceptabelt mange forskelle i forhold til sidstnævnte. Det forklarer hvorfor personens ledighedsstatus i føromtalte panel ikke harmonerer mellem de to. I \texttt{SOCSTIL} er han sat med beskæftigelsesværdien \emph{Lønmodtagere på grundniveau}, mens han i \texttt{SOCIO} er kategoriseret som på kontanthjælp. \texttt{SOCSTIL} understøtte dermed \texttt{DISCOALLE\_INDK}, mens \texttt{SOCIO} ikke gør det. What to do? Vores vurdering i dette for det videre arbejde helt centrale spørgsmål, har været følgende: Det er sandsynligt, at personen både har haft et arbejde og har været på kontanthjælp i 2002. Derfor mener vi netop, at hvis den ene af de to variable kategoriserer ham som kontanthjælpsmodtager som primær socioøkonimisk status i 2002, bør vi vurdere ham som ledig i år 2002 - eller i hvert fald \emph{primært} som ledig.

Nedenstående tabel viser hvor mange tvivlstilfælde, der er tale om. 



% kom med tal på hvor mange tvivlstilfælde der er tale om.


% som nævnt er det også muligt at have en \texttt{DISCO}-værdi selvom både \texttt{SOCIO} og \texttt{SOCSTIL} mener man ikke er i en beskæftigelseskategori. Det betyder sandsynligvis at der er tale om et job, der ikke fylder meget i forhold til de forskellige overførselsindkomster, som de to aggregerede variable baserer sig på. Det forekommer derfor rimeligt at ignorere denne beskæftigelse.}
% % 
% \begin{table}[H] \centering
% \caption{Antallet af ledige i perioden 1996 til 2009. Kilde: DST}
% \label{tab_SOCSTIL}
% \begin{tabular}{@{}lll@{}} \toprule
% Årstal & Vores ledige i ledighedsperiode & Vores ledige i beskæftigelse \\ \midrule
% 1996  & 0 & ? \\ 
% 1997  & ? & ? \\ 
% 1998  & ? & ? \\ 
% 1999  & ? & ? \\ 
% 2000  & ? & ? \\ 
% 2001  & ? & ? \\ 
% 2002  & ? & ? \\ 
% 2003  & ? & ? \\ 
% 2004  & ? & ? \\ 
% 2005  & ? & ? \\ 
% 2006  & ? & ? \\ 
% 2007  & ? & ? \\ 
% 2008  & ? & ? \\ 
% 2009  & 0 & ? \\  \bottomrule
% \end{tabular} \end{table}
% % 



%Local Variables: 
%mode: latex
%TeX-master: "report"
%End:
% \input{tex/4.0_teori_om_segmenter}
% \input{tex/5.0_metode_om_netvaerksanalyse_og_moneca}
% \input{tex/6.0_metode_om_disco}
\input{tex/7.0_analyse_hovedkort}
% \input{tex/7.1_analyse_faglaerte_og_ufaglaerte}
% \input{tex/7.2_analyse_akademikerne}
% \input{tex/8.0_baggrund}
% \input{tex/9.0_diskussion}
% \input{tex/10.0_konklusion}


%%% -------------- Appendiks ----------------- %%%
\begin{appendices}
\setcounter{secnumdepth}{3}
% \input{tex/appendiks_figurer}
% \input{tex/appendiks_metode_arbejdsloeshed_og_skift}
% \input{tex/appendiks_metode_disco}
% \input{tex/appendiks_metode_baggrundsvariable}


\end{appendices}

%%% -------------- Bibliografi ----------------- %%%
% biblatex definerer alt i preamblet
%\nocite{*} % brug denne for at tage referencer med der ikke citeres i teksten
% \printbibliography

\chapter{BIBLIOGRAFI} \label{biblio}

\printbibliography[heading=subbibliography,title={Hovedbibliografi},filter=notonline]
\printbibliography[heading=subbibliography,title={Online kilder},type=online]
% \printbibliography[heading=subbibliography,title={DST manualer},filter=DST]
\printbibliography[heading=subbibliography,title={Danmarks Statistiks Manualer},type=manual]


%%% -------------- slut det hele er slut ----------------- %%%
\end{document}




