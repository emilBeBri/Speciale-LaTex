%!TEX TS-program = pdflatex
%!TEX TS-program = pdflatex

%%% -------------- Preamble ----------------- %%%

%!TEX root = ./report.tex
%-*- coding: utf-8 -*-
%!TeX encoding = UTF-8


%%%%%%%%%%%%%%%%%%%%%%%%%%%%%%%%%%%%%
% ----------- Preamble ------------ %
%%%%%%%%%%%%%%%%%%%%%%%%%%%%%%%%%%%%%



% ----------- dokumentclass ------------ %

\documentclass[11pt,a4paper,report,danish,oneside]{memoir} % for a short document
%brug article til kortere opgaver, hvor en af effekterne er at chapters og sections har samme status


% ----------- Font og sprog ------------ %

\usepackage[utf8]{inputenc} % set input encoding to utf8
\usepackage[T1]{fontenc}
\usepackage{kpfonts}
\usepackage[danish]{babel} %dansk sprog
\renewcommand{\danishhyphenmins}{22} % gør ordopdeling bedre men den er lort i forvejen så hvor slemt kan det være hvis det her er det bedste?! Prøv at eksperimentér med det der tal, 22, se hvad der sker.


% \usepackage[avantgarde]{quotchap}   % Pænere kapitler

\usepackage[scaled]{helvet}
\renewcommand\familydefault{\sfdefault} 
% \usepackage[T1]{fontenc}


% ----------- Sidelayout------------ %
% Set up the paper to be as close as possible to both A4 & letter:
\setulmarginsandblock{3cm}{3cm}{*} % 50pt upper margins
\setlrmarginsandblock{3cm}{3cm}{*} % golden ratio again for left/right margins
\checkandfixthelayout 
% Ovenstående er fra the memoir manual

\pagestyle{ruled} % try also: empty , plain , headings , ruled , Ruled , companion

%\SingleSpacing
\OnehalfSpacing
%\DoubleSpacing
%\setstretch{1.1}
\setcounter{secnumdepth}{2}
\setcounter{tocdepth}{2}

%indholdsfortegnelse
\renewcommand\cftappendixname{\appendixname~} %gør at der står "bilag A" i stedet for bare "A"

% fjerner sidenr fra \part-sider
\makeatletter
\renewcommand\part{%
  \if@openright
    \cleardoublepage
  \else
    \clearpage
  \fi
  \thispagestyle{empty}%
  \if@twocolumn
    \onecolumn
    \@tempswatrue
  \else
    \@tempswafalse
  \fi
  \null\vfil
  \secdef\@part\@spart}
\makeatother




% Ændre skrifttypen:
% \renewcommand{\familydefault}{\ttdefault} % typewriter font i hele documentet, men fucker med linjebrydningen/hypernation, ved ikke hvorfor.


% ----------- Diverse pakker - rækkefølgen er IKKE ligegyldig ------------ %

%\usepackage{longtable}  For at kunne lave tabeller fx i Stata der rækker over flere sider

\usepackage[table,xcdraw]{xcolor} % farver i tabeller
\usepackage{enumitem} %en ekstra slags liste 
\usepackage{float} % for at kunne placere en floater PRÆCIS her med \begin{figure}[H]
\usepackage{pdfpages} % understøtter inkluderingen af pdf-filer
\usepackage[plainpages=false,pdfpagelabels,pageanchor=false,hidelinks]{hyperref}   % aktive links
\usepackage{memhfixc}       % rettelser til hyperref - ved ikke hvad den gør
\usepackage{booktabs} %fancy tabeller
\usepackage{tabularx} % til statas tabout
\usepackage{multirow} % flere rækker i tabeller
\usepackage{graphicx} % indsæt billeder
\usepackage{rotating} %tillader figurer der står sidelæns
\usepackage{csquotes}
\usepackage{enumitem} %mindre mellemrum mellem items
\usepackage{dashrule} %stiblede linier i tabeller
%\usepackage{comment} % kunne sætte sectioner er ikke helt overbevidst om den her er brugbar, måske bare ifelse-løsningen
\usepackage{nameref} %giver mulighed for at refere til chapters etc med navn i stedet for bare en counter eller sidetal. 
% \usepackage{bytefield} % til at skabe datafigurer, se http://tex.stackexchange.com/questions/108257/implementing-a-way-to-show-data-structures-in-latex
\usepackage[normalem]{ulem} %striketrough
\usepackage{array} %centering 
\usepackage{wallpaper}
\usepackage{wrapfig} % wrap figure
\usepackage[font=small]{caption}

% Test pakker
\usepackage{appendix} %[toc,page]
\usepackage{capt-of} % til at sættte figurer og tabeller side-by-side
\usepackage[table,xcdraw]{xcolor} % farver til tabeller
\usepackage{courier} %sætter \texttt{} command til courier i stedet for den anden. courier er angiveligt bedre fordi den visuelt er mere forskellig fra standardfonten.
\usepackage{tcolorbox}
% \usepackage{siunitx}
% \sisetup{locale = DA}
\usepackage{multirow} 
\usepackage{geometry}
\usepackage{nicefrac}
\usepackage{flowchart}
\usetikzlibrary{arrows}
% \usepackage{pgfplotstable}


% ----------- Bibliografi ------------ %



%%bibliografi med biblatex
\usepackage[citestyle=authoryear-ibid, sorting=nty,backend=biber,bibencoding=UTF8,maxcitenames=2,defernumbers=true]{biblatex}
%% options til ovenstaaende:
%% maxcitenames=x styrer antal af forfattere før der står et.al/m.fl.
%% uniquelist=false/true styrer om den alligevel skal skrive flere navne end specifieret i maxcitenames i tvivlspørgsmål, dvs. hvis en forfatter optræder i flere referencer med andre forfattere.


%\usepackage[style=authoryear,backend=biber]{biblatex}
%bibtex-fil

%lave filtre så bibliografien kan deles op
% \defbibfilter{notonline}{
%   not type=Electronic, 
%   % and not type=manual
%   not keyword={DSTmanual}
% }

% \defbibfilter{DSTmanual}{%
% keyword=DSTmanual %keyword er et felt i en bib-entry
% }

% \defbibfilter{onlinekilder}{%
% type={Electronic}, 
% notkeyword={DSTmanual}
% }



\addbibresource{biblio/bibliografi.bib}



% ----------- Floaters ------------ %


%%% Ikke sikker på effekten af følgende. Måske slet, eksperimenter dig frem når du står i en situation hvor du kan se floaters opfører sig underligt.
%Alter some LaTeX defaults for better treatment of figures:
    % See p.105 of "TeX Unbound" for suggested values.
    % See pp. 199-200 of Lamport's "LaTeX" book for details.
    %   General parameters, for ALL pages:
    \renewcommand{\topfraction}{0.9}	% max fraction of floats at top
    \renewcommand{\bottomfraction}{0.8}	% max fraction of floats at bottom
    %   Parameters for TEXT pages (not float pages):
    \setcounter{topnumber}{2}
    \setcounter{bottomnumber}{2}
    \setcounter{totalnumber}{4}     % 2 may work better
    \setcounter{dbltopnumber}{2}    % for 2-column pages
    \renewcommand{\dbltopfraction}{0.9}	% fit big float above 2-col. text
    \renewcommand{\textfraction}{0.07}	% allow minimal text w. figs
    %   Parameters for FLOAT pages (not text pages):
    \renewcommand{\floatpagefraction}{0.7}	% require fuller float pages
	% N.B.: floatpagefraction MUST be less than topfraction !!
    \renewcommand{\dblfloatpagefraction}{0.7}	% require fuller float pages
	% remember to use [htp] or [htpb] for placement


% ----------- Egne kommandoer ------------ %

% \newcommand{\5.1}{5.1} (bruges ikke længere men for eksemplets skyld)

% \newcommand{\antalkat}{150} (bruges ikke længere men for eksemplets skyld)

% \fref og venner er defineret saaledes i memoir

% \renewcommand{\fref}[1]{\figurerefname~ef{#1}} 
% \renewcommand{\tref}[1]{\tablerefname~ef{#1}} 
% \renewcommand{\pref}[1]{\pagerefname~\pageref{#1}} 
% \renewcommand{\Pref}[1]{\partrefnameef{#1}} 
% \renewcommand{\Cref}[1]{\chapterrefnameef{#1}} 
% \renewcommand{\Sref}[1]{\sectionrefnameef{#1}} 

% og de navne som er anvendt er defineret som det her, dvs her kan man ændre om det er på dansk eller engelsk:

% \renewcommand*{\figurerefname}{Figur} 
% \renewcommand*{\tablerefname}{Tabel} 
\renewcommand*{\pagerefname}{side} 
% \renewcommand*{\partrefname}{Part~} 
% \renewcommand*{\chapterrefname}{Chapter~} 
% \renewcommand*{\sectionrefname}{\S} 

% \renewcommand*{\pref}{side} 


%Local Variables: 
%mode: latex
%TeX-master: "report"
%End: 







%%% -------------- Dokument ----------------- %%%

\begin{document} \selectlanguage{danish}


% %%% -------------- forside ----------------- %%%
% \begin{titlingpage}
% % KU wallpaper
% \ThisLRCornerWallPaper{1}{fig/forside/samf-farve.pdf}
% % start
% % newcommand{\HRule}{\rule{\linewidth}{0.5mm}} % Defines a new command for the horizontal lines, change thickness here
% \center % Center everything on the page
% %Heading sections
% \begin{minipage} {1\textwidth}
% \begin{flushleft} \large
% \textsc{Københavns Universitet} 
% \end{flushleft}
% \end{minipage}
% \vspace{4,5cm}
% %Title section
% \HRule \\[0.4cm]
% { \huge \bfseries \textsc{Veje til Beskæftigelse}}\\[0.4cm] % Title of your document
% { \large \bfseries \textsc{En kortlægning af arbejdsmarkedsparates beskæftigelsesmobilitet på det danske arbejdsmarked fra 1996 til 2009}\\[0.4cm] % Title of your document
% \HRule \\[1.5cm]
% \vspace{12cm}
% %Authour section
% \begin{minipage}{1\textwidth}
% \begin{flushleft} \large
% Emil Erik Pula Bellamy \textsc{Begtrup-Bright} 
% \newline Søren \textsc{Nielsen-Gravholt} 
% \end{flushleft}
% \end{minipage}
% \end{titlingpage}

% %%% -------------- kolofon ----------------- %%%
% \begin{titlingpage}
% \begin{minipage}{1\textwidth}
% \vspace{11cm}
% \begin{flushleft} \large
% Sociologisk Institut, Københavns Universitet
% \newline September 2015
% \end{flushleft}
% \vspace{0,5cm}
% \begin{flushleft} \large
% \emph{Forfattere:}\\
% Emil Erik Pula Bellamy Begtrup-Bright
% \newline Søren Nielsen-Gravholt
% \end{flushleft}
% \begin{flushleft} \large
% \emph{Vejleder:}\\
% Jens Arnholtz
% \end{flushleft}
% \vspace{0,5cm}
% \begin{flushleft} \large
% \emph{Antal tegn:} xxx
% \newline \emph{Antal tegn i fodnoter:} xxx
% \end{flushleft}
% \end{minipage}
% \end{titlingpage}


%%% -------------- Indholdsfortegnelse ----------------- %%%

\pagenumbering{roman} %til preface etc.
\tableofcontents*		% Indholdsfortegnelse the asterisk means that the contents itself isn't put into the ToC
\newpage
%\thispagestyle{empty}
%\listoffigures
%\newpage 
%\listoftables"C:\\texlive\\2011\\bin\\win32;$PATH"
\pagenumbering{arabic}
%måske ikke nødvendigt


%%% -------------- Afsnit ----------------- %%%

\chapterstyle{southall} %hvordan kapitler skal se ud.
%Nogle valgmuligheder, se flere i the memoir manual:
% - madsen thatcher verville southall ger companion

%!TEX root = ../report.tex

%%%%%%%%%%%%%%%%%%%%%%%%%%%%%%%%%%%%%%%%%%%%%%%%%%%%%%%%%%%
\chapter{Indledning \label{indledning}}
%%%%%%%%%%%%%%%%%%%%%%%%%%%%%%%%%%%%%%%%%%%%%%%%%%%%%%%%%%%

Dette speciale er en undersøgelse af hvorvidt arbejdsmarkedet er delt op i flere forskellige segmenter, og i så fald, hvordan en sociologisk analyse kan belyse disse segmenters mulige forskellige funktionsmåder.

Hvorfor er det interessant? Der meget lidt sociologisk teori, der beskæftiger sig med arbejdsmarkedets funktionsmåder på et mesoniveau, og samtidig er åben overfor en empirisk kortlægning. For det meste beskæftiger den arbejdsmarkedssociologiske teori (mig bekendt) enten med et kvalitativt orienteret mikroniveau, eller de helt brede penselstrøg på makroniveau.

Hvorfor er det vigtigt med undersøgelse af arbejdsmarkedet på et mesoniveau, når der laves undersøgelser på et mikro- og makroniveau? Viden om arbejdsmarkedet på et mikroniveau giver et billedet af et bestemt og snævert sted på arbejdsmarkedet og altså ikke viden hele arbejdsmarkedet. Viden om arbejdsmarkedet på et makroniveau giver et billede af abstrakte lovmæssigheder, der skal forklare en eller anden tendens i moderne samfund generelt. Dette er ikke en ny kritik (se GER og Merton find henvisninger \#todo), men noget, der tilsyneladende alligevel ofte glemmes. Med en undersøgelse på mesoniveau fås et billede af hvordan arbejdsmarkdet i et givet land fungerer lige nu.

Den teori, der bedst egner sig til en sådan undersøgelse, er (for mig at se) arbejdsmarkedssegmenteringsteorierne, med elementer fra moderne empirisk klasseforskning. Arbejdsmarkedssegmenteringsteorierne blev udviklet i 1970'erne, men er kun i begrænset omfang er blevet benyttet de sidste 20 år. Denne afhandling vil benytte denne teoriretning, med udgangspunkt i registerdata fra Danmarks Statistik i perioden fra 1996 til 2009.


%%%%%%%%%%%%%%%%%%%%%%%%%%%%%%%%%%%%%%%%%%%%%%
\section{Problemformulering}
%%%%%%%%%%%%%%%%%%%%%%%%%%%%%%%%%%%%%%%%%%%%%%
%
Med udgangspunkt i arbejdsmarkedssegmenteringsteorien, moderne klasseteori og empiri fra registerdata i Danmarks Statistik, har dette speciale følgende problemformulering:
%
\vspace{\baselineskip}
%
\begin{tcolorbox}[title=\textbf{Problemformulering}]
Findes der segmenter på det danske arbejdsmarked, og hvordan kan forskelle i sociale processer være med til at forklare sådanne forskelle i segmentstrukturen?
\end{tcolorbox}
%
\vspace{\baselineskip}
Det vil jeg undersøge med følgende forskningsspørgsmålspørgsmål:
\vspace{\baselineskip}
\begin{tcolorbox}[title=Forskningspørgsmål,
subtitle style={boxrule=0.4pt} ]
	\tcbsubtitle{1.} Er der en opdeling af arbejdsmarkedet for arbejdstagere i delmarkeder, hvor mobilitet indenfor delmarkederne er hyppig, og mellem delmarkederne sjælden?
	\tcbsubtitle{2.} Kan forskelle i de sociale processer vise, at der er tale om segmenter, og ikke blot delmarkeder?
	\tcbsubtitle{3.} Kan klasseteori belyse denne segmentering?
\end{tcolorbox}

Teorier med blik for segmenteringsprocesser på arbejdsmarkedet har en række forskellige indfaldsvinkler: det, der primært binder dem sammen, er ideen om at arbejdsmarkedet mere er karakteriseret ved \emph{opdeling} end af \emph{enshed}, hvilket ellers ville være det basale præmis i neoklassisk økonomisk teori. Dette speciale læner sig op af Thomas Bojes udformning af dette præmis \parencite[174]{Boje1986}.


%%%%%%%%%%%%%%%%%%%%%%%%%%%%%%%%%%%%%%%%%%%%%%
\subsection{Forskningsspørgsmål 1: Opdeling af arbejdsmarkedet i delmarkeder}
%%%%%%%%%%%%%%%%%%%%%%%%%%%%%%%%%%%%%%%%%%%%%%

For Thomas Boje er det første kriterie for arbejdsmarkedssegmentering, at arbejdsmarkedet er delt op i delmarkeder, med begrænset mobilitet mellem de enkelte delmarkeder. Det betyder, at der i mellem visse typer jobs forekommer hyppige skift, og andre jobtyper, hvor der sjældent, eller aldrig, observeres skifte fra det ene til det andet.

Delmarkeder i denne tradition har ofte tilknyttet en forståelse af samfundet som opdelt i forskellige sociale klasser, med forskellige strukturelle livsbetingelser for individerne i dem. Det er denne afhandlings formål at benytte en sådan tilgang, for at vurdere om den kan belyse arbejdsmarkedets struktur.

En klassebaseret forståelse har været genstand for sociologiske analyser siden sociologiens første store teoretikere, Weber, Marx og Durkheim%
%
\footnote{Durkheim er ikke bredt kendt som klasseteoretikere, men elementer i hans sociologi, samt især videreudviklinger af denne, har disse elementer i sig (citer Harrits \#todo)}%
%
. Weber var en af de af de første til at se på forholdet mellem klasse og mobilitet, og til at beskrive klasser ud fra de sociale mobilitetsmønstre, individerne var en del af. Han definerer klasse således: “\emph{A »social class« makes up the totality of those class situations within which individual and generational mobility is easy and typical.}” \parencite[302]{Weber1978}. Weberiansk orienterede sociologer såsom Goldthorpe (henvisning \#todo) og Oesch ( henvisning \#todo) har bibeholdt dette fokus på social mobilitet i nyere forskning. Spørgsmålet om både intra- og intergenerationel mobilitet på arbejdsmarkedet, som det observeres empirisk, har en væsentlig rolle at spille i klasseforskningen både indenfor den marxistiske, weberianske og durkheimianske tradition (henvisning: Wright, Goldthorpe, Grusky).

Den marxistiske tradition er optaget af klasser som et spørgsmål om udbytning, baseret på et socialt forhold til produktionsmidlerne. Nymarxistiske sociologer som Olin-Wright benytter også en stratifikation af lønmodtagerne, hvor positioner på arbejdmarkedet, forstået som arbejderklassens interne sammensætning, også har stor betydning for den faktiske fordeling af goder, og forskellige individers mulighed for opnåelse af disse goder. Arbejdsmarkedet er med andre ord også her delt op ud fra andre kriterier end blot arbejdsgiver-arbejder dikotomien. Den faktiske klassestruktur, udover den grundlæggende modsætning mellem arbejde og kapital, skal undersøges empirisk (henvisning til Wright \#todo). Den weberianske og durkheimianske tradition har ikke samme fokus på blah blah (skriv måske ud: tvivlsomt om afsnittet skal være sådan her? \#todo)

Af nyere sociologisk teori kan desuden nævnes Pierre Bourdieus sammensmeltning og nytænkning af marxistisk og weberiansk funderet teori, i hans forståelse af samfundet som et socialt rum, opdelt i felter, der opererer ud fra forskellige sociale logikker (\#henvisning). Anthony Giddens understreger nødvendigheden af empirisk funderede analyser, for at forstå klassestrukturen i et bestemt samfund, og de muligheder individet har, alt efter "hvor det kommer fra" \parencite[48,110]{Giddens1973}. (inkluder eventuelt Gruskys "mikroklasser")

Indenfor arbejdsmarkedssegmenteringsteorierne har man i en amerikansk optik beskæftiget sig med det såkaldte "dual labour market", det vil arbejdsmarkedet opdelt i to overordnede delmarkeder, hvor det primære indeholder faste stillinger med tryghed i ansættelse og favorable lønninger og gode arbejdsvilkår, hvorimod det sekundære delmarked består af midlertidige ansættelser til lavere lønninger og dårligere arbejdsvilkår \parencite{Piore1980}. Boje karakteriserer denne opdeling et amerikansk fænomen, da en række institutionelle forhold samt en produktionsstruktur, med få, store virksomheder i landet, har skabt denne struktur, og kan ikke generaliseres til lande med andre instutionelle forhold. Dette understreger behovet for en analyse af hvilke danske delmarkeder, der findes. Her finder Boje, at det danske samfund består af langt flere delmarkeder, blandt andet på grund af kollektive overenskomster, sikring af arbejdsløshedsunderstøttelse, samt den langt større rolle, små og mellemstore firmaer spiller, hvilket skaber en anden social dynamik \parencite[36]{Boje1985}

I Danmark er en nyere kortlægning af arbejdsmarkedets delmarkeder allerede påbegyndt, ligeledes med udgangspunkt i arbejdsmarkedssegmenteringsteorien, af Toubøl og Grau Larsen \parencite{Touboel2013}, ved brug af social netværksanalyse. Denne afhandling benytter sig af den af Toubøl og Grau Larsen nyligt udviklede metode til at finde kliker i et socialt netværk, til at finde segmenter i det danske arbejdsmarked, baseret på mobiliet.  

%%%%%%%%%%%%%%%%%%%%%%%%%%%%%%%%%%%%%%%%%%%%%%
\subsection{Forskningsspørgsmål 2: Forskellen på et delmarked og et segment er påvisningen af særegne (specifikke? partikulære) sociale processer indenfor delmarkedet}
%%%%%%%%%%%%%%%%%%%%%%%%%%%%%%%%%%%%%%%%%%%%%%

Det er imidlertidig ikke nok at påvise høj intern mobilitet for at kunne påstå at der er tale om et selvstændigt segment. Det skal være muligt at påvise forskelle i de sociale processer, som findes i delmarkedet i forhold til andre delmarkeder. Et delmarked, hvor forskellen i mobilitet primært skyldes faglige eller geografiske forskelle, men andre væsentlige sociale processer ellers er ens, kan ikke karakteriseres som et segment \parencite[41]{Boje1985}. Det er bare et delmarked, da allokeringen af arbejdskraft i væsentligt grad sker uden (determinerende) sociale stratifikationsmekanismer. Hvis eksempelvis løndannelse og kønsforskelle viser sig tydeligt mellem to ellers sammenlignelige delmarkeder, kan man begynde at tale om segmenter i Bojes forstand. Sammenlignelige skal her forstås som, at uddannelseslængde og 

Sociale processer er et noget abstrakt begreb for den praksis, hvori livet på arbejdsmarkedet udspiller sig for den enkelte lønmodtager. Det teoretiske indhold, samt naturligvis empiriske målbarhed, vil blive udpenslet efterfølgende, her skal blot nævnes to kortfattede perspektiver på sådanne processer.

Frank Parkin definerede i slut 60'erne begrebet \emph{social lukning} som en måde at forstå opdeling mellem sociale grupper, baseret på forhåndenværende distinktionskriterier i et samfund. Begrebet bruges her til at definere de processer, hvorpå forskellige faggrupper sørger for at beskytte egne privilegier, på en sådan vis at det ekskluderer andre, og hæmmer mobiliteten ud fra andre hensyn end (åbenlyst) faglig eller uddannelsesmæssige \parencite{Parkin1994}. 
% Mark  Granovetters benytter social netværksteoris metodik(?? \#todo) om svage og stærke sociale forbindelser for at forstå social lukning. Hans hovedargument er individernes mulighed mobilitet af nye kanaler, gennem deres såkaldte "svage forbindelser" - det er igennem de mennesker, man ikke kender så godt, at der er adgang til nye muligheder indenfor for eksempel arbejdslivet \parencite{Granovetter1973}. (\emph{de her to eksempler skal strammes op, I know - E }) Giver ikke rigtig så god mening her. 



%%%%%%%%%%%%%%%%%%%%%%%%%%%%%%%%%%%%%%%%%%%%%%
\subsection{Forskningsspørgsmål 3:  Et klasseteoretisk perspektiv på segmentering}
%%%%%%%%%%%%%%%%%%%%%%%%%%%%%%%%%%%%%%%%%%%%%%

Her adskiller min afhandling sig væsentligt fra Toubøl \& Grau Larsen, Nielsen-Gravholt og Boje, da dette speciale udover en beskrivelse af delmarkederne, har et klasseteoretisk perspektiv på segmentering og de sociale processer der definerer dem, der derefter undersøges empirisk. Formålet er at bevæge sig over i et klasseteoretisk perspektiv på segmentering, for at undersøge klassebegrebets anvendelig i at forstå disse sociale processer, som jeg ser dem komme til udtryk i min empiri. 

Forbindelsen til segmenteringsteori er ikke kontroversiel, og teoriretningen er bestemt ikke fremmed overfor teorier om sociale klasser, omend det ofte frames anderledes og i mere arbejdsmarkedsfokuserede termer. 
Boje er optaget af, at forskelle i sociale processer skaber ulighed på arbejdsmarkedet, og ses i ulige vilkår for forskellige (segmenterede) delmarkeder, hvorved forskelle i livsvilkår fører til øget seggregering. Dette er tæt beslægtet med tankegangen i klasseteori. Arbejdsmarkedet kan, som den marxistiske arbejdsmarkedssegmentteoretiker Richard Edwards bemærker, ses som et helt særgent marked, der tydeliggør styrkeholdende i produktionen og i den arbejdende befolkning som helhed {\parencite[177]{Edwards1979}. 

Moderne klasseteori, som den kommer til udtryk hos Daniel Oesch og John Goldthorpe, mener jeg har frugtbare teoretiske såvel som empiriske indsigter i, hvad man kan kalde den over-tid segmenterede arbejdsmarkedsstruktur. Deres klasseinddeling er næsten udelukkende baseret på position på arbejdsmarkedet. En årsag til Oesch og Goldthorpes anvendelig i min kortlægning af arbejdsmarkedet, er deres stringente - næsten ydmyge - fokus på \emph{økonomiske} klasse, fremfor det mere vidtløftige begreb \emph{social} klasse. Det vil jeg komme nærmere ind på senere, foreløbigt skal det bare konstateres, at det er yderst anvendeligt, når man som mig har fokus på arbejdsmarkedets struktur. Det betyder, at deres teori og empiri om differentieringer på arbejdsmarkedet er yderst anvendelige for mig.






%%%%%%%%%%%%%%%%%%%%%%%%%%%%%%%%%%%%%%%%%%%%%%
\section{Fremgangsmåde}
%%%%%%%%%%%%%%%%%%%%%%%%%%%%%%%%%%%%%%%%%%%%%%

I de to næste kapitler gennemgår jeg teori. Andet kapitel handler således om arbejdsmarkedssegmenteringsteori og tredje kapitel handler om klasseteori.

I de to efterfølgende kapitler gennemgår jeg metode. Fjerde kapitel handler således om social netværksanalyse og femte kapitel handler om registerdatamaterialet fra Danmarks Statistik.

De tre efterfølgende kapitler er analysen opdelt tre delanalyser, som passer overens med de tre forskningsspørgsmål. Sjette kapitel er en delanalyse af opdeling af arbejdsmarkedet i delmarkeder. Syvende kapitel er en delanalyse af sociale processer i delmarkeder og segmenter. Ottende kapitel er en delanalyse af det klasseteoretiske perspektiv på segmentering.

Niende kapitel er en diskussion af analysen på baggrund af teori og metode.

De tiende og afsluttende kapitel er konklusion.


Denne afhandling vil fokusere på, hvorledes sociale processer kan ses afvige fra hinanden i delmarkederne, på sådan vis at vi kan tale om segmenter. Jeg vil benytte social stratifikationsteori, til at forklare forskellen i sociale processer, som de kommer til udtryk på et empirisk niveau, gennem intern mobilitet i delmarkederne, indkomst, uddannelse og køn. 

%!TEX root = ../report.tex

%%%%%%%%%%%%%%%%%%%%%%%%%%%%%%%%%%%%%%%%%%%%%%%%%%%%%%%%%%%
\chapter{Udkast til teoriafsnit samt overordnet udkast til analysen \label{teori}}
%%%%%%%%%%%%%%%%%%%%%%%%%%%%%%%%%%%%%%%%%%%%%%%%%%%%%%%%%%%
%klasse er et analytisk framework til at forstå segmenteringsprossesen


 Arbejdsmarkedssegmenteringsteorien er det overordnede framework der benyttes i specialet til at forstå arbejdsmarkedets indretning. Centralt i denne står mobiliteten på arbejdsmarkedet samt barrierer for mobiliteten, det vil sige arbejdsmarkedets opdeling i delmarkeder, og hvordan opkomsten af disse delmarkeder skal forstås. Thomas Boje skelner mellem delmarkeder og segmenter, hvor det sidste ses som mere stabile sociale strukturer.

 Thomas Boje benytter følgende 3 kriterier:

\begin{itemize}
  \item 1. kriterie - Delmarkeder og mobilitet
  \item 2. kriterie - Forskellen på et delmarked og et segmenter er forskelle i sociale processeser
  \item 3. kriterie - Sociale processer fører til segmentering
\end{itemize}

Sociale processer er et meget vagt begreb, hvilket også betyder at segmenter bliver tilsvarende vagt. Ellers bliver det bare empiriske observationer, der viser "klynger af jobtyper med hyppig udveksling." Dette behov for forståelse af de sociale processer der ligger bag segmentering, motiverer følgende: Hvordan kan sociologiens begreber om stratifikation og differentiering bruges til at forstå arbejsmarkedets opdeling i segmenter og delmarkeder? Der findes massere sociologisk teori om stratifikation af mange typer, men fokus vil her være på arbejdsmarkedet og hvad der virker ulighedsskabende på selve arbejdsmarkedet. 

Altså: Stratifikation i samfundet, med udgangspunkt i beskæftigelse. Jeg vil tage udgangspunkt i klassebegrebet, som det analytisk framework til at forstå segmenteringsprossesen  og de forskellige bud på klasser, da intet begreb i sociologien er så nært knyttet op på en forståelse af de sociale processer arbejdslivet  består af. 
Som Eric Olin Wright fremhæver, gælder det om at benytte den klasseteori, der er bedst til det man gerne vil udtale sig om (find henvisningen). Jeg vil i det følgende gennemgå forskellige klasseteorier, for at finde frem til de forklaringer, der hjælper til at forstå segmenteringsopdelingen, som jeg finder den empirisk ved hjælp af Moneca (mere elegant introduktion af den empiriske metode senere, skal bindes bedre an). Ud af denne gennemgang bliver det også klart, hvori Moneca er i stand til at bidrage til klasseforsknigens nuværende diskussioner. 

Gitte Harrits (2014) benytter en opdeling mellem klasseteoriens \emph{beskrivende} og \emph{forklarende} spørgsmål. Klasse som beskrivende spørgsmål er hvad klasse "er", hvordan teorien fastsætter de sociale definitionskampe, der afgrænser og definerer de forskellige klasser. Klasseteoriens forklarende spørgsmål er hvad klasse "gør": Hvilke effekter har klasserelationerne, og hvilke mekanismer er vigtige, eksempelvis værdier eller interesser \parencite[19]{Harrits2014}. Denne vil blive brugt som overordnet skelnen i teorierne (til Jens: det er så ikke rigtigt gjort endnu. Men det kommer! Ideen er god synes jeg.)

Det vil jeg gøre med følgende fremgangsmåde. 

Først vil jeg gennemgå Marx, der definerer relationerne til produktionsmidlerne som afgørende, og den udbytning, de danner basis for, og som definerer de sociale relationer. Derefter vil jeg gå ind i nymarxisternes hovedpine med basis-overbygningsmodellen, eftersom klassesamfundet ikke blev polariseret i to store klasser, sat imod hinanden, men i stedet fremkomsten af nye middelklasser samt nye serviceerhverv, der ikke ser ud til at være midlertige, men tværtimod helt nødvendige dele af den moderne kapitalisme. Jeg vil slutte min gennemgang af marxismens bud på stratifikation på arbejdsmarkedet ved det for mig at se mest empirisk anvendelige bud Eric Olin Wrights klassemodel.

Derefter vil jeg kort gennemgå Weber, der skelner mellem social status, klasse og stænder. Han har inspireret John Goldthorpes meget anvendte såkaldte EGP-klasseskema, der vil blive gennemgået som det næste, med dets fokus på kontrol over arbejdsprocessen og specifiteten af faglige kompetencer som de centrale differentieringsmekanismer. Goldthorpe har en analytisk-teoretisk funderet klasseforståelse, som bliver udfordret af David Grusky, der er fortaler for en noget anderledes klasseforståelse, baseret på erhvervsgrupper og de reelt oplevede fællesskaber, der findes her. Han benytter sig af Frank Parkins begreb om sociale lukningsmekanismer, som også bliver gennemgået (til Jens: sikkert for grundigt gennemgået, skal skæres til). I forhold til MONECA er hans bud spændende, da han udfordrer ideen om store klasser, og er fortaler for disse mikroklasser, som han mener drukner i de i sammenligning grovkornede klasseinddelinger, baseret på teoretiske principper. For mig at se kan Moneca netop være en måde at opdele i større klasser, baseret på empiriske bevægelsesmønstre, det vil sige give plads til Gruskys anke mod "big classes", som for mig at se er fornuftig nok, i og med en undersøgelse på mikroniveau kan afsløre klassedynamikker man ikke fanger på et højere plan. Jeg synes dog Gruskys teoretisk grundlag og affejning af teoretiske klasseinddelinger er problematisk, og hans afvisning af at der findes effekter af at tilhøre større klasser end erhvervsgrupper er problematisk. Her er Moneca en metode, der nærmest virker som skabt til at fungere på begge niveauer, det vil sige som beskrivelse af erhvervsgrupper på meget detaljeret niveua, samt mulighed for at afdække fællestræk i deres "sociale processer" på segmentniveau.

%til sidst vil jeg lede det over i metodeafsnittet, da både Grusky og Goldthorpe benytter økonometrisk funderet regressionsanalyse, og jeg vil mene at en relationelt funderet social netværksanalyse har mulighed for at klasser som relationelle størrelser, fremfor som udtryk for variable.



%DET VIL JEG VIL GØRE MED FØLGENDE FREMGANGSMÅDE...

% Forskellige bud på klasser
% 	Marx
% 		-forholdet til produktionsmidlerne, udbytning
% 	Nymarxisme	
% 		- Problemerne i basis-overbygning og middelklassens position
% 		- Redskaber mere nuancerede end kapital-arbejde - hvordan, og stadig beholde den marxistiske kerne? 
% 	Wright
% 		- klasse som forklarende og nominel (tror jeg)
% 		- Tre dimensioner: Ejerskab over produktionsmidler, autoritet, færdigheder
% 	Weber
% 		- Klasse, social status og stænder
% 	Goldthorpe
% 		- Klasse som forklarende og nominel
% 		- Tre dimensioner: Ejerskab over produktionsmidlerne (knapt så centralt), specifiteten af kompetencer, kontrol over arbejdsprocessen 
% 	Grusky
% 		- Klasse som beskrivende og realistisk
% 		- kritik af nominel klasseinddeling
% 			- for grovkornet
% 			- misser det centrale ved definition udenfor de reelt oplevede fælleskaber
% 		- erhvervsgrupper som klasser, sociale lukningsmekanismer

%DET SKAL ALT SAMMEN BRUGES TIL...

	% Opsamling: 
	% 	- Kobling til delanalyse 1: Moneca som måde at inddele arbejdsmarkedet i segmenter (klasser)
	% 	- Kobling til delanalyse 2: Forklare sociale processer i segmenter (klasser) vha. klasseforskere
	% 	- Kobling til delanalyse 3: Jeg  placere Moneca som et oplagt bud på at mediere og vurdere de to positioner (den realistiske og nominelle klasseindling)

Denne teoretiske gennemgang skal bruges til følgende i forbindelse med det overordnede spørgsmål om arbejdsmarkedets segmentering, i 3 analyser:

\begin{itemize}
  \item Delanalyse 1: Moneca som måde at inddele arbejdsmarkedet i segmenter
  \item Delanalyse 2: Forklare sociale processer i segmenter (klasser) vha. klasseforskere
  \item Delanalyse 3: Moneca som på at mediere og vurdere de to positioner (den realistiske og nominelle klasseindeling hos henholdsvis Grusky og Goldthorpe)
\end{itemize}



%Motivation for at inddrage klasseteori er, at analysen af delmarkeder/segmenter og mobilitet ligger inden for en tradition af klasseteori. Dette ses såvel i Wrights analyse/teori om bla bla bla såvel som i Gruskys analyse/teori om...

%Arbejdsmarkedssegmenteringsteorien, som ligger til grund for Toubøl, Larsen og Strøbys udvikling af en a posteori arbejdsmarkedssegmenteringsanalyse, ligger inden for en amerikansk/anglo-saksisk økonomisk diskussion op i mod den neoklassiske økonomiske teori \parencite[2]{Touboel2013}. Thomas Boje opsummerer kritikkken således:

%\begin{displayquote} “\textit{The core of this criticism is first that neoclassical labour-market theories in their analogies to the commodity market do not take into account the heterogeneous character of the labour market and the lack of homogeneity in the labour force. Secondly the theories do not take into account that the behaviour of individuals on the labour market is not governed by economic utility maximization, but, on the contrary, determined by social relations and institutions of which the individual is an integral part.}” \parencite[171]{Boje1986}. \end{displayquote}

%Segmenteringsteorien fremkomst primært i USA opstod ifølge Cain i 1960'erne og 1970'erne som en reaktion på kampen mod fattigdom og fuld deltagelse i økonomien for blandt andet minoritetsgrupper og kvinder \parencite[1216]{Cain1976}. Doeringer og Piore opdelte arbejdsmarkedet i primære og sekundære markeder med “gode” jobs og sidstnævnte indeholder alle de “dårlige” jobs med henblik på at forklare den ulige fordeling af lønninger \parencite[70]{Doeringer1971}, mens Reich, Gordan og Edwards mere radikale tilgang tilgang beskrev, hvordan den historiske proces, hvorved politisk-økonomiske kræfter fremmer opdelingen af arbejdsmarkedet i seperate delmarkeder \parencite[359]{Reich1973}. Hvor denne segmenteringsteori tager udgangspunkt i forandringerne i sin tid og en økomisk kritik af økonomien, så er mit udgangspunkt og fokus et andet. Mit udgangspunkt er den sociologiske tradition, klasseteori og klassediskussion.

%Derfor vil jeg frem for at dykke ned i segmenteringsteoretikerne tage fat i de sociologiske klaseteoretikere såsom Marx, Weber, Gitte Harrits, Goldthorpe, Wright, Parkin og Grusky. Disse teoretikere er både med til at forklare det første kritikere i forhold til delmarkeder og mobilitet (fx Weber og Goldthorpe), hvilket var bla bla bla... Samtidig ligger der også inden for andet kriterie det vil sige, hvorfor der er forskellen på et delmarked og et segmenter er forskelle i sociale processeser (fx Wright og Goldthorpe). Alt sammen er det interessant også for det tredjhe kriterie oim hvorfor sociale processer fører til segmentering.




%%%%%%%%%%%%%%%%%%%%%%%%%%%%%%%%%%%%%%%%%%%%%%
\section{Karl Marx \label{2_marx}}
%%%%%%%%%%%%%%%%%%%%%%%%%%%%%%%%%%%%%%%%%%%%%%

Marx selv nåede aldrig at færdiggøre sin klasseteori, eller skitsere den i en sammenhængende form. Kapitel 52 i tredje bind af \emph{Kapitalen} har titlen "Klasserne", men efter cirka en side stopper skriften, og Engels, der udgav værket efter Marx' død, skriver: "Her afbrydes manuskriptet" \parencite[22]{Harrits2014}. Som samlet klasseteori må man derfor benytte sig af Marx' generelle teoriapparat, for at forstå hvad en klasse er hos Marx. Det er værd at huske på, at netop fordi klasserne ikke nåede at modtage en specifik teoretisk behandling, må klasseteorien konstrueres ud fra hans generelle historieforståelse.

Formålet her er ikke en grundig gennemgang, men en skitsering, der skal tydeliggøre fundamentet i den neomarxistiske klasseforståelse.

Klasserne hos Marx er defineret ud fra den tilgang, at det vigtigste menneskelige forhold omhandler produktionen af livsfornødenheder, altså de materielle forhold. De værktøjer - forstået i bredest mulige forstand - som gør det muligt at producere disse livsfornødenheder, defineres som produktionsmidlerne. Menneskers forhold til produktionsmidlerne er dermed den centrale sociale relation i et samfund, og igennem denne primære sociale relatio udgår alle andre sociale relationer. Det vil sige, at mennesker med samme sociale relation, lad os kalde det ejerskabsforhold, til produktionsmidlerne, kan karakteriseres som en klasse, da de deler det mest centrale grundvilkår, nemlig betingelsen der tillader dem at reproducere deres livsgrundlag. Marx peger på det forhold, at mennesker er i stand til at arbejde mere, end det der skal til for at holde dem i live, og det arbejde, der ligger udover hvad der skal til for at reproducere dem som arbejdskraft, kaldes merarbejdet. Det er dette arbejde, der er grundlag for udbytningsrelationer mellem forskellige klasser, hvor nogle klasser i kraft af deres ejerskab over produktionsmidlerne kan nyde godt af de primære producenters arbejdskraft. Det er således både en dominansrelation og en udbytningsrelation på spil, i forholdet mellem de eller den herskende klasse og den eller de klasser, der tvinges til at arbejde for dem.

Kapitalistiske samfund er kort sagt karakteriseret ved, at penge bliver den centrale mediator mellem forskellige former for arbejde. Hvad enten den er indlejret som arbejdstid i en fysisk form som en stol, et kilo sukker, eller i et menneske, som levende arbejdskraft. Det er denne "universelle ækvivalent", som Marx kalder den, der tillader det moderne kapitalistiske marked, hvori alle former for menneskeligt arbejde kan udveksles frit som varer. Frit er det dog ikke helt, da det er Marx påstand, at kun en enkelt vare rummer en ganske særlig kvalitet, nemlig evnen til at avle mere arbejde end det, der er indlejret i den. Den vare er mennesket. Et kapitalistiske samfund er derved karakteriseret ved lønarbejde, hvori en klasse, borgerskabet, ejer produktionsmidlerne, mens størstedelen fungerer som den arbejderklasse, der bliver betalt mindre, end deres arbejde er værd, men nok til at reproducere deres arbejdskraft. Det er Marx forudsigelse, at i overgangen fra feudalsamfundet til det kapitalistiske samfund, vil størstedelen af feudalsamfundets udbyttede klasser blive tvunget fra at have, eller i det mindste være i besiddelse af, deres egne produktionsmidler, til at arbejde direkte for borgerskabet i form af løn. Det kapitalistiske samfund vil derfor blive karakteriseret ved en tiltagende polariseret klassekamp mellem arbejdere og kapitalister, bourgeosi og proletariat.

Det behøver ikke nødvendigvis tolkes sådan, at der findes to klasser, der empirisk stringent kan deles op, selvom det, særligt i mere polemisk og politiske øjemed, er blevet benyttet sådan. I Marx' egne konkrete klasseanalyser benytter han en række termer for klasser og klassefraktioner, ikke kun bourgeosiet og proletariatet. Snarere er påstanden, at der findes en grundlæggende modsætning mellem arbejde og kapital, og denne modsætning er afgørende for interesserne og handlemønstrene for dem, der står på henholdsvis den ene og den anden side af denne modsætning. Her benytter Marx sig af begreberne "klasse i sig" og "klasse for sig", hvor det første er en klasse, der findes objektivt, bestemt af de sociale relationer, og det andet er en klasse, der forstår sig selv som klasse. Der er altså to niveauer i en klasseanalyse:

- et abstrakt niveau, der handler om kampen om merværdi og den grundlæggende klassemodsætning i samfundet
- et konkret niveau, om de observerbare klassefraktioner og grupperinger, der indgår i de politiske kampe i samfundet \parencite[26]{Harrits2014}. %lav til liste

Marx' teori indeholder derfor, som Harris fremhæver, ikke nødvendigvis en påstand om arbejderklassens enhed, men kan ligeså vel tolkes som en invitation til at lave historiske analyser med fokus på samfundsforandring, klassedannelse og politiske magtkampe \parencite[28]{Harrits2014}. I den marxistiske tradition ligger der altså et fokus på at bestemme klasser ud fra deres forhold til produktionsmidlerne, sagt med lidt mere almindelige termer: Hvilken overordnet beskæftigelse, man er i. Den fokuserer derfor på relationer indenfor produktionen, og ikke markedsrelationer, med særligt fokus på den \emph{udbytning} der findes i disse relationer.


%%%%%%%%%%%%%%%%%%%%%%%%%%%%%%%%%%%%%%%%%%%%%%
\section{Problemet med middelklassen og det marxistiske svar \label{2_nymarxisme}}
%%%%%%%%%%%%%%%%%%%%%%%%%%%%%%%%%%%%%%%%%%%%%%

Ikke desto mindre udviklede de moderne samfund sig ikke sådan, at kapitalismens centrifugalkræfter delte samfundet i to, med kapitalister og arbejdere sat overfor hinanden, grundet deres ensartede relation til produktionsmidlerne, som besiddende og ikke-besiddende. Den basale to-klasse model synes ikke at fange de interessemodsætninger og klasseformationer, som opstod i stabile kapitalistiske samfund. En række klasser i det sociale stratum med ejerskab over egne produktionsmidler, såsom artians, small traders, shopkeepers and småbønder, %hvordan fanden oversætter man det
forsvandt godt nok (evt. Parkin henvisning). Men denne polarisering i den socioøkonomiske struktur - altså en koncentration af produktionsmidlerne i færre hænder - blev historisk ikke til to store klasser, sat i mod hinanden, og den teoretiske skillelinje fra den marxistiske politiske økonomi mellem arbejde - kapital kunne tydeligvis ikke omsættes til en tilsvarende opdeling i klasser med egentlig ageren efter denne opdeling. Det vigtige i denne afhandling er, at selvom denne modsætning mellem arbejde - kapital måske nok findes - som påvist af bl.a. Thomas Pikkety, og strukturerer indkomstfordeling på afgørende vis, fungerer denne grundlæggende struktur ikke på nogen entydig måde, der gør denne opdeling særlig brugbar til at undersøge de klasseformationer, vi kan observere empirisk. 

En række andre forhold må medtænkes, da det er tydeligt at middelklassen, der erstattede det gamle småborgerskab, ikke er ekstern i forhold til den kapitalistiske produktionsmåde, men er en central del af den, som Parkin påpeger i sit værk \emph{Marxism and Class Theory: A Bourgeois Critique} fra 1979 \parencite[16]{Parkin1979}. %kan man henvise bare til siden når værket allerede er nævnt?

Der må altså være en række forhold, der muliggør eksistensen af den langt mere komplekse klassestruktur i senmoderne, stabile, kapitalistiske samfund, og som er af central betydning for at forstå klasseformationerne i dette.

Hvis arbejde-kapital ikke er en skillelinje, der giver særlig meget forklaringskraft i de samfund, der findes, hvad gør man så, for at finde de politisk relevante brudflader mellem klasserne? Og hvad skal man gøre med den føromtalte middelklasse? Det er hvad Parkin kalder \emph{the boundary problem in Marxism and sociology} \parencite[11]{Parkin1979}. 

Parkin identificerer tre nymarxistiske svar på dette problem. De to første skal kun kort skitseres, mens den sidste, repræsenteret ved Eric Olin Wright, bliver gennemgået i dybden. 

Det første svar er, hvad Parkin kalder \emph{den minimalistiske} definition af arbejderklassen, repræsenteret ved Poulantzas. Her drages skillelinjen mellem produktivt og uproduktivt arbejde, baseret på en fortolkning af hvilken form for arbejde, der skaber merværdi. Lønarbejde, der ikke skaber merværdi, er også udbytning, og indgår derfor ikke i proletariatet, altså den politisk relevante samfundsgruppe i marxistisk optik. Det giver en forklaring på stabile udbytningsrelationer, også lønarbejdere i mellem. 

Det andet svar er den \emph{maksimalistiske} version, repræsenteret ved Poul Baran, der benytter sig af Marx' lighedstegn, visser steder i sine skrifter, mellem \emph{folket} og arbejderklassen. De mange mod de få. Her benyttes et politisk defineret kriterie baseret på hvilke sociale grupper, hvis sociale funktion også ville være nødvendige i et socialistisk samfund \parencite[20]{Parkin1979}. Det er politisk i den forstand, at det handler om at skabe forening mod en overklasse, og analytisk i den forstand, at det prøver at forudsige, hvilket arbejde der kan siges at være \emph{samfundsmæssigt nødvendigt}, for på den måde at klassificere dem, der lever af andres arbejde.  

Begge disse svar forekommer mig ikke særlig anvendelige i en konkret klasseanalyse, og nævnes derfor kun som forsøg på at løse “\emph{the problem of the middle-classes}”, som som Wright kalder det \parencite[15]{Wright2000}. I stedet vil jeg fokusere på den nymarxistiske retning Parkin kalder \emph{mellempositionen}, som Parkin karakteriserer ved Erik Olin Wright selv \parencite[17]{Parkin1979}.


%%%%%%%%%%%%%%%%%%%%%%%%%%%%%%%%%%%%%%%%%%%%%%
\section{Erik Olin Wright \label{2_wright}}
%%%%%%%%%%%%%%%%%%%%%%%%%%%%%%%%%%%%%%%%%%%%%%

Erik Olin Wright introducerer to dimensioner for at forstå hvad han kalder \emph{modsætningsfyldte klassepositioner}, normalt omtalt som middelklassen: “\emph{people who do not own their own means of production, who sell their labor power on a labor market, and yet do not seem part of the working class.}” \parencite[15]{Wright2000}. Det giver samtidig anledning til at differentiere arbejderklassen og klassesamfundet generelt, i en mere nuanceret klassebaseret stratifikationsteori, der viser \emph{hvad klasse gør}, som Harris formulerede det. 

Centralt står stadig forholdet til produktionsmidlerne og de udbytningsrelationer, der hersker mellem mennesker, alt efter deres relation til produktionsmidlerne. Wright sætter to dimensioner på udbytningsforholdet, der udtrykker mulighederne for at tilegnelse af samfundets ressourcer. Klassepositioner ses som udtryk for en kombination af disse to akser samtidig med, at han bibeholder overordnede skelnen mellem ejer-ikke ejer.

Den første akse kalder Wright \emph{autoritetspositioner}, og omhandler det forhold, at kapitalejere uddelegerer deres lovmæssige ret til at lede og fordele arbejdet til andre. Det sætter dem i en modsætningsfyldt klasseposition, da de ikke er kapitalister selv, men står over arbejderne, og modtager højere løn for deres ansvar i en virksomheds daglige ledelse. Den anden akse kalder Wright \emph{færdigheder}, der beskriver de faglige og tekniske færdigheder, der er nødvendige for en given produktion. Desto mere specialiserede, og dermed monopoliserede, en færdighed er, desto mere tager den karakter af ekspertviden, som også muliggør større belønning for det udførte arbejde. % Søren: Dette kan bruges i forhold til sociale processer i segmenterne på sammen måde som "socia closure".
Udover disse akser har Wright, traditionen tro fra Marx, en skillelinje mellem ejere og ansatte, altså de, der selv ejer produktionsmidler, og de, der ikke gør. Disse to akser demonstrerer dermed en skala af klassepositioner, der muliggør forskellige slags tilhørsforhold og identiter, der ikke nødvendigvis polariseres i den enkeltdimensionerede arbejder-kapitalist relation, omend denne grundlæggende modsætning bibeholdes i modellen. Middelklassen bliver dermed den gruppe mennesker, der befinder sig i disse mellempositioner, og Wrights model muliggør også at positionere ansatte indenfor service- og informations industrien, der traditionelt har voldt marxistiske teoretikere en del genvordigheder.

Wright benytter denne udbyggede model til at forstå, hvordan man på det samfundsmæssige plan ser bestemte klasseformationer og politiske kampe, der gerne skulle kunne forklares ud fra denne klassemodel. Spørgsmålet om hvorvidt man skal kortlægge klassestrukturerer “bag om ryggen” på individerne står altså stadig centralt. Wright fastholder på den side ene side en teoretisk model, baseret på udbytningsrelationer i økonomien, samtidig med at den konkrete udformning af denne skal testes empirisk og revurderes løbende, baseret på dens evne til at forklare politiske kampe og observerede klasseformationer s. 75. 

% brug evt. henvisning fra Harris s. 67 til to konkrete danske analyser med brug af Wrights klassemodel 


%%%%%%%%%%%%%%%%%%%%%%%%%%%%%%%%%%%%%%%%%%%%%%
\section{Max Weber \label{2_weber}}
%%%%%%%%%%%%%%%%%%%%%%%%%%%%%%%%%%%%%%%%%%%%%%

Selvom Weber heller aldrig nåede at færdiggøre sin klasseteori, fremstår den i noget mere færdig form end tilfældet er hos Marx \parencite[28]{Harrits2014}. Webers klasseteori medtager stratifikation, der ligger udover hans klassebegreb, og indeholder derfor en mere pluralistisk tilgang til samfundets opdeling i forskellige sociale grupperinger. Hans klassebegreb er dog tydeligt inspireret af Marx, i den forstand at det er bundet op på økonomiske interesser og et bestemt forhold til produktionsmidlerne:
%
\begin{displayquote} “\textit{Vi vil tale om "klasse" når 1. livsbetingelserne for en flerhed af mennesker er betinget af en specifik fælles komponent, for så vidt 2. denne komponent ene og alene udgøres af økonomiske ejendoms- og erhvervsinteresser, og det foregår på 3. (vare- eller arbejds)markedets betingelser ("klassesituation"}” \parencite[302]{Weber1978}. \end{displayquote}
%
Til forskel fra Marx er klasser også defineret ved deres status på arbejdsmarkedet, fremfor relationen til produktionsmidlerne, og der ligger derfor et større fokus på markedsrelationer, altså “\emph{arten af éns chancer på markedet}” (weber 31 2003, i \parencite[29]{Harrits2014}). For at begribe disse chancer på markedet, der jo i høj grad er bestemt af hvilken økonomisk klasse, man er i, siger Weber, at det er ens evne til at tilegne sig indtægter og goder (Weber 1976 i \parencite[29]{Harrits2014}). Det åbner op for et kulturelt perspektiv, da alle mulige egenskaber, der kan omsættes til økonomiske ressourcer, bliver en del af et menneskes livschancer på markedet. Hos Weber finder vi et klassebegreb, der i udgangspunktet indeholder en typologi baseret på ejerskab/ikke-ejerskab over produktionsmidlerne, men derudover indeholder en langt bredere forståelse af klassefraktioner indkapslet i konceptualisering af hvad der definerer klasser. Og i øvrigt ligger tæt på Marx egne konkrete analyser af klasser, som omtalt tidligere.

Weber peger også på mobilitet mellem klasser som et centralt spørgsmål for klasseforskning. Så bevæger vi os hen til at beskrive, hvad klasse \emph{gør} parencite[302]{Weber1978}. Der er hos Weber et større fokus på den subjektive oplevelse af klassetilhørsforhold. Selvom folk individuelt reagerer ens på deres klassesituation, er det ikke ensbetydende med en deterministisk bevægelse hen imod en klasse-for-sig, som hos Marx. Weber er dermed mere interesseret i alle de aspekter, som en klasse \emph{gør}, fremfor Marx fokus på politiske kampe og social forandring. Klassebegrebet kan bruges generelt til undersøgelse af interessekampe, mobilitet og al mulig anden adfærd. (Harris 31). % Søren: Dette skal du vende tilbage til i analysen - intern mobilitet som en weberiansk klasse-/segmentopdeling.

Det sidste centrale begreb hos Weber er hans begreb om stænder. Det vil sige hans forståelse af, at der udover en økonomisk orden, også er en social orden, "den måde hvorpå social ære" fordeler sig i et fælleskab" (Harris 31). Denne sociale orden er en \emph{selvstændig} kilde til magt i et fælleskab, hvilket er en væsentlig forskel fra den marxistiske klasseanalyse. Det er knyttet op på subjektive oplevelser af fælleskab og anerkendelse af bestemte former for livsførsel om særligt attråværdige. Selvom Weber understreger stænders selvstændighed i forhold til klassesituationen, bliver spørgsmålet om sammenfaldet mellem klasse og stænder, og den måde, de kommer til udtryk på i en bestemt praksis, et genstandsfelt for klasseforskning \parencite[32]{Harrits2014}.


%%%%%%%%%%%%%%%%%%%%%%%%%%%%%%%%%%%%%%%%%%%%%%
\section{John Goldthorpe \label{2_goldthorpe}}
%%%%%%%%%%%%%%%%%%%%%%%%%%%%%%%%%%%%%%%%%%%%%%

Den engelske sociolog John Goldthorpe fremhæves ofte som den neoweberianske traditions vægtigste bidrag til klasseforskning, selvom han selv værger sig at blive sat i hartkorn med en bestemt tradition \parencite[90]{Harrits2014}. Goldthorpes formål med sin klasseteori er, “\emph{evnen til at bruge et par få, velvalgte begreber såsom klasseposition, klaseoprindelse og klassemobilitet - immobilitet til at forklare en stor del både af hvad der sker, og ikke sker, med individer på tværs af forskellige aspekter af deres liv}” (Goldthorpe i \parencite[90]{Harrits2014}). Formålet er derfor at benytte klasse \emph{forklarende} overfor sociale forskelle, bredt forstået i samfundet. Her benytter Goldthorpe ofte Webers forståelse af livschancer og social mobilitet som centrale fokuspunkter for klasseforskning, og da han samtidig tager afstand fra det marxistiske fokus på udbytning, Marx historiefilosofi samt forståelsen af politik som klasseinteresser (goldthorpe i \parencite[90]{Harrits2014}), er det forståeligt hvorfor han ofte indskrives i en neoweberiansk tradition. % Søren: Dette skal du vende tilbage til i analysen - intern mobilitet som en weberiansk klasse-/segmentopdeling.

Selvom Goldthorpe tager afstand fra strukturen og måske især historieforståelsen i den marxistiske klasseforståelse, er hans stratifikationsprincip i klasseopdelingen ligesom i den nymarxistiske tradition baseret på placering på arbejdsmarkedet som den grundlæggende klassestratifikationsmekanisme, i Goldthorpes terminologi \emph{ansættelsesrelationer}. Forskellige ansættelsesrelationer implicerer forskellige klassepositioner og klasseinteresser. (Goldthorpe i \parencite[92]{Harrits2014}). Goldthorpe ser 3 typer af ansættelsesrelationer:
%
\begin{enumerate}
 \item ejerskab over produktionsmidlerne
 \item specificiteten af kompetencer
 \item mulighed for kontrol over arbejdsprocessen
\end{enumerate}
%
Specificiteten af kompetencer har samme status som Wrights færdighedsakse, da værdien af ens færdigheder på markedet er højere, hvis udbuddet af dem er lavt og er svært opnåelige. Mulighed for kontrol over arbejdsprocessen, eller mangel på samme, udmønter i to slags arbejsforhold: som \emph{arbejdskontrakt} eller som \emph{servicerelation}. Hvis arbejdets resultater er nemt målbare, og arbejdets karakter ikke kræver sjældne kompetencer, vil arbejdsforholdene typisk baseres på timeløn eller akkordlønning. Mere diffust målbart arbejde, hvis udførelse kræver en mere specifik, og derfor typisk sjælden, uddannelseskompetence, bliver typisk aflønnet på længere bane såsom månedsbasis eller andet, og her vil kombinationen af svært målbart arbejde, samt specifikke kompetencer, give disse lønarbejdere bedre mulighed for forhandle gode arbejdsforhold. (Goldthorpe i \parencite[93]{Harrits2014}). Kombinationen af disse faktorer indebærer forskellige vilkår for forskellige klasser, hvoraf de tre primære er: 
%
\begin{enumerate}
 \item Forskellige begivenheder indtræffer med forskellig styrker alt efter klasseposition
 \item Forskel i interessedrejning hos forskellige klasser
 \item Omkostninger forbundet med at forfølge en given interesse
\end{enumerate}
%
(Goldthorpe i \parencite[93]{Harrits2014})

Det er ikke nødvendigvis sådan, at alle individer indenfor samme klasse forfølger samme interesser. Snarere sådan at individer i samme klasseposition vil have større affinitet overfor de samme interesser. Goldthorpe benytter Webers forståelse af social orden, forbundet med ære og hierarkisk værdisætning, som selvstændig kilde til social stratifikation og livsanskuelse. Disse værdisæt har ifølge Goldthorpe antaget en løsere form som det 20. århundrede skred frem, uden at det dog ser ud til at opløse dets effekter - snarere er denne form for “verdensanskuelse” langt mere betydningsfuld for at forstå eksempelvis kulturforbrug, og de politiske holdninger på en værdipolitisk akse, gående fra liberitær til autoritær (Goldthorpe i \parencite[95]{Harrits2014}. 

Selvom Goldthorpe understreger denne sociale status kilde til at forstå individerne og deres subjektivitet, er hans EGP-klasseskema dog strengt baseret på position på arbejdsmarkedet, da han laver denne skelnen mellem de to niveauer \parencite[95]{Harrits2014}. 


%%%%%%%%%%%%%%%%%%%%%%%%%%%%%%%%%%%%%%%%%%%%%%
\section{Grusky og Parkin \label{gruskyogparkin}}
%%%%%%%%%%%%%%%%%%%%%%%%%%%%%%%%%%%%%%%%%%%%%%

Grusky har to kritikker af klasseanalyse som den almindeligvis bedrives: For det første at de er nominelt funderede, fremfor at tage reelt oplevede, sociale fællesskaber som grundprincip for klasseinddelingerne. Og for det andet, at de derved ender med en grovkornet klasseinddeling, der skjuler udbytnings- og stratifikationsmekanismer, der burde være i centrum af klasseforskningen. 

%brug eventuelt: basis skal være institutionaliserede, "aktøroplevede" (dårligt ord) erhvervsgruppers i samfundet. Det er på dette niveau, at klasse reelt "fungerer" (Grusky 2001 eller 2002 find henvisning)

Grusky påpeger at menneskers identifikation med klassebaserede identiteter historisk er blevet svagere og svagere, samt den manglende massehandlen, der burde følge med positioner i arbejderklassen, hvilket har ledt til en hel del \emph{nymarxistisk håndvridning}, som han kalder det \parencite[205]{Grusky2001}. 

Det er ikke problemet at benytte beskæftigelse som grundlag for klasseinddeling. Problemet er niveauet af aggregering, der sker ved at placere en lang række forskellige professioner i eksempelvis kategorien \emph{nedre serviceklasse}, som hos Goldthorpe, baseret på teoretisk funderede magtdimensioner, i Goldthorpes tilfælde: Kontrol over arbejdsprocessen samt specificiteten af færdigheder. 

Klasseanalyse bør tage udgangspunkt i hvad Grusky kalder \emph{realistiske} fælleskaber, altså fælleskaber der opleves som sådan af individerne i dem. Det er på dette (bevidste) niveau, at sociale stratifikationsmekanismer reelt opererer \parencite[212]{Grusky2001}. Det er ikke sådan, at den funktionelle enshed i arbejdet, som er fundamentet for erhvervsklassifikationsskemaer, skal tages for pålydende. Funktionel enshed i arbejdet er ikke nødvendigvis lig den sociale enshed, Grusky ser som udgangspunktet for en klasseanalyse. Han mener at denne "\emph{technicist vision}" %
\label{gruskytechnicistvision}%
også bør medtage reelt oplevede sociale distinktioner, fremfor udelukkende tekniske hensyn \parencite[215:fodnote 5]{Grusky2001}%
%
\footnote{Han nævner bl.a. ISCO, der benyttes af Danmarks Statistik, og som jeg også benytter i denne afhandling. Mere om det senere.}%
%
. Men på et tilstrækkeligt detaljeret niveau, hvor vi nærmer os rene professioner, er sandsynligheden for reelt oplevede fælleskaber størst \parencite[207]{Grusky2001}. Ud fra dette niveau kan aggregation i større klasser \emph{eventuelt} finde sted, omend man fornemmer at guldstandarden for Grusky stadig er de realistiske fællesskaber. Det er hans to validitetskriterier: Det første er at medlemmerne i den skal forstå sig selv som tilhørende en gruppe, baseret på beskæftigelse. Det andet er at kollektiv handlen, der sikrer denne gruppe privilegier, forstået som \emph{social luknings}-strategier. Det er et begreb fra Parkins professionssociologi, Grusky ser ud til at overføre direkte. Jeg vil derfor gennemgå Parkins definition af det nu. 

For Parkin er \emph{social lukning} som basis for gruppedannelse en måde at tilbyde et alternativ til den marxistiske nominelle klasseforståelse. Alternativet er baseret på en omfattende kritik af den særstatus, relationen til produktionsmidlerne har i marxistisk klasseinddeling. Især de ikke-holdbare krumspring, han mener nymarxister gør sig for at bevare denne særstatus, samtidig med at klasserne "ikke gør det, det var meningen de skulle gøre". Denne kritik vil ikke blive gennemgået nærmere, jeg vil i stedet fokusere på hans egen forklaringsmodel på stratifikation og udbytningsmekanismer. I forlængelse af den weberianske tradition mener Parkin, at en autoritet- og statussrelationer, der fungerer på niveauet af \emph{distributionen af goder} i samfundet er central. Det bør tillægges lige så stor vægt som ejerskabet til produktionsmidlerne \parencite[24f]{Parkin1979}. Dermed sidestilles markedsmuligheder og livschancer med den direkte relation til produktionsmidlerne%
%
\footnote{Parkin anerkender, at nymarxister også har måtte indoptage denne dimension af klassekamp i deres teoriapparat. "inside every neo-Marxist there seeems to be a Weberian struggling to get out", som Parkin triumerende og ikke uden skadefro formulerer det. \parencite[25]{Parkin1979}}. Ingen akademisk diskussion værd at beskæftige sig med uden lidt skyttegravskrig med sennepsgas.%
%
. Parkin er ikke interesseret i at revitalisere Webers klasseanalyse, men i at benytte den weberianske forståelse af sammenhængen mellem produktionsrelationer, symbolske relationer og markedseffekter, til foreslå et nyt framework til at forstå sociale gruppers kamp om goder i samfundet. Hans pointe er, at hvis man skal forstå sociale afgrænsninger, er (produktionsorienteret) klassebaseret afgrænsning kun en dimension, og det giver ikke mening at tillægge denne større vægt end andre historiske diffenteringsmekanismer, hvoraf han fremhæver etnicitet som et markant eksempel \parencite[38f]{Parkin1979}. Han vil derfor gerne skabe en teori, der kan rumme \emph{enhver slags} social afgræsning, i kampen om samfundets goder \parencite[42]{Parkin1979}. 

Parkin tager fat i Webers begreb \emph{social lukning} i fundamentet af denne teori. Han skriver:
%
\begin{displayquote} “By social closure Weber means the process by which social collectivities seek to maximize rewards by restricting access to resources and opportunities to a limited circle of eligbles. This entails the singling out of certain social or physical attribute as the justificatory basis of exclusion. (...) the nature of these exlusionary practices, and the completeness of social closure, determine the general character of the distributive system” \parencite[44]{Parkin1979}. \end{displayquote}
%
I samme åndedrag skriver Parkins, at tilegnelsen af økonomiske ressourcer er central \parencite[44]{Parkin1979}. Konsekvensen af ovenstående citat,for Parkins, er at udbytning ikke længere defineres som udtryk for en relation til produktionsmidlerne, men at ejerskab over produktionsmidlerne ses som en specifik, og dog helt central social lukningstrategi, der muliggør udbytning \parencite[53]{Parkin1979}. 

Parkins peger på en anden væsentlig kilde til udbytning i moderne samfund, som er de højere, bogligt funderede uddannelser%
%
\footnote{Parkin benytter den engelske betegnelse \emph{the professions}, hvilket jeg vil mene kan oversættes til ovenstående på dansk [er du enig Jens?]}%
%
. Ligesom en række andre sociologer såsom Bourdieu og Collins (\#henvisningmangler) peger Parkin på professionalisering gennem uddannelse og juridisk legitime uddannelsesdiplomer som central differentieringsmekanisme i moderne samfund, hvilket han kalder \emph{kredentialisme}. I Parkins teori er der tale om udbytning, da der tilegnes ressourcer i form af social lukning: En strategi til at hæve markedsværdien \parencite[54]{Parkin1979}. I Thomas Bojes optik kunne vi sige, at det er erhvervsgruppers måde at kontrollere udbuddet af arbejdskraft på et givent område, for dermed at øge markedsværdien. Parkin adskiller denne kredentialistiske strategi indenfor de længere uddannelser, med de strategier, fagforeninger bruger for at beskytte sig imod udbytning fra arbejdsgivere, da det bevidste mål ikke er at reducere de materielle muligheder for andre i arbejdsstyrken \parencite[57]{Parkin1979}. Dette vidensmonopol for de længerevarende uddannelser mener han er så omfattende, at overklassen i dag udgøres af ejerne og (deres ansatte) ledere af produktionsmidlerne, samt de lærde klasser med monopol en særlig viden \parencite[58]{Parkin1979}. 

Parkin skelner mellem social lukning som \emph{eksklusion}, hvor målet er at tilegne sig ressourcer ved at ekskludere andre, og social lukning som \emph{usurpation}. Det sidste foregår som respons på en i samfundet befæstet, muligvis \emph{formaliseret}, eksklusion. For eksempel organiserede lønarbejderes kamp for bedre løn og arbejdsvilkår, eller etniske gruppers kamp for udvidelse af deres juridiske borgerrettigheder \parencite[74]{Parkin1979}. 

Ekslusionsbaserede social lukningsstrategier er, som eksempelvis etnicitetsbaserede kriterier viser, ikke kun noget, der hører klasser som helhed til. Parkin peger også på eksklusitionsbaseret social lukning indenfor klasser, i hvad han kalder \emph{dobbelt lukning}. Han bruger arbejderklassen som eksempel, uden at definere den nærmere, men man antage at han mener de lønmodtagere, der ikke tilhører de professionaliserede højtuddannede. Disse benytter også eksklutionsbaseret social lukning mod andre dele af klassen. Det kan ske baseres på køn, etnicitet eller andre sociale kriterier der giver mening i den sociale kontekst \parencite[89]{Parkin1979}. Når en indfødt arbejderklasse lukker sig overfor en arbejderklasse fra et andet land, handler det ofte om at sikre de sikrede rettigheder overfor udvidelsen af arbejdsudbuddet til dårligere forhold \parencite[91]{Parkin1979}. Det virker ganske genkendeligt i diskussion af østarbejderes indtog í Danmark. Som han siger: "\emph{workers who opt for closure against a minority can hardly de beclared guilty of irrationality in choosing to retain the proven benefits of exclusion in preference to the uncertain or doubtful pay-off resulting from combined usupartion.}" s. 95". Ligeledes er fragmentering i klasserne langs beskæftigelsesmæssige linjer et vigtigt fokuspunkt \parencite[81]{Parkin1979}.

I en tilbagevenden til Grusky kan vi sige, at Parkin er helt enig i Gruskys 2. kriterie for klasseinddeling: For Parkin er kollektiv handlen det, der definerer en klasse, og ikke en a priori teoretisk inddeling, som han ikke har meget til overs overfor \parencite[113]{Parkin1979}. De skøjter simpelhen over potentielt vigtige lukningsmekanismer internt i deres nominelt inddelte klasser, fremfor at fokusere på sociale grupper, der viser deres eksistens ved deres sociale agens. 

Det spørgsmål, man kan stille til Parkins teori er, hvad status klassebegrebet overhovedet har i hans teori, om det ikke tømmes for substans? Hvis en klasse er omdannet til en hvilken som helst social gruppe, der er involveret i sociale lukningsmekanismer - dvs. semitautologisk, har skabt sig selv som social gruppe - til hvad nytte er så et klassebegreb? I Parkins bog fra '79 bruger han lang tid på at fjerne klasses \emph{nødvendige} relation til produktionsmidlerne, blot for at genindsætte det som central social lukningsmekanisme i moderne samfund, og alligevel benytte beskæftigelsesstatus til at beskrive de dominerende klasser i samfundet som ejere/administratorer af produktionsmidlerne samt de højtuddannede professioner. I den proces mister klasse sin betydning som et bestemt analytisk greb, og bliver til en hvilken som helst social gruppe, der har organiseret sig som sådan. Det er også formålet, for at slippe uden om den teoretiske a priori inddeling, men man kan spørge, om det er at udvande klassebegrebet i det omfang, Parkin gør det, for at slippe af med hvad man kunne kalde den marxistiske blindgyde om økonomisk basis- ideologisk overbygning%
%
\footnote{"\emph{(...) and more maddening talk about steam mills}" som Parkin hvast formulerer det \parencite[6]{Parkin1979}.}%
%
. 

Grusky giver beskæftigelsesstatus en priviligeret position, eftersom han i sin argumentation for "disaggregerede" mikroklasser benytter beskæftigelsesstatus uden at argumentere yderligere for det. %(det gør han nok et andet sted, det bliver du nok nødt til at finde. sucks. \#todo) 
Ham og Parkins indtager samme standpunkt i deres holdning til klasseinddelingskriterier.

Man kan med Harrits sige, at fokus for særligt Grusky ligger på hvad klasse er, omend han ved hjælk af Parkin m.fl. trækker på teorier, der handler om hvad klasser gør. Det primære fokus - eller første skridt - er dog spørgsmålet om hvad klasse er. 

Goldthorpe har i \emph{Acta Sociologica} svaret på Gruskys kritik, og diskussionen mellem de to for mig at se ligger i hjertet af hvad en moderne klasseanalyse bør indeholde og hvad den kan tilbyde. Jeg vil derfor referere Goldthorpes kritik, og kommentere på denne, samt se på hvad min afhandlings bidrag kan være i denne diskussion. 


%%%%%%%%%%%%%%%%%%%%%%%%%%%%%%%%%%%%%%%%%%%%%%
\subsection{analysepunkter (skitseagtigt)}
%%%%%%%%%%%%%%%%%%%%%%%%%%%%%%%%%%%%%%%%%%%%%%

I indledningen præsenterede jeg følgende 3 delanalyser:

\begin{itemize}
  \item Delanalyse 1: Moneca som måde at inddele arbejdsmarkedet i segmenter
  \item Delanalyse 2: Forklare sociale processer i segmenter (klasser) vha. klasseforskere
  \item Delanalyse 3: Moneca som på at mediere og vurdere de to positioner (den realistiske og nominelle klasseindeling hos henholdsvis Grusky og Goldthorpe)
\end{itemize}


Her er en en nogle uddybelser af ovenstående, hvad der skal undersøges

\begin{itemize}
  \item Fælles ting indenfor segmenterne (arbejdsløshed, indkomst etc?)
  \item stærk intern mobilitet (de kender hinanden - realistiske fællesskaber som hos Grusky, MEN ikke nok!)
  \item Mønstre i segmenterne, hvilke findes der? Tilnærmer de sig Goldthorpes teoretiske inddeling? Bourdieu og felter?
   \item reproduktion over tid (vigtigt men måske for omfangsrigt - Moneca kunne sagtens lave et netværk ikke på individers bevægelser, men på mobilitet fra højest uddannede forældres arbejde til barns arbejde)
\end{itemize}

%Kan disse anses som klasser? Hvad skal der til for at de kan anses som klasser?
%		- Fælles ting indenfor segmenterne (arbejdsløshed, indkomst etc?)
%		- stærk intern mobilitet (de kender hinanden - realistiske fællesskaber som hos Grusky, MEN ikke nok!)
%		- Mønstre i segmenterne, hvilke findes der? Tilnærmer de sig Goldthorpes teoretiske inddeling? Bourdieu og felter?	
%		- Reproduktion over tid (vigtigt)
%		- Boje diskuterer det ikke: Folk gifter sig, familien som enhed (ham polakken og Wright), større klynger? Hvilken konsekvens har det for segmententinddelingen? Skal de så ligges sammen hvor de gifter sig meget? Hvor langt skal "realistiske fælleskaber" strækkes? Er der ikke et hul i Gruskys argument her: Hvis far er i jordbeton segmentet og mor i omsorgsklyngen, strukturerer det ikke deres "oplevede fælleskab" på en måde der kræver en slags underliggende logik for at kunne forstås?


% Diskussionen om de to positioner 




% Den store forventning 

% Nominalist view 





% Ved hjælp 11,8 \% procent a












% søskendekorrelationer i langtidsindkomst, .  


% grusky
% grusky kritik
% grusky svar
% egne overvejelser om anvendelighed
% mere grusky fordi han kommer med anvendelige ideer
% (læs mere goldthorpe og wright måske, og andet. Konkrete mekanismer der kan undersøges, ikke mere teori om hvorfor klasser ser ud som de gør)





% Therborn er inde på et lignende argument om globalisering (Therborn 2002:224)

% i takt med globaliserings effekt på klasserelationer, at et skift i fokus til erhvervsgrupper, defineret som mikroklasser, vil betyde et manglende blik for denne essentielle mekanisme, der fremstår tydeligst på et makroniveau. 


% Der er en god teoretisk begrundelse for denne kritik, som jeg vil vende tilbage til, men jeg mener Grusky i sit svar på

% Therborn (ref!)	 og Birkelund (ref!) mener på et metodisk niveau, at makroklasser overskueliggør ten


% - for meget information
% - at have en foruddefineret nominalistisk inddeling 











% Therbon: "An unnecessary violation of language", lol, s. 222



% . Konceptet udbytning er eksempelvis centralt i klasseanalyse, omend det ikke nødvendigvis, som hos Wright,  



% I et lignende argument påpeger Therborn 


% Dette er uafhængigt af om de definerer sig som klasse. 


% de interesser, det giver, og det andet: de omkostninger, der er forbundet med at forfølge visse interesser Harris s. 93.





% definerer klasse anderledes, i rationelle termer - at undergrupper af en klasse spiller "free-rider" på de andre, grundet lokale tilhørsforhold, uden at realisere deres klasseinteresser. Men hvad er klasseinteresser så, på niveau der ikker er arbejde-kapital, men heller ikker erhvervsgrupper? s. 215



% rationelle handlingsmodeller, hvor forskelle i belønninger kontra risici kan forklares ud fra forskelle i ressourcer. Især to forhold: Forskelle i begivenheder, der indtræder for forskellige klasser, fx arbejdsløshed, de interesser, det giver, og det andet: de omkostninger, der er forbundet med at forfølge visse interesser s. 93



% Hvad Grusky harcelerer imod, er a priori teoriske inddelinger af klasse, der står udenfor den sociale praksis, og forekommer på et abstraktionsniveau, der helt har mistet følingen med den sociale praksis, den ellers postulerer at beskæftige sig med på et "væsensniveau" fremfor i "fremtrædelsesform". En kritik, der bestemt har noget på sig i de endeløse marxistiske diskussioner om klasse i dogmatisk forstand, men er lige lovlig polemisk at skyde Goldthorpe og særligt Bourdieu i skoene \parencite[206]{Grusky2001}. Snarere er 

% Som Bourdieu fremhæver, er klasse en analytisk kategori, og bør også også defineres som sådan (FIND HENVISNING). 


% Som salige Marx sagde: Sie wissen das nicht, aber sie tun es. %find reference, søg på google, det er vist nok fra Kapitalens 1. bind. 
% Der findes social logik, der virker strukturerende for aktørens handlinger, uden at denne logik umiddelbart opfattes bevidst af aktøren som en sådan logik. 



% Goldthorpe mener, at der  



% Goldthorpe afviser decideret 

% Goldthorpe mener at klasse fundamentalt adskiller sig fra erhvervsgrupper, da der er tale om en strukturation på et niveau over den præcise beskæftigelse: Det er ikke klasseanalysens formål at vise hvorfor lægesønner bliver læger, men hvorfor lægesønner bliver akademikere eller managers eller andet. Det vil sige noget bag individerne, noget der strukturerer uden nødvendigvis at være en lokal strukturation i Giddens forstand, og Durkheims gemeinschaft. Aktør/struktur. s. 214




% % Søren: Du bruger jo ikke "bare" technicist vision - altså a priori tekniske kategorier. Du tager 143 tekniske kategorier og kigger på den faktiske mobilitet. Den faktiske mobilitet har vel netop en relation til det faktiske fællesskab - man tager vel job de steder, hvor man kan se sig selv arbejde. Derfor er du bedre en Grusky. Du tager Grusky et niveau længere op.



% At fokusere på autoritetsrelationer der fungerer på et niveau af *distribution af goder* fremfor udelukkende på relationen til produktionsapparatet, siger Parker, s. 25. 
% → Er det en nyttig skelnen? Jeg er i tvivl. 

% "inside every neo-Marxist there seeems to be a Weberian struggling to get out", som Parkin triumerende og ikke uden en vis brod formulerer det. s. 25

% snak om etnicitet, ikke super brugbart tænker jeg. Hans pointe er at klasse og kulturel stratifikation som etnicitet skal fungere på samme konceptuelle niveau, hvilket leder til hans social closure teori. 






% Såfremt at en klasse er en klasse-for-sig, er den en klasse, 



% Grusky indrømmer, at alle  













% udviklet den marxistiske klasseanalys

% s

% Det spørgsmål som jeg vil trække frem, er hvorvidt en nominel inddeling af klasse, på baggrund af visse karakteristiska, kan benyttes til at beskrive hvilke klasser et samfund består af. 






% kun findes to klasser, selvom ovenstående beskrivelse unægteligt peger i den retning. Andre har 







%%%%%%%%%%%%%%%%%%%% noter %%%%%%%%%%%%%%%%%%%%%%%%%%%%%%%%


%	både Boje og Grusky pointerer at big classes i høj grad findes i skandinavien. men det er vel netop tegn på, at historiske formationer har formet dem sådan, dvs at klasse netop er et flydende begreb, der har brug for forklaringer på det niveau, fremfor blot på mikro niveau? Netop analytiske kategorier er vel vigtige her.
%	
%	
%	
%	
%	 
%	
% Hvad jeg taler om når jeg taler om klasse
%
%Marx afsnittet: Måske find stedet i kapitalen hvor han taler om "bagerklassen"? Det ved du er der et sted, brug det som bevis på hans forholdsvis løse klassebegreb.



	\input{tex/2.1_teori_bourdieu}
	% -*- coding: utf-8 -*-
% !TeX encoding = UTF-8
% !TeX root = ../report.tex


%%%%%%%%%%%%%%%%%%%%%%%%%%%%%%%%%%%%%%%%%%%%%%%%%%%%%%%%%%%
\newpage \section{\textsc{Arbejdsløshed \label{teori_arbejdsloeshed}}}
%%%%%%%%%%%%%%%%%%%%%%%%%%%%%%%%%%%%%%%%%%%%%%%%%%%%%%%%%%%

Formålet med dette teoretiske afsnit om arbejdsløshed er at kontekstualisere arbejdsløshed som problemstilling og forskningsfelt. Denne kontekstualisering bliver anvendt i vores operationalisering af arbejdsløsheds påvirkning på mobilitet i afsnit \ref{teori_operationalisering}. Dette teoretiske afsnit om arbejdsløshed indleder med at beskrive arbejdsløshed som problemstilling. Efterfølgende beskrives økonomiske forståelser af arbejdsløshedsproblemet. Bagefter følger en beskrivelse af sociologiske forståelser af arbejdsløshedsproblemet.  Afslutningsvis foreligger en afsluttende opsamling.








%%%%%%%%%%%%%%%%%%%%%%%%%%%%%%%%%%%%%%%%%%%%%%%%%%%%%%%%%%%
\subsection{\textsc{Arbejdsløshed som problemstilling}}
%%%%%%%%%%%%%%%%%%%%%%%%%%%%%%%%%%%%%%%%%%%%%%%%%%%%%%%%%%%

C. Wright Mills konstaterer: “No problem can be adequately formulated unless the values involved and the apparent threat to them are stated.” \textbf{\parencite[129]{Mills1959}}. Med inspiration fra Mills vil vi gribe arbejdsløshed som problemstilling ved at kontekstualisere kontekstualisere arbejdsløshed som et socialt fænomen med fokus på hvilke værdier, som er involverede og truslerne herimod. Først vil vi beskrive en kort historik af arbejdsløshed som et socialt fænomen. Derefter vil vi beskrive dagens syn på arbejdsløshed, herunder beskrive de dominerende diskurser forbundet med arbejde og arbejdsløshed. %%%% Emil: Uddyb citatet. Det kunne være bedre med et Bourdieu-citat


%%%%%%%%%%%%%%%%%%%%%%%%%%%%%%%%%%%%%%%%%%%%%%%%%%%%%%%%%%%
\subsubsection{Kort historik over arbejdsløshed i et dansk perspektiv}

Arbejdsløshed anskues i dag som et socialt problem. Som begreb kom arbejdsløshed dog først til verden i løbet af det 19. århundrede. I dette århundrede blev arbejdsløshed ligeledes også et adskilt fænomen fra fattigdom \textbf{\parencite[3]{Halvorsen1999}}. I Danmark blev det op igennem 1800-tallet og omkring århundredeskiftet en dominerende tanke at anskue arbejdsløshed som et kollektivt og statsligt anliggende, som delvist var forårsaget af forhold arbejdstagerne ikke havde kontrol over. Dette bliver slået fast i 1907 med den første danske arbejdsløshedsforsikringslov\footnote{Hermed blev arbejdsløshedskasserne, staten og kommunerne de centrale aktører i  danske arbejdsløshedsforsikringssystem. Dette kendetegner den såkaldte Gent-model, hvor staten anerkender og yder tilskud til arbejdsløshedskasser organiseret af forsikringstagere (i praksis fagbevægelsen), og at det for det enkelte individ er frivilligt om denne vil forsikre sig mod arbejdsløshed \parencite{Jensen2007a}.}, som blev vedtaget med et bredt flertal i Landstinget og Folketinget. Baggrunden herfor var en kommission, som anbefalede, at samfundet måtte træde ind over for arbejdsløshed, fordi kommissionen kunne konstatere, at arbejdsløshed var et socialt onde, som ramte arbejdstagerne uden, at de havde skyld heri \parencite[69]{Pedersen2007}. Med indførelsen af arbejdsløshedsforsikringen får den danske velfærdsstat som rolle at administrere arbejdsløshed som et socialt problem. Det var dog først i 1970, at staten overtog den primære økonomiske byrde i arbejdsløshedsforsikringen gik fra at være en primært privat forsikringsordning til at være en primært statsfinansieret velfærdsordning \parencite[83]{Pedersen2007}. Fra 1907 til 1970 blev de arbejdsløses vilkår løbende styrket, men beskæftigelseskrisen som fremgår af figur \ref{fig_udvikl.arbejdsloeshed} fra 1970'erne og frem til midten af 1990'erne medfører en mere aktiv arbejdsmarkedspolitik end tidligere, der blandt andet indebar løbende besparelser. %%%% Emil: Kan fyldes en smule mere på. Måske Foucault hevisnng til dannelse af befolkning. Kan suppleres med Hornemann Møller som har gode inddelinger. Figuren skal forklares noget mere.
% 
\begin{figure}[H]
\begin{centering}
	\caption{Udviklingen i arbejdsløshedsprocenten: Kilde AE-Rådet}
	\includegraphics[width=0.8\textwidth]{fig/teori/historiskudvikling.png}

	\footnotesize{Bruttoledigheden opgør Danmarks Statistik kun tilbage til 2007. Finansministeriet har dog foretaget en tilbageføring til 1996, som er anvendt til figuren. Kilde: AE på baggrund af Danmarks Statistik, Finansministeriet og OECD \parencite[2]{Bjoersted2012}.}
	\label{fig_udvikl.arbejdsloeshed}
\end{centering}
\end{figure}
% 
Eksempelvis indføres efterlønsordningen i 1977 under Anker Jørgensens socialdemokratiske regering \parencite[86]{Pedersen2007}, dagpengesatsen fastfryses fra 1982 til 1986 under Poul Schlüters firkløverregering \parencite[88]{Pedersen2007}, vægten på ret, pligt og individuel behovsorientering øges i 1993 under Poul Nyrup Rasmussens socialdemokratisk ledede regering \parencite[92]{Pedersen2007}, muligheden for tværfaglige a-kasser oprettes for at øge konkurrencen under Anders Fogh Rasmussens VKO-regering \parencite[97]{Pedersen2007} og senest dagpengereformen fra 2010, hvor dagpengeperioden blev halveret fra 4 til 2 år og genoptjeningspligten fordobles fra 26 til 52 uger \parencite{lov_dagpenge}. %%% Søren: Henvisningen til Halvorsen er god nok, men man kunne godt overveje at anvende en anden eller flere andre henvisninger fx. Bauman

Ifølge Keane og Owens er udviklingen af velfærdsstaten i Danmark og andre lande bygget på et normativt grundsyn om at alle som udgangspunkt skal forsørge sig selv gennem et arbejde \textbf{\parencite[18]{Keane1986}}. Lønarbejdet bidrager til at sikre social integration blandt velfærdsstatens medlemmer. Og velfærdsstatens aktive arbejdsmarkedspolitik er med til at opretholde en vis levestandard for dem, som ikke har et arbejde samtidig med at benytte en “gulerod” til at få folk i arbejde gennem økonomiske incitamenter \textbf{\parencite[7]{Halvorsen1999}}. %%% Henvisningerne til Keane og Halvorsen er gode nok, men man kunne godt overveje at anvende en andre henvisninger fx. nogle fra FAOS


%%%%%%%%%%%%%%%%%%%%%%%%%%%%%%%%%%%%%%%%%%%%%%%%%%%%%%%%%%%
\subsubsection{Dagens syn på arbejdsløshed i et dansk perspektiv} 

Arbejdsløshed regnes ifølge Halvorsen for en af nutidens største udfordringer for velfærdsstaten både nationalt og internationalt \textbf{\parencite[8]{Halvorsen1999}}. Som socialt problem indgår arbejdsløshed i diskurser i forbindelse med arbejdets betydning og i den forbindelse også betydning af \textit{fravær} af arbejde. De diskurser, som er knyttet til arbejdsløshedsfænomenet er både med til at påvirke, hvordan arbejdsløse klassificeres, og hvordan arbejdsløse forstår dem selv og deres situation \textbf{\parencite[12]{Halvorsen1999}}. %%% Henvisningen til Halvorsen er god nok, men man kunne godt overveje at anvende en anden eller flere andre henvisninger fx. Bauman

Halvorsen skelner mellem tre forskellige diskurser om lønarbejde \parencite[13]{Halvorsen1999}. Den første diskurs knytter sig til retten til arbejde, hvor lønarbejdet både er lig med selvrealisering og er en forudsætning for, at man kan fungere som en god samfundsborger\footnote{Lars Svendsen skelner inden for den europæiske idéhistorie mellem to grundlæggende forskellige arbejdsopfattelser. Indtil reformationen blev arbejdet anset som en \textit{meningsløs forbandelse}, og efter reformationen blev arbejdet anset som et \textit{meningsfyldt kald} \parencite[13]{Svendsen2008}. I det moderne samfund beskriver Bauman, at arbejdet bliver æstetisk, fordi den enkelte eksempelvis skal kunne identificere sig med sit arbejde eller at ens arbejde skal være autentisk \textbf{\parencite[169-215]{Baum2006}}.}. Den anden diskurs knytter sig til arbejdspligt, hvor lønarbejde er lig med den grundlæggende værdiskabende aktivitet i samfundet\footnote{Efter anden verdenskrig gik velfærdsstaterne ind i en ny historisk fase, hvor regeringerne forsøgte at skaffe fuldtidsjobs til alle voksne igennem en politik som havde til formål for det første at stimulere privat og offentlig vækst \parencite[17]{Keane1986}.}. Den tredje diskurs knytter sig ligeledes til arbejdspligt, hvor lønarbejdet er lig med et nødvendigt onde, som er nødvendigt for at få samfundet til at fungere og et onde, fordi det enkelte individ er tvunget til at arbejde\footnote{Den dominerende opfattelse af arbejdet som et \textit{meningsløs forbandelse} \parencite[13]{Svendsen2008} kan stadigvæk siges at være gældende i dag (\textbf{henvisning mangler}).}.

Halvorsen skelner mellem tre ækvivalente arbejdsløshedsdiskurser \parencite[13]{Halvorsen1999}. \textit{Elendighedsdiskursen} handler om, at arbejdsløshed er lig med social død. Utallige historier i medierne knytter sig til denne diskurs med overskrifter som for eksempel: “Knæk. Arbejdsløshed rammer hele familien” (Politiken, 19.04.2013), “Arbejdsløse rammes af stress” (Politiken, 18.07.2010), “Ekspert: Unge arbejdsløse risikerer ar mange år frem” (Berlingske Tidende, 26.11.2010) og “Arbejdsløse frygter aldrig at finde job igen” (Politiken, 01.01.2011). \textit{Beskæftigelsesdiskursen} handler om, at arbejdsløshed er lig med sløseri med ressourcer. Dette fylder også en del i finansnyhederne, som eksempelvis “Arbejdsløse koster kassen” (Ekstra Bladet, 14.11.2008), “Ledighed sender folk i sygesengen” (Jyllands-Posten, 18.03.2013) og “Høj ledighed truer EUs økonomi” (Berlingske, 03.07.2014). \textit{Moraldiskursen} handler om, at arbejdsløshed skyldes dovenskab og manglende arbejdsmotivation. Denne diskurs har fyldt meget i mediedebatten med overskrifter som “Joachim B. til arbejdsløs: Du er for slap” (Politiken.dk, 19.04.2012), “Dovne Robert på kontanthjælp i 11 år: Hellere kontanthjælp end et lortejob” (Ekstra Bladet, 11.09.2012) og “Vi arbejdsløse bliver opfattet som dumme, dovne og dårlige mødre” (Politiken, 06.11.2014). Alle tre diskurser er fælles om, at arbejdsløshed anskues som et onde. Den danske realpolitik er domineret af de to sidstnævnte diskurser ved, at arbejdsløse mødes med de “økonomiske realiteter” eller “nødvendighedens politik” (\textbf{henvisning mangler}) med den føromtalte dagpengereform fra 2010 samtidig med, at denne reform og kommende reformer bakkes op af udsagn som “Det skal kunne betale sig at arbejde” \parencite{Stoejberg2015}. %%%% Emil: kort analyse hvem/hvad diskurserne tjener. Der mangler en Enhedslisten/venstrefløjdiskurs. Skrives mere ud.


%%%%%%%%%%%%%%%%%%%%%%%%%%%%%%%%%%%%%%%%%%%%%%%%%%%%%%%%%%%
\subsubsection{Opsummering} 

I dette afsnit har vi kontekstualiseret arbejdsløshed som problemstilling ved kort at skitsere arbejdsløshed i en historisk og en nutidig kontekst i et dansk perspektiv. Historisk er danske arbejdsløses forhold løbende blevet styrket i perioden fra den første lov om arbejdsløse i 1907 til 1970'erne. Siden har arbejdsløses forhold i højere grad været til debat og under under pres. Ifølge Halvorsen dominerer diskurserne “arbejdsløshed som social død” (elendighedsdiskursen), “arbejdsløshed som sløseri med ressourcer” (beskæftigelsesdiskursen) og “arbejdsløshed som dovenskab” (moraldiskursen) i større eller mindre grad synet på arbejdsløshed i hverdagen og den offentlige debat. Disse diskurser knytter sig ligeledes til de forskellige videnskabsdiscipliner, hvilket vi vil komme ind på i afsnittene om henholdsvis økonomiske og sociologiske forståelser af arbejdsløshed.








%%%%%%%%%%%%%%%%%%%%%%%%%%%%%%%%%%%%%%%%%%%%%%%%%%%%%%%%%%%
\newpage \subsection{\textsc{Økonomiske forståelser af arbejdsløshedsproblemet}}
%%%%%%%%%%%%%%%%%%%%%%%%%%%%%%%%%%%%%%%%%%%%%%%%%%%%%%%%%%%

Fra 1970'erne og fremefter har økonomisk teori haft en betydelig gennemslagskraft i arbejdsløshedsforskning samtidig med at være det ideologiske grundlag for de lovændringer på arbejdsløshedsområdet, som har været med til at sænke arbejdsløshedsydelserne og været med til at kontrollere de arbejdsløses bestræbelser på at få et arbejde i Danmark jævnfør \ref{teori_arbejdsloeshed} \parencite[19]{Andersen2003} \parencite[1679]{Atkinson1991}. De økonomiske modeller er orienteret mod at sikre en effektiv økonomi og et effektivt fungerende arbejdsmarked. Disse modeltyper som økonomerne benytter sig af svarer til \textit{covering law-modellen}\footnote{Carl Hempel udviklede covering law-modellen for at forstå naturvidenskabelige forklaringer \parencite[15]{Hedstroem2005}.}, hvor et foreliggende faktum forklares ud fra andre udsagn, herunder minimum en almen lov, som det pågældende faktum derfor er underordnet. Det vil sige, at for at forklare et faktum, bygger økonomerne oftest deres modeller på en eller flere antagelser.

Dette teoretiske afsnit om økonomiske forståelser af arbejdsløshed indeholder først en definition af arbejdsløshed. Herefter følger arbejdsløshed på lang sigt og modellerne trade-off mellem arbejde og fritid, basal jobsøgningsteori og principal-agent-modellen  Afslutningsvis følger arbejdsløshed på kort sigt samt modellen hysterese.


%%%%%%%%%%%%%%%%%%%%%%%%%%%%%%%%%%%%%%%%%%%%%%%%%%%%%%%%%%%
\subsubsection{Definition på arbejdsløshed} %%% Emil: vigtigt afsnit, men det er alt for indforstået

Arbejdsløshed defineres typisk inden for den økonomiske disciplin ud fra et anvendelsesorienteret sigte. Den ene af de to mest udbredte definitioner af arbejdsløshed kommer af, at det er muligt at måle antallet af individer på arbejdsløshedsunderstøttelse på et givet tidspunkt i løbet af året \parencite[594]{Mankiw2011}. Den anden definition anvendes i arbejdsmarkedsundersøgelser, stammer fra \textit{International Labour Organisation} og består af antallet af personer som står uden beskæftigelse samtidig med at være til rådighed for arbejdsmarkedet og aktivt arbejdssøgende \parencite{ILO1982}. Fordelen ved at anvende arbejdsmarkedsundersøgelser til at måle arbejdsløshed kontra at måle antallet af individer på arbejdsmarkedsunderstøttelse er, at lovændringer kan resultere i ændringer i, hvem som har ret til arbejdsløshedsunderstøttelse. %%% Emil: Uddybes

De arbejdsløse er sammen med beskæftigede per definition arbejdsstyrken, mens den resterende ikke økonomisk aktive del  befolkningen i den arbejdsdygtige alder per definition står uden for arbejdsstyrken\footnote{Denne opdeling anvendes til at udregne arbejdsløshedsraten, som er lig med procentdelen af arbejdsstyrken som er arbejdsløse, og arbejdsmarkedsdeltagelsesraten, som er lig med procentdelen af den befolkningen i den arbejdsdygtige alder som er en del af arbejdsstyrken. \parencite[595]{Mankiw2011}. Forholdet mellem arbejdsløshed og beskæftigelse er kompleks, fordi det er muligt for arbejdsløshedsraten at falde samtidig med at beskæftigelsen ikke stiger, hvis for eksempelvis arbejdsstyrken skrumper \parencite[449]{Cahuc2004}.}. Atkinson har kritiseret de fleste arbejdsmarkedsmodeller for at antage, at arbejdsløshedstilstande både starter og slutter med beskæftigelse uden at tage højde for at de i mange tilfælde både starter og slutter uden for arbejdsstyrken \parencite[1683]{Atkinson1991}. Atkinson er endvidere kritisk over for at behandle beskæftigede, arbejdsløse og personer uden for arbejdsstyrken som homogene kategorier \parencite[1683]{Atkinson1991}. Når man skelner mellen arbejdsløse og personer uden for arbejdsstyrken, kan man have dem som er aktivt jobsøgende, men som ikke modtager eller ikke har ret til arbejdsløshedsydelse. Omvendt kan man også have dem, som modtager arbejdsløshedsydelser, men som ikke er aktivt jobsøgende. Som et tredje eksempel er der også de såkaldte “modløse arbejdere”, som hverken søger job eller modtager arbejdsløshedsydelser\footnote{Beskæftigede er også heterogen kategori, hvor man kan komme i beskæftigelse efter at have været arbejdsløs som selvstændig eller lønmodtager, fuldtid eller deltid og hvad Atkinson kalder almindelige eller marginale jobs \parencite[1685]{Atkinson1991}, hvilket minder om Guy Standings definition af prækære jobs \parencite[168]{Standing2011}.}. %%% svært at forstå

Fuld beskæftigelse anvendes i forbindelse med et konkurrencedygtigt arbejdsmarked, hvor lønnen er i equilibirum samtidig med at være lig med værdien af arbejdskraftens marginalprodukt. Equilibrium modellen, som dette udsagn også kaldes, hænger sammen med de grundlæggende antagelser om \textit{udbud}, \textit{efterspørgsel} og \textit{arbejdskraftens marginalprodukt}. På arbejdsmarkedet er arbejdskraft, jord og kapital de inputs som anvendes til produktion af varer og ydelser. Efterspørgslen på arbejdskraft bygger på antagelserne om, at virksomheder er \textit{konkurrencedygtige} og \textit{profitmaksimerende} \parencite[383]{Mankiw2011}. Når den konkurrencedygtige virksomhed skal ansætte en person, skal der tages højde for antallet af ansattes påvirkning af produktionen af varer. Hvis arbejdskraftens marginalprodukt er profitabelt, kan det betale sig for virksomheden at ansætte en ny person \parencite[384]{Mankiw2011}. Når arbejdstagere er ansat til at arbejde til en løn i perfekt balance mellem udbud og efterspørgsel samtidig med, at lønnen er lig med værdien af arbejdskraftens marginalprodukt, så vil der være fuld beskæftigelse. Da fuld beskæftigelse reelt ikke eksisterer, skelner økonomiske arbejdsløshedsstudier mellem arbejdsløshed på lang sigt og arbejdsløshed på kort sigt, som vi henholdsvis vil beskæftige os med i det kommende to afsnit. %%% Emil: Forklar equilibrium + hvad betyder "Lønnen er lig med værdien af arbejdskraftens marginalprodukt"


%%%%%%%%%%%%%%%%%%%%%%%%%%%%%%%%%%%%%%%%%%%%%%%%%%%%%%%%%%%
\subsubsection{Arbejdsløshed på lang sigt}

Arbejdsløshed på lang sigt kaldes også den naturlige arbejdsløshedsrate, som er lig med den mængde arbejdsløshed en økonomi normalt er udsat for \parencite[592]{Mankiw2011}. På lang sigt skyldes arbejdsløshed friktionsledighed, strukturel lighed og sæsonledighed\footnote{I praksis er det vanskeligt at udskille friktions- og sæsonledighed fra strukturledighed, hvilket betyder, at de alle sammen oftest optræder som strukturarbejdsløshed \parencite{2015}.}. %%%% Emil: hvad betyder normalt
\textit{Friktionsledighed} som også kaldes skifteledighed opstår ofte i forbindelse med indtræden på arbejdsmarkedet eller ved jobskifte. Det tager tid at matche den rette arbejdstager til det rette job, fordi arbejdstagerne blandt andet har forskellige præferencer og færdigheder og jobs kræver forskellige færdigheder\footnote{Det bryder med equilibrium-modellen som antager, at alle arbejdstagere kan passe hvilket som helst jobs, fordi alle arbejdstagere og alle jobs er identiske. Hvis dette var sandt, og arbejdsmarkedet var i equilibrium, så ville et jobtab ikke medfører arbejdsløshed, fordi en fyret arbejdstager ville finde et nyt job til markedsløn \parencite[163]{Mankiw2007}.} \parencite[163]{Mankiw2007}.
\textit{Strukturel ledighed} opstår ved manglende faglig eller geografisk fleksibilitet på arbejdsmarkedet. Dette betyder, at lønnens balance mellem udbud og efterspørgsel ikke er i stand til at sikre fuld beskæftigelse\footnote{Dette bryder ligeledes med equilibrium-modellen, hvor de reelle lønninger tilpasser sig i perfekt balance med udbud og efterspørgsel. Men lønninger er ikke altid fleksible, fordi de reelle lønninger kan være fastsat over markedsniveau. Denne lønstivhed medfører strukturel arbejdsløshed, fordi den udbudte mængde af arbejdskraft er større end den efterspurgte mængde af arbejdskraft. Årsagerne her til er blandt andet minimumslønninger, fagforeningerne og effektivitetslønninger, som har til fælles at skabe lønninger over equilibrium \parencite[165]{Mankiw2007}.} \parencite[600]{Mankiw2011}.
\textit{Sæsonledighed} opstår på grund af sæsonmæssige svingninger i produktionen, hvor en stor del af sæsonarbejdsløsheden forekommer er midlertidig hjemsendelsesledighed, idet den arbejdsløse efter en kortere periode som arbejdsløs kommer tilbage til den samme arbejdsgiver \parencite{2015}.

Modellerne trade-off mellem arbejde og fritid, den basale jobsøgningsteori og principal-agent-modellen, som vil blive gennemgået i forbindelse med arbejdsløshed på lang sigt beskæftiger sig primært med arbejdsløshed som friktionsledighed.

%%%%%%%%%%%%%%%%%%%%%%%%%%%%%%%%%%%%%%%%%%%%%%%%%%%%%%%%%%%
\textbf{Trade-off'et mellem arbejde og fritid} bygger på en antagelse fra den neoklassiske økonomiske teori om, at for at have et arbejde skal man have valgt at tage et. Individet har begrænset tid, og står derfor over for et trade-off mellem arbejde og fritid\footnote{Implicit er dette også et trade-off mellem at forbruge goder og forbruge fritid \parencite[5]{Cahuc2004}.}. Det betyder, at når en handling vælges frem for andre mulige handlinger er der omkostninger herved \parencite[389]{Mankiw2011}. Eksempelvis, hvis hvis man vælger at holde en times fri frem for at arbejde den samme time til en timeløn på 100 kroner, så er omkostningerne herved altså 100 kroner\footnote{Trade-off'et mellem arbejde og fritid er senere hen blevet mere kompleks. Eksempelvis er den tid man ikke arbejder ikke nødvendigvis blot fritid, fordi denne tid kan bruges på produktion i husholdningen, som jo også er et supplement for ens lønindtægter \parencite[14]{Cahuc2004}.}. Ifølge Halvorsen hænger trade-offet mellem arbejde og fritid sammen med, at den økonomiske disciplin ser arbejdet som et onde samtidig med, at individet er rationelt og nyttemaksimerende \parencite[26]{Halvorsen1999}.

%%%%%%%%%%%%%%%%%%%%%%%%%%%%%%%%%%%%%%%%%%%%%%%%%%%%%%%%%%%
\textbf{Den basale jobsøgningsmodel} kaldes også for den partielle model og blev udviklet i 1970'erne af blandt andet McCall og Mortensen \parencite[109]{Cahuc2004}. Teorien forklarer, hvorfor friktionsledighed forekommer og bryder hermed med to væsentlige neoklassiske perspektiver. Den første perspektiv er, at der er fuld beskæftigelse, når arbejdsmarkedet er i equilibrium. Her vil en fyret arbejdstager automatisk vil finde et nyt job til markedsløn, fordi alle arbejdstagere kan varetage alle jobs \parencite[163]{Mankiw2007}. Det anden perspektiv er, at individer har \textit{fuldkommen information} om arbejdsmarkedet, og derfor ikke har brug for tid til at søge arbejde. Jobsøgningsteorien tager udgangspunkt i jobsøgning som en proces, hvor individer under ufuldkommen information har brug for tid til at finde de rette jobs i forhold til deres præferencer og færdigheder med henblik på at få den højeste løn som betaling for sin ydelse \parencite[108]{Cahuc2004}. Den optimale søgestrategi for en person som leder efter arbejde består i at vælge en \textit{reservationsløn}. En reservationsløn er den laveste løn en person er villig til at acceptere, hvilket betyder, at alle jobtilbud under dette beløb afvises \parencite[114]{McCall1970}. Og jo højere reservationslønnen er sat til, jo længere vil den gennemsnitlige arbejdsløshedsperiode være \parencite[848]{Mortensen1970}. Valget af reservationsløn kan betragtes som en bestræbelse på at maksimere egennytte. Det vil sige, at fordelene man kan opnå ved at acceptere et job dags dato må opvejes med fordelene i form af bedre job på længere sigt. Et bedre job på længere sigt kan både give en bedre løn, men kan også medfører indkomsttab mens søgningen foregår \parencite[1698]{Atkinson1991}.


%%%%%%%%%%%%%%%%%%%%%%%%%%%%%%%%%%%%%%%%%%%%%%%%%%%%%%%%%%%
\textbf{Principal-agent-modellen} udvikles fra slutningen af 1970'ernes af blandt andet Baily og Flemming på baggrund af den basale jobsøgningsmodel. Her tages der udgangspunkt i en kontrakt mellem en principal og en agent, som gør, at agenten kan tage flere risici, fordi principalen bærer byrden af disse risici. Adfærden hos principalen og agenten er afhængig af om jobsøgningsindsatsen er kontrollerbar eller ej, hvilket vil sige, om der er bevis for, at agenten virkelig har foretaget en indsats for at søge job \parencite[134]{Cahuc2004}. Når indsatsen \textit{er} kontrollerbar, kan principalen gøre udbetalingen af arbejdsløshedsydelsen afhængig af agenternes jobsøgningsindsats, hvilket betyder, at den optimale kontrakt mellem principal og agent giver agenten fuld kompensationsgrad for at søge arbejde, når agenten er arbejdsløs \parencite[138]{Cahuc2004}. Når indsatsen \textit{ikke} er kontrollerbar er der mulighed kan der være mulighed for, at agenten modtager en arbejdsløshedsydelse uden at gøre en indsats for at søge arbejde. Derfor er det nødvendigt for principalen, at den optimale kontakt \textit{ikke} giver fuld kompensationsgrad for arbejdsløshed med henblik på at give agenten større incitament for at søge arbejde\footnote{Baily definerer det optimale arbejdsløshedsforsikringsystem som en afvejning mellem jobsøgningsincitament og arbejdsløshedsforsikringen.
Den marginale gevinst af arbejdsløshedsforsikring er lig den marginale omkostning ved øget arbejdsløshed. Derved påviser Baily som en af de første sammenhængen mellem incitament for at søge jobs og størrelsen på arbejdsløshedsforsikring \parencite[379]{Baily1978}.} \parencite[379]{Baily1978}. %%% Baily er uforståelig

%%%%%%%%%%%%%%%%%%%%%%%%%%%%%%%%%%%%%%%%%%%%%%%%%%%%%%%%%%%
Ifølge Atkinson bygger politiske valg om at sænke ydelsesniveauerne for arbejdsløse på baggrund af en oversimplificering af arbejdsmarkedet, hvor økonomer typisk ser arbejdsløshedsydelser som havende en negativ effekt på arbejdsmarkedet med høje ydelser som forårsager, at arbejdsløse er mindre villige til at tage et arbejde \parencite[1680]{Atkinson1991}. Atkinson selv kritiserer modellerne for typisk at antage, at arbejdsløshedens effekt kan reduceres til beløbet på arbejdsløshedsydelsen uden at forholde sig til de institutionelle forhold i arbejdsløshedssystemet\footnote{I OECD-lande er betingelserne for modtagelse af arbejdsløshedsydelser eksempelvis typisk, at personen ikke må være frivillig arbejdsløs, der skal gøres en reel jobsøgningsindsats, man må ikke blive ved med at afvise jobtilbud og der er ydelsen dækker over en begrænset periode \parencite[1689]{Atkinson1991}.} \parencite[1688]{Atkinson1991}. De økonomiske modeller er altså ikke gode nok til at forklare institutionelle forhold som for eksempel kommunikation med jobcentret og a-kassen og deres påvirkning på den arbejdsløse på trods af forsøg på at tage højde for hvor længe der kan modtages ydelser, forskelle på arbejdsløshedsydelser og sociale ydelser og dem som ikke er berettiget til arbejdsløshedsydelser \parencite[1692]{Atkinson1991} \parencite[33-34]{Halvorsen1999} \parencite[114]{Cahuc2004}.

De økonomiske modeller forudsætter typisk en positiv sammenhæng mellem søgeintensitet og antallet af jobtilbud man modtager. Modellerne har dog typisk ikke fokus på, at jobsøgningen ikke er enstationær tilstand og jo længere arbejdsløsheden varer, jo færre jobtilbud vil der for det meste komme \parencite[119]{Cahuc2004}. Ifølge Halvorsen er et væsentligt problem med jobsøgningsmodellerne, at arbejdsløse næsten altid vil tage imod det første tilbud de modtager, hvilket betyder at variationer i arbejdsløshedslængden primært opstår ud fra sandsynligheden for at modtage et jobtilbud \parencite[28]{Halvorsen1999}.

Omvendt er en høj arbejdsløshedsydelsen også blevet sat i et mere positivt lys, fordi det kan bidrage til at have en positiv effeekt på søgeaktiviteten. Ifølge Tatsiramos er fordelene ved at modtage arbejdsløshedsforsikring større end omkostningerne. Selvom en høj arbejdsløshedsforsikring kan medføre en længere arbejdsløshedsperiode, viser et europæisk studie, at de arbejdsløse som modtager arbejdsløshedsydelse forbliver 2-4 måneder længere i det efterfølgende jobs end arbejdsløse som ikke modtager en arbejdsløshedsydelse \parencite[602]{Mankiw2011}. Dette perspektiv åbner op for, at et job ikke bare er et job, hvilket kommer af at beskæftigelse som tidligere nævnt ligesom arbejdsløshed og personer uden for arbejdsstyrken er en heterogen kategori. %%% Det ligger måske endu mere op til Bourdieu

 Afslutningsvis skal det nævnes, at de økonomiske modeller typisk ser også bort fra ikke-økonomiske incitamenter, hvor lønarbejde typisk foretrækkes frem for arbejdsløshed, fordi det giver selvrespekt, anderkendelse og man lever op til de sociale forventninger. Disse sociale og psykologiske forhold dominerer den sociologiske arbejdsløshedsforskning som vi som før nævnt vil gå i dybden med senere \parencite{Jahoda1971, Eisenberg1938, Ezzy1993, Halvorsen1999, Baum2001, Noerup2014}. %%% Emil: Senere - udnyt latex


%%%%%%%%%%%%%%%%%%%%%%%%%%%%%%%%%%%%%%%%%%%%%%%%%%%%%%%%%%%
\subsubsection{Arbejdsløshed på kort sigt} %%%% Emil: Bedre overgang måske

Arbejdsløshed på kort sigt kaldes også konjunkturledighed, og karakteriseres ved økonomiske udsving fra år til år \parencite[592]{Mankiw2011}. John Maynard Keynes forsøgte som den første at forklare disse udsving i forbindelse med den depressionen i 1930'erne.  Ifølge Keynes ville lønninger og priser ikke nødvendigvis tilpasse sig på kort sigt. På den måde ville økonomien være i en position, hvor efterspørgslen ikke var tilstrækkelig til at skabe en beskæftigelse svarende til fuld beskæftigelse. Når økonomien ikke ville kunne tilpasses på kort sigt, mente Keynes, at staten burde gribe ind og administrere efterspørgslen for at opnå det ønskede beskæftigelsesniveau \parencite[707]{Mankiw2011}. Det kan staten typisk gøre ved finanspolitisk at øge forbruget for at booste den økonomiske aktivitet\footnote{Et eksemepl herpå kunne være, når staten laver en kontrakt på 10 milliarder for at bygge tre nye atomkraftværker. Hermed skabes beskæftigelse og profit hos byggefirmaet, som resulterer i beskæftigelse og profit hos underleverandørerne. Alt i alt skaber der et forbrug, hvor hvert pund som er brugt får den samlede efterspørgsel på varer og ydelser til at stige for mere end et pund. Dette kaldes også multiplikatoreffekten som er defineret ved at være supplerende ændringer i den samlede efterspørgsel som finder sted, når  ekspansiv finanspolitik øger indkomst og dermed øger privatforbruget. \parencite[709]{Mankiw2011}.} eller ved pengepolitik, hvis centralbanken beslutter sig for at udvide mængden af penge \parencite[718]{Mankiw2011}. Udover Keynes insisteren på vigtigheden af arbejdsløshed på kort sigt spiller \textit{Phillipskurven} en stor rolle. Phillipskurven viser en negativ sammenhæng mellem arbejdsløshedsraten og inflationsraten, hvilket vil sige at år med lav arbejdsløshed har høj inflation, og år med høj arbejdsløshed har lav inflation \parencite[783]{Mankiw2011}. Finanspolitik og pengepolitik kan flytte økonomien langs Phillipskurven. Øgninger af pengemængden, øger det offentlige forbrug eller skattelettelser udvider den samlede efterspørgsel og flytter økonomien til et punkt i Phillipskruven, som giver et trade-off med lav arbejdsløshed for høj inflation \parencite[785]{Mankiw2011}.

%%%%%%%%%%%%%%%%%%%%%%%%%%%%%%%%%%%%%%%%%%%%%%%%%%%%%%%%%%%
\textbf{Hysterese} kendetegner midlertidige chok på økonomien, som har permanente eftervirkninger på arbejdsløshedsniveauet. Hermed er det muligt for, at arbejdsløshed på kort sigt har en effekt på arbejdsløsheden på lang sigt. Det kan ske ved, at visse arbejdsløse bliver ved med at være ekskluderet fra arbejdsmarkedet, fordi deres produktivitet er så lav, at det ikke er profitabelt at ansatte dem selv til en lav løn. Hvis der ikke er nogen måde at reintegrerer de arbejdsløse på, så vil de have en vedholdende effekt på arbejdsløshedsraten \parencite[477]{Cahuc2004}. Dette hænger sammen med den lave beskæftigelse af langtidsledige. De langtidsledige har svært ved at komme i beskæftigelse på grund af manglende påskønnelse af deres human kapital, manglende motivation i jobsøgning og kendsgerningen, at en lang arbejdsløshedsperiode kan fortolkes som en signal om at arbejdstagerens kvalitet ved ansættelsen kan forklare den dårlige præstation hos den langtidsledige \parencite[479]{Cahuc2004}.


%%%%%%%%%%%%%%%%%%%%%%%%%%%%%%%%%%%%%%%%%%%%%%%%%%%%%%%%%%%
\subsubsection{Opsummering}

Dette afsnit har behandlet arbejdsløshed i lyset af, at økonomisk teori har domineret arbejdsløshedsforskningen samtidig med at være det ideologiske grundlag for politiske beslutninger på arbejdsløshedsområdet. Inden for den økonomiske disciplin inddeles befolkningen typisk i arbejdsstyrken, som består af beskæftigede og arbejdsløse, og i den ikke-økonomisk aktive del af befolkningen, som står uden for arbejdsstyrken. I dette afsnit har vi beskæftiget os med en række modeller i relation til arbejdsløshed på kort og lang sigt. Arbejdsløshed på lang sigt kan opdeles i friktionsledighed, strukturel ledighed og sæsonledighed. Trade-off'et mellem arbejde og fritid, den basale jobsøgningsmodel og principal-agent-modellen er med til at forklare, hvorfor arbejdsløshed opstår, og hvordan man kan få arbejdsløse i beskæftigelse. Arbejdsløshed på kort sigt  opstår i forbindelse med konjunkturændringer, hvor hysterese er med til at forklare arbejdsløshed som konsekvens heraf. De økonomiske modeller er blevet kritiseret for at oversimplificere arbejdsløshedsproblemet ved at anvende for simple kategorier og for at reducere incitament til et økonomisk spørgsmål uden at forholde sig til institutionelle, sociale og psykologiske spørgsmål.







%%%%%%%%%%%%%%%%%%%%%%%%%%%%%%%%%%%%%%%%%%%%%%%%%%%%%%%%%%%
\newpage \subsection{\textsc{Sociologiske forståelser af arbejdsløshedsproblemet}}
%%%%%%%%%%%%%%%%%%%%%%%%%%%%%%%%%%%%%%%%%%%%%%%%%%%%%%%%%%%

Sociologien står i skarp kontrast til økonomien. Granovetter som er en af de mest førende sociologiske kritikere af den økonomiske diciplin definerer den økonomiske aktør som undersocialiseret: “Actors do not behave or decide as atoms outside a social context, nor do they adhere slavishly to a script written for them by the particular intersection of social categories that they happen to occupy. Their attempts at purposive action are instead embedded in concrete, ongoing systems of social relations.” \parencite[487]{Granovetter1985}. Hvor økonomien mangler psykologisk og social realisme, beskæftiger sociologien sig med identitet, social integration, normer og menneskelig adfærd såvel som et statistisk baseret styringsvidenskab efter et mere økonomisk mønster \parencite[36]{Halvorsen1999}. Sociologisk arbejdsløshedsforskning er desuden domineret af empirisk variabelanalyse, årsags-virkningsforhold og mekanismer som kan fører til dårligt mentalt helbred \parencite[38]{Halvorsen1999}. Størstedelen af arbejdsløshedsforskerne benytter sig af middle-range teorier \parencite[9]{Hedstroem2005}, som beskriver en begrænset del af den sociale virkelighed i modsætning grand theories. %%%% EmIL: "Sociologisk arbejdsløshedsforskning er desuden..." er en lidt uklar formulering

Dette teoretiske afsnit om sociologiske forståelser af arbejdsløshed indeholder først en strukturbaserede teorier som primært beskæftiger sig med sociale og psykologiske konsekvenser af arbejdsløshed herunder, stadiemodellen, Jahodas funktionelle deprivationsteori, Warrs vitaminmodel og marginaliseringsperspektivet. Herefter følger en række aktørbaserede teorier herunder rehabiliteringstilgangen, Fryers agency kritik, status passagemodellen og Halvorsens mestringsperpektiv. 


%%%%%%%%%%%%%%%%%%%%%%%%%%%%%%%%%%%%%%%%%%%%%%%%%%%%%%%%%%%
\subsubsection{Strukturbaserede teorier}

De negative konsekvenser af arbejdsløshed blev først beskrevet Marienthal-studiet af Jahoda, Lazarsfeld og Zeizel som foregår i den østrigske by Marienthal, hvor en voldsom arbejdsløshed havde ført til apati blandt de arbejdsløse \parencite[vii]{Lazarsfeld1971}. Sidenhen er der foretaget en række strukturbaserede teorier og modeller som er særlig relevante i forhold til sociale og psykologiske konsekvenser af arbejdsløshed herunder stadiemodellen, Jahodas funktionelle model, Warr's vitaminmodel og marginaliseringsperspektivet.

Lazarsfeld udvikler sammen med Eisbenberg en model som trækker erfaringerne fra Marienthal og en lang række andre arbejdsløshedsstudier. Jahoda, Lazarsfeld og Zeizel skelnede mellem det at være knækket, resigneret, fortvivlet og apatisk som reaktioner på arbejdsløshed i Marienthal \parencite[56]{Jahoda1971}. \textbf{Stadiemodellen} konkluderer, at arbejdsløse gennemgår tre forskellige stadier: Først oplever den arbejdsløse et chok opfulgt af en aktiv indsats for at søge job, hvor den arbejdsløse stadigvæk er optimistisk og ikke knækket. Dernæst, når al indsats er mislykket, bliver den arbejdsløse pessimistisk, fortvivlet og føler sig i nød. Til sidst bliver den arbejdsløse knækket \parencite[378]{Eisenberg1938}. Stadiemodellen er dog blevet kritiseret for at være metodisk problematisk, teorien modsætningsfuld og kategorierne hule. Ifølge Ezzy er modellen ikke så meget en teori som et deskriptivt framework, hvor den operationelle variabel er længden på arbejdsløshed \parencite[44]{Ezzy1993}.

Det centrale i \textbf{Jahodas funktionelle deprivationsteori} er, at arbejdsløshed medfører en række afsavn, som den arbejdsløse ville have fået ved at være i beskæftigelse \parencite[44]{Ezzy1993}. Udgangspunktet er Mertons opdeling af sociale funktioner som manifeste og latente. Arbejdets manifeste funktion består i at opfylde et økonomisk behov for indtægt, mens arbejdets latente funktion består i at opfylde et psykologisk behov: “First, employment imposes a time structure on the waking day; second, employment implies regularly shared experiences and contacts with people outside the nuclear family; third, employment links individuals to goals and purposes that transcend their own; fourth, employment defines aspects of personal status and identity; and finally, employment enforces activity.” \parencite[188]{Jahod1981}. Som en parallel til trade-offet mellem arbejde og fritid, kan “fritiden” hos den arbejdsløse ikke udfylde de samme funktioner som arbejdet har \parencite[189]{Jahod1981}. Ifølge Ezzy imødekommer lønarbejdet ikke automatisk individets grundlæggende psykologiske og sociale behov, Jahoda romantiserer derfor og ser bort fra, at lønarbejde for nogle kan være isolerende og ubehageligt \parencite[45]{Ezzy1993}.

I modsætning til Jahoda giver Warrs \textbf{vitaminmodel} mulighed for at skelne mellem at have et godt eller dårligt mentalt helbred både som beskæftiget og som arbejdsløs. Warr foreslår, at på samme måde som vitaminer har en effekt på det fysiske helbred, så har forskellige forhold i ens omgivelser en effekt på det mentale helbred. Disse forhold eller “vitaminer” består af muligheden for at have kontrol over ens tilværelse, muligheden for at anvende ens færdigheder, ydre mål, variation, klarhed i forhold til omgivelserne, penge, fysisk sikkerhed, social kontakt og værdsat social position \parencite[45]{Ezzy1993}. Det vil sige, at når vitaminniveauerne er lave i en utilfredsstillende jobtilværelse eller som arbejdsløs, går det ud over det mentale helbred. Samtidig er der også mulighed for at skelne mellem vitaminniveauet hos eksempelvis arbejdsløse middelalderende mænd med større problemer i forhold til penge, sikkerhed, personlig kontakt og en værdsat social position end hos arbejdsløse teenagere, som er mindre afhængige af arbejde for at have tilstrækkelige niveauer inden for de nævnte vitamintyper \parencite[46]{Ezzy1993}.

\textbf{Marginaliseringsperspektivet} anvendes både på arbejdsmarkedet, men også en række andre områder. Ifølge Elm larsen skal marginaliseringsperspektivet ses som en midterkategori mellem inklusion og eksklusion. Larsen definerer eksklusion som en ufrivillig ikke-deltagelse gennem forskellige typer af udelukkelsesmekanismer og -processer, som det ligger uden for indvidets og gruppens muligheder at få kontrol over \parencite[137]{Larsen2009}. Hermed er Larsen kritisk over for Luhmanns dikotomi inklusion/eksklusion, som Larsen i dens binære form ikke mener er særlig hensigtsmæssig i forhold til at beskrive virkeligheden \parencite[130]{Larsen2009}. Derfor argumenter han for, at marginalisering kan anvendes som en midtergruppe mellem de to, hvor individet bevæger sig i en proces mod inklusion eller eksklusion. På baggrund af dette har vi tegnet følgende model:\footnote{Modellen er også inspireret af lignende modeller benyttet af Lars Svedberg \parencite[44]{Svedberg1995} og Catharina Juul Kristensen \parencite[18]{Kristensen1999}.} som fremgår af tabel \ref{tab_marginaliseringsmodel_1}. 
% 
\begin{table}[H] \centering
\caption{Model over marginalisering}
\label{tab_marginaliseringsmodel_3}
\begin{tabular}{@{} m{3,4cm} c m{3,6cm} c m{3,6cm} @{}} \toprule
\textbf{Inkluderet} & & \textbf{Marginaliseret} & & \textbf{Ekskluderet} \\ \midrule
\end{tabular} \end{table} %%%%%
\begin{table}[H] \centering
\label{tab_marginaliseringsmodel}
\begin{tabular}{@{} m{5,9cm} m{5,9cm} @{}} 
  \textbf{Marginaliseringsproces} & \textbf{Eksklusionsproces} \\  
  --------------------------------------------> & --------------------------------------------> \\ 
\end{tabular} \end{table} %%%%%
\begin{table}[H] \centering
\label{tab_marginaliseringsmodel}
\begin{tabular}{@{} m{12,3cm} @{}} 
  \textbf{Inklusionsproces} \\  
  <--------------------------------------------------------------------------------------------- \\ \bottomrule
\end{tabular} \end{table}
%
Den første proces består af individer som går fra at være inkluderet til at indgå i en proces i retning mod marginalisering. Den anden proces består af individer som går fra at være marginaliseret til at indgå i en proces i retning mod inklusion. Marginaliseringsperspektivet giver hermed mulighed for at se de  bevægelser på arbejdsmarkedet inden for kategorier som ikke rigide.


%%%%%%%%%%%%%%%%%%%%%%%%%%%%%%%%%%%%%%%%%%%%%%%%%%%%%%%%%%%
\subsubsection{Aktørbaserede teorier}

De aktørbaserede teorier beskæftiger sig mere mindre med de negative sociale og psykologiske konsekvenser for den arbejdsløse og mere med, hvad agenten gør, ikke gør eller kan gøre for at komme ud af arbejdsløshedssituationen herunder rehabiliteringstilgangen, Fryers agency kritik, status passagemodellen og Halvorsens mestringsperpektiv.

\textbf{Fryers agency kritik} er ikke så meget en teori, men mere en direkte kritik af at beskæftige sig med passive aktører i arbejdsløshedsteorier såsom Jahoda og Warrs. Fryer foreslår, at det individet bringer til situationen er lige så vigtigt som konsekvenserne af arbejdsløshed. De proaktive arbejdsløse i Fryers studie oplever ikke sociale og psykologiske afsavn på trods af, at de lider af de materielle afsavn, som kommer af at miste indtægterne fra deres arbejde. Ifølge Fryer er disse proaktive arbejdsløse eksempler på aktive sociale agenter, som forsøger at få mening ud af deres situation and agerer ud fra målsætninger \parencite[47]{Ezzy1993}. Omvendt er Fryer blevet kritiseret for at lægge for meget vægt på kognitive processer og ignorerer de institutionelle begrænsninger som arbejdsløsheden oftest medfører \parencite[47]{Ezzy1993}.

Tiffany, Cowan og Tiffanys udvikler i 1970 \textbf{rehabiliteringstilgangen} på baggrund af et studie, som viser, at størstedelen af de arbejdsløse er arbejdsløse på grund af psykologiske årsager: “they show avoidance behavior pattern or what has been referred to as “work inhibition” which implies that they are physically capable of work but are prevented from working because of psychological disabilities” \parencite[43]{Ezzy1993}. Som konsekvens bør man ifølge Tiffany, Cowan og Tiffany rehabiliterer de arbejdsløse gennem terapi eller træning, så de kan vende tilbage på arbejdsmarkedet. Ifølge Ezzy ligner denne tilgang den historiske skelnen mellem dem, som var fysisk ude af stand til at arbejde og fortjente støtte, og dem, som var i fysisk stand til at arbejde, men som ikke arbejdede, fordi de var dovne, og derfor ikke fortjente støtte. Ezzy påpeger, at årsagen til, at tilgangen fokuser på individuelle forklaringer frem for strukturelle forklaringer er, at tilgangen har været mest toneangivende i perioder med relativ lav arbejdsløshed, mens de fleste andre sociologiske og psykologiske arbejdsløshedsstudier er foretaget i perioder med høj arbejdsløshed \parencite[43]{Ezzy1993}.
 
Ezzy anvender begrebet jobtab i stedet for arbejdsløshed\footnote{Denne problematik adskiller sig på den ene side fra andre veje til arbejdsløshed såsom dimittendledighed eller et vende tilbage til arbejdsstyrken igen og veje ud af beskæftigelse såsom pensionering, orlov eller at tage en uddannelse \parencite[48]{Ezzy1993}.} og sammenligner det at miste sit job med at gå i gennem et skilsmisseforløb, et sygdomsforløb eller opleve et dødsfald i familien. Disse overgange skal forstås i forlængelse af Glaser og Strauss' definition af en \textbf{status passage} som et individs ”movement into a different part of a social structure, or loss or gain of privilige, influence, or power, and changed behaviour” \parencite[48]{Ezzy1993}. En status passage er en del af individets biografi, hvilket involverer samtidige og tidligere erfaringer, som påvirker meningen der tillægges arbejdsløsheden. Ezzy skelner mellem integrative passager og afhændelsespassager\footnote{Denne skelnen er foretaget med udgangspunkt i Van Gennep skelnen mellem separation (for eksempel begravelse), transition (for eksempel jobskifte) og integration (for eksempel ægteskab) som kategorier for sociale passager\parencite[48]{Ezzy1993}.}. Integrative passager er oftest en overgangsperiode efterfulgt af integration i en ny status gennem en ceremonial proces som for eksempel bryllupper. Afhændelsespssager er en separation fra en status som oftest er en længerevarende overgangsfase med en usikker varighed for eksempel skilsmisser, sygdomsforløb, arbejdsløshed eller dødsfald \parencite[49]{Ezzy1993}. En afhændelsespassager fører ikke nødvendigvis i sig selv til mentale problemer, mistrivsel eller eksklusion, da det afhænger af den enkeltes identitet og selvopfattelse i relation til andre og samfundet og en lang række andre faktorer samt om afhændelsespassagen efterfølges af en reintegrativ passage, hvor den enkelte får en ny status eksempelvis et nyt job \parencite[32]{Noerup2014}.

Knut Halvorsen anskuer med \textbf{mestring}-perspektivet ligeledes de arbejdsløse som aktive aktører. De arbejdsløse kan dermed påvirke deres situation og være med til at forandre den i stedet for at være ofre for omstændighederne. For at kapere en negativ hændelse - som det at miste sit job er - indgår den arbejdsløse i forskellige psykiske, fysiske og sociale aktiviteter. Den problemorienterede mestring udgør konkrete strategier med formål at fjerne belastningen af den marginaliserede position for eksempel jobsøgning. Den emotionsorienterede mestring handler om, hvordan man ser verdenen for eksempel kan det at redefinere sig som hjemmegående ved at måde at søge ny basis for mening for at opretholde selvrespekten \parencite[47]{Halvorsen1999}. Ifølge Nørup ligger fokus i perspektivet, hvordan individet håndterer det, at stå midlertidigt uden for arbejdsmarkedet. På den måde forholder Halvorsen sig ikke til, hvad der sker med, når arbejdsløsheden bliver permanent eller langsigtet \parencite[30]{Noerup2014}. Nørup kritiserer endvidere Halvorsen for i sin ontologiske individualisme at fokuserer på, hvordan individet mestrer bestemte livssituationer under bestemte rammer frem for på hvordan de samfundsmæssige strukturer og sociale relationer påvirker eksklusionen \parencite[37]{Noerup2014}.


%%%%%%%%%%%%%%%%%%%%%%%%%%%%%%%%%%%%%%%%%%%%%%%%%%%%%%%%%%%
\subsubsection{Opsummering}
%%%%%%%%%%%%%%%%%%%%%%%%%%%%%%%%%%%%%%%%%%%%%%%%%%%%%%%%%%%

Vi har valgt at dele de sociologiske forståelser af arbejdsløshedsproblemet op i struktur- og aktørbaserede teorier. Fælles for de strukturbaserede teorier såsom stadiemodellen, Jahodas funktionelle deprivationsteori og Warrs vitaminmodel som i høj grad fokuserer på de sociale og psykologiske konsekvenser af arbejdsløshed. De aktørbaserede teorier såsom Fryers agency kritik, rehabiliteringstilgangen og Halvorsens mestringsperpektiv beskæftiger sig mere med, hvad agenten gør, ikke gør eller kan gøre for at håndtere arbejdsløsheden eller komme tilbage i beskæftigelse. Jahoda, Warr og andre strukturbaserede teoretikere er blevet kritiseret for at gøre de arbejdsløse til passive individer, mens Fryer, Halvorsen og andre agentbaserede teoretikere omvendt er blevet kritiseret for at se bort fra de strukturelle og institutionelle begrænsninger i arbejdsløshedsperspektivet. Jahoda er i særdeleshed blevet kritiseret for at fremhæve  lønarbejdet som et ubetinget gode, hvilket hænger sammen med det førnævnte elendighedsperspektiv på arbejdsløsheden. Denne beskrivelse af lønarbejdet står i modsætning til den økonomisk teori, der som nævnt så lønarbejdet som en byrde\footnote{Dette perspektiv har imidlertidigt være udfordret af den marxistiske tradition inden for sociologien, hvor lønarbejdet anset for at være fremmedgørende \parencite[48]{Halvorsen1999}.}. Status massagemodellen og marginaliseringsperspektivet giver mulighed for at anskue arbejdsløshed i bevægelse og dermed også mere dynamisk end flere af de sociologiske og økonomiske teorier.





%%%%%%%%%%%%%%%%%%%%%%%%%%%%%%%%%%%%%%%%%%%%%%%%%%%%%%%%%%%
\newpage \subsection{SFI - en case}
%%%%%%%%%%%%%%%%%%%%%%%%%%%%%%%%%%%%%%%%%%%%%%%%%%%%%%%%%%%

Det Nationale Forskningscenter for Velfærd (SFI) er storproducent af policy-studier på arbejdsmarkeds- og beskæftigelsesområdet med rapporter som typisk er bestilt af Beskæftigelesministeriet eller forskellige ministerier og kommuner. Rapporterne tager typisk udgangspunkt i en grupper af personer. Hvis man kigger på SFIs rapporter de sidste 20 år er målgrupperne hovedsageligt ledige\footnote{Også kaldet udsatte ledige, arbejdsmarkedsparate ledige, ikke-arbejdsmarkedsparate ledige, langtidsledige og forsikrede ledige}, dagpengemodtagere\footnote{Også kaldet sygedagpengemodtagere og aktiverede dagpengemodtagere.}, kontanthjælpsmodtagere\footnote{Også kaldet ikke arbejdsmarkedsparatemodtagere, de svageste kontanthjælpsmodtagere og aktiverede kontanthjælpsmodtagere.}, sygemeldte og arbejdsskadede\footnote{Også kaldet skadeslidte beskæftigede og personer som har nedsat arbejdsevne efter en ulykke i fritiden.} samt pensionister og efterlønsmodtagere\footnote{Der er også foretaget en hel del undersøgelse om handicappede, indvandrere, efterkommere, mænd, kvinder, ældre og højtuddannede.}. Fokus handler hovedsageligt om at få dem i beskæftigelse eksempelvis ved at bringe de langtidsledige tættere på arbejdsmarkedet i \textit{Tættere på arbejdsmarkedet} (2011), ved at måle beskæftigelseseffekten af dagpengeophør i \textit{Dagpengemodtagers situation omkring dagpengeophør} (2014), ved at kigge på indsatser over for ikke arbejdsmarkedsparate kontanthjælpsmotagere i \textit{Veje til beskæftigelse} (2010), ved at måle effekten af den beskæftigelesrettede indsats for sygemeldte i \textit{Effekten af den beskæftigelsesrettede indsats for sygemeldte} (2012) eller ved at kigge på pensionisters og efterlønsmodtageres genindtræden op arbejdsmarkedet i \textit{Pensionisters og efterlønsmodtageres arbejdskraftspotentiale} (2012). Det som er kendetegnede ved denne typer rapporter er et grundlæggende fokus på at få de pågældende personer tilbage i beskæftigelse hvad man kan gøre og ikke hvad deres situation egentlig betyder for deres liv\footnote{Der skal ikke menes med, at disse rapporter slet ikke forholder sig til de pågældende personers liv. I \textit{Veje til Beskæftigelse} (2010) fortælles der igennem 30 kvalitative interviews med sagsbehandlere, at de oplever de ikke-arbejdsmarkedsparate kontanthjælpsmodtagere som værende en heterogen gruppe som har gavn af forskellige typer indsatser alt efter, hvilke udfordringer de har. Nogle har for eksempel helbredsproblemer, mens andre har brug for hjælp til daglige gøremål.}.





%%%%%%%%%%%%%%%%%%%%%%%%%%%%%%%%%%%%%%%%%%%%%%%%%%%%%%%%%%%
\newpage \subsection{\textsc{Afsluttende opsamling}}
%%%%%%%%%%%%%%%%%%%%%%%%%%%%%%%%%%%%%%%%%%%%%%%%%%%%%%%%%%%

\textbf{Marginaliseringsperspektivet i stedet for arbejdsløshed:} Arbejdsløse fremtræder forskelligt og et ikke et problem. Arbejdsløse fremtræder i dagligdagen, den offentlige debat og blandt forskerne på forskellige måder. Det kan være som ledige, dagpengemodtagere, kontanthjælpsmodtagere. Dertil kommer der en masse tillægsord som udsatte, arbejdsmarkedsparate, ikke-arbejdsmarkedsparate, langtids, forsikrede, ikke-forsikrede, sæson/friktion/førstegang, syge, aktiverede, svage, sygemeldte, arbejdsskadede, skadeslidte. Dertil kommer der undergrupper som handikappede, indvandrere, efterkommere, mænd, kvinder, ældre, højtuddannede\footnote{Denne lange beskrivelse af forskellige typer arbejdsløse kommer fra en gennemgang af Det Nationale Forskningscenter for Velfærd (SFI) rapporter fra de sidste 20 år.}. Dertil kommer der en masse forskellige definitioner af arbejdsløse fra forskellige forskere og institutioner. En af de mest gængse definitioner kommer fra International Labour Organization som definerer arbejdsløse som det antal personer som står uden beskæftigelse samtidig med at være til rådighed for arbejdsmarkedet og aktivt arbejdssøgende \parencite{ILO1982}. Men hvad har de arbejdsløse egentlig tilfælles ud over at de står uden lønnet arbejde. Det er netop, hvad Halvorsen spørger sig selv og konkluderer, at arbejdsløse ikke er en ensartet gruppe, men en kategori sammensat af forskellige mennesker med forskellige udfordringer.  Arbejdsløshed er en institutionel konstruktion, som påvirkers af sociale ordninger og arbejdsmarkedets organisering: “The unemployed are not a group of people, but an economic and adminsitrative category” \parencite{Kelvin1985} \parencite[18]{Halvorsen1999}. Overordnet kan man sige, at arbejdsløshed ikke bare et socialt problem, men som flere til dels adskilte sociale problemer. Samtidig er arbejdsløse ikke bare en ensartet gruppe, men forskellige mennesker med forskellige problemstillinger . Til sidst fremstår arbejdsløse - om man kalder dem ledige eller noget tredje som en kategori forbundet med negative beskrivelser \parencite[12]{Halvorsen1999}. Definition af arbejdsløshed. Det centrale at have med - inddeling af arbejdsmarkedet - kategorierne beskæftiget, arbejdsløs og uden for arbejdsstyrken - jobsøgning - aktiv indsats/handling for at søge jobs - incitament - motivation for at handle. Atkinson, Jørgen Elm, Ezzy mv. Kernen i vores teoretiske og empiriske arbejde er arbejdsløshed. Arbejdsløshed defineres i \textit{Den Store Danske} som den manglende overensstemmelse på arbejdsmarkedet mellem udbud af arbejdskraft og efterspørgsel efter arbejdskraft \parencite{2015}. I de almindelige statistiske definitioner opdeles den danske befolkning i folk der er inden for og uden for arbejdsstyrken. Dem der er inden for arbejdsstyrken er enten i beskæftigede eller arbejdsløse, mens alle andre per definition betragtes som værende uden for arbejdsstyrken \parencite{2015a}. Problemet med denne opdeling er, at personer der klassificeres som uden for arbejdsstyrken ofte kommer fra beskæftigelse og bevæger sig tilbage i beskæftigelse. De følger måske ikke standarddefinitionen på arbejdsløshed\footnote{\textit{International Labour Organization} definerer arbejdsløse som det antal personer som står uden beskæftigelse samtidig med at være til rådighed for arbejdsmarkedet og aktivt arbejdssøgende \parencite{ILO1982}. Denne definition fremgår både af \textit{Den Store Danske}, Danmarks Statistik, Beskæftigelsesministeriet med flere.}, men kan i et lidt bredere perspektiv godt betragtes som arbejdsløse. Eksempelvis kan personer på revalideringsydelse, kontanthjælp og førtidspension godt vende tilbage i beskæftigelse igen. For at få en bedre forståelse af arbejdsløses sociale mobilitet på arbejdsmarkedet må vi derfor bryde med de almindelige definitioner af arbejdsløshed\footnote{Vi er vel opmærksomme på den symbolske kamp, der ligger i at gøre dette, hvilket fremgår af de politiske diskussioner om arbejdsløse og kontanthjælpsmodtagere, der er ’skjult’ i aktiveringsforløb...}. For at bryde med standarddefinitionerne af arbejdsløshed vil først og fremmest redegøre for sociologisk-videnskabelige og økonomisk-videnskabelige tilgang til arbejdsløse og arbejdsløshed for så til sidst at anvende Bourdieu og marginaliseringsbegrebet til at lave en teoretisk operationalisering af arbejdsløshed. Når en stor gruppe får sværere ved at konkurrere om de ledige jobs, vil arbejdsløsheden i mindre grad lægge pres på løndannelsesprocessen, og løntilpasningen vil således ikke kunne sikre en tilbagevenden til høj beskæftigelse. Jo længere tid arbejdsmarkedet er præget af høj arbejdsløshed, jo flere af de arbejdsløse vil blive marginaliserede, og jo større bliver den strukturelle ledighed \parencite{2015}.  Marginaliseringbegrebet åbner op for arbejdsløshed i forhold til at inkluderer dem som står uden for arbejdsstyrken. Flere måde at anskue de arbejdsløse på som helhed, det vil sige alle eller som dele, hvilket både kan være i grupper (for eksempel ledige, langtidsledige og kontanthjælpsmodtagere) eller som årsag (friktionsledighed, konjunkturledighed, strukturledighed og sæsonledighed). Blik for uden for arbejdsstyrken, hvilket vil sige studerende, efterløn, pensionister mv. Vores definition af arbejdsløse er så bred som overhovedet muligt. Med arbejdsløs har vi som udgangspunkt den bredeste definition overhovedet, hvilket er det at stå uden arbejde. Den teoretiske pointe er at arbejdsløshed defineres og behandles forskelligt alt efter om det er økonomer, sociologer mv.. Vores fokus ligger i forlængelse af marginaliseringsbegrebet \parencite{Larsen2009} samt Bourdieus perspektiver om at være placeret et specifikt sted i det sociale rum og at være på kanten af arbejdsmarkedet. Arbejdsløse versus ledige: Fokus på arbejdsmarkedsparate arbejdsløse, men værd opmærksom på, at arbejdsmarkedsparathed kan have en mening (økonomiske incitamenter) vi ikke ønsker. Arbejdsløse versus ledige... Arbejdsløshed er en person uden arbejde, mens ledig er en person som står til rådig på arbejdsmarkedet. Spørgsmål om skyld.

\textbf{Psykiske plager og selvrespekt:} Fælles for de strukturbaserede teorier er, at de behandle de sociale og psykologiske konsekvenser af arbejdsløshed, når man er arbejdsløs. Her er der tale om Rehabiliteringstilgangen repræsenteret ved Tiffany, Cowan og Tiffany, Stadiemodellen repræsenteret ved Lazarsfed og Eisenberg, Jahodas funktionelle deprivationsteori og Warrs vitaminmodel. Hysterese - det at være arbejdsløs - De aktørbaserede teorier beskæftiger sig mere med, at arbejdsløshed ikke nødvendigvis er negativt i sig selv og beskæftiger sig mere med den proces som får individer til at opleve psykologiske og materielle afsavn samtidig med, hvad agenten gør, ikke gør eller kan gøre for at komme videre. Her er der tale om Fryers agency kritik, Ezzys status passagemodel og Halvorsens mestringsperpektiv.

\textbf{Incitamenter.} - det at bevæge sig mod beskæftigelse Arbejdsløshed på lang sigt: friktionsledighed, strukturel lighed og sæsonledighed. Trade-off mellem arbejde og fritid. Basal jobsøgningsteori. Trade-off mellem forsikring og incitament.Incitamenter. - det at bevæge sig mod beskæftigelse Arbejdsløshed på lang sigt: friktionsledighed, strukturel lighed og sæsonledighed. Trade-off mellem arbejde og fritid. Basal jobsøgningsteori. Trade-off mellem forsikring og incitament - Økonomiske vanskeligheder og det offentlige og private sikkerhedsnet 

Den økonomiske gennemgang bidrager med at vise den dominerende perspektiv på arbejdsløse. Fokus ligger på at få folk i beskæftigelse (fra marginalisering til inklusion). Den sociologiske og socialpsykologiske gennemgang bidrager med at få et indblik i de arbejdsløses vilkår. Fokus på arbejdsløshed (marginalisering og eksklusion). Deprivation (Jahoda, Lazarsfeld og Zeizel 1971; Eisenberg og Lazarsfeld 1938; Tiffany, Cowan og Tiffany 1970; Warrs (987) er delvist brugbart i forståelsen af arbejdsløshed som havende en social og psykologisk påvirkvning på arbejdsløse. Halvorsen er delvis brugbar i forståelsen af at individerne bliver aktører som kan handle og har ressourcer. Social passage (Glaser og Strauss 1971)er relevant i vores definition af arbejdsløshed som midlertidigt.

% 
\begin{table}[H]
\centering
\caption{Oversigt over økonomiske og sociologiske teoriperspektiver}
\label{tab_spellrun}
\resizebox{0.6\textwidth}{!}{%
\begin{tabular}{@{}|l|l|l|@{}} \toprule
Tematik & Økonomi & Sociologi \\ \midrule
Begrebet &  &  \\ \midrule
Incitamenter & Søgeteori & Mestring \\ \midrule
Mentalt helbred & Hysterese & Deprivationsteori \\ \midrule
Det offentlige system & Principal-agent-model & \\ \bottomrule
\end{tabular} }
\end{table}
% 

Litteratur som gerne må læses ifm. med dette afsnit: 
%
 \begin{enumerate} [topsep=6pt,itemsep=-1ex]
   \item \sout{\parencite{Halvorsen1999}}
   \item (Dencker Larsen 2014, s. 17-38)
   \item \parencite{Baum2001, Baum2006}
   \item Goul Andersen
   \item ny forskning (web of science)
 \end{enumerate}




%%%%%%%%%%%%%%%%%%%%%%%%%%%%%%%%%%%%%%%%%%%%%%%%%%%%%%%%%%%

%Local Variables: 
%mode: latex
%TeX-master: "report"
%End:
	\input{tex/2.3_teori_mobilitet}
	\input{tex/2.4_teori_opsamling}

\input{tex/3.0_metode}
	% \input{tex/3.1_metode_arbejdsloeshed}
	% \input{tex/3.2_metode_netvaerksanalyse}
	% \input{tex/3.3_metode_disco}
	% \input{tex/3.4_metode_opsamling}

\input{tex/4.0_analyse}
	% \input{tex/4.1_analyse_hovedkort}
	% % -*- coding: utf-8 -*-
% !TeX encoding = UTF-8
% !TeX root = ../report.tex


% %%%%%%%%%%%%%%%%%%%%%%%%%%%%%%%%%%%%%%%%%%%%%%%%%%%%%%%%%%%
% \section{\textsc{Faglærte og ufaglræte} \label{cluster5.3}}
% %%%%%%%%%%%%%%%%%%%%%%%%%%%%%%%%%%%%%%%%%%%%%%%%%%%%%%%%%%%

% Segment 5.3 er med sine 37 disco-kategorier det absolut største, og indeholder 24 \% af alle disco-kategorierne. Eftersom størrelsen på kategorierne varierer ganske betragteligt, som beskrevet i afsnit \ref{fig_hist_beskaeftigede_allekategorier}, er et mere sigende mål hvor mange beskæftigede, der i gennemsnit er tale om over perioden. Her har segment 5.3 en andel på 36 \%,hvilket svarer til 237.411 personer. De to største disco-kategorier befinder sig i dette segment, og de står sammen for 12 \% af arbejdsmarkedet for ledige. Taget i betragtning af at det næststørste segment kun har en andel på 16,5 \%, må man sige at segmentet fylder ganske meget på det danske arbejdsmarked. 

% Som kort XX viser i oversigten over de 1-cifrede Disco-koder, består størstedelen af segmentet af arbejder indenfor \texttt{Operatør- og monteringsarbejde smat transportarbejde}, samt \emph{Andet manuelt arbejde}. Faktisk indeholder dette segment alle disco-kategorier indenfor \emph{9: Andet manuelt arbejde} undtagen medhjælp indenfor landbrug, gartneri fiskeri og skovbrug, hvilket tæller blandt andet skovhuggere, plantagearbejdere, land-og staldarbejder og frugtplukker. Indenfor gruppen af manuelt arbejde kan vi dermed forstå karakteren af dette arbejde som så distinkt anderledes, at det åbenbart ikke ligger lige for at skifte fra det industrielle manuelle arbejde til det landlige ditto.

% At der er tale om den industrielle, manuelle klynge, bekræftiges når man ser hvilke \texttt{disco(8)}-kategorier der er inkluderet, og hvilke der ikke er. Udaf de 22 


% %husk at nævn at "andet arbejde" bare ikke har så mange udfald som mange af de andre, derfor er den sådan. skriv ind at gennemsnittet ikke er brugbart, men det er medianen tilgengæld

% %Husk at skriv Collins på om "the credential society"

% 6.43+5.51

% Det betyder at lidt over 

% 33+32+22+21+14+9+8+6+5


%noter

% brug segmenter til at finde ud af hvor mange SOC_STIL etc der er i hvert cluster

%%%%%%%%%%%%%%%%%%%%%%%%%%%%%%%%%%%%%%%%%%%%%%%%%%%%%%%%%%%
% Trash
%%%%%%%%%%%%%%%%%%%%%%%%%%%%%%%%%%%%%%%%%%%%%%%%%%%%%%%%%%%


%Local Variables: 
%mode: latex
%TeX-master: "report"
%End: 
	% % -*- coding: utf-8 -*-
% !TeX encoding = UTF-8
% !TeX root = ../report.tex


% %%%%%%%%%%%%%%%%%%%%%%%%%%%%%%%%%%%%%%%%%%%%%%%%%%%%%%%%%%%
% \section{\textsc{Akademikerne} \label{}}
% %%%%%%%%%%%%%%%%%%%%%%%%%%%%%%%%%%%%%%%%%%%%%%%%%%%%%%%%%%%

% Når man taler om uddannelsesgrupper opdeles de typisk i faglærte, ufaglærte og videregående uddannelser, de videregående uddannelser opdeles i de korte, de mellemlange og de lange, de lange videregående uddannelser i hmuanistiske, naturvidenskabelig, sundhedsvidenskabelige, samfundsvidenskabelige, tekniske, farmaceutiske, teologiske og så videre \parencite{Groes2014}. 

% For at skelne mellem på den ene side de forskellige udddannelsesgrupper og på den anden side opdelingen af de lange videregående uddannelser i faggrupper, kan man sige at førstnævnte tager udgangspunkt i den internationale uddannelsesklassifikation\footnote{International Standard Classification of Education (\texttt{ISCED}) placerer uddannelse på ti niveauer: førskoleniveau (børnehaveklasse), grundskoleniveau I (1.-6. klasse), grundskoleniveau II (7.-10. klasse/årgang), gymnasialt niveau I (10. uddannelsesår), gymnasialt niveau II (11.-12. uddannelsesår), korte videregående uddannelser (13.-14. uddannelsesår), mellemlange videregående uddannelser (15.-16. uddannelsesår), lange videregående uddannelser (17.-18. uddannelsesår) og forskerniveau (19.- uddannelsesår) (henvisning).} og sidstnævnte følger de danske universiteters inddeling på baggrund af hvilket fakultet, man er på. Sidstnævnte er problematisk, fordi psykologi på Aarhus Universitet for eksempel ligger under det sundhedsvidenskabelige fakultet, fordi det har rødder i sundhedsvidenskaben, mens det på Københavns Universitet ligger under det samfundsvidenskabelige fakultet. Vore indelinger af akademikre på arbejdsmarkedet bryder med begge selninger. Som det fremgår af kortet, kan man se alle de mørkeblå noder er akademikerarbejdsstillinger (eller viden på højeste niveau). Her er der 12 segmenter ud af de 32 forskellige som indeholder personer med viden på højeste niveau, som fremadrettet vil blive kaldt for akademisk arbejdskraft. De tre segmenter, vi ønsker at fremhæve er karakteriseret ved, at det er tre clusters med flest akademikere og som vi kalder for magisterclusteret, djøfclusteret og kreaclusteret\footnote{De ni andre segmenter er: 1) \emph{Udvikling og analyse af software og applikationer} (\texttt{2130}) og \emph{Edb teknisk arbejde, primaert programmoer} (\texttt{3120}), 2) \emph{Ingenioerer og arkitekter} (\texttt{2141}) og Fire forskllige typer af teknikerarbejde (\texttt{3112, 3115, 3118, 3181}), 3) \emph{Laege} (\texttt{2221}), 4) \emph{Tandlaege} (\texttt{2222}), 5) \emph{Jordemoder, overordnet sygepleje mv} (\texttt{2230}) og \emph{Sygeplejearbejde} (\texttt{3230}), 6) \emph{Kulturformidling og informationsarbejde, primaert bibliotekar} (\texttt{2430}), 7) \emph{Overordnet revisions og regnskabsarbejde, herunder registeret revisor og statsautoriseret revisor} (\texttt{2411}) og otte andre \texttt{DISCO}-kategorier med kontor-, administrations- og revisionsarbejde mv., 8) \emph{Religioest arbejde} (\texttt{2482}) og 9) \emph{Overordnet socialraadgivningsarbejde} (\texttt{2446}) og \emph{Administrativt arbejde vedr. offentlige ydelser og afgifter} (\emph{3440}).}

% I perioden 1996 til 2009 har de lange videregående uddannelser haft den største  procentvise tilvækst, men de talte alligevel mindre en halvdelen af antallet med en mellemlang mellemlang videregående uddannelse. %%%% Den samlede arbejdsstyrke har været nogenlunde konstant. 
% Nettoledigheden for de lange videregående uddannelser bevæger sig parallelt med de andre faggrupper i perioden 1996 til 2009 er nettoledighedsgennemsnittet 6000 personer. 
% % 
% \begin{table}[H] \centering
% \caption{Arbejdsstyrken fordelt på uddannelsesgrupper. Kilde: DST}
% \label{tab_uddannelse}
% \begin{tabular}{lrrrrrrr} \toprule
% 	& \multicolumn{1}{c}{Grundskole} & \multicolumn{1}{c}{GYM} & \multicolumn{1}{c}{EUD} & \multicolumn{1}{c}{KVU}	& \multicolumn{1}{c}{MVU} & \multicolumn{1}{c}{LVU} & Alle	\\ \midrule
% 1996	&	1.572.425	&	310.367	&	1.294.603	&	118.561	&	419.763	&	162.520	&	\\
% 1997	&	1.556.567	&	317.815	&	1.314.644	&	123.732	&	434.072	&	170.017	&	\\
% 1998	&	1.536.264	&	323.421	&	1.335.701	&	130.000	&	449.752	&	177.948	&	\\
% 1999	&	1.524.624	&	325.966	&	1.348.020	&	135.721	&	465.883	&	185.261	&	\\
% 2000	&	1.510.944	&	325.431	&	1.364.746	&	140.052	&	482.040	&	192.667	&	\\
% 2001	&	1.499.835	&	324.567	&	1.379.370	&	145.075	&	498.671	&	201.119	&	\\
% 2002	&	1.488.688	&	322.904	&	1.391.768	&	150.452	&	515.013	&	210.416	&	\\
% 2003	&	1.480.263	&	320.959	&	1.399.158	&	156.772	&	530.979	&	220.133	&	\\
% 2004	&	1.473.175	&	321.088	&	1.406.980	&	159.774	&	545.612	&	230.323	&	\\
% 2005	&	1.466.360	&	321.403	&	1.411.090	&	162.922	&	559.638	&	239.798	&	\\
% 2006	&	1.455.210	&	322.732	&	1.414.756	&	165.711	&	572.289	&	249.919	&	\\
% 2007	&	1.439.702	&	324.260	&	1.419.023	&	168.987	&	584.270	&	261.475	&	\\
% 2008	&	1.517.184	&	329.613	&	1.428.161	&	173.227	&	598.317	&	273.095	&	\\
% 2009	&	1.460.590	&	329.557	&	1.439.554	&	177.572	&	613.044	&	285.460	&	\\  \bottomrule
% \end{tabular} \end{table}
% %


% %%%%%%%%%%%%%%%%%%%%%%%%%%%%%%%%%%%%%%%%%%%%%%%%%%%%%%%%%%%
% \subsubsection{Magisterclusteren \label{}}
% %%%%%%%%%%%%%%%%%%%%%%%%%%%%%%%%%%%%%%%%%%%%%%%%%%%%%%%%%%%

% % 
% Magisterclusteren består primært af meget forskelligt arbejde. Clusteren indeholder
%  \begin{enumerate} [topsep=6pt,itemsep=-1ex]
%    \item \emph{Arbejde med emner inden for fysik, kemi, astronomi, meteorologi, geologi og geofysik} (\texttt{2110})
%    \item \emph{Arbejde med emner inden for de biologiske grene af naturvidenskab} (\texttt{2210})
%    \item \emph{Dyrlaege} (\texttt{2223})
%    \item \emph{Farmaceut} (\texttt{2224})
%    \item \emph{Arbejde med emner inden for medicin, odontologi, veterinaervidenskab og farmaci i oevrigt} (\texttt{2229})
%    \item \emph{Undervisning paa universiteter og andre hoejere laereanstalter} (\texttt{2311})
%    \item \emph{Undervisning paa gymnasier, erhvervsskoler mv} (\texttt{2321})
%    \item \emph{Folkeskolelaerer} (\texttt{2331})
%    \item \emph{Undervisning af handicappede mennesker} (\texttt{2341})
%    \item \emph{Arbejde vedr. undervisning i oevrigt, primaert kursusvirksomhed} (\texttt{2350})
%    \item \emph{Samfundsvidenskabeligt arbejde og historie} (\texttt{2442})
%    \item \emph{Sprogvidenskabeligt arbejde} (\texttt{2444})
%    \item \emph{Psykolog} (\texttt{2445})
%    \item \emph{Arbejde med administration af lovgivningen inden for den offentlige sektor} (\texttt{2470})
%    \item \emph{Blandet undervisning i folkeskoler, erhvervsskoler, gymnasier og hoejere laereanstalter samt forskningstilrettelaeggelse og kontrol af undervisningsarbejde} (\texttt{2930}) 
%  \end{enumerate}
% % 


%%%%%%%%%%%%%%%%%%%%%%%%%%%%%%%%%%%%%%%%%%%%%%%%%%%%%%%%%%%
% \subsubsection{Kreaclusteren \label{}}
%%%%%%%%%%%%%%%%%%%%%%%%%%%%%%%%%%%%%%%%%%%%%%%%%%%%%%%%%%%

% % 
% Kreaclusteren består af forskelligt kreativt arbejde. Clusteren indeholder:
%  \begin{enumerate} [topsep=6pt,itemsep=-1ex]
%    \item \emph{Ledelse af virksomhed faerre end 10 ansatte} (\texttt{1300})
%    \item \emph{Alment journalistisk arbejde og skribentarbejde} (\texttt{2451}) 
%    \item \emph{Illustrationsgrafisk arbejde vedr. formidling og kunstnerisk arbejde vedr. billedkunst og formgivning} (\texttt{2452}) 
%    \item \emph{Kunsterisk arbejde indenfor dans, musik, koreografi, skuespil eller film} (\texttt{2481}) 
%    \item \emph{Blandet journalist, kunst og skribentarbejde} (\texttt{2945}) 
%    \item \emph{Arbejde med lyd, lys og billeder ved film og teaterforestillinger mv samt betjening af medicinsk udstyr} (\texttt{3130}) 
%    \item \emph{Arbejde inden for kunst, underholdning og sport} (\texttt{3470})
%  \end{enumerate}
% % 


%%%%%%%%%%%%%%%%%%%%%%%%%%%%%%%%%%%%%%%%%%%%%%%%%%%%%%%%%%%
% \subsubsection{Djøfferclusteren \label{}}
%%%%%%%%%%%%%%%%%%%%%%%%%%%%%%%%%%%%%%%%%%%%%%%%%%%%%%%%%%%

% % 
% Djøfferclusteren består primært af forskelligt økonomi- og juridisk arbejde. Clusteren indeholder
%  \begin{enumerate} [topsep=6pt,itemsep=-1ex]
%    \item \emph{Lovgivningsarbejde samt ledelse i offentlig administration og interesseorganisationer} (\texttt{1100}) 
%    \item \emph{Arbejde med matematik, aktuariske og statistiske metoder} (\texttt{2120}) 
%    \item \emph{Udvikling og planlaegning af personalespoergsmaal} (\texttt{2412}) 
%    \item \emph{Ledelsesraadgivning og andre specialfunktioner indenfor organisation} (\texttt{2419}) 
%    \item \emph{Advokat, dommer og andet juridisk arbejde} (\texttt{2420}) 
%    \item \emph{Oekonomi} (\texttt{2441}) 
%  \end{enumerate}

%%%%%%%%%%%%%%%%%%%%%%%%%%%%%%%%%%%%%%%%%%%%%%%%%%%%%%%%%%%

% % 
% \begin{table}[H] \centering
% \caption{Arbejdsløshed fordelt på uddannelsesgrupper. Kilde: DST}
% \label{tab_uddannelse_arbejdsloeshed}
% \begin{tabular}{lrrrrrr} \toprule
% 	& \multicolumn{1}{c}{Grundskole} & \multicolumn{1}{c}{STX} & \multicolumn{1}{c}{EUD} & \multicolumn{1}{c}{KVU}	& \multicolumn{1}{c}{MVU} & \multicolumn{1}{c}{LVU} & Alle	\\ \midrule
% 1996	&	81.151	&	13.622	&	60.666	&	6.145	&	11.871	&	6.418	&	\\
% 1997	&	68.797	&	11.806	&	58.219	&	5.610	&	11.304	&	6.641	&	\\
% 1998	&	55.107	&	8.872	&	45.082	&	4.204	&	8.799	&	5.221	&	\\
% 1999	&	47.700	&	7.363	&	41.245	&	3.868	&	8.517	&	5.090	&	\\
% 2000	&	47.203	&	7.031	&	42.168	&	4.250	&	8.964	&	5.039	&	\\
% 2001	&	43.330	&	6.396	&	40.099	&	4.143	&	8.315	&	4.906	&	\\
% 2002	&	43.826	&	7.018	&	43.365	&	5.310	&	9.853	&	6.620	&	\\
% 2003	&	52.604	&	8.981	&	53.668	&	6.801	&	13.079	&	8.639	&	\\
% 2004	&	48.517	&	8.307	&	47.637	&	5.850	&	12.629	&	8.075	&	\\
% 2005	&	39.159	&	7.106	&	35.808	&	4.706	&	10.888	&	6.975	&	\\
% 2006	&	30.184	&	5.715	&	24.019	&	3.430	&	8.451	&	5.776	&	\\
% 2007	&	23.436	&	4.133	&	16.806	&	2.459	&	5.888	&	4.813	&	\\
% 2008	&	15.707	&	2.669	&	15.353	&	1.916	&	3.824	&	3.297	&	\\
% 2009	&	29.452	&	5.283	&	39.438	&	4.581	&	8.227	&	6.459	&	\\ \bottomrule
% \end{tabular} \end{table}
% %

%%%%%%%%%%%%%%%%%%%%%%%%%%%%%%%%%%%%%%%%%%%%%%%%%%%%%%%%%%%

% Toubøl, Larsen og Jensen \parencite[3]{TouboelLarsenJensen2013} \parencite[4]{TouboelLarsen2015} kan beskæftigelsesmønstre have en funktionel, en institutionel og en normativ form. Den funktionelle form opstår, når man skal have de rette færdigheder for at have mulighed for at indtage en bestemt arbejdsstilling. Den institutionelle form opstår, når man skal have det rette certifikat for at komme i betragtning til en bestemt arbejdsstilling. Og den normativ form opstår, når der ekskluderes personer med et bestemt køn eller en bestemt race fra visse arbejdsstillinger.

%%%%%%%%%%%%%%%%%%%%%%%%%%%%%%%%%%%%%%%%%%%%%%%%%%%%%%%%%%%
% Trash
%%%%%%%%%%%%%%%%%%%%%%%%%%%%%%%%%%%%%%%%%%%%%%%%%%%%%%%%%%%


%Local Variables: 
%mode: latex
%TeX-master: "report"
%End: 

\input{tex/5.0_diskussion}

\input{tex/6.0_konklusion}


%% -------------- Bibliografi ----------------- %%%
% biblatex definerer alt i preamblet
% \nocite{*} % brug denne for at tage referencer med der ikke citeres i teksten
% \printbibliography

\chapter{\textsc{Bibliografi} \label{biblio}}

\printbibliography[heading=subbibliography,title={\textsc{Hovedbibliografi}},filter=notonline]
\printbibliography[heading=subbibliography,title={\textsc{Online kilder}},type=online]
% \printbibliography[heading=subbibliography,title={\textsc{DST manualer}},filter=DST]
\printbibliography[heading=subbibliography,title={\textsc{Danmarks Statistiks Manualer}},type=manual]



%%% -------------- Appendiks ----------------- %%%
% \begin{appendices}
% \setcounter{secnumdepth}{3}
% \input{tex/appendiks_figurer}
% \input{tex/appendiks_metode_arbejdsloeshed_og_skift}
% \input{tex/appendiks_metode_disco}
% \input{tex/appendiks_metode_baggrundsvariable}
% \end{appendices}


%%% -------------- slut det hele er slut ----------------- %%%
\end{document}




