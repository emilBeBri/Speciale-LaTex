





Grusky har to kritikker af klasseanalyse som den almindeligvis bedrives: For det første at de er nominelt funderede, fremfor at tage reelt oplevede, sociale fællesskaber som grundprincip for klasseinddelingerne. Og for det andet, at de derved ender med en grovkornet klasseinddeling, der skjuler udbytnings- og stratifikationsmekanismer, der burde være i centrum af klasseforskningen. 

Grusky påpeger at menneskers identifikation med klassebaserede identiteter historisk er blevet svagere og svagere, samt den manglende massehandlen, der burde følge med positioner i arbejderklassen, hvilket har ledt til en hel del \emph{nymarxistisk håndvridning}, som han kalder det \parencite[205]{Grusky2001}.  

Det er ikke problemet at benytte beskæftigelse som grundlag for klasseinddeling. Problemet er niveauet af aggregering, der sker ved at placere en lang række forskellige professioner i eksempelvis kategorien \emph{nedre serviceklasse}, som hos Goldthorpe, baseret på teoretisk funderede magtdimensioner, i Goldthorpes tilfælde: Kontrol over arbejdsprocessen samt specificiteten af færdigheder.  

Klasseanalyse bør tage udgangspunkt i hvad Grusky kalder \emph{realistiske} fælleskaber, altså fælleskaber der opleves som sådan af individerne i dem. Det er på dette (bevidste) niveau, at sociale stratifikationsmekanismer reelt opererer \parencite[212]{Grusky2001}. Det er ikke sådan, at den funktionelle enshed i arbejdet, som er fundamentet for erhvervsklassifikationsskemaer, skal tages for pålydende. Funktionel enshed i arbejdet er ikke nødvendigvis lig den sociale enshed, Grusky ser som udgangspunktet for en klasseanalyse. Han mener at denne "\emph{technicist vision}" %
\label{gruskytechnicistvision}%
også bør medtage reelt oplevede sociale distinktioner, fremfor udelukkende tekniske hensyn \parencite[215]{Grusky2001}%
%
\footnote{Han nævner bl.a. ISCO, der benyttes Danmarks Statistik, og som jeg også benytter i denne afhandling. Mere om det senere.}%
%
. Men på et tilstrækkeligt detaljeret niveau, hvor vi nærmer os rene professioner, er sandsynligheden for reelt oplevede fælleskaber størst \parencite[207]{Grusky2001}. Ud fra dette niveau kan aggregation i større klasser \emph{eventuelt} finde sted, omend man fornemmer at guldstandarden for Grusky stadig er de realistiske fællesskaber. Det er hans to validitetskriterier: Det første er at medlemmerne i den skal forstå sig selv som tilhørende en gruppe, baseret på beskæftigelse. Det andet er at kunne observere kollektiv handlen, der sikrer denne gruppe privilegier, forstået som \emph{social luknings}-strategier.  Det er et begreb fra Parkins professionssociologi, Grusky ser ud til at overføre direkte. Jeg vil derfor gennemgå Parkins definition af det nu.







