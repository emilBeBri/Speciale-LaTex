%!TEX root = ../report.tex


TEORI

Arbejdsmarkedssegmenteringsteori 

Arbejdsmarkedsmarkedet definerer økonomisk klasse ifølge Goldthorpe + Oesch
 - ikke sociale klasse
 - a priori / a posteriori indeling
 - arbejdskontrakt, overvågning, kompetencer, arbejdslogik, køn, uddannelse
Goldthorpe + Oesch diskuterer, hvordan arbejdet differentieres

1. er klynger de delmarkeder? -> ja
2. er klynger de segmenter? -> hm
3. differentieringslogikker


EMPIRI

Deskriptiv analyse (kriterie 1-3):
 - Er det segmenter? Hvad kendetegner segmenter? Foregår det kun inden for segmentet?
 - Intern mobilitet
 - Løn
 - Køn
 - Ledighedsgrad
 - (Politik og værdier?)

Klasseteori (kriterie 4)
 - EGP-11
 - Oesch





%%%%%%%%%%%%%%%%%%%%%%%%%%%%%%%%%%%%%%%%%%%%%%%%%%%%%%%%%%%
\chapter{Indledning \label{indledning}}
%%%%%%%%%%%%%%%%%%%%%%%%%%%%%%%%%%%%%%%%%%%%%%%%%%%%%%%%%%%

%%%%%%%%%%%%%%%%%%%%%%%%%%%%%%%%%%%%%%%%%%%%%%
\emph{Teaser, aktualisering, kommer senere.}
%%%%%%%%%%%%%%%%%%%%%%%%%%%%%%%%%%%%%%%%%%%%%%

Udgangspunktet for dette speciale er en undersøgelse af om der findes segmenter på det danske arbejdsmarked i i perioden 1996-2009. Det overordnede forhold, der søges belyst, er hvorvidt arbejdsmarkedet er delt op i flere forskellige delmarkeder, og i så fald, hvordan en sociologisk analyse kan belyse disse delmarkeders mulige forskellige funktionsmåder.

Der er meget lidt sociologisk teori, der beskæftiger sig med arbejdsløshed og arbejdsmarkedets funktionsmåder på et mesoniveau, og samtidig er åben overfor empiriske kortlæggelser på dette niveau: For det meste er den arbejdsmarkedssociologiske teori mig bekendt beskæftiget med et kvalitativt orienteret mikroniveau, eller de helt brede penselstrøg på makroplan. Den teori, der bedst egner sig til en sådan undersøgelse, er for mig at se arbejdsmarkedssegmenteringsteorien, der blev udviklet i 70'erne, men kun i meget begrænset omfang er blevet benyttet de sidste 20 år.
Denne afhandling vil benytte denne teoriretning, med udgangspunkt i registerdata fra Danmarks Statistik i den føromtalte periode, 1996-2009.

Teorier med blik for segmenteringsprocesser på arbejdsmarkedet har en række forskellige indfaldsvinkler. Disse vil blive diskuteret i de kommende afsnit, men udgangspunktet er Thomas Bojes udformning af teoriretningen. Man kan benytte følgende tre kriterier udformet af Boje \parencite[174]{Boje1986}, for at skabe et overblik.


Findes der segmenter på det danske arbejdsmarked, og hvordan kan forskelle i sociale processer være med til at forklare sådanne forskelle i segmentstrukturen?
1. Er der en opdeling af arbejdsmarkedet for arbejdstagere i delmarkeder, hvor mobilitet indenfor delmarkederne er hyppig, og mellem delmarkederne sjælden?
2. Kan forskelle i de sociale processer vise, at der er tale om segmenter, og ikke blot delmarkeder?
3. Kan klasseteori belyse denne segmentering?

%%%%%%%%%%%%%%%%%%%%%%%%%%%%%%%%%%%%%%%%%%%%%%
\subsubsection{Problemformulering og forskningsspørgsmål}
%%%%%%%%%%%%%%%%%%%%%%%%%%%%%%%%%%%%%%%%%%%%%%
%
Ovenstående kriterier udmønter sig i følgende problemformulering:
%
\vspace{\baselineskip}
%
\begin{tcolorbox}[title=\textbf{Problemformulering}]
Findes der segmenter på det danske arbejdsmarked, og hvordan kan forskelle i sociale processer være med til at forklare sådanne forskelle i segmentstrukturen?
\end{tcolorbox}
%
\vspace{\baselineskip}
Det vil jeg undersøge med følgende forskningsspørgsmålspørgsmål:
\vspace{\baselineskip}
\begin{tcolorbox}[title=Forskningspørgsmål,
subtitle style={boxrule=0.4pt} ]
\tcbsubtitle{1.}
Er der en opdeling af arbejdsmarkedet for arbejdstagere i delmarkeder, hvor mobilitet indenfor delmarkederne er hyppig, og mellem delmarkederne sjælden?
\tcbsubtitle{2.}
Kan forskelle i de sociale processer vise, at der er tale om segmenter, og ikke blot delmarkeder?
\tcbsubtitle{3.}
Kan klasseteori belyse denne segmentering?
\end{tcolorbox}


%%%%%%%%%%%%%%%%%%%%%%%%%%%%%%%%%%%%%%%%%%%%%%
\subsubsection{1. Kriterie - Delmarkeder og mobilitet}
%%%%%%%%%%%%%%%%%%%%%%%%%%%%%%%%%%%%%%%%%%%%%%

Det første kriterie er, at arbejdsmarkedet er delt op i delmarkeder, med begrænset mobilitet mellem de enkelte delmarkeder. Det betyder, at der i mellem visse typer jobs forekommer hyppige skift, og andre jobtyper, hvor der sjældent, eller aldrig, observeres skifte fra det ene til det andet.

Delmarkeder kan relatere sig til en forståelse af samfundet som opdelt i forskellige sociale klasser, med forskellige strukturelle livsbetingelser for individerne i dem. Det er denne afhandlings formål at benytte en sådan tilgang, for at vurdere om den kan belyse forhold for arbejdsløse på arbejdsmarkedet.

Denne forståelse har været genstand for sociologiske analyser siden sociologiens første store teoretikere, Weber, Marx og Durkheim. Weber var en af de af de første til at se på forholdet mellem klasse og mobilitet, og til at beskrive ud klasser ud fra de sociale bevægelser, individerne var en del af. Han definerer klasse således: “\emph{A »social class« makes up the totality of those class situations within which individual and generational mobility is easy and typical.}” \parencite[302]{Weber1978}. Weberiansk orienterede sociologer såsom Goldthorpe (\#henvisning) og (flere henvisninger) har bibeholdt dette fokus på social mobilitet i nyere sociologisk forskning. Hvad der definerer en klasse, er i denne optik i høj grad et empirisk spørgsmål om hvilke sociale grupper, hvor det kan påvises, at social mobilitet kun i lav grad forekommer.

Den marxistiske tradition er mere optaget af klasser som et strukturelt forhold til produktionsmidlerne, omend nymarxistiske sociologer som Olin-Wright også benytter en opdeling af lønmodtagere i forskellige hierarkiske positioner (\#henvisning), hvor positioner på arbejdmarkedet, forstået som arbejderklassens interne sammensætning, også har stor betydning for den faktiske fordeling af goder, og forskellige individers mulighed for opnåelse af disse goder. Arbejdsmarkedet er med andre ord også her delt op ud fra andre kriterier end blot arbejdsgiver-arbejder dikotomien, den faktiske klassestruktur skal også her undersøges empirisk.

Af nyere sociologisk teori kan desuden nævnes Pierre Bourdieus sammensmeltning og nytænkning af marxistisk og weberiansk funderet teori, i hans forståelse af samfundet som et socialt rum, opdelt i felter, der opererer ud fra forskellige sociale logikker (\#henvisning). Anthony Giddens understreger nødvendigheden af empirisk funderede analyser, for at forstå klassestrukturen i et bestemt samfund, og de muligheder individet har, alt efter "hvor det kommer fra" \parencite[48,110]{Giddens1973}. (inkluder eventuelt Gruskys "mikroklasser")

Indenfor arbejdsmarkedssegmenteringsteorierne har man i en amerikansk optik beskæftiget sig med det såkaldte "dual labour market", det vil arbejdsmarkedet opdelt i to overordnede delmarkeder, hvor det primære indeholder faste stillinger med tryghed i ansættelse og favorable lønninger og gode arbejdsvilkår, hvorimod det sekundære delmarked består af midlertidige ansættelser til lavere lønninger og dårligere arbejdsvilkår \parencite{Piore1980}. Som Boje argumenterer for, er denne opdeling et amerikansk fænomen, da en række institutionelle forhold samt en produktionsstruktur, med få, store virksomheder i landet, har skabt denne struktur, og kan ikke generaliseres til lande med andre instutionelle forhold. Dette understreger behovet for en analyse af hvilke danske delmarkeder, der findes. Her finder Boje, at det danske samfund består af langt flere delmarkeder, blandt andet på grund af kollektive overenskomster, sikring af arbejdsløshedsunderstøttelse, samt den langt større rolle, små og mellemstore firmaer spiller, hvilket skaber en anden social dynamik \parencite[36]{Boje1985}

I Danmark er en nyere kortlægning af arbejdsmarkedets delmarkeder allerede påbegyndt, ligeledes med udgangspunkt i arbejdsmarkedssegmenteringsteorien, af Toubøl og Grau Larsen \parencite{Touboel2013}, ved brug af social netværksanalyse. Denne afhandling benytter sig af den af Toubøl og Grau Larsen nyligt udviklede metode til at finde kliker i et socialt netværk, til at finde delmarkeder for arbejdsløse. De mennesker, der i perioder har været arbejdsløse, er naturligvis ikke løsrevet fra resten af arbejdsmarkedet, og en sammenligning mellem sammensætningen af delmarkeder og mobilitet mellem jobs for henholdsvis arbejdsløse og dem, der går direkte fra job til job, er derfor en vigtig del af afhandlings formål. Dette har også været dækket i Søren Nielsen-Gravholts speciale fra tidligere i år, som jeg har samarbejdet med i databearbejdningen (\#Henvisning SNGs speciale).  


%%%%%%%%%%%%%%%%%%%%%%%%%%%%%%%%%%%%%%%%%%%%%%
\subsubsection{2. kriterie - Forskellen på et delmarked og et segmenter er forskelle i sociale processeser}
%%%%%%%%%%%%%%%%%%%%%%%%%%%%%%%%%%%%%%%%%%%%%%

For at et delmarked defineres som et selvstændigt segment indenfor arbejdsmarkedet, skal det være muligt at påvise forskelle i de sociale processer, som karakteriserer delmarkedet. Et delmarked, hvor forskellen i mobilitet primært skyldes faglige eller geografiske forskelle, men andre væsentlige sociale processer ellers er ens, kan ikke karakteriseres som et segment \parencite[41]{Boje1985}. Det er bare et delmarked, da allokeringen af arbejdskraft i væsentligt grad sker uden (determinerende) sociale stratifikationsmekanismer. Hvis eksempelvis løndannelse og kønsforskelle viser sig tydeligt mellem to ellers sammenlignelige delmarkeder, kan man begynde at tale om segmenter i Bojes forstand. 

Sociale processer er et noget abstrakt begreb for den praksis, hvori livet på arbejdsmarkedet udspiller sig for den enkelte lønmodtager. Det teoretiske indhold, samt naturligvis empiriske målbarhed, vil blive udpenslet efterfølgende, her skal blot nævnes to kortfattede perspektiver på sådanne processer. 

Parkins begreb om social closure er en sådan måde. Det handler om de processer, hvorpå forskellige faggrupper sørger for at beskytte egne privilegier, på en sådan vis at det ekskluderer andre, og hæmmer mobiliteten ud fra andre hensyn end uddannelsesmæssige og faglige hensyn \parencite{Parkin1994}. Et andet eksempel er Granovetters begreb fra social netværksteori om svage og stærke sociale forbindelser, hvor hans hovedargument er individernes mulighed mobilitet af nye kanaler, gennem deres såkaldte "svage forbindelser" - det er igennem de mennesker, man ikke kender så godt, at der er adgang til nye muligheder indenfor for eksempel arbejdslivet \parencite{Granovetter1973}. (\emph{de her to eksempler skal strammes op, I know - E})


%%%%%%%%%%%%%%%%%%%%%%%%%%%%%%%%%%%%%%%%%%%%%%
\subsubsection{3. Kriterie - Sociale processer fører til segmentering}
%%%%%%%%%%%%%%%%%%%%%%%%%%%%%%%%%%%%%%%%%%%%%%

Det tredje kriterium er, at sociale processer på arbejdsmarkedet medfører differiering mellem ellers sammenligne grupper af arbejdstagere. Disse sociale processer skaber ulighed på arbejdsmarkedet, og kommer til udslag i ulige vilkår for forskellige delmarkeder, hvorved forskelle i livsvilkår fører til øget seggregering mellem forskellige delmarkeder. 

Det tredje kriterie hænger tæt sammen med socialklassetænkningen fra 1. kriterie, og kan betegnes som den teoretiske forklaring på manglende social mobilitet (1. kriterie) og forskelle i sociale processer (2. kriterie), der eventuelt vil vise sig i den empiriske kortlægning af arbejdsmarkedet. Udover det generelt interessante i en sådan kortlægning for arbejdsmarkedssociologi, er det relevant for mig, fordi jeg viser, at hvis der er forskelle på segmenterne hos de beskæftigede og de arbejdsløse, må der også være forskelle på de sociale processer. 

Her begynder min afhandling at adskille sig fra Toubøl og Grau Larsen, samt Nielsen-Gravholt. Toubøl og Grau Larsen har indtil videre i deres studier haft begrænset fokus på de sociale processer, da deres fokus har været på at beskrive de \"faktiske\" delmarkeder på det dansker arbejdsmarked. Nielsen-Gravholt har ligeledes haft et begrænset fokus på de sociale processer, da han har været været optaget af at diskutere segmenteringstilgangen i forhold til en nyklassiske økonomiske teori, kaldet jobsøgningsteorien. 

Denne afhandling vil fokusere på, hvorledes sociale processer kan ses afvige fra hinanden i delmarkederne, på sådan vis at vi kan tale om segmenter. jeg vil benytte social stratifikationsteori, til at forklare forskellen i sociale processer, som de kommer til udtryk på et empirisk niveau, gennem intern mobilitet i delmarkederne, indkomst, uddannelse og køn. 


%%%%%%%%%%%%%%%%%%%%%%%%%%%%%%%%%%%%%%%%%%%%%%
\subsubsection{4. Kan klasseteori informerer om den mere overordnede opdeling der er ud over segmenter}
%%%%%%%%%%%%%%%%%%%%%%%%%%%%%%%%%%%%%%%%%%%%%%


arbejdsmarkedets funktionsmåde og segmentering målt gennem mobilitet, og hvordan arbejdsmarkedet skal forstås som en del af klassestrukturen. 



%%%%%%%%%%%%%%%%%%%%%%%%%%%%%%%%%%%%%%%%%%%%%%
\emph{Disposition}
%%%%%%%%%%%%%%%%%%%%%%%%%%%%%%%%%%%%%%%%%%%%%%


\paragraph{1. Indledning}

	Teaser
	Delmarkeder og mobilitet (1. kriterie)
	Forskellen på et delmarked og et segmenter er forskelle i sociale processeser (2. kriteri)
	Sociale processer fører til segmentering (3 kriterier)
	Problemformulering og forskningsspørgsmål


\paragraph{2. Teori}

	Intro
		- Arbejdsmarkedssegmenteringsteori (kobling til Toubøl/Bojes 3 kriterier)
		- Movitation: Hvordan kan sociologien benyttes til at forstå barrierer på arbejdsmarkedet, især segmenteringen? Hvordan kan sociale processer forstås, og hvad er segmentering udtryk for?
		- Stratifikation i samfundet (overgang til min teori)
			* mange bud på klasseteori
			* klasse som beskrivende og forklarende værktøj (Harrits)
			* Nominel og realistisk klasseopdeling

	Marx
		-forholdet til produktionsmidlerne, udbytning

	Nymarxisme	
		- Problemerne i basis-overbygning og middelklassens position
		- Redskaber mere nuancerede end kapital-arbejde - hvordan, og stadig beholde den marxistiske kerne? 

	Wright
		- klasse som forklarende og nominel (tror jeg)
		- Tre dimensioner: Ejerskab over produktionsmidler, autoritet, færdigheder

	Weber
		- Klasse, social status og stænder

	Goldthorpe
		- Klasse som forklarende og nominel
		- Tre dimensioner: Ejerskab over produktionsmidlerne (knapt så centralt), specifiteten af kompetencer, kontrol over arbejdsprocessen 

	Grusky
		- Klasse som beskrivende og realistisk
		- kritik af nominel klasseinddeling
			- for grovkornet
			- misser det centrale ved definition udenfor de reelt oplevede fælleskaber
		- erhvervsgrupper som klasser, sociale lukningsmekanismer

	Opsamling: 
		- Kobling til delanalyse 1: Moneca som måde at inddele arbejdsmarkedet i segmenter (klasser)
		- Kobling til delanalyse 2: Forklare sociale processer i segmenter (klasser) vha. klasseforskere
		- Kobling til delanalyse 3: Jeg  placere Moneca som et oplagt bud på at mediere og vurdere de to positioner (den realistiske og nominelle klasseindling)


\paragraph{3. Metode}

	Netværksanalyse
	Videnskabsteori
	Operationalisering


\paragraph{4. Analyse}

	Delanalyse 1: Dannelse af delmarkeder
		- svar på problemformulering (findes der segmenter? - ja) + forskningsspørgsmål 1
		- beskrivelse af hovedkort og nøgletal (antal segmenter, intern mobilitet, ekstern mobilitet)
		- zoom ind på udvalgte segmenter (beskrivelse såsom hvilke er afgrænsede ifht andre, hvor er de fleste mennesker placeret etc.) 

	Delanalyse 2: Delmarkedernes sociale processer (er delmarkedernes sociale processer udtryk for segmenter)
		- svar på problemformulering (hvordan kan forskelle i sociale processer være med at forklare forskelle i segmentstrukturen) + forskningsspørgsmål 2 og 3
		- kønskort
		- lønkort
		- ledighedsgradskort
		- uddannelsesår (måske)
		- arbejdsløshed (ikke langt)

	Delanalyse 3: Min min empiri møder klasseteorien
		- svar på forskningsspørgsmål 4 (kan teori med udgangspunkt i social stratifikation belyse denne segmentering)
		- sammenligning med Goldthorpes klasseskemaer
		- sammenligning med Gruskys klasseskema

	Delanalyse 4 (måske'r)
		- social mobilitet mellem generationer
		- ægteskab/familie ties


\paragraph{5. Diskussion}




\paragraph{6. Konklusion}













% 	Teoretisk motivation

% 	1. kriterie - delmarkeder og mobilitet
% 		Klasse: Marx, Weber, Gitte Harris, Goldthorpe, Wright, Parkin, Grusky
% 		arbejdsmarkedssegmenteringsteori: Dual labour markets-dudes, Boje, Toubøl, Grau Larsen \& Strøhby

% 	2. kriterie - Forskellen på et delmarked og et segmenter er forskelle i sociale processeser
% 		Institutioner og strukturer
% 		institutionalisme
% 		Parkins social closure
% 		Granovetters weak ties/strong ties
% 		Gruskys "holding power" 

% 	3. kriterie - Sociale processer fører til segmentering
% 		Goldthorpe - Grusky standoff'et 


% Teoriafsnit

% 	beskæftigelse basal stratifikationsmekanisme (Scott, )
% 	Mobilitet mellem beskæftigelseskategorier central for at forstå segmenter, og i bredere forstand klasser.  (Wright, Goldthorpe, Grusky, Parkin)
% 		netop klasseteori kan være en måde at forstå de sociale processer, der fører til segmentering på arbejdsmarkedet, og som klassisk økonomisk teori ikke tager højde for. 
% 	Klassebegrebet er dog omdiskuteret, og findes i en række udgaver. 
% 		Det vigtige er, hvad man skal bruge det til (Wright)
% 		Harris måde at opdele hvad klasse er (beskrivende) og hvad klasse gør (forklarende). Denne skelnen vil blive benyttet for at forstå de forskellige bud på klasse som stratifikationsmekanisme på arbejdsmarkedet. 
% 		Bojes 2. kriterie er hvad klasse er, på overordnet plan
% 		Bojes 3. kriteri er hvad klasse gør - eller måske mere, hvad indholdet er i klasse, hvis man vælger at benytte det som en måde at forstå segmenter på. 
% 	Hvad er indholdet i klasse, og hvordan kan vi bruge det til at forstå segmenter på arbejdsmarkedet? Vi skal have forskellige bud på klasseteori og mekanismer på plads, og derigennem kan vi se på hvad en empirisk kortlægning af mobiliteten mellem beskæftigelseskategorier kan sige om segmentering og de sociale processer, der skaber dem. Den empiriske kortlægning er altså tofold: Den forsøger både at hente indhold fra og give indhold til klasse som beskrivende, og det samme med klasse som forklarende værktøj. 

% Marxistisk klasseanalyse

% Weberiansk klasseanalyse

% Sammensmeltning og empirisk det samme

% (Bourdieu?)

% Durkheimiansk klasseanalyse

% Standoff


% %
%  \begin{enumerate} [topsep=6pt,itemsep=-1ex]
%    \item Er der en opdeling af arbejdsmarkedet for arbejdsløse i delmarkeder, hvor mobilitet indenfor delmarkederne er hyppig, og mellem delmarkederne sjælden?
%    \item Er der forskelle i den struktur vi ser blandt de beskæftigede, der går fra job til job, og de arbejdsløse?
%    \item Kan forskelle i de sociale processer vise, at der er tale om segmenter, og ikke blot delmarkeder?
%    \item Kan teori med udgangspunkt i social stratifikation belyse denne segmentering?   
%  \end{enumerate}

