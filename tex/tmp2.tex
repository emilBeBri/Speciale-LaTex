

Arbejdsmarkedssegmenteringsteori (overordnet framework)
	- mobilitet på arbejdsmarkedet af interesse for samfundets økonomi (læs Boje)
	1. kriterie - Delmarkeder og mobilitet
	2. kriterie - Forskellen på et delmarked og et segmenter er forskelle i sociale processeser
	3. kriterie - Sociale processer fører til segmentering

Giver motivation til følgende:
	- Hvordan kan sociologien benyttes til at forstå barrierer på arbejdsmarkedet, især segmenteringen? 
	- Hvordan kan sociale processer forstås, og hvad er segmentering udtryk for?

Stratifikation i samfundet, særligt: Beskæftigelse
	Klasse som social proces der fører til segmentering - diffentiering/stratifikation
		- Mange bud på klasseteori: Gælder om at finde den, den passer til det man vil lave (Wright)
		- gennemgang af bud på klasse med følgende fokuspunkter:
			- Klasse som beskrivende og forklarende værktøj (Harrits)
			- Nominel og realistisk klasseopdeling
		Fokus er på det forklarende, jvf arbejdsmarkededssegmenteringsteorien: Det er arbejdsmarkedets interne opdeling, der har primær interesse. 

Forskellige bud på klasser
	Marx
		-forholdet til produktionsmidlerne, udbytning
	Nymarxisme	
		- Problemerne i basis-overbygning og middelklassens position
		- Redskaber mere nuancerede end kapital-arbejde - hvordan, og stadig beholde den marxistiske kerne? 
	Wright
		- klasse som forklarende og nominel (tror jeg)
		- Tre dimensioner: Ejerskab over produktionsmidler, autoritet, færdigheder
	Weber
		- Klasse, social status og stænder
	Goldthorpe
		- Klasse som forklarende og nominel
		- Tre dimensioner: Ejerskab over produktionsmidlerne (knapt så centralt), specifiteten af kompetencer, kontrol over arbejdsprocessen 
	Grusky
		- Klasse som beskrivende og realistisk
		- kritik af nominel klasseinddeling
			- for grovkornet
			- misser det centrale ved definition udenfor de reelt oplevede fælleskaber
		- erhvervsgrupper som klasser, sociale lukningsmekanismer

Hemmelig agenda: Placere Moneca som et oplagt bud på at mediere og vurdere de to positioner, da den starter disaggregeret og giver bud på segmenter. IKKE løsningen, men et bud på yderligere udforskning af segmenter/klasseopdeling. Kan disse anses som klasser? Hvad skal der til for at de kan anses som klasser?
		- Fælles ting indenfor segmenterne (arbejdsløshed, indkomst etc?)
		- stærk intern mobilitet (de kender hinanden - realistiske fællesskaber som hos Grusky, MEN ikke nok!)
		- Mønstre i segmenterne, hvilke findes der? Tilnærmer de sig Goldthorpes teoretiske inddeling? Bourdieu og felter?	
		- Reproduktion over tid (vigtigt)
		- Boje diskuterer det ikke: Folk gifter sig, familien som enhed (ham polakken og Wright), større klynger?
			- Hvilken konsekvens har det for segmententinddelingen? Skal de så ligges sammen hvor de gifter sig meget? Hvor langt skal "realistiske fælleskaber" strækkes? Er der ikke et hul i Gruskys argument her: Hvis far er i jordbeton segmentet og mor i omsorgsklyngen, strukturerer det ikke deres "oplevede fælleskab" på en måde der kræver en slags underliggende logik for at kunne forstås?

Sociale processer hermed begrebsligjort. Videre til analysen.






spm til Jens: Bourdieu?




