%%%%%%%%%%%%%%%%%%%%%%%%%%%%%%%%%%%%%%%%%%%%%%%%%%%%%%%%%%%
\chapter{Problemformulering og forskningsspørgsmål}
%%%%%%%%%%%%%%%%%%%%%%%%%%%%%%%%%%%%%%%%%%%%%%%%%%%%%%%%%%%

Hej Jens, her den nyeste udgave af min problemformulering og forskningsspørgsmål. 

%
\vspace{\baselineskip}
%
\begin{tcolorbox}[title=\textbf{Problemformulering}]
Findes der segmenter på det danske arbejdsmarked, og hvordan kan forskelle i sociale processer være med til at forklare denne stratifikation af arbejdsmarkedet?
\end{tcolorbox}
%
\vspace{\baselineskip}
Det vil jeg undersøge med følgende undersøgelsesspørgsmål:
\vspace{\baselineskip}
\begin{tcolorbox}[title=Forskningspørgsmål,
subtitle style={boxrule=0.4pt} ]
\tcbsubtitle{1.}
Er der en opdeling af det danske arbejdsmarkede i delmarkeder, hvor mobilitet indenfor delmarkederne er hyppig, og mellem delmarkederne sjælden?
\tcbsubtitle{2.}
Kan forskelle i de sociale processer vise, at der er tale om segmenter, og ikke blot delmarkeder?
\tcbsubtitle{4.}
Hvordan teori med udgangspunkt i social stratifikation belyse denne segmentering? Hvad kan det sige om stratifikationen på det danske arbejdsmarked og i det danske samfund?
\end{tcolorbox}