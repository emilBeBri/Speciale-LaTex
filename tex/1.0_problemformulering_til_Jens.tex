%%%%%%%%%%%%%%%%%%%%%%%%%%%%%%%%%%%%%%%%%%%%%%%%%%%%%%%%%%%
\chapter{Problemformulering og forskningsspørgsmål}
%%%%%%%%%%%%%%%%%%%%%%%%%%%%%%%%%%%%%%%%%%%%%%%%%%%%%%%%%%%


%
Ovenstående kriterier udmønter sig i følgende problemformulering:
%
\vspace{\baselineskip}
%
\begin{tcolorbox}[title=\textbf{Problemformulering}]
Findes der segmenter på det danske arbejdsmarked, og hvordan kan forskelle i sociale processer være med til at forklare sådanne forskelle i segmentstrukturen?
\end{tcolorbox}
%
\vspace{\baselineskip}
Det vil jeg undersøge med følgende forskningsspørgsmålspørgsmål:
\vspace{\baselineskip}
\begin{tcolorbox}[title=Forskningspørgsmål,
subtitle style={boxrule=0.4pt} ]
\tcbsubtitle{1.}
Er der en opdeling af arbejdsmarkedet for arbejdstagere i delmarkeder, hvor mobilitet indenfor delmarkederne er hyppig, og mellem delmarkederne sjælden?
\tcbsubtitle{2.}
Er der forskelle i den struktur vi ser blandt de beskæftigede, der går fra job til job, og de arbejdsløse?
\tcbsubtitle{3.}
Kan forskelle i de sociale processer vise, at der er tale om segmenter, og ikke blot delmarkeder?
\tcbsubtitle{4.}
Kan teori med udgangspunkt i social stratifikation belyse denne segmentering?
\end{tcolorbox}