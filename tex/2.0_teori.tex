%!TEX root = ../report.tex

%%%%%%%%%%%%%%%%%%%%%%%%%%%%%%%%%%%%%%%%%%%%%%%%%%%%%%%%%%%
\chapter{Udkast til teoriafsnit samt overordnet udkast til analysen \label{teori}}
%%%%%%%%%%%%%%%%%%%%%%%%%%%%%%%%%%%%%%%%%%%%%%%%%%%%%%%%%%%
%klasse er et analytisk framework til at forstå segmenteringsprossesen


 Arbejdsmarkedssegmenteringsteorien er det overordnede framework der benyttes i specialet til at forstå arbejdsmarkedets indretning. Centralt i denne står mobiliteten på arbejdsmarkedet samt barrierer for mobiliteten, det vil sige arbejdsmarkedets opdeling i delmarkeder, og hvordan opkomsten af disse delmarkeder skal forstås. Thomas Boje skelner mellem delmarkeder og segmenter, hvor det sidste ses som mere stabile sociale strukturer.

 Thomas Boje benytter følgende 3 kriterier:

\begin{itemize}
  \item 1. kriterie - Delmarkeder og mobilitet
  \item 2. kriterie - Forskellen på et delmarked og et segmenter er forskelle i sociale processeser
  \item 3. kriterie - Sociale processer fører til segmentering
\end{itemize}

Sociale processer er et meget vagt begreb, hvilket også betyder at segmenter bliver tilsvarende vagt. Ellers bliver det bare empiriske observationer, der viser "klynger af jobtyper med hyppig udveksling." Dette behov for forståelse af de sociale processer der ligger bag segmentering, motiverer følgende: Hvordan kan sociologiens begreber om stratifikation og differentiering bruges til at forstå arbejsmarkedets opdeling i segmenter og delmarkeder? Der findes massere sociologisk teori om stratifikation af mange typer, men fokus vil her være på arbejdsmarkedet og hvad der virker ulighedsskabende på selve arbejdsmarkedet. 

Altså: Stratifikation i samfundet, med udgangspunkt i beskæftigelse. Jeg vil tage udgangspunkt i klassebegrebet, som det analytisk framework til at forstå segmenteringsprossesen  og de forskellige bud på klasser, da intet begreb i sociologien er så nært knyttet op på en forståelse af de sociale processer arbejdslivet  består af. 
Som Eric Olin Wright fremhæver, gælder det om at benytte den klasseteori, der er bedst til det man gerne vil udtale sig om (find henvisningen). Jeg vil i det følgende gennemgå forskellige klasseteorier, for at finde frem til de forklaringer, der hjælper til at forstå segmenteringsopdelingen, som jeg finder den empirisk ved hjælp af Moneca (mere elegant introduktion af den empiriske metode senere, skal bindes bedre an). Ud af denne gennemgang bliver det også klart, hvori Moneca er i stand til at bidrage til klasseforsknigens nuværende diskussioner. 

Gitte Harrits (2014) benytter en opdeling mellem klasseteoriens \emph{beskrivende} og \emph{forklarende} spørgsmål. Klasse som beskrivende spørgsmål er hvad klasse "er", hvordan teorien fastsætter de sociale definitionskampe, der afgrænser og definerer de forskellige klasser. Klasseteoriens forklarende spørgsmål er hvad klasse "gør": Hvilke effekter har klasserelationerne, og hvilke mekanismer er vigtige, eksempelvis værdier eller interesser \parencite[19]{Harrits2014}. Denne vil blive brugt som overordnet skelnen i teorierne (til Jens: det er så ikke rigtigt gjort endnu. Men det kommer! Ideen er god synes jeg.)

Det vil jeg gøre med følgende fremgangsmåde. 

Først vil jeg gennemgå Marx, der definerer relationerne til produktionsmidlerne som afgørende, og den udbytning, de danner basis for, og som definerer de sociale relationer. Derefter vil jeg gå ind i nymarxisternes hovedpine med basis-overbygningsmodellen, eftersom klassesamfundet ikke blev polariseret i to store klasser, sat imod hinanden, men i stedet fremkomsten af nye middelklasser samt nye serviceerhverv, der ikke ser ud til at være midlertige, men tværtimod helt nødvendige dele af den moderne kapitalisme. Jeg vil slutte min gennemgang af marxismens bud på stratifikation på arbejdsmarkedet ved det for mig at se mest empirisk anvendelige bud Eric Olin Wrights klassemodel.

Derefter vil jeg kort gennemgå Weber, der skelner mellem social status, klasse og stænder. Han har inspireret John Goldthorpes meget anvendte såkaldte EGP-klasseskema, der vil blive gennemgået som det næste, med dets fokus på kontrol over arbejdsprocessen og specifiteten af faglige kompetencer som de centrale differentieringsmekanismer. Goldthorpe har en analytisk-teoretisk funderet klasseforståelse, som bliver udfordret af David Grusky, der er fortaler for en noget anderledes klasseforståelse, baseret på erhvervsgrupper og de reelt oplevede fællesskaber, der findes her. Han benytter sig af Frank Parkins begreb om sociale lukningsmekanismer, som også bliver gennemgået (til Jens: sikkert for grundigt gennemgået, skal skæres til). I forhold til MONECA er hans bud spændende, da han udfordrer ideen om store klasser, og er fortaler for disse mikroklasser, som han mener drukner i de i sammenligning grovkornede klasseinddelinger, baseret på teoretiske principper. For mig at se kan Moneca netop være en måde at opdele i større klasser, baseret på empiriske bevægelsesmønstre, det vil sige give plads til Gruskys anke mod "big classes", som for mig at se er fornuftig nok, i og med en undersøgelse på mikroniveau kan afsløre klassedynamikker man ikke fanger på et højere plan. Jeg synes dog Gruskys teoretisk grundlag og affejning af teoretiske klasseinddelinger er problematisk, og hans afvisning af at der findes effekter af at tilhøre større klasser end erhvervsgrupper er problematisk. Her er Moneca en metode, der nærmest virker som skabt til at fungere på begge niveauer, det vil sige som beskrivelse af erhvervsgrupper på meget detaljeret niveua, samt mulighed for at afdække fællestræk i deres "sociale processer" på segmentniveau.

%til sidst vil jeg lede det over i metodeafsnittet, da både Grusky og Goldthorpe benytter økonometrisk funderet regressionsanalyse, og jeg vil mene at en relationelt funderet social netværksanalyse har mulighed for at klasser som relationelle størrelser, fremfor som udtryk for variable.



%DET VIL JEG VIL GØRE MED FØLGENDE FREMGANGSMÅDE...

% Forskellige bud på klasser
% 	Marx
% 		-forholdet til produktionsmidlerne, udbytning
% 	Nymarxisme	
% 		- Problemerne i basis-overbygning og middelklassens position
% 		- Redskaber mere nuancerede end kapital-arbejde - hvordan, og stadig beholde den marxistiske kerne? 
% 	Wright
% 		- klasse som forklarende og nominel (tror jeg)
% 		- Tre dimensioner: Ejerskab over produktionsmidler, autoritet, færdigheder
% 	Weber
% 		- Klasse, social status og stænder
% 	Goldthorpe
% 		- Klasse som forklarende og nominel
% 		- Tre dimensioner: Ejerskab over produktionsmidlerne (knapt så centralt), specifiteten af kompetencer, kontrol over arbejdsprocessen 
% 	Grusky
% 		- Klasse som beskrivende og realistisk
% 		- kritik af nominel klasseinddeling
% 			- for grovkornet
% 			- misser det centrale ved definition udenfor de reelt oplevede fælleskaber
% 		- erhvervsgrupper som klasser, sociale lukningsmekanismer

%DET SKAL ALT SAMMEN BRUGES TIL...

	% Opsamling: 
	% 	- Kobling til delanalyse 1: Moneca som måde at inddele arbejdsmarkedet i segmenter (klasser)
	% 	- Kobling til delanalyse 2: Forklare sociale processer i segmenter (klasser) vha. klasseforskere
	% 	- Kobling til delanalyse 3: Jeg  placere Moneca som et oplagt bud på at mediere og vurdere de to positioner (den realistiske og nominelle klasseindling)

Denne teoretiske gennemgang skal bruges til følgende i forbindelse med det overordnede spørgsmål om arbejdsmarkedets segmentering, i 3 analyser:

\begin{itemize}
  \item Delanalyse 1: Moneca som måde at inddele arbejdsmarkedet i segmenter
  \item Delanalyse 2: Forklare sociale processer i segmenter (klasser) vha. klasseforskere
  \item Delanalyse 3: Moneca som på at mediere og vurdere de to positioner (den realistiske og nominelle klasseindeling hos henholdsvis Grusky og Goldthorpe)
\end{itemize}



%Motivation for at inddrage klasseteori er, at analysen af delmarkeder/segmenter og mobilitet ligger inden for en tradition af klasseteori. Dette ses såvel i Wrights analyse/teori om bla bla bla såvel som i Gruskys analyse/teori om...

%Arbejdsmarkedssegmenteringsteorien, som ligger til grund for Toubøl, Larsen og Strøbys udvikling af en a posteori arbejdsmarkedssegmenteringsanalyse, ligger inden for en amerikansk/anglo-saksisk økonomisk diskussion op i mod den neoklassiske økonomiske teori \parencite[2]{Touboel2013}. Thomas Boje opsummerer kritikkken således:

%\begin{displayquote} “\textit{The core of this criticism is first that neoclassical labour-market theories in their analogies to the commodity market do not take into account the heterogeneous character of the labour market and the lack of homogeneity in the labour force. Secondly the theories do not take into account that the behaviour of individuals on the labour market is not governed by economic utility maximization, but, on the contrary, determined by social relations and institutions of which the individual is an integral part.}” \parencite[171]{Boje1986}. \end{displayquote}

%Segmenteringsteorien fremkomst primært i USA opstod ifølge Cain i 1960'erne og 1970'erne som en reaktion på kampen mod fattigdom og fuld deltagelse i økonomien for blandt andet minoritetsgrupper og kvinder \parencite[1216]{Cain1976}. Doeringer og Piore opdelte arbejdsmarkedet i primære og sekundære markeder med “gode” jobs og sidstnævnte indeholder alle de “dårlige” jobs med henblik på at forklare den ulige fordeling af lønninger \parencite[70]{Doeringer1971}, mens Reich, Gordan og Edwards mere radikale tilgang tilgang beskrev, hvordan den historiske proces, hvorved politisk-økonomiske kræfter fremmer opdelingen af arbejdsmarkedet i seperate delmarkeder \parencite[359]{Reich1973}. Hvor denne segmenteringsteori tager udgangspunkt i forandringerne i sin tid og en økomisk kritik af økonomien, så er mit udgangspunkt og fokus et andet. Mit udgangspunkt er den sociologiske tradition, klasseteori og klassediskussion.

%Derfor vil jeg frem for at dykke ned i segmenteringsteoretikerne tage fat i de sociologiske klaseteoretikere såsom Marx, Weber, Gitte Harrits, Goldthorpe, Wright, Parkin og Grusky. Disse teoretikere er både med til at forklare det første kritikere i forhold til delmarkeder og mobilitet (fx Weber og Goldthorpe), hvilket var bla bla bla... Samtidig ligger der også inden for andet kriterie det vil sige, hvorfor der er forskellen på et delmarked og et segmenter er forskelle i sociale processeser (fx Wright og Goldthorpe). Alt sammen er det interessant også for det tredjhe kriterie oim hvorfor sociale processer fører til segmentering.




%%%%%%%%%%%%%%%%%%%%%%%%%%%%%%%%%%%%%%%%%%%%%%
\section{Karl Marx \label{2_marx}}
%%%%%%%%%%%%%%%%%%%%%%%%%%%%%%%%%%%%%%%%%%%%%%

Marx selv nåede aldrig at færdiggøre sin klasseteori, eller skitsere den i en sammenhængende form. Kapitel 52 i tredje bind af \emph{Kapitalen} har titlen "Klasserne", men efter cirka en side stopper skriften, og Engels, der udgav værket efter Marx' død, skriver: "Her afbrydes manuskriptet" \parencite[22]{Harrits2014}. Som samlet klasseteori må man derfor benytte sig af Marx' generelle teoriapparat, for at forstå hvad en klasse er hos Marx. Det er værd at huske på, at netop fordi klasserne ikke nåede at modtage en specifik teoretisk behandling, må klasseteorien konstrueres ud fra hans generelle historieforståelse.

Formålet her er ikke en grundig gennemgang, men en skitsering, der skal tydeliggøre fundamentet i den neomarxistiske klasseforståelse.

Klasserne hos Marx er defineret ud fra den tilgang, at det vigtigste menneskelige forhold omhandler produktionen af livsfornødenheder, altså de materielle forhold. De værktøjer - forstået i bredest mulige forstand - som gør det muligt at producere disse livsfornødenheder, defineres som produktionsmidlerne. Menneskers forhold til produktionsmidlerne er dermed den centrale sociale relation i et samfund, og igennem denne primære sociale relatio udgår alle andre sociale relationer. Det vil sige, at mennesker med samme sociale relation, lad os kalde det ejerskabsforhold, til produktionsmidlerne, kan karakteriseres som en klasse, da de deler det mest centrale grundvilkår, nemlig betingelsen der tillader dem at reproducere deres livsgrundlag. Marx peger på det forhold, at mennesker er i stand til at arbejde mere, end det der skal til for at holde dem i live, og det arbejde, der ligger udover hvad der skal til for at reproducere dem som arbejdskraft, kaldes merarbejdet. Det er dette arbejde, der er grundlag for udbytningsrelationer mellem forskellige klasser, hvor nogle klasser i kraft af deres ejerskab over produktionsmidlerne kan nyde godt af de primære producenters arbejdskraft. Det er således både en dominansrelation og en udbytningsrelation på spil, i forholdet mellem de eller den herskende klasse og den eller de klasser, der tvinges til at arbejde for dem.

Kapitalistiske samfund er kort sagt karakteriseret ved, at penge bliver den centrale mediator mellem forskellige former for arbejde. Hvad enten den er indlejret som arbejdstid i en fysisk form som en stol, et kilo sukker, eller i et menneske, som levende arbejdskraft. Det er denne "universelle ækvivalent", som Marx kalder den, der tillader det moderne kapitalistiske marked, hvori alle former for menneskeligt arbejde kan udveksles frit som varer. Frit er det dog ikke helt, da det er Marx påstand, at kun en enkelt vare rummer en ganske særlig kvalitet, nemlig evnen til at avle mere arbejde end det, der er indlejret i den. Den vare er mennesket. Et kapitalistiske samfund er derved karakteriseret ved lønarbejde, hvori en klasse, borgerskabet, ejer produktionsmidlerne, mens størstedelen fungerer som den arbejderklasse, der bliver betalt mindre, end deres arbejde er værd, men nok til at reproducere deres arbejdskraft. Det er Marx forudsigelse, at i overgangen fra feudalsamfundet til det kapitalistiske samfund, vil størstedelen af feudalsamfundets udbyttede klasser blive tvunget fra at have, eller i det mindste være i besiddelse af, deres egne produktionsmidler, til at arbejde direkte for borgerskabet i form af løn. Det kapitalistiske samfund vil derfor blive karakteriseret ved en tiltagende polariseret klassekamp mellem arbejdere og kapitalister, bourgeosi og proletariat.

Det behøver ikke nødvendigvis tolkes sådan, at der findes to klasser, der empirisk stringent kan deles op, selvom det, særligt i mere polemisk og politiske øjemed, er blevet benyttet sådan. I Marx' egne konkrete klasseanalyser benytter han en række termer for klasser og klassefraktioner, ikke kun bourgeosiet og proletariatet. Snarere er påstanden, at der findes en grundlæggende modsætning mellem arbejde og kapital, og denne modsætning er afgørende for interesserne og handlemønstrene for dem, der står på henholdsvis den ene og den anden side af denne modsætning. Her benytter Marx sig af begreberne "klasse i sig" og "klasse for sig", hvor det første er en klasse, der findes objektivt, bestemt af de sociale relationer, og det andet er en klasse, der forstår sig selv som klasse. Der er altså to niveauer i en klasseanalyse:

- et abstrakt niveau, der handler om kampen om merværdi og den grundlæggende klassemodsætning i samfundet
- et konkret niveau, om de observerbare klassefraktioner og grupperinger, der indgår i de politiske kampe i samfundet \parencite[26]{Harrits2014}. %lav til liste

Marx' teori indeholder derfor, som Harris fremhæver, ikke nødvendigvis en påstand om arbejderklassens enhed, men kan ligeså vel tolkes som en invitation til at lave historiske analyser med fokus på samfundsforandring, klassedannelse og politiske magtkampe \parencite[28]{Harrits2014}. I den marxistiske tradition ligger der altså et fokus på at bestemme klasser ud fra deres forhold til produktionsmidlerne, sagt med lidt mere almindelige termer: Hvilken overordnet beskæftigelse, man er i. Den fokuserer derfor på relationer indenfor produktionen, og ikke markedsrelationer, med særligt fokus på den \emph{udbytning} der findes i disse relationer.


%%%%%%%%%%%%%%%%%%%%%%%%%%%%%%%%%%%%%%%%%%%%%%
\section{Problemet med middelklassen og det marxistiske svar \label{2_nymarxisme}}
%%%%%%%%%%%%%%%%%%%%%%%%%%%%%%%%%%%%%%%%%%%%%%

Ikke desto mindre udviklede de moderne samfund sig ikke sådan, at kapitalismens centrifugalkræfter delte samfundet i to, med kapitalister og arbejdere sat overfor hinanden, grundet deres ensartede relation til produktionsmidlerne, som besiddende og ikke-besiddende. Den basale to-klasse model synes ikke at fange de interessemodsætninger og klasseformationer, som opstod i stabile kapitalistiske samfund. En række klasser i det sociale stratum med ejerskab over egne produktionsmidler, såsom artians, small traders, shopkeepers and småbønder, %hvordan fanden oversætter man det
forsvandt godt nok (evt. Parkin henvisning). Men denne polarisering i den socioøkonomiske struktur - altså en koncentration af produktionsmidlerne i færre hænder - blev historisk ikke til to store klasser, sat i mod hinanden, og den teoretiske skillelinje fra den marxistiske politiske økonomi mellem arbejde - kapital kunne tydeligvis ikke omsættes til en tilsvarende opdeling i klasser med egentlig ageren efter denne opdeling. Det vigtige i denne afhandling er, at selvom denne modsætning mellem arbejde - kapital måske nok findes - som påvist af bl.a. Thomas Pikkety, og strukturerer indkomstfordeling på afgørende vis, fungerer denne grundlæggende struktur ikke på nogen entydig måde, der gør denne opdeling særlig brugbar til at undersøge de klasseformationer, vi kan observere empirisk. 

En række andre forhold må medtænkes, da det er tydeligt at middelklassen, der erstattede det gamle småborgerskab, ikke er ekstern i forhold til den kapitalistiske produktionsmåde, men er en central del af den, som Parkin påpeger i sit værk \emph{Marxism and Class Theory: A Bourgeois Critique} fra 1979 \parencite[16]{Parkin1979}. %kan man henvise bare til siden når værket allerede er nævnt?

Der må altså være en række forhold, der muliggør eksistensen af den langt mere komplekse klassestruktur i senmoderne, stabile, kapitalistiske samfund, og som er af central betydning for at forstå klasseformationerne i dette.

Hvis arbejde-kapital ikke er en skillelinje, der giver særlig meget forklaringskraft i de samfund, der findes, hvad gør man så, for at finde de politisk relevante brudflader mellem klasserne? Og hvad skal man gøre med den føromtalte middelklasse? Det er hvad Parkin kalder \emph{the boundary problem in Marxism and sociology} \parencite[11]{Parkin1979}. 

Parkin identificerer tre nymarxistiske svar på dette problem. De to første skal kun kort skitseres, mens den sidste, repræsenteret ved Eric Olin Wright, bliver gennemgået i dybden. 

Det første svar er, hvad Parkin kalder \emph{den minimalistiske} definition af arbejderklassen, repræsenteret ved Poulantzas. Her drages skillelinjen mellem produktivt og uproduktivt arbejde, baseret på en fortolkning af hvilken form for arbejde, der skaber merværdi. Lønarbejde, der ikke skaber merværdi, er også udbytning, og indgår derfor ikke i proletariatet, altså den politisk relevante samfundsgruppe i marxistisk optik. Det giver en forklaring på stabile udbytningsrelationer, også lønarbejdere i mellem. 

Det andet svar er den \emph{maksimalistiske} version, repræsenteret ved Poul Baran, der benytter sig af Marx' lighedstegn, visser steder i sine skrifter, mellem \emph{folket} og arbejderklassen. De mange mod de få. Her benyttes et politisk defineret kriterie baseret på hvilke sociale grupper, hvis sociale funktion også ville være nødvendige i et socialistisk samfund \parencite[20]{Parkin1979}. Det er politisk i den forstand, at det handler om at skabe forening mod en overklasse, og analytisk i den forstand, at det prøver at forudsige, hvilket arbejde der kan siges at være \emph{samfundsmæssigt nødvendigt}, for på den måde at klassificere dem, der lever af andres arbejde.  

Begge disse svar forekommer mig ikke særlig anvendelige i en konkret klasseanalyse, og nævnes derfor kun som forsøg på at løse “\emph{the problem of the middle-classes}”, som som Wright kalder det \parencite[15]{Wright2000}. I stedet vil jeg fokusere på den nymarxistiske retning Parkin kalder \emph{mellempositionen}, som Parkin karakteriserer ved Erik Olin Wright selv \parencite[17]{Parkin1979}.


%%%%%%%%%%%%%%%%%%%%%%%%%%%%%%%%%%%%%%%%%%%%%%
\section{Erik Olin Wright \label{2_wright}}
%%%%%%%%%%%%%%%%%%%%%%%%%%%%%%%%%%%%%%%%%%%%%%

Erik Olin Wright introducerer to dimensioner for at forstå hvad han kalder \emph{modsætningsfyldte klassepositioner}, normalt omtalt som middelklassen: “\emph{people who do not own their own means of production, who sell their labor power on a labor market, and yet do not seem part of the working class.}” \parencite[15]{Wright2000}. Det giver samtidig anledning til at differentiere arbejderklassen og klassesamfundet generelt, i en mere nuanceret klassebaseret stratifikationsteori, der viser \emph{hvad klasse gør}, som Harris formulerede det. 

Centralt står stadig forholdet til produktionsmidlerne og de udbytningsrelationer, der hersker mellem mennesker, alt efter deres relation til produktionsmidlerne. Wright sætter to dimensioner på udbytningsforholdet, der udtrykker mulighederne for at tilegnelse af samfundets ressourcer. Klassepositioner ses som udtryk for en kombination af disse to akser samtidig med, at han bibeholder overordnede skelnen mellem ejer-ikke ejer.

Den første akse kalder Wright \emph{autoritetspositioner}, og omhandler det forhold, at kapitalejere uddelegerer deres lovmæssige ret til at lede og fordele arbejdet til andre. Det sætter dem i en modsætningsfyldt klasseposition, da de ikke er kapitalister selv, men står over arbejderne, og modtager højere løn for deres ansvar i en virksomheds daglige ledelse. Den anden akse kalder Wright \emph{færdigheder}, der beskriver de faglige og tekniske færdigheder, der er nødvendige for en given produktion. Desto mere specialiserede, og dermed monopoliserede, en færdighed er, desto mere tager den karakter af ekspertviden, som også muliggør større belønning for det udførte arbejde. % Søren: Dette kan bruges i forhold til sociale processer i segmenterne på sammen måde som "socia closure".
Udover disse akser har Wright, traditionen tro fra Marx, en skillelinje mellem ejere og ansatte, altså de, der selv ejer produktionsmidler, og de, der ikke gør. Disse to akser demonstrerer dermed en skala af klassepositioner, der muliggør forskellige slags tilhørsforhold og identiter, der ikke nødvendigvis polariseres i den enkeltdimensionerede arbejder-kapitalist relation, omend denne grundlæggende modsætning bibeholdes i modellen. Middelklassen bliver dermed den gruppe mennesker, der befinder sig i disse mellempositioner, og Wrights model muliggør også at positionere ansatte indenfor service- og informations industrien, der traditionelt har voldt marxistiske teoretikere en del genvordigheder.

Wright benytter denne udbyggede model til at forstå, hvordan man på det samfundsmæssige plan ser bestemte klasseformationer og politiske kampe, der gerne skulle kunne forklares ud fra denne klassemodel. Spørgsmålet om hvorvidt man skal kortlægge klassestrukturerer “bag om ryggen” på individerne står altså stadig centralt. Wright fastholder på den side ene side en teoretisk model, baseret på udbytningsrelationer i økonomien, samtidig med at den konkrete udformning af denne skal testes empirisk og revurderes løbende, baseret på dens evne til at forklare politiske kampe og observerede klasseformationer s. 75. 

% brug evt. henvisning fra Harris s. 67 til to konkrete danske analyser med brug af Wrights klassemodel 


%%%%%%%%%%%%%%%%%%%%%%%%%%%%%%%%%%%%%%%%%%%%%%
\section{Max Weber \label{2_weber}}
%%%%%%%%%%%%%%%%%%%%%%%%%%%%%%%%%%%%%%%%%%%%%%

Selvom Weber heller aldrig nåede at færdiggøre sin klasseteori, fremstår den i noget mere færdig form end tilfældet er hos Marx \parencite[28]{Harrits2014}. Webers klasseteori medtager stratifikation, der ligger udover hans klassebegreb, og indeholder derfor en mere pluralistisk tilgang til samfundets opdeling i forskellige sociale grupperinger. Hans klassebegreb er dog tydeligt inspireret af Marx, i den forstand at det er bundet op på økonomiske interesser og et bestemt forhold til produktionsmidlerne:
%
\begin{displayquote} “\textit{Vi vil tale om "klasse" når 1. livsbetingelserne for en flerhed af mennesker er betinget af en specifik fælles komponent, for så vidt 2. denne komponent ene og alene udgøres af økonomiske ejendoms- og erhvervsinteresser, og det foregår på 3. (vare- eller arbejds)markedets betingelser ("klassesituation"}” \parencite[302]{Weber1978}. \end{displayquote}
%
Til forskel fra Marx er klasser også defineret ved deres status på arbejdsmarkedet, fremfor relationen til produktionsmidlerne, og der ligger derfor et større fokus på markedsrelationer, altså “\emph{arten af éns chancer på markedet}” (weber 31 2003, i \parencite[29]{Harrits2014}). For at begribe disse chancer på markedet, der jo i høj grad er bestemt af hvilken økonomisk klasse, man er i, siger Weber, at det er ens evne til at tilegne sig indtægter og goder (Weber 1976 i \parencite[29]{Harrits2014}). Det åbner op for et kulturelt perspektiv, da alle mulige egenskaber, der kan omsættes til økonomiske ressourcer, bliver en del af et menneskes livschancer på markedet. Hos Weber finder vi et klassebegreb, der i udgangspunktet indeholder en typologi baseret på ejerskab/ikke-ejerskab over produktionsmidlerne, men derudover indeholder en langt bredere forståelse af klassefraktioner indkapslet i konceptualisering af hvad der definerer klasser. Og i øvrigt ligger tæt på Marx egne konkrete analyser af klasser, som omtalt tidligere.

Weber peger også på mobilitet mellem klasser som et centralt spørgsmål for klasseforskning. Så bevæger vi os hen til at beskrive, hvad klasse \emph{gør} parencite[302]{Weber1978}. Der er hos Weber et større fokus på den subjektive oplevelse af klassetilhørsforhold. Selvom folk individuelt reagerer ens på deres klassesituation, er det ikke ensbetydende med en deterministisk bevægelse hen imod en klasse-for-sig, som hos Marx. Weber er dermed mere interesseret i alle de aspekter, som en klasse \emph{gør}, fremfor Marx fokus på politiske kampe og social forandring. Klassebegrebet kan bruges generelt til undersøgelse af interessekampe, mobilitet og al mulig anden adfærd. (Harris 31). % Søren: Dette skal du vende tilbage til i analysen - intern mobilitet som en weberiansk klasse-/segmentopdeling.

Det sidste centrale begreb hos Weber er hans begreb om stænder. Det vil sige hans forståelse af, at der udover en økonomisk orden, også er en social orden, "den måde hvorpå social ære" fordeler sig i et fælleskab" (Harris 31). Denne sociale orden er en \emph{selvstændig} kilde til magt i et fælleskab, hvilket er en væsentlig forskel fra den marxistiske klasseanalyse. Det er knyttet op på subjektive oplevelser af fælleskab og anerkendelse af bestemte former for livsførsel om særligt attråværdige. Selvom Weber understreger stænders selvstændighed i forhold til klassesituationen, bliver spørgsmålet om sammenfaldet mellem klasse og stænder, og den måde, de kommer til udtryk på i en bestemt praksis, et genstandsfelt for klasseforskning \parencite[32]{Harrits2014}.


%%%%%%%%%%%%%%%%%%%%%%%%%%%%%%%%%%%%%%%%%%%%%%
\section{John Goldthorpe \label{2_goldthorpe}}
%%%%%%%%%%%%%%%%%%%%%%%%%%%%%%%%%%%%%%%%%%%%%%

Den engelske sociolog John Goldthorpe fremhæves ofte som den neoweberianske traditions vægtigste bidrag til klasseforskning, selvom han selv værger sig at blive sat i hartkorn med en bestemt tradition \parencite[90]{Harrits2014}. Goldthorpes formål med sin klasseteori er, “\emph{evnen til at bruge et par få, velvalgte begreber såsom klasseposition, klaseoprindelse og klassemobilitet - immobilitet til at forklare en stor del både af hvad der sker, og ikke sker, med individer på tværs af forskellige aspekter af deres liv}” (Goldthorpe i \parencite[90]{Harrits2014}). Formålet er derfor at benytte klasse \emph{forklarende} overfor sociale forskelle, bredt forstået i samfundet. Her benytter Goldthorpe ofte Webers forståelse af livschancer og social mobilitet som centrale fokuspunkter for klasseforskning, og da han samtidig tager afstand fra det marxistiske fokus på udbytning, Marx historiefilosofi samt forståelsen af politik som klasseinteresser (goldthorpe i \parencite[90]{Harrits2014}), er det forståeligt hvorfor han ofte indskrives i en neoweberiansk tradition. % Søren: Dette skal du vende tilbage til i analysen - intern mobilitet som en weberiansk klasse-/segmentopdeling.

Selvom Goldthorpe tager afstand fra strukturen og måske især historieforståelsen i den marxistiske klasseforståelse, er hans stratifikationsprincip i klasseopdelingen ligesom i den nymarxistiske tradition baseret på placering på arbejdsmarkedet som den grundlæggende klassestratifikationsmekanisme, i Goldthorpes terminologi \emph{ansættelsesrelationer}. Forskellige ansættelsesrelationer implicerer forskellige klassepositioner og klasseinteresser. (Goldthorpe i \parencite[92]{Harrits2014}). Goldthorpe ser 3 typer af ansættelsesrelationer:
%
\begin{enumerate}
 \item ejerskab over produktionsmidlerne
 \item specificiteten af kompetencer
 \item mulighed for kontrol over arbejdsprocessen
\end{enumerate}
%
Specificiteten af kompetencer har samme status som Wrights færdighedsakse, da værdien af ens færdigheder på markedet er højere, hvis udbuddet af dem er lavt og er svært opnåelige. Mulighed for kontrol over arbejdsprocessen, eller mangel på samme, udmønter i to slags arbejsforhold: som \emph{arbejdskontrakt} eller som \emph{servicerelation}. Hvis arbejdets resultater er nemt målbare, og arbejdets karakter ikke kræver sjældne kompetencer, vil arbejdsforholdene typisk baseres på timeløn eller akkordlønning. Mere diffust målbart arbejde, hvis udførelse kræver en mere specifik, og derfor typisk sjælden, uddannelseskompetence, bliver typisk aflønnet på længere bane såsom månedsbasis eller andet, og her vil kombinationen af svært målbart arbejde, samt specifikke kompetencer, give disse lønarbejdere bedre mulighed for forhandle gode arbejdsforhold. (Goldthorpe i \parencite[93]{Harrits2014}). Kombinationen af disse faktorer indebærer forskellige vilkår for forskellige klasser, hvoraf de tre primære er: 
%
\begin{enumerate}
 \item Forskellige begivenheder indtræffer med forskellig styrker alt efter klasseposition
 \item Forskel i interessedrejning hos forskellige klasser
 \item Omkostninger forbundet med at forfølge en given interesse
\end{enumerate}
%
(Goldthorpe i \parencite[93]{Harrits2014})

Det er ikke nødvendigvis sådan, at alle individer indenfor samme klasse forfølger samme interesser. Snarere sådan at individer i samme klasseposition vil have større affinitet overfor de samme interesser. Goldthorpe benytter Webers forståelse af social orden, forbundet med ære og hierarkisk værdisætning, som selvstændig kilde til social stratifikation og livsanskuelse. Disse værdisæt har ifølge Goldthorpe antaget en løsere form som det 20. århundrede skred frem, uden at det dog ser ud til at opløse dets effekter - snarere er denne form for “verdensanskuelse” langt mere betydningsfuld for at forstå eksempelvis kulturforbrug, og de politiske holdninger på en værdipolitisk akse, gående fra liberitær til autoritær (Goldthorpe i \parencite[95]{Harrits2014}. 

Selvom Goldthorpe understreger denne sociale status kilde til at forstå individerne og deres subjektivitet, er hans EGP-klasseskema dog strengt baseret på position på arbejdsmarkedet, da han laver denne skelnen mellem de to niveauer \parencite[95]{Harrits2014}. 


%%%%%%%%%%%%%%%%%%%%%%%%%%%%%%%%%%%%%%%%%%%%%%
\section{Grusky og Parkin \label{gruskyogparkin}}
%%%%%%%%%%%%%%%%%%%%%%%%%%%%%%%%%%%%%%%%%%%%%%

Grusky har to kritikker af klasseanalyse som den almindeligvis bedrives: For det første at de er nominelt funderede, fremfor at tage reelt oplevede, sociale fællesskaber som grundprincip for klasseinddelingerne. Og for det andet, at de derved ender med en grovkornet klasseinddeling, der skjuler udbytnings- og stratifikationsmekanismer, der burde være i centrum af klasseforskningen. 

%brug eventuelt: basis skal være institutionaliserede, "aktøroplevede" (dårligt ord) erhvervsgruppers i samfundet. Det er på dette niveau, at klasse reelt "fungerer" (Grusky 2001 eller 2002 find henvisning)

Grusky påpeger at menneskers identifikation med klassebaserede identiteter historisk er blevet svagere og svagere, samt den manglende massehandlen, der burde følge med positioner i arbejderklassen, hvilket har ledt til en hel del \emph{nymarxistisk håndvridning}, som han kalder det \parencite[205]{Grusky2001}. 

Det er ikke problemet at benytte beskæftigelse som grundlag for klasseinddeling. Problemet er niveauet af aggregering, der sker ved at placere en lang række forskellige professioner i eksempelvis kategorien \emph{nedre serviceklasse}, som hos Goldthorpe, baseret på teoretisk funderede magtdimensioner, i Goldthorpes tilfælde: Kontrol over arbejdsprocessen samt specificiteten af færdigheder. 

Klasseanalyse bør tage udgangspunkt i hvad Grusky kalder \emph{realistiske} fælleskaber, altså fælleskaber der opleves som sådan af individerne i dem. Det er på dette (bevidste) niveau, at sociale stratifikationsmekanismer reelt opererer \parencite[212]{Grusky2001}. Det er ikke sådan, at den funktionelle enshed i arbejdet, som er fundamentet for erhvervsklassifikationsskemaer, skal tages for pålydende. Funktionel enshed i arbejdet er ikke nødvendigvis lig den sociale enshed, Grusky ser som udgangspunktet for en klasseanalyse. Han mener at denne "\emph{technicist vision}" %
\label{gruskytechnicistvision}%
også bør medtage reelt oplevede sociale distinktioner, fremfor udelukkende tekniske hensyn \parencite[215:fodnote 5]{Grusky2001}%
%
\footnote{Han nævner bl.a. ISCO, der benyttes af Danmarks Statistik, og som jeg også benytter i denne afhandling. Mere om det senere.}%
%
. Men på et tilstrækkeligt detaljeret niveau, hvor vi nærmer os rene professioner, er sandsynligheden for reelt oplevede fælleskaber størst \parencite[207]{Grusky2001}. Ud fra dette niveau kan aggregation i større klasser \emph{eventuelt} finde sted, omend man fornemmer at guldstandarden for Grusky stadig er de realistiske fællesskaber. Det er hans to validitetskriterier: Det første er at medlemmerne i den skal forstå sig selv som tilhørende en gruppe, baseret på beskæftigelse. Det andet er at kollektiv handlen, der sikrer denne gruppe privilegier, forstået som \emph{social luknings}-strategier. Det er et begreb fra Parkins professionssociologi, Grusky ser ud til at overføre direkte. Jeg vil derfor gennemgå Parkins definition af det nu. 

For Parkin er \emph{social lukning} som basis for gruppedannelse en måde at tilbyde et alternativ til den marxistiske nominelle klasseforståelse. Alternativet er baseret på en omfattende kritik af den særstatus, relationen til produktionsmidlerne har i marxistisk klasseinddeling. Især de ikke-holdbare krumspring, han mener nymarxister gør sig for at bevare denne særstatus, samtidig med at klasserne "ikke gør det, det var meningen de skulle gøre". Denne kritik vil ikke blive gennemgået nærmere, jeg vil i stedet fokusere på hans egen forklaringsmodel på stratifikation og udbytningsmekanismer. I forlængelse af den weberianske tradition mener Parkin, at en autoritet- og statussrelationer, der fungerer på niveauet af \emph{distributionen af goder} i samfundet er central. Det bør tillægges lige så stor vægt som ejerskabet til produktionsmidlerne \parencite[24f]{Parkin1979}. Dermed sidestilles markedsmuligheder og livschancer med den direkte relation til produktionsmidlerne%
%
\footnote{Parkin anerkender, at nymarxister også har måtte indoptage denne dimension af klassekamp i deres teoriapparat. "inside every neo-Marxist there seeems to be a Weberian struggling to get out", som Parkin triumerende og ikke uden skadefro formulerer det. \parencite[25]{Parkin1979}}. Ingen akademisk diskussion værd at beskæftige sig med uden lidt skyttegravskrig med sennepsgas.%
%
. Parkin er ikke interesseret i at revitalisere Webers klasseanalyse, men i at benytte den weberianske forståelse af sammenhængen mellem produktionsrelationer, symbolske relationer og markedseffekter, til foreslå et nyt framework til at forstå sociale gruppers kamp om goder i samfundet. Hans pointe er, at hvis man skal forstå sociale afgrænsninger, er (produktionsorienteret) klassebaseret afgrænsning kun en dimension, og det giver ikke mening at tillægge denne større vægt end andre historiske diffenteringsmekanismer, hvoraf han fremhæver etnicitet som et markant eksempel \parencite[38f]{Parkin1979}. Han vil derfor gerne skabe en teori, der kan rumme \emph{enhver slags} social afgræsning, i kampen om samfundets goder \parencite[42]{Parkin1979}. 

Parkin tager fat i Webers begreb \emph{social lukning} i fundamentet af denne teori. Han skriver:
%
\begin{displayquote} “By social closure Weber means the process by which social collectivities seek to maximize rewards by restricting access to resources and opportunities to a limited circle of eligbles. This entails the singling out of certain social or physical attribute as the justificatory basis of exclusion. (...) the nature of these exlusionary practices, and the completeness of social closure, determine the general character of the distributive system” \parencite[44]{Parkin1979}. \end{displayquote}
%
I samme åndedrag skriver Parkins, at tilegnelsen af økonomiske ressourcer er central \parencite[44]{Parkin1979}. Konsekvensen af ovenstående citat,for Parkins, er at udbytning ikke længere defineres som udtryk for en relation til produktionsmidlerne, men at ejerskab over produktionsmidlerne ses som en specifik, og dog helt central social lukningstrategi, der muliggør udbytning \parencite[53]{Parkin1979}. 

Parkins peger på en anden væsentlig kilde til udbytning i moderne samfund, som er de højere, bogligt funderede uddannelser%
%
\footnote{Parkin benytter den engelske betegnelse \emph{the professions}, hvilket jeg vil mene kan oversættes til ovenstående på dansk [er du enig Jens?]}%
%
. Ligesom en række andre sociologer såsom Bourdieu og Collins (\#henvisningmangler) peger Parkin på professionalisering gennem uddannelse og juridisk legitime uddannelsesdiplomer som central differentieringsmekanisme i moderne samfund, hvilket han kalder \emph{kredentialisme}. I Parkins teori er der tale om udbytning, da der tilegnes ressourcer i form af social lukning: En strategi til at hæve markedsværdien \parencite[54]{Parkin1979}. I Thomas Bojes optik kunne vi sige, at det er erhvervsgruppers måde at kontrollere udbuddet af arbejdskraft på et givent område, for dermed at øge markedsværdien. Parkin adskiller denne kredentialistiske strategi indenfor de længere uddannelser, med de strategier, fagforeninger bruger for at beskytte sig imod udbytning fra arbejdsgivere, da det bevidste mål ikke er at reducere de materielle muligheder for andre i arbejdsstyrken \parencite[57]{Parkin1979}. Dette vidensmonopol for de længerevarende uddannelser mener han er så omfattende, at overklassen i dag udgøres af ejerne og (deres ansatte) ledere af produktionsmidlerne, samt de lærde klasser med monopol en særlig viden \parencite[58]{Parkin1979}. 

Parkin skelner mellem social lukning som \emph{eksklusion}, hvor målet er at tilegne sig ressourcer ved at ekskludere andre, og social lukning som \emph{usurpation}. Det sidste foregår som respons på en i samfundet befæstet, muligvis \emph{formaliseret}, eksklusion. For eksempel organiserede lønarbejderes kamp for bedre løn og arbejdsvilkår, eller etniske gruppers kamp for udvidelse af deres juridiske borgerrettigheder \parencite[74]{Parkin1979}. 

Ekslusionsbaserede social lukningsstrategier er, som eksempelvis etnicitetsbaserede kriterier viser, ikke kun noget, der hører klasser som helhed til. Parkin peger også på eksklusitionsbaseret social lukning indenfor klasser, i hvad han kalder \emph{dobbelt lukning}. Han bruger arbejderklassen som eksempel, uden at definere den nærmere, men man antage at han mener de lønmodtagere, der ikke tilhører de professionaliserede højtuddannede. Disse benytter også eksklutionsbaseret social lukning mod andre dele af klassen. Det kan ske baseres på køn, etnicitet eller andre sociale kriterier der giver mening i den sociale kontekst \parencite[89]{Parkin1979}. Når en indfødt arbejderklasse lukker sig overfor en arbejderklasse fra et andet land, handler det ofte om at sikre de sikrede rettigheder overfor udvidelsen af arbejdsudbuddet til dårligere forhold \parencite[91]{Parkin1979}. Det virker ganske genkendeligt i diskussion af østarbejderes indtog í Danmark. Som han siger: "\emph{workers who opt for closure against a minority can hardly de beclared guilty of irrationality in choosing to retain the proven benefits of exclusion in preference to the uncertain or doubtful pay-off resulting from combined usupartion.}" s. 95". Ligeledes er fragmentering i klasserne langs beskæftigelsesmæssige linjer et vigtigt fokuspunkt \parencite[81]{Parkin1979}.

I en tilbagevenden til Grusky kan vi sige, at Parkin er helt enig i Gruskys 2. kriterie for klasseinddeling: For Parkin er kollektiv handlen det, der definerer en klasse, og ikke en a priori teoretisk inddeling, som han ikke har meget til overs overfor \parencite[113]{Parkin1979}. De skøjter simpelhen over potentielt vigtige lukningsmekanismer internt i deres nominelt inddelte klasser, fremfor at fokusere på sociale grupper, der viser deres eksistens ved deres sociale agens. 

Det spørgsmål, man kan stille til Parkins teori er, hvad status klassebegrebet overhovedet har i hans teori, om det ikke tømmes for substans? Hvis en klasse er omdannet til en hvilken som helst social gruppe, der er involveret i sociale lukningsmekanismer - dvs. semitautologisk, har skabt sig selv som social gruppe - til hvad nytte er så et klassebegreb? I Parkins bog fra '79 bruger han lang tid på at fjerne klasses \emph{nødvendige} relation til produktionsmidlerne, blot for at genindsætte det som central social lukningsmekanisme i moderne samfund, og alligevel benytte beskæftigelsesstatus til at beskrive de dominerende klasser i samfundet som ejere/administratorer af produktionsmidlerne samt de højtuddannede professioner. I den proces mister klasse sin betydning som et bestemt analytisk greb, og bliver til en hvilken som helst social gruppe, der har organiseret sig som sådan. Det er også formålet, for at slippe uden om den teoretiske a priori inddeling, men man kan spørge, om det er at udvande klassebegrebet i det omfang, Parkin gør det, for at slippe af med hvad man kunne kalde den marxistiske blindgyde om økonomisk basis- ideologisk overbygning%
%
\footnote{"\emph{(...) and more maddening talk about steam mills}" som Parkin hvast formulerer det \parencite[6]{Parkin1979}.}%
%
. 

Grusky giver beskæftigelsesstatus en priviligeret position, eftersom han i sin argumentation for "disaggregerede" mikroklasser benytter beskæftigelsesstatus uden at argumentere yderligere for det. %(det gør han nok et andet sted, det bliver du nok nødt til at finde. sucks. \#todo) 
Ham og Parkins indtager samme standpunkt i deres holdning til klasseinddelingskriterier.

Man kan med Harrits sige, at fokus for særligt Grusky ligger på hvad klasse er, omend han ved hjælk af Parkin m.fl. trækker på teorier, der handler om hvad klasser gør. Det primære fokus - eller første skridt - er dog spørgsmålet om hvad klasse er. 

Goldthorpe har i \emph{Acta Sociologica} svaret på Gruskys kritik, og diskussionen mellem de to for mig at se ligger i hjertet af hvad en moderne klasseanalyse bør indeholde og hvad den kan tilbyde. Jeg vil derfor referere Goldthorpes kritik, og kommentere på denne, samt se på hvad min afhandlings bidrag kan være i denne diskussion. 


%%%%%%%%%%%%%%%%%%%%%%%%%%%%%%%%%%%%%%%%%%%%%%
\subsection{analysepunkter (skitseagtigt)}
%%%%%%%%%%%%%%%%%%%%%%%%%%%%%%%%%%%%%%%%%%%%%%

I indledningen præsenterede jeg følgende 3 delanalyser:

\begin{itemize}
  \item Delanalyse 1: Moneca som måde at inddele arbejdsmarkedet i segmenter
  \item Delanalyse 2: Forklare sociale processer i segmenter (klasser) vha. klasseforskere
  \item Delanalyse 3: Moneca som på at mediere og vurdere de to positioner (den realistiske og nominelle klasseindeling hos henholdsvis Grusky og Goldthorpe)
\end{itemize}


Her er en en nogle uddybelser af ovenstående, hvad der skal undersøges

\begin{itemize}
  \item Fælles ting indenfor segmenterne (arbejdsløshed, indkomst etc?)
  \item stærk intern mobilitet (de kender hinanden - realistiske fællesskaber som hos Grusky, MEN ikke nok!)
  \item Mønstre i segmenterne, hvilke findes der? Tilnærmer de sig Goldthorpes teoretiske inddeling? Bourdieu og felter?
   \item reproduktion over tid (vigtigt men måske for omfangsrigt - Moneca kunne sagtens lave et netværk ikke på individers bevægelser, men på mobilitet fra højest uddannede forældres arbejde til barns arbejde)
\end{itemize}

%Kan disse anses som klasser? Hvad skal der til for at de kan anses som klasser?
%		- Fælles ting indenfor segmenterne (arbejdsløshed, indkomst etc?)
%		- stærk intern mobilitet (de kender hinanden - realistiske fællesskaber som hos Grusky, MEN ikke nok!)
%		- Mønstre i segmenterne, hvilke findes der? Tilnærmer de sig Goldthorpes teoretiske inddeling? Bourdieu og felter?	
%		- Reproduktion over tid (vigtigt)
%		- Boje diskuterer det ikke: Folk gifter sig, familien som enhed (ham polakken og Wright), større klynger? Hvilken konsekvens har det for segmententinddelingen? Skal de så ligges sammen hvor de gifter sig meget? Hvor langt skal "realistiske fælleskaber" strækkes? Er der ikke et hul i Gruskys argument her: Hvis far er i jordbeton segmentet og mor i omsorgsklyngen, strukturerer det ikke deres "oplevede fælleskab" på en måde der kræver en slags underliggende logik for at kunne forstås?


% Diskussionen om de to positioner 




% Den store forventning 

% Nominalist view 





% Ved hjælp 11,8 \% procent a












% søskendekorrelationer i langtidsindkomst, .  


% grusky
% grusky kritik
% grusky svar
% egne overvejelser om anvendelighed
% mere grusky fordi han kommer med anvendelige ideer
% (læs mere goldthorpe og wright måske, og andet. Konkrete mekanismer der kan undersøges, ikke mere teori om hvorfor klasser ser ud som de gør)





% Therborn er inde på et lignende argument om globalisering (Therborn 2002:224)

% i takt med globaliserings effekt på klasserelationer, at et skift i fokus til erhvervsgrupper, defineret som mikroklasser, vil betyde et manglende blik for denne essentielle mekanisme, der fremstår tydeligst på et makroniveau. 


% Der er en god teoretisk begrundelse for denne kritik, som jeg vil vende tilbage til, men jeg mener Grusky i sit svar på

% Therborn (ref!)	 og Birkelund (ref!) mener på et metodisk niveau, at makroklasser overskueliggør ten


% - for meget information
% - at have en foruddefineret nominalistisk inddeling 











% Therbon: "An unnecessary violation of language", lol, s. 222



% . Konceptet udbytning er eksempelvis centralt i klasseanalyse, omend det ikke nødvendigvis, som hos Wright,  



% I et lignende argument påpeger Therborn 


% Dette er uafhængigt af om de definerer sig som klasse. 


% de interesser, det giver, og det andet: de omkostninger, der er forbundet med at forfølge visse interesser Harris s. 93.





% definerer klasse anderledes, i rationelle termer - at undergrupper af en klasse spiller "free-rider" på de andre, grundet lokale tilhørsforhold, uden at realisere deres klasseinteresser. Men hvad er klasseinteresser så, på niveau der ikker er arbejde-kapital, men heller ikker erhvervsgrupper? s. 215



% rationelle handlingsmodeller, hvor forskelle i belønninger kontra risici kan forklares ud fra forskelle i ressourcer. Især to forhold: Forskelle i begivenheder, der indtræder for forskellige klasser, fx arbejdsløshed, de interesser, det giver, og det andet: de omkostninger, der er forbundet med at forfølge visse interesser s. 93



% Hvad Grusky harcelerer imod, er a priori teoriske inddelinger af klasse, der står udenfor den sociale praksis, og forekommer på et abstraktionsniveau, der helt har mistet følingen med den sociale praksis, den ellers postulerer at beskæftige sig med på et "væsensniveau" fremfor i "fremtrædelsesform". En kritik, der bestemt har noget på sig i de endeløse marxistiske diskussioner om klasse i dogmatisk forstand, men er lige lovlig polemisk at skyde Goldthorpe og særligt Bourdieu i skoene \parencite[206]{Grusky2001}. Snarere er 

% Som Bourdieu fremhæver, er klasse en analytisk kategori, og bør også også defineres som sådan (FIND HENVISNING). 


% Som salige Marx sagde: Sie wissen das nicht, aber sie tun es. %find reference, søg på google, det er vist nok fra Kapitalens 1. bind. 
% Der findes social logik, der virker strukturerende for aktørens handlinger, uden at denne logik umiddelbart opfattes bevidst af aktøren som en sådan logik. 



% Goldthorpe mener, at der  



% Goldthorpe afviser decideret 

% Goldthorpe mener at klasse fundamentalt adskiller sig fra erhvervsgrupper, da der er tale om en strukturation på et niveau over den præcise beskæftigelse: Det er ikke klasseanalysens formål at vise hvorfor lægesønner bliver læger, men hvorfor lægesønner bliver akademikere eller managers eller andet. Det vil sige noget bag individerne, noget der strukturerer uden nødvendigvis at være en lokal strukturation i Giddens forstand, og Durkheims gemeinschaft. Aktør/struktur. s. 214




% % Søren: Du bruger jo ikke "bare" technicist vision - altså a priori tekniske kategorier. Du tager 143 tekniske kategorier og kigger på den faktiske mobilitet. Den faktiske mobilitet har vel netop en relation til det faktiske fællesskab - man tager vel job de steder, hvor man kan se sig selv arbejde. Derfor er du bedre en Grusky. Du tager Grusky et niveau længere op.



% At fokusere på autoritetsrelationer der fungerer på et niveau af *distribution af goder* fremfor udelukkende på relationen til produktionsapparatet, siger Parker, s. 25. 
% → Er det en nyttig skelnen? Jeg er i tvivl. 

% "inside every neo-Marxist there seeems to be a Weberian struggling to get out", som Parkin triumerende og ikke uden en vis brod formulerer det. s. 25

% snak om etnicitet, ikke super brugbart tænker jeg. Hans pointe er at klasse og kulturel stratifikation som etnicitet skal fungere på samme konceptuelle niveau, hvilket leder til hans social closure teori. 






% Såfremt at en klasse er en klasse-for-sig, er den en klasse, 



% Grusky indrømmer, at alle  













% udviklet den marxistiske klasseanalys

% s

% Det spørgsmål som jeg vil trække frem, er hvorvidt en nominel inddeling af klasse, på baggrund af visse karakteristiska, kan benyttes til at beskrive hvilke klasser et samfund består af. 






% kun findes to klasser, selvom ovenstående beskrivelse unægteligt peger i den retning. Andre har 







%%%%%%%%%%%%%%%%%%%% noter %%%%%%%%%%%%%%%%%%%%%%%%%%%%%%%%


%	både Boje og Grusky pointerer at big classes i høj grad findes i skandinavien. men det er vel netop tegn på, at historiske formationer har formet dem sådan, dvs at klasse netop er et flydende begreb, der har brug for forklaringer på det niveau, fremfor blot på mikro niveau? Netop analytiske kategorier er vel vigtige her.
%	
%	
%	
%	
%	 
%	
% Hvad jeg taler om når jeg taler om klasse
%
%Marx afsnittet: Måske find stedet i kapitalen hvor han taler om "bagerklassen"? Det ved du er der et sted, brug det som bevis på hans forholdsvis løse klassebegreb.


