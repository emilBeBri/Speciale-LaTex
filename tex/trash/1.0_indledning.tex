% -*- coding: utf-8 -*-
% !TeX encoding = UTF-8
% !TeX root = ../report.tex

\chapter{INDLEDNING} \label{intro}

\begin{quote} \small %\raggedright %(bloktekst on/off)
\emph{En sociologisk analyse, der udelukker al psykologi, undtagen enkelte vredesubrud} \sourceatright{- Pierre Bourdieu, “Udkast til en selvanalyse”}
\end{quote}
%


%%%%%%%%%%%%%%%%%%%%%%%%%%%%%%%%%%%%%%%%%%%%%%%%%%%%%%%%%%%%%%%%%%%%%%%%%%%%%%%%%%%%%%%%%%%%%%%%


\section{Teaser \label{}}

Den danske beskæftigelsespolitik beskæftiger sig med at få ledige i job. I 2013 nedsatte regeringen det seneste ekspertudvalg på området med tidligere skatteminister Carsten Koch i spidsen. Udvalgets anbefalinger er den første af to rapporter (2014) var blandt andet en ny, individuel og jobrettet indsats for den enkelte, målrettet brug af opkvalificering i beskæftigelsesindsatsen, styrket fokus på virksomhedernes behov, styrkede økonomiske incitamenter for beskæftigelsessystemet og mindre bureaukrati. Beskæftigelsesanbefalingerne fra Carsten Koch-udvalget ligger sig i forlængelse af en lang række af reguleringer på beskæftigelsesområdet, hvor dansk beskæftigelsespolitik siden Wechselmann-udvalgets i 1960'erne med små skridt af gangen er gået mere og mere væk fra at være tryghedsorienterede velfærdspolitik som sikrer arbejdstagernes økonomisk og social tryghed (Pedersen 2007).

Den historie vi med vores speciale ønsker at fortælle handler om at forstå arbejdsløshed som et multifacetteret begreb, der tydeliggør distinktioner mellem grupper, alt efter deres position i samfundsstrukturen, og som spiller en afgørende rolle for fordelingen af goder i samfundet. Der er grundlæggende tre ting på spil i konstitueringen af et centralt socialt fænomen som arbejdsløshed. Det handler lige så meget om den symbolske magt til at definere, hvad \emph{problemet med arbejdsløshed} i det hele taget \emph{er}, som det handler om, hvad der er \emph{de bedste midler} til at løse problemerne forbundet med det. Og uadskilleligt forbundet med disse to størrelser er beskrivelsen af forskelle og ligheder mellem dem, der står uden job. Vi mener at kunne yde et unikt bidrag til det sidste, på en måde som også kaster lys over de to første.


%%%%%%%%%%%%%%%%%%%%%%%%%%%%%%%%%%%%%%%%%%%%%%%%%%%%%%%%%%%%%%%%%%%%%%%%%%%%%%%%%%%%%%%%%%%%%%%%


\section{Problemformulering og undersøgelsesspørgsmål \label{}}

Vores problemformulering er:
\begin{quote} %\small %\raggedright %(bloktekst on/off)
  \textbf{Hvad betyder kortere og længere perioder med ledighed for arbejdstageres sociale mobilitet på det danske arbejdsmarked?}%
\end{quote}
%
Dette overordnede spørgsmål besvarer vi med følgende underspørgsmål:
%
 \begin{enumerate} [topsep=6pt,itemsep=-1ex]
   \item Hvilke former for beskæftigelse kommer arbejdstagere fra og går til før og efter perioder med ledighed?
   \item Hvilke sammenhænge er der sammenhængen mellem hvilke dispositioner disse arbejdstagere har til rådighed og hvilke strategier de benytter sig af for at komme ud af ledighed og tilbage i beskæftigelse.
   \item Kan man med vores tilgang til ledighed se noget andet end andre tilgange, og hvordan kan vores empiri og andres empiri forstås som en del af en konstant symbolsk kamp om konstituering af arbejdsløshed?
 \end{enumerate}

Vi er særligt interesserede i mobilitet i forbindelse med ledighed. Hvilke jobs får man efter perioder med ledighed? Derved vil vi undersøge hvilken praksis der hænger sammen med hvilke felter, og hvad det siger om hvilke felter der ligger nær hinanden samtidig med hvad der skal til for at bevæge sig ud over det felt man oprindeligt kommer fra. Dette betyder, at vi ønsker at diskutere, hvad der strukturerer folk der oplever arbejdsløsheds opfattelse af handlingsrum, som vi ser det komme til udtryk igennem deres praksis mellem forskellige typer af jobs. Specialets primære metode til at fortælle denne historie er en netværksanalyse af registerdata fra Danmarks Statistiks arbejdsmarkedsdatabaser i perioden 1996 til 2009. Som det centrale redskab til at analysere ledighed bruger vi den den såkaldte MONECA-algoritme, der er en måde at lave en datadrevet skitse af hvilke beskæftigelsestyper der ligger tæt på hinanden frem for en teoretisk og institutionel a priori-inddeling. Det vil sige når en arbejdstager er ledig i en periode, kan MONECA bruges til at se hvilken type job vedkommende kommer fra og går til før og efter den pågældende periode med ledighed.

Når vores speciale beskæftiger sig med en bestemt gruppe - arbejdstagere - som er i beskæftigelse efter at have oplevet periode med ledighed - og hvordan de i praksis - handler - og kommer ud af ledighed, er vores fokus et sted mellem at se arbejdsløshed som et økonomisk problem og at se på arbejdsløshedens sociale konsekvenser i et historisk perspektiv. Vi læner os op af søgeteoriens økonomiske fokus på hvad der skal til for at en arbejdsløs tager imod et job på baggrund af forventninger om løn (Rosholm 2009:162) uden at give et entydigt mål for hvornår det kan betale sig at arbejde, som mere abstrakt rationelle undersøgelser har for vane. I stedet viser vi, hvilke kanaler der eksisterer mellem forskellige jobtyper og derved hvilken praksis som er mulig for hvilke arbejdsløse. Vi lægger os samtidig op af at den sociologiske tilgang til den enkeltes arbejdsløshed skal ses i lyset af bestemte økonomiske markedskræfter og inden for en bestemt historisk udvikling uden, at fokus bliver på arbejdsløse i en slags passiv offerrolle (Goul Andersen 2003:35). I modsætning til det, anskuer vi arbejdsløse som forskelligartede og handlende i et socialt rum, der så til gengæld er betinget af omstændighederne, med sammenfald mellem mentale og objektive strukturer, der kommer til udtryk i “hvad der er muligt”.
Dette følger hen til det tredje og sidste undersøgelsesspørgsmål som fokuserer på et videnskabeligt spørgsmål om hvilken måde vores tilgang til sociale mobilitet på arbejdsmarkedet åbner op at anskue ledighedsbegrebet på end anderledes måde end de nævnte metoder.


%%%%%%%%%%%%%%%%%%%%%%%%%%%%%%%%%%%%%%%%%%%%%%%%%%%%%%%%%%%%%%%%%%%%%%%%%%%%%%%%%%%%%%%%%%%%%%%%


\section{Arbejdsløshed og arbejdsløse - introduktion og afgrænsning \label{}}

Vi trækker på et marginaliseringsbegreb anvendt af Jørgen Elm Larsen, Catharina Juul Kristensen og Nils Mortensen. Nils Mortensen skelner mellem dikotomierne integration/differentiering og inklusion/eksklusion. Integration og differentiering i samfundet kan beskrives som samspil og modspil mellem eksisterende hierarkier som lægger op til strategier for fastholdelse eller forandring af ulighed i magt, privilegier og prestige (Bourdieu 1986), udviklingen af funktionel differentiering mellem forskellige subsystemer svarende til en arbejdsdeling i sektorer, erhverv, fag og professioner (Luhmann 2002) og en opdeling mellem befolkningsgrupper med forskellige regionale og religiøse udgangspunkter (Mortensen 2009: 26).

Jørgen Elm Larsen giver en mere specifik definition af social eksklusion som en ufrivillig ikke-deltagelse gennem forskellige typer af udelukkelsesmekanismer og -processer, som det ligger uden for indvidets og gruppens muligheder at få kontrol over (Elm Larsen 2009: 237). Inklusion/eksklusion har hos Luhmann en binær form, som ikke er særlig hensigtsmæssig i forhold til virkeligheden, og derfor er det mere hensigtsmæssigt at anvende begrebet marginalisering. Marginalisering kan derfor anvendes som en midtergruppe, hvor nogle er i en proces mod inklusion, nogle er i en proces mod eksklusion og nogle forbliver i midtergruppen  (Mortensen 2009: 27; Elm Larsen 2009: 130f). Til vores operationalisering af begrebet marginalisering er vi inspireret af Lars Svedbergs model (Svedberg 1995:44 i Juul Kristensen 1999:16).
\begin{figure}[h]
\begin{centering}
	\caption{marginaliseringsfigur (draft)}
  	\includegraphics[width=\textwidth]{fig/midlertidige/marginaljoergen.pdf}
  	\label{fig_marginaljoergen}
\end{centering}
\end{figure}
%---------------------------------------------------------------------------------
%
%		 	Inkluderet 		<-> 	Marginaliseret 		<-> 	Ekskluderet
%---------------------------------------------------------------------------------
%		kort afstand til 									stor afstand til
%		arbejdsmarkedet										arbejdsmarkedet
%---------------------------------------------------------------------------------
%		beskæftigelse 										vender ikke tilbage
%															på arbejdsmarkedet
%---------------------------------------------------------------------------------
%								delvis beskæftigelse
%								arbejdsløs
%								dagpenge
%								kontanthjælp
%								(efterløn)
%								(førtidspension)
%								(folkepension)
%---------------------------------------------------------------------------------
%							korttidsledig <-> Langtidsledig
%							 (under 1 år)	   (over 1 år)
%---------------------------------------------------------------------------------

Vi anvender DISCOALLE\_INDK i perioden 1996 til 2009 til en variabel på 167 udfald.
\begin{center}
\begin{tabular} { p{3cm} l p{6cm} l p{3cm} } \hline
  Inkluderet & <-> & Marginaliseret & <-> & Ekskluderet \\ \hline
  Lille afstand til arbejdsmarkedet & & & & Stor afstand til arbejdsmarkedet \\ \hline
  I beskæftigelse & & & & Vender ikke tilbage på arbejdsmarkedet \\ \hline
  & & delvis beskæftigelse, arbejdsløs, dagpenge, kontanthjælp, (efterløn, førtidspension, folkepension) & & \\ \hline
  & & korttidsledig <-> Langtidsledig & & \\
  & & (under 1 år)     (over 1 år) & & \\ \hline
\end{tabular}
\end{center}

Vores model viser forskellige grader af marginalisering på arbejdsmarkedet i et spektrum mellem inkluderet og ekskluderet. Med inkluderet på arbejdsmarkedet forstår vi ingen eller kort afstand til arbejdsmarkedet og med ekskluderet forstår vi en stor afstand til arbejdsmarkedet. De inkluderede er i fuld beskæftigelse, beskæftigelse deltid eller beskæftigelse mere end halvdelen af året, og er således \emph{ikke} en del af vores population%
%
\footnote{ovenstående er også defineret ud fra muligheder i DSTs data og det vil vi naturligvis beskrive mere dybdegående i den færdige tekst. Foreløbigt kan man sige at vores (deskriptive) ledighedsbegreb minder om Danmarks Statistiks opgørelse af nettoledighed, som omfatter alle modtagere af arbejdsløshedsdagpenge samt såkaldt arbejdsmarkedsparate kontanthjælpsmodtagere. Danmarks Statistik arbejder også med opgørelse bruttoledighed, som  omfatter alle nettoledige samt alle personer, der er i aktivering og modtager en aktiveringsydelse, og AKU-ledige i form af en Arbejdskraftundersøgelsen.}%
%
. De totalt ekskluderede vender ikke eller sjældent tilbage på arbejdsmarkedet, og vil derfor optræde som dem, der kun optræder en enkelt gang i vores data fra 1996 til 2009. I et spektrum mellem det at være inkluderet og ekskluderet ligger forskellige grader af marginalisering, hvilket kan være beskæftigede mindre end halvdelen af året, arbejdsløs i perioder længere end et halvt år, dagpengemodtagere, kontanthjælpsmodtagere. Hermed træffer vi også et fravalg. Arbejdsløse med mere en et halv års beskæftigelse i det indeværende år vælges fra sammen med arbejdstagere i forskellige prækære og usikre arbejdsstillinger, der er konstant, omend usikkert, beskæftiget. Afslutningsvis skelner vi i vores kortlægning mellem korttidsledige og langtidsledige for at se mønstre.

En tentativ vurdering på baggrund af det foreløbige dataarbejde er, at mange fra vores population af ledige har disco-koder der viser at de roder rundt i det administrative mørke (99-koder), eller har at gøre med job i usikre beskæftigelser. Det siger meget om vores population at staten har så svært ved at kategorisere dem, og når de kan kategoriserer dem er det ofte forventelige beskæftigelser som rengøringsarbejde eller andet prekært.   



%%%%%%%%%%%%%%%%%%%%%%%%%%%%%%%%%%%%%%%%%%%%%%%%%%%%%%%%%%%%%%%%%%%%%%%%%%%%%%%%%%%%%%%%%%%%%%%%


\section{Skitse af metode og delanalyser \label{}} 

Vores tilgang er inspireret af Pierre Bourdieus måde at anskue den sociale verden objektivt på. Det handler om at synliggøre de principper, der ligger til grund for de valg og fravalg af metode, operationalisering, fremstilling mv. Undervejs har det ikke været helt klart, hvilken vej den netværksanalytiske metode har ville føre os hen, da den netop er datadrevet og på mange måder induktiv%
%
\footnote{Det er klart at dataen i sig selv er struktureret på måder der synliggør visse ting og skjuler andre, samt netværksmetoden selv giver et bestemt blik. Men derfor er det stadig svært at gisne om kortenes udformning på nuværende stadie.}%
%
\ .Derfor hjælper disse principper til at klargøre hvad det er for valg og fravalg vi har taget undervejs i specialeprocessen og ultimativt hvad vi har valgt at fokusere på og hvorledes vi har valgt at fremstille det. Med inspiration i en relationelt orienteret sociologi vil vi benytte moneca-algoritmen til at afgrænse individer i klynger alt efter deres tilknytning til arbejdsmarkedet. 

Vi betragter de arbejdsløse som sociale agenter, hvis handlinger er udtryk for en praktisk logik, der er indskrevet i deres kroppe gennem de livsbaner, de har haft, det vil sige de felter de er formet af. Den logik som disse handlinger er udtryk for er derfor feltspecifikke, så for at forstå disse handlinger, må vi afdække den logik de er udtryk for.  Netværksanalysen hjælper os med kortlægge strukturen i handlinger omhandlende jobskift, men vores analyse kræver også at gøre forudsætningerne for disse handlinger forståelige. Det er afgørende at vise de generative mekanismer%
\footnote{Dermed ikke sagt, at der er noget som helst \emph{mekanisk} ved det.}%
%
, der skaber denne struktur. Det vil finde sted i forlængelse af netværksanalysen, og har flere elementer.

%
\begin{description}
  \item[1.] En historisk analyse af feltets%
  %
  \footnote{Man kan diskutere - eller rettere vi \emph{vil} diskutere - om det kan betegnes som et felt. Det er klart at man kan betegne de forskellige institutioner og jobs, der beskæftiger sig med arbejdsløshed, som en del af samme felt, med alle de kampe og regelmæssigher i konstueringen som deraf følger. Men eftersom vi også i høj grad ser på hvad der karakteriserer de arbejdsløse, må man spørge sig om de er en del af et felt som sådan, når de fleste så flygtigt bevæger sig rundt på det?}%
  %
   udvikling og forholdet til andre felter, hvilke aktører der kendetegner feltet og hvilken politiske kampe der foregår. Det betyder også at redegør for for deskriptive sandheder og forstå hvorfor de er blevet til sandheder. Det handler ganske simpelt om antallet af arbejdsløse, økonomiske og juridiske vilkår for arbejdsløse og så videre. Af særlig vigtighed er det at forstå hvorfor disse opfattelser findes, hvem der producerer dem og hvilke formål det tjener for forskellige interessenter. Det indeholder også at forholde sig til andre teoretiske retninger indenfor arbejdsløshesforskningen og undersøgelser om arbejde og arbejdsløshed, for derved at vise hvilke andre fortællinger, der produceres indenfor såvel samfundsvidenskab som i feltet selv, og hvordan de spiller sammen. Det vil give et indtryk af den symbolske konstituering hos forskellige interessenter, samt disses bud på at forstå hvad der kendetegner feltet. Det vil vi gøre ved at læse udgivne forskningsartikler, statitikker fra DST, samt artikler fra diverse fagblade, arbejdsgiverforeninger og politiske grupper som partier og tænketanke%
%
\footnote{Det kan meget hurtigt blive uoverskueligt og der vil ligge en del arbejde i at finde de rigtige dokumenter fra forskellige interessenter, som du f.eks. har gjort ved at se i fagblade fra den tidsperiode du kiggede på.}%
%
\ 
  \item[2.] Inddrage EVS og ESS surveydata koblet op på registerdata, til at sige noget mere om den distinktioner, der adskiller arbejdsløse i form af livssyn og sociale omstændigher. Det vil muligvis gøre os i stand til at konkludere med en vis vægt, at der er tale om vidt forskellige felter, og hvad der så kendetegner disse felter. Det gøres gennem krydstabeller og muligvis en korrespondanceanalyse. Regression kan også komme på tale, hvis det giver mening, men har klart lavest prioritet og skal have en god grund.
  \item[3.] Interviews med mennesker der kan puste liv i vores kvantitative betragtninger, så vi kan forstå de mønstre vi observerer i statistikken som de levede erfaringer de i sidste instans er. Det kunne også være at se på de officielle dokumenter som a-kasser og jobcentre tilbyde til arbejdsløse.
  \item[4.] mere dybdegående inddeling af ledighed i netværksanalysen, hvor vi udnytter at vi har adgang til information om beskæftigelse også internt i året. Uklart præcis hvor brugbare disse oplysninger er.
\end{description}
%

\subsection{Gisninger om resultater} %gisninger er et fedt ord!!

Da moneca-metoden er datadrevet og vi endnu ikke har fået de første netværkskort i hus svarer det til, at vi lige er gået i gang med vores etnografiske feltarbejde. Vi ved altså endnu ikke, hvilke kanaler der bliver etableret, og eftersom vi arbejder med 100+ beskæftigelseskategorier, er det svært at komme med andet end meget rudimentære gæt.

Basalt set mener vi at forskellige jobtyper ligger indenfor forskellige felter, og de sociale agenters habitus er tilpasset disse forskellige felter, så nogle job ligger tættere på nogle typer mennesker end andre. , hvilket kommer til udtryk ved at nogle jobtyper ligger tættere på og længere væk fra hinanden end andre. Det er tæt på at være en truisme, men ved at tænke denne opdeling gennem Bourdieus forståelse af felter og habitus, mener vi at kunne frame disse opdelinger på måder, som gør det muligt at forstå de generative mekanismer bedre. 

Dernæst har vi en antagelse om, at længden på ledighedsperioden siger noget om habitus på arbejdsmarkedsfeltet. I den sammenhæng er det særlig interessant at se på om langtidsledige har større disposition til at bevæge sig længere væk fra de jobtyper der normalt ligger tættere på der hvor man kommer fra før end ledighedsperiode sammenlignet med korttidsledige og arbejdstagere som ikke har oplevet ledighed.

Vender man tilbage til arbejde i samme felt, eller bevæger man sig ind på et nyt? Derved vil vi diskutere, hvilken praksis der hænger sammen med hvilke felter, og hvad det siger om hvilke felter der ligger nær hinanden. Eller måske siger noget om hvor desperat man skal være, for at bevæge sig ud over det felt man er trænet ind i. Vi vil gerne diskutere hvad der strukturerer folk, der oplever arbejdsløsheds opfattelse af handlingsrum, som vi ser det komme til udtryk igennem deres praksis mellem forskellige typer af jobs efter perioder med ledighed.

Vi er desuden interesseret i at diskutere om arbejdstagere som vender tilbage til en anden type arbejde end det de havde i forvejen opretholder arbejdsmarkedets doxa om at arbejde ikke bare er noget man varetager af nødvendighed, men fordi man synes at det er meningsfuldt eller om det sker et brud med arbejdslivets illusio. I den sammenhæng kunne det være interessant at passe på med at betegne det med at komme i en beskæftigelse igen som \emph{at komme videre} eller som en udelukkende positiv ting. For eksempel kan beskæftigelsen personen kommer i være sæsonarbejde eller være et plaster på sået, der gør, at personen inden alt for længe bliver ledighed igen. Derfor kunne det alternativ være bedre eksempelvis som akademiker at vente på \emph{et mere passende job} eller som ufaglært at tage en uddannelse (Arnholz 2004:24). Dette kan ud fra Bourdieu være et udtryk for at fokusere på og komme ud af  \emph{den store elendighed} som arbejdsløsheden og alle de problemer man har som arbejdsløshed (både dem som er forårsaget af arbejdsløshed og dem som ikke er forårsaget af arbejdsløsheden) og derved ignorerer *den lille elendighed* som består i at få forringet sine arbejdsvilkår, blive tvunget til at ofre sig lidt før man bliver ofret og samtidig være taknemmelig over, at man ikke hører til blandt de svage. Hermed tilpasser man sig arbejdsmarkedets umiddelbare behov ved at man som ledige accepterer samfundets objektive strukturer  (Arnholz 2004:27) og får derved nogle realistiske forventninger i forhold til deres position i den sociale verden (*De arbejdsløse må tage hvad de kan få*) (Arnholz 2004: 35).


%%%%%%%%%%%%%%%%%%%%%%%%%%%%%%%%%%%%%%%%%%%%%%%%%%%%%%%%%%%%%%%%%%%%%%%%%%%%%%%%%%%%%%%%%%%%%%%%


\section{Disposition \label{}}
%
 \begin{enumerate} [topsep=6pt,itemsep=-1ex]
   \item Indledning
		 \begin{itemize} [topsep=0pt,itemsep=0ex]
		 	\item Problemformulering	
	 	 \end{itemize}
   \item Analysedel 1: Kortlægning af arbejdsmarkedsfeltet (også metodeafsnit)
   		 \begin{itemize} [topsep=0pt,itemsep=0ex]
		 	\item Vores umiddelbare findings
		 	\item Operationalisering af beskæftigelseskategorier
		 	\item Operationalisering af ledighed
		 	\item Videnskabsteori
	 	 \end{itemize}
   \item Analysedel 2: Subfelt 1 -> faglært/ufaglærte-segment
 		 \begin{itemize} [topsep=0pt,itemsep=0ex]
		 	\item Empirisk analyse
		 	\item Teori og historicering af feltet
		 	\item evt. EVS, ESS, interviews, vejledninger/guides
	 	 \end{itemize}
   \item Analysedel 3: Subfelt 2 -> de 3 akademiker-segmenter
   		 \begin{itemize} [topsep=0pt,itemsep=0ex]
 		 	\item Empirisk analyse
		 	\item Teori og historicering af feltet
		 	\item evt. EVS, ESS, interviews, vejledninger/guides
	 	 \end{itemize}
   \item Diskussion af teorier og teorikritik
   		 \begin{itemize} [topsep=0pt,itemsep=0ex]
   		 	\item Sociologi vs. Økonomi
  	 	 \end{itemize}
   \item Konklusion
 \end{enumerate}
%


%%%%%%%%%%%%%%%%%%%%%%%%%%%%%%%%%%%%%%%%%%%%%%%%%%%%%%%%%%%%%%%%%%%%%%%%%%%%%%%%%%%%%%%%%%%%%%%%


\section{Læsevejledning \label{}}

1) Brugen af monotype skrift for at understrege at der er tale om \texttt{variable} og \texttt{udfald} eller generelle \texttt{kategoriseringer} af data. Formålet er at forholde sig til at vi har at gøre med staten og vores egne statistiske kategorier. Det skulle gerne fremkalde en fremmedgørelseseffekt, som minder læseren om hvad det er vi ser verden igennem, når vi ser verden igennem registerdata, og det er struktureret i forhold et forsknings- og administrativt system. Det er for at vi skal se de briller vi ser med, med andre ord. Et mål er også at nå udover denne administrative tilgang, at give et indtryk af praksis, og derfor vil vi også forsøge at tale mere direkte om de job som kategorierne dækker. så ovenstående forsøg på at forholde sig ærligt til tilgangen skulle gerne vekselvirke med at kunne sige noget om den praksis kategorierne på den ene side afslører med muligheden for at vise (det der står det andet sted om de objektive strukturer), samtidig med den tilslører i samme bevægelse. Læseren må dømme om det er lykkedes blah blah.

2) En komplet liste over vores disco-kategorisering findes i et excelark, der forefindes her. Vi har valgt dette fordi vi ikke mener det ville skabe større overblik at reducere excelarket til et format, der kunne være på et A4-ark. Vi har i stedet skabt et overskueligt dokument, der kan hentes her: http blah blah. En vejledning til at bruge dokumentet findes i bilag XXX (LAV GUIDE PÅ ET TIDSPUNKT)




%Local Variables: 
%mode: latex
%TeX-master: "report"
%End: 