

## d. 23 marts 

Med et speciale om arbejdsløses vej tilbage på arbejdsmarkedet spiller *arbejde* selvfølgelig en afgørende rolle. Ifølge Lars Svendsen findes der inden for europæisk idéhistorie to grundlæggende forskellige opfattelser af hvad arbejde er for en størrelse. Op til reformationen blev arbejde anset som en *meningsløs forbandelse*, og efter reformationen blev arbejdet anset som et *meningsfuldt kald* (Svendsen 2010:13). Det kræver en omfattende social bearbejdning af de kognitive strukturer i individerne, og når vi i dag betegner arbejde som *meningsfuldt kald* et udtryk for arbejdslivets illusio, hvilket vil sige troen på, at arbejdet er en vigtig del af livet. Hvis man som social agent efterlever dette, har man tillært sig arbejdsmarkedets doxa om at arbejde ikke bare er noget man varetager af nødvendighed, men fordi man synes, at det er meningsfuldt i sig selv (Arnholz 2005:21). 


. Dette har vi gjort med inspiration i Bourdieus relationelle metode til at afgrænse individer i klynger efter smag og dispositioner i *Distinktionen* (Bourdieu 1984) og Jonas Toubøl og Anton Grau Larsen kortlægning af klassestrukturer med MONECA-Algoritmen (Toubøl og Larsen).


(inddrage EVS eller ESS surveydata og kvalitative interviews med udvalgte arbejdsløse eller en kvalitative analyse af *guides* og vejledninger til at komme i job*) og ikke mindst inddrage teoretiske forklaringsmodeller. Som primær teoretisk forklaringsmodel inddrager vi Bourdieus begreberne habitus og felt. Bourdieu udvikler begrebet habitus som udtryk for sociale agenters dispositioner som indlejres i den fysiske krop (Bourdieu 1990:56 i Jensen \& Pless 1999; 36). Det betyder, at mennesker skal betrages som socialt strukturerede agenter som hjælper til at forstå, hvorfor vores arbejdstagere har forskellige forudsætninger. I det sociale rum som er arbejdsmarkedsfeltet bliver agenterne placeret ud fra deres habitus.  
%det her skal der arbejdes på


Med udgangspunkt i Niklas Luhmann betyder inklusion, at en person eller grupper anerkendes som deltager i et socialt system og eksklusion betyder det omvendte (Mortensen 2009: 27). 


Derfor ligger ... tættere på ... end ... i det sociale rum. Her forestiller vi os, at uddannelse, indkomst i tidligere jobs, køn og geografi spiller en afgørende rolle.













ARBEJDSLØSHEDENS HISTORISKE UDVIKLING - TRE NEDSLAG
Arbejdsløshedsforsikringsloven, 1907
- Det danske arbejdsløshedsforsikringsystem bygger på den første danske lov om arbejdsløshed i *Lov om anerkendte Arbejdsløshedskasser* fra 1907 (Pedersen 2007: 71). Loven opstår i en tid med et tiltagende pres fra den socialdemokratisk aktivisme, samfundsdebat om arbejdsløshed i almindelighed, oprettelsen af fagforeninger og arbejdsløshedskasser og vedtagelsen af en række socialreformer (Jensen 2007, Pedersen 2007).
- Det danske arbejdsløshedsforsikringsystem bygger på den såkaldte Gent-model efter den belgiske by, hvor systemet blev en realitet som det første sted allerede i 1901 (Due & Madsen 2007: 201). Gent-modellen går på, at staten anerkender og yder tilskud til arbejdsløshedskasser organiseret af forsikringstagere (i praksis fagbevægelsen), og at det for det enkelte individ er frivilligt om denne vil forsikre sig mod arbejdsløshed, som det er tilfældet i Danmark, Severige og Finland. I modsætning her til står den obligatoriske forsikringsform som går på, at det for alle lønmodtagere er obligatorisk at være forsikret mod arbejdsløshed i et statsligt organiseret forsikringssystem, som er tilfældet i eksempelvis Italien og Storbrittanien (Jensen 2007: 33f).
- Arbejdsløshed er et begreb som opstår i takt med at arbejdsstyrken vokser, hvor urbaniseringen resulterer i, at der i Danmark opstår et egentlig arbejdsmarked, hvor et stigende antal lønarbejdere udbød deres arbejdskraft i en etableret pengeøkonomi. Indvaliditets- og Alderdomsforsikringskommisionenen, som var den kommusion som fik til at opgave at undersøge mulighederne for etablering af en arbejdsløshedsforsikring, kom til enighed om, at arbejdsløshed var et socialt onde, der ramte arbejdere uden at de selv var skyld i det og anså sæsonarbejdsløshed, konjunkturarbejdsløshed og tilfældig arbejdsløshed som de tre hovedårsager til arbejdsløshed (Pedersen 2007: 67f).
Reformen, 1967-70
- Selvom der mellem arbejdsløshedsforsikringslovens oprettelse i 1907 er foretaget op mod 200 ændringer (Huulgaard 2007: 123) som eksempelvis en række tiltag under krisen i starten af 1930'erne, så foretages den første vigtige kontinuitetsbrud med reformen 1967-70 (Pedersen 2007: 83).
- Wechselmann-udvalget efter formanden, amt- og forligsmand Sigurd Wechselmann fik i 1964 til opgave at komme med et forslag til en revision af arbejdsløshedsforsikringen. Deres forslag blev i sort set uændret form gennemført i 1970. Som noget afgørende nyt overtog staten den marginale risiko ved ledighedsstigninger, hvilket vil sige, at staten fuldt ud betalte medudgifter ved stigende ledighed, hvilket få år efter med oliekrisen blev særdeles mærkbart. Derudover etables for første gang en samlet offentlig arbejdsanvisning *Arbejdsformidlingen*, som skulle formidle både ledige jobs og ledig arbejdskraft.  
Indførelsen af tværfaglige a-kasser, 2002 
- I perioden efterfølgende sker der også en række væsentlige tiltag for at modvirke stigningen i ledigheden fra oliekrisen til midten af 190'erne herunder i særdeleshed indførelse af efterlønsordningen i 1979. I 2002 sker der dog det andet deciderede kontinuitetsbrud med den hidtidige linje i arbejdsløshedsforsikring med vedtagelsen af lovforslaget om mulighed for at oprette flere tværfaglige a-kasser. Ind til da havde kun Kristelig A-kasse været tværfaglig. Formålet var at fremme fleksibilitet på arbejdsmarkedet og konkurrence på effektivitet, serviceniveau og pris (Pedersen 2007: 97).

ARBEJDSLØSHED SOM ØKONOMISK PROBLEM - TRE TEORIER DOMINERER
- Jørgen Goul Andersen identificerer et skifte inden for en bred strømning af økonomisk teori fra slutningen af 1970'erne, hvor velfærdsstaten blev anskuet som et middel at afbøde *markedsfejl* til i stigende grad at fokusere på *politikfejl* og *forvridninger* på markedet, hvilket kan karakteriseres som et skifte fra efterspørgselsiden til udbudssiden og et skifte fra makro til mikro. Økonomen Milton Friedman udviklede allerdede i slutningen af 1960'erne begrebet *naturlig arbejdsløshed* og mulighed for at den forskydes opaf, hvis løndannelsen ikke foregik i sammenhæng med modellen for fri konkurrence (Goul Andersen 2003: 19).
Søgeteorien
- Søgeteorien blev grundlagt i 1960'erne og kendte teoretikere er John J. McCall, George Stigler, Peter Diamon, Dalte T. Mortensen og Chrisstopher A. Pissarides og den basale søgemodel handler om arbejdsløse som søger beskæftigelse.
- Forestil en arbejdsløs som af og til  får jobtilbud. Den arbejdsløse kender først lønnen på jobtilbuddet, når det modtages. Efter at have modtaget tilbudet, skal den arbejdsløse beslutter sig for, om tilbuddet accepteres eller afslås. Dette gør den arbejdsløse på baggrund af en reservationsløn, som er fastsat af forventningerne til lønnen og viden om lønfordelingerne på  arbejdsmarkedet. Hvis det modtagne tilbudd er større end reservationslønnen accepteres tilbuddet, og ellers afslås det (Rosholm 2009: 159f).
- Marginalisering kan i denne sammenhæng forklares med, at den arbejdsløses kvalifikationer bliver mindre værd efterhånden som ledighedsperioden bliver længere, at den arbejdsløse gradvist mister forbindelsen til gamle kolleger eller at den arbejdsløse dømmes på baggrund af sin langtidsledighed (Rosholm 2009: 160f, Goul Andersen 2003: 20).
Matching-teorien
- Matching-teorien deler arbejdsmarkedet op i to typer agenter: arbejdsgivere og lønmodtagere og er kendt for teoretikere som Dale T. Mortensen, Christopher A. Pissarides Peter A. Diamond, Alvin E. Roth and Lloyd Shapley.
- Her leder lønmodtagerne efter job i de perioder, hvor de er arbejdsløse, mens arbjdsgivere slår stillinger op, så længe de vurderer, at det kan betale sig. Ledige lønmodtagere og job mødes på arbejdsmarkedet i en matching-proces. Når der skabes kontakt mellem en arbejdsgiver og lønmodtagere, opstår der en forhandling mellem arbejdsgiver og lønmodtager om fordelingen af overskuddet i en eventuel ansættelse. Arbejdsgiveren kan på den ene side presse lønmodtageren til at acceptere en løn, som ligger under arbejdskraftens marginalprodukt, fordi lønmodtageren ikke uden videre kan finde nyt job på anden vis end ved at vente på det næste jobtilbud. Lønmodtageren kalkulerer på baggrund af forskellen mellem den tilbudte løn og værdien af at være ledig (*outside option*).
- Matching-teoren kan bruges til at analyse den umiddelbare effekt af at ændre i ledighedsydeleser som eksempelvis dagpenge eller kontakthjælp (Rosholm 2009: 162f).
Insider-outsider-teorien
- Insider-outsider-terien er udviklet af Assar Lindbeck og Dennis Snower i 1980'erne og fokuserer på omkomstningerne forbundet med ansættelser og afskedigelser af arbejdskraft.
- Omkomstningerne kan være i forbindelse med søge- og optræning, afskedigelse, uproduktiv konkurrence mellem to grupper af lønmodtagere. Insidere/outsidere kan blandt andet defineres som beskæftige/arbejdsløse, fagforeningsmedlemmer/ikke-medlemmer, ansatte i gode jobs/dårlige jobs. Insidere vil forsøge at forhandle sig til så høje lønninger som muligt og afholde andre for at underbyde dem på markedet. Insidernes magt består blandt andet i, at arbejdsgiverne har omkostninger vil at afskedige dem og kan skabe omkostninger ved strejke, aktioner og mobning af nyansatte.
- Langstidsledige, i modsætning til beskæftige og korttidsledige, kan tilbyde sin arbejdskraft til reduceret løn og forsøge at overvbevise en arbejdsgiver om at blive ansat så længere arbejdsgiverens omkostninger ved at ansætte en outsider ikke stiger gevensten ved at gøre det (Rosholm 2009: 164, Goul Andersen 2003: 20).

ARBEJDSLØSHEDENS SOCIALE KONSEKVENSER - TRE PERSPEKTIVER
Deprivation
- Deprivationsteorien stammer fra det klassiske Marientha-studie af de sociale konsekvenser af arbejdsløshed i et lille samfund gennemført af Marie Jahoda i samarbejde med Paul Lazarsfeld og Hans Zeizel. 
- Marienthal var et industriby som led af høj arbejdsløshed i 1920'erne, og studiet undersøger hvad der sker med arbejderne i den østrigske by Marienthal, når de oplever arbejdsløshed. Hovedargumentet i deprivationsperspektivet er, at arbejdsløshed medfører social eksklusion og isolation, tab af struktur i hverdagen og selvtillid og en betydelig øget risiko for psykiske problemer. Det antages at deltagelse på arbejdsmarkedet opfylder både et psykologisk behov for individet og et økonomisk behov for indtægt. Uknækket vilje, resignation, fortvivlelse og apati som fire stadier eller reaktioner den arbejdsløse gennemgår.  Arbejdsdeltagelsen har ifølge Jahoda fem funktioner: tidsmæssig struktur i dagligdagen, sociale kontakter, deltagelse i kollektive formål, status og identitet og regelmæssig aktivitet (Nørup 2012:23f).
Mestring 
- Knut Halvorsen udvikler i sit forfatterskab en modpol til Jahoda, Lazarsfeld og Zeizel og deres efterfølgere (fx Eisenberg og Lazarsfeld 1938; Tiffany, Cowan & Tiffany 1970; Warrs 1987).
- I stedet for et passive og ensartede individperspektiv, betragter Halvorsen arbejdsløse som forskelligartede og handlende aktører, der kan påvirke og forandre deres situation i stedet for at være ofre for omstændighederne. Den arbejdsløse vil indgå i forskellige fysiske, psykiske og sociale aktiviteter for at afbøde effekter af arbejdsløshed og minimere stress og mental belastning (Nørup:28)
Sociale passager
- Barney Glaser og Anselm Strauss definerer en status passage som et individs ”movement into a different part of a social structure, or loss or gain of privilige, influence, or power, and changed behaviour”. 
- I den sammenhæng kan arbejdsløshed anskues som  exit fra arbejdsmarkedet frem for en enten-eller tilstand og sammenlignes med andre status passage som eksempelvis skilsmisse, sygdom eller dødsfald i familien. En sådan passage fører ikke nødvendigvis i sig selv til mentale problemer, mistrivsel eller eksklusion, da det afhænger af den enkeltes identitet og selvopfattelse i relation til andre og samfundet og en lang række andre faktorer samt om passagen efterfølges af en reintegrativ passage, hvor den enkelte får en ny status eksempelvis et nyt job (Nørup:31).




GAMMEL PROBLEMFORMULERING:
Hvordan kan man forstå ledighed gennem beskæftigelsesmobilitet på arbejdsmarkedet,  og hvordan kan vi se arbejdsløshed som en mangfoldighed af praktikker alt efter hvilke sociale ressourcer man har tilrådighed?
- Hvordan forstår vi ledighed? (begreb - historisk/teoretisk/krydstabeller/MCA/interviews)
- Hvordan bevæger ledige sig ind og ud af arbejdsmarkedet? (praksis - MONECA)
- Hvilke ressourcer kan man se komme til udtryk i bevægelser ind og ud af arbejdsmarkedet? (ressourcer - MONECA/krydstabeller/MCA/interviews)
- Hvordan kommer lediges strategier, praksis og ressourcer til udtryk for forskellige typer af ledige? 

KONKLUSION
1) Personer med en ledighedsperiode ender hverken altid i den samme jobtype eller bevæger sig fuldstændig tilfældigt imellem forskellige jobtyper. I stedet for bevæger de sig imellem bestemte jobtyper, hvilket kommer til udtryk i en række clusters. Når man fokuserer på jobtype x1 (fx sociolog), kan man se det denne jobtype ligger i cluster y1 (administration), som er karakteriseret ved karaktertræk z1 (administration, regnskab og offentligt arbejde mv.). På trods af, at jobtype x1 (sociolog) ligger i cluster y1 (administration), er der også en række personer som bevæger sig mod jobtype x2 (rengøringspersonale) og jobtype x3 (kommunikation) som ligger i hhv. cluster y2 (service) og cluster y3 (kommunikation og ledelse).
2) 
3) Konsekvenserne af ledighed for beskæftigelsesmobilitet på arbejdsmarkedet er, at personer med bestemte jobtyper (fx x1), vælger efter en periode med ledighed, får jobs i inden for jobtyper, som ligger uden for deres cluster inden for andre jobtyper/clusters, som er karakteriseret ved en anden type jobs og en anden type lønninger.
4) Metodisk og teoretisk en anden måde at anskue ledighed på en den gængse historie, sociologi og økonomi.

KRITIK AF PROBLEMFORMULERING
- Hvad er sammenhængen mellem ledighed og det at være på kanten af arbejdsmarkedet? Hvordan produceres de socialt udsatte?
- Afdækker vi de generative mekanismer der skaber praksis eller viser vi bare et aspekt og kalder det for praksis?
- Handler det mere om metode end om emnet?




