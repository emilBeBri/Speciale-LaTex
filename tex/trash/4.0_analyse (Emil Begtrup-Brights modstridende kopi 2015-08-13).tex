% -*- coding: utf-8 -*-
% !TeX encoding = UTF-8
% !TeX root = ../report.tex


%%%%%%%%%%%%%%%%%%%%%%%%%%%%%%%%%%%%%%%%%%%%%%%%%%%%%%%%%%%
\chapter{\textsc{Analyse} \label{analyse}}
%%%%%%%%%%%%%%%%%%%%%%%%%%%%%%%%%%%%%%%%%%%%%%%%%%%%%%%%%%%

\begin{table}[h] 
\begin{tabular}{@{}||l||l||@{}} \hline \hline 
 4.	 & \textbf{Analyse} \\ \hline \hline
 4.1 & \textbf{...} \\ 
 	 & \textit{ - ...} \\
	 & \textit{ - ...} \\ \hline \hline
\end{tabular} \end{table}


%%%%%%%%%%%%%%%%%%%%%%%%%%%%%%%%%%%%%%%%%%%%%%%%%%%%%%%%%%%
% Trash
%%%%%%%%%%%%%%%%%%%%%%%%%%%%%%%%%%%%%%%%%%%%%%%%%%%%%%%%%%%

% I dette analysekapitel vil vi gennemgå hovedkortet og en række bikort.

% For hovedkortet vil vi gennemgå først gennemgå, hvordan man aflæser et Moneca-netværkskort. Dernæst vil beskrive, hvad der er på spil i hovedkortet ved første øjekost. Derefter vil vi gennemgå de forskellige segmenter i hovedkortet. Efterfølgende viser vi hovedkort med nodernes interne mobilitet, længden på arbejdsløshedsperioder og antallet af arbejdsløshedsperioder.

% Til sidst vil vi gennemgå en række bikort, herunder DST's definition af arbejdsløse som nettoledige, vores definition af arbejdsløshed med minimum seks ugers ledighed og alle beskæftigede på arbejdsmarkedet.

%%%%%%%%%%%%%%%%%%%%%%%%%%%%%%%%%%%%%%%%%%%%%%%%%%%%%%%%%%%

%Local Variables: 
%mode: latex
%TeX-master: "report"
%End: