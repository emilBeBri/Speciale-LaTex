% -*- coding: utf-8 -*-
% !TeX encoding = UTF-8
% !TeX root = ../report.tex


%%%%%%%%%%%%%%%%%%%%%%%%%%%%%%%%%%%%%%%%%%%%%%%%%%%%%%%%%%%
\newpage \section{\textsc{Arbejdsløshed \label{teori_arbejdsloeshed}}}
%%%%%%%%%%%%%%%%%%%%%%%%%%%%%%%%%%%%%%%%%%%%%%%%%%%%%%%%%%%

Formålet med dette teoretiske afsnit om arbejdsløshed er at kontekstualisere arbejdsløshed som problemstilling og forskningsfelt. Denne kontekstualisering bliver anvendt i vores operationalisering af arbejdsløsheds påvirkning på mobilitet i afsnit \ref{teori_operationalisering}. Dette teoretiske afsnit om arbejdsløshed indleder med at beskrive arbejdsløshed som problemstilling. Efterfølgende beskrives økonomiske forståelser af arbejdsløshedsproblemet. Bagefter følger en beskrivelse af sociologiske forståelser af arbejdsløshedsproblemet.  Afslutningsvis foreligger en afsluttende opsamling.








%%%%%%%%%%%%%%%%%%%%%%%%%%%%%%%%%%%%%%%%%%%%%%%%%%%%%%%%%%%
\subsection{\textsc{Arbejdsløshed som problemstilling}}
%%%%%%%%%%%%%%%%%%%%%%%%%%%%%%%%%%%%%%%%%%%%%%%%%%%%%%%%%%%

C. Wright Mills konstaterer: “No problem can be adequately formulated unless the values involved and the apparent threat to them are stated.” \textbf{\parencite[129]{Mills1959}}. Med inspiration fra Mills vil vi gribe arbejdsløshed som problemstilling ved at kontekstualisere kontekstualisere arbejdsløshed som et socialt fænomen med fokus på hvilke værdier, som er involverede og truslerne herimod. Først vil vi beskrive en kort historik af arbejdsløshed som et socialt fænomen. Derefter vil vi beskrive dagens syn på arbejdsløshed, herunder beskrive de dominerende diskurser forbundet med arbejde og arbejdsløshed. %%%% Emil: Uddyb citatet. Det kunne være bedre med et Bourdieu-citat


%%%%%%%%%%%%%%%%%%%%%%%%%%%%%%%%%%%%%%%%%%%%%%%%%%%%%%%%%%%
\subsubsection{Kort historik over arbejdsløshed i et dansk perspektiv}

Arbejdsløshed anskues i dag som et socialt problem. Som begreb kom arbejdsløshed dog først til verden i løbet af det 19. århundrede. I dette århundrede blev arbejdsløshed ligeledes også et adskilt fænomen fra fattigdom \textbf{\parencite[3]{Halvorsen1999}}. I Danmark blev det op igennem 1800-tallet og omkring århundredeskiftet en dominerende tanke at anskue arbejdsløshed som et kollektivt og statsligt anliggende, som delvist var forårsaget af forhold arbejdstagerne ikke havde kontrol over. Dette bliver slået fast i 1907 med den første danske arbejdsløshedsforsikringslov\footnote{Hermed blev arbejdsløshedskasserne, staten og kommunerne de centrale aktører i  danske arbejdsløshedsforsikringssystem. Dette kendetegner den såkaldte Gent-model, hvor staten anerkender og yder tilskud til arbejdsløshedskasser organiseret af forsikringstagere (i praksis fagbevægelsen), og at det for det enkelte individ er frivilligt om denne vil forsikre sig mod arbejdsløshed \parencite{Jensen2007a}.}, som blev vedtaget med et bredt flertal i Landstinget og Folketinget. Baggrunden herfor var en kommission, som anbefalede, at samfundet måtte træde ind over for arbejdsløshed, fordi kommissionen kunne konstatere, at arbejdsløshed var et socialt onde, som ramte arbejdstagerne uden, at de havde skyld heri \parencite[69]{Pedersen2007}. Med indførelsen af arbejdsløshedsforsikringen får den danske velfærdsstat som rolle at administrere arbejdsløshed som et socialt problem. Det var dog først i 1970, at staten overtog den primære økonomiske byrde i arbejdsløshedsforsikringen gik fra at være en primært privat forsikringsordning til at være en primært statsfinansieret velfærdsordning \parencite[83]{Pedersen2007}. Fra 1907 til 1970 blev de arbejdsløses vilkår løbende styrket, men beskæftigelseskrisen som fremgår af figur \ref{fig_udvikl.arbejdsloeshed} fra 1970'erne og frem til midten af 1990'erne medfører en mere aktiv arbejdsmarkedspolitik end tidligere, der blandt andet indebar løbende besparelser. %%%% Emil: Kan fyldes en smule mere på. Måske Foucault hevisnng til dannelse af befolkning. Kan suppleres med Hornemann Møller som har gode inddelinger. Figuren skal forklares noget mere.
% 
\begin{figure}[H]
\begin{centering}
	\caption{Udviklingen i arbejdsløshedsprocenten: Kilde AE-Rådet}
	\includegraphics[width=0.8\textwidth]{fig/teori/historiskudvikling.png}

	\footnotesize{Bruttoledigheden opgør Danmarks Statistik kun tilbage til 2007. Finansministeriet har dog foretaget en tilbageføring til 1996, som er anvendt til figuren. Kilde: AE på baggrund af Danmarks Statistik, Finansministeriet og OECD \parencite[2]{Bjoersted2012}.}
	\label{fig_udvikl.arbejdsloeshed}
\end{centering}
\end{figure}
% 
Eksempelvis indføres efterlønsordningen i 1977 under Anker Jørgensens socialdemokratiske regering \parencite[86]{Pedersen2007}, dagpengesatsen fastfryses fra 1982 til 1986 under Poul Schlüters firkløverregering \parencite[88]{Pedersen2007}, vægten på ret, pligt og individuel behovsorientering øges i 1993 under Poul Nyrup Rasmussens socialdemokratisk ledede regering \parencite[92]{Pedersen2007}, muligheden for tværfaglige a-kasser oprettes for at øge konkurrencen under Anders Fogh Rasmussens VKO-regering \parencite[97]{Pedersen2007} og senest dagpengereformen fra 2010, hvor dagpengeperioden blev halveret fra 4 til 2 år og genoptjeningspligten fordobles fra 26 til 52 uger \parencite{lov_dagpenge}. %%% Søren: Henvisningen til Halvorsen er god nok, men man kunne godt overveje at anvende en anden eller flere andre henvisninger fx. Bauman

Ifølge Keane og Owens er udviklingen af velfærdsstaten i Danmark og andre lande bygget på et normativt grundsyn om at alle som udgangspunkt skal forsørge sig selv gennem et arbejde \textbf{\parencite[18]{Keane1986}}. Lønarbejdet bidrager til at sikre social integration blandt velfærdsstatens medlemmer. Og velfærdsstatens aktive arbejdsmarkedspolitik er med til at opretholde en vis levestandard for dem, som ikke har et arbejde samtidig med at benytte en “gulerod” til at få folk i arbejde gennem økonomiske incitamenter \textbf{\parencite[7]{Halvorsen1999}}. %%% Henvisningerne til Keane og Halvorsen er gode nok, men man kunne godt overveje at anvende en andre henvisninger fx. nogle fra FAOS


%%%%%%%%%%%%%%%%%%%%%%%%%%%%%%%%%%%%%%%%%%%%%%%%%%%%%%%%%%%
\subsubsection{Dagens syn på arbejdsløshed i et dansk perspektiv} 

Arbejdsløshed regnes ifølge Halvorsen for en af nutidens største udfordringer for velfærdsstaten både nationalt og internationalt \textbf{\parencite[8]{Halvorsen1999}}. Som socialt problem indgår arbejdsløshed i diskurser i forbindelse med arbejdets betydning og i den forbindelse også betydning af \textit{fravær} af arbejde. De diskurser, som er knyttet til arbejdsløshedsfænomenet er både med til at påvirke, hvordan arbejdsløse klassificeres, og hvordan arbejdsløse forstår dem selv og deres situation \textbf{\parencite[12]{Halvorsen1999}}. %%% Henvisningen til Halvorsen er god nok, men man kunne godt overveje at anvende en anden eller flere andre henvisninger fx. Bauman

Halvorsen skelner mellem tre forskellige diskurser om lønarbejde \parencite[13]{Halvorsen1999}. Den første diskurs knytter sig til retten til arbejde, hvor lønarbejdet både er lig med selvrealisering og er en forudsætning for, at man kan fungere som en god samfundsborger\footnote{Lars Svendsen skelner inden for den europæiske idéhistorie mellem to grundlæggende forskellige arbejdsopfattelser. Indtil reformationen blev arbejdet anset som en \textit{meningsløs forbandelse}, og efter reformationen blev arbejdet anset som et \textit{meningsfyldt kald} \parencite[13]{Svendsen2008}. I det moderne samfund beskriver Bauman, at arbejdet bliver æstetisk, fordi den enkelte eksempelvis skal kunne identificere sig med sit arbejde eller at ens arbejde skal være autentisk \textbf{\parencite[169-215]{Baum2006}}.}. Den anden diskurs knytter sig til arbejdspligt, hvor lønarbejde er lig med den grundlæggende værdiskabende aktivitet i samfundet\footnote{Efter anden verdenskrig gik velfærdsstaterne ind i en ny historisk fase, hvor regeringerne forsøgte at skaffe fuldtidsjobs til alle voksne igennem en politik som havde til formål for det første at stimulere privat og offentlig vækst \parencite[17]{Keane1986}.}. Den tredje diskurs knytter sig ligeledes til arbejdspligt, hvor lønarbejdet er lig med et nødvendigt onde, som er nødvendigt for at få samfundet til at fungere og et onde, fordi det enkelte individ er tvunget til at arbejde\footnote{Den dominerende opfattelse af arbejdet som et \textit{meningsløs forbandelse} \parencite[13]{Svendsen2008} kan stadigvæk siges at være gældende i dag (\textbf{henvisning mangler}).}.

Halvorsen skelner mellem tre ækvivalente arbejdsløshedsdiskurser \parencite[13]{Halvorsen1999}. \textit{Elendighedsdiskursen} handler om, at arbejdsløshed er lig med social død. Utallige historier i medierne knytter sig til denne diskurs med overskrifter som for eksempel: “Knæk. Arbejdsløshed rammer hele familien” (Politiken, 19.04.2013), “Arbejdsløse rammes af stress” (Politiken, 18.07.2010), “Ekspert: Unge arbejdsløse risikerer ar mange år frem” (Berlingske Tidende, 26.11.2010) og “Arbejdsløse frygter aldrig at finde job igen” (Politiken, 01.01.2011). \textit{Beskæftigelsesdiskursen} handler om, at arbejdsløshed er lig med sløseri med ressourcer. Dette fylder også en del i finansnyhederne, som eksempelvis “Arbejdsløse koster kassen” (Ekstra Bladet, 14.11.2008), “Ledighed sender folk i sygesengen” (Jyllands-Posten, 18.03.2013) og “Høj ledighed truer EUs økonomi” (Berlingske, 03.07.2014). \textit{Moraldiskursen} handler om, at arbejdsløshed skyldes dovenskab og manglende arbejdsmotivation. Denne diskurs har fyldt meget i mediedebatten med overskrifter som “Joachim B. til arbejdsløs: Du er for slap” (Politiken.dk, 19.04.2012), “Dovne Robert på kontanthjælp i 11 år: Hellere kontanthjælp end et lortejob” (Ekstra Bladet, 11.09.2012) og “Vi arbejdsløse bliver opfattet som dumme, dovne og dårlige mødre” (Politiken, 06.11.2014). Alle tre diskurser er fælles om, at arbejdsløshed anskues som et onde. Den danske realpolitik er domineret af de to sidstnævnte diskurser ved, at arbejdsløse mødes med de “økonomiske realiteter” eller “nødvendighedens politik” (\textbf{henvisning mangler}) med den føromtalte dagpengereform fra 2010 samtidig med, at denne reform og kommende reformer bakkes op af udsagn som “Det skal kunne betale sig at arbejde” \parencite{Stoejberg2015}. %%%% Emil: kort analyse hvem/hvad diskurserne tjener. Der mangler en Enhedslisten/venstrefløjdiskurs. Skrives mere ud.


%%%%%%%%%%%%%%%%%%%%%%%%%%%%%%%%%%%%%%%%%%%%%%%%%%%%%%%%%%%
\subsubsection{Opsummering} 

I dette afsnit har vi kontekstualiseret arbejdsløshed som problemstilling ved kort at skitsere arbejdsløshed i en historisk og en nutidig kontekst i et dansk perspektiv. Historisk er danske arbejdsløses forhold løbende blevet styrket i perioden fra den første lov om arbejdsløse i 1907 til 1970'erne. Siden har arbejdsløses forhold i højere grad været til debat og under under pres. Ifølge Halvorsen dominerer diskurserne “arbejdsløshed som social død” (elendighedsdiskursen), “arbejdsløshed som sløseri med ressourcer” (beskæftigelsesdiskursen) og “arbejdsløshed som dovenskab” (moraldiskursen) i større eller mindre grad synet på arbejdsløshed i hverdagen og den offentlige debat. Disse diskurser knytter sig ligeledes til de forskellige videnskabsdiscipliner, hvilket vi vil komme ind på i afsnittene om henholdsvis økonomiske og sociologiske forståelser af arbejdsløshed.








%%%%%%%%%%%%%%%%%%%%%%%%%%%%%%%%%%%%%%%%%%%%%%%%%%%%%%%%%%%
\newpage \subsection{\textsc{Økonomiske forståelser af arbejdsløshedsproblemet}}
%%%%%%%%%%%%%%%%%%%%%%%%%%%%%%%%%%%%%%%%%%%%%%%%%%%%%%%%%%%

Fra 1970'erne og fremefter har økonomisk teori haft en betydelig gennemslagskraft i arbejdsløshedsforskning samtidig med at være det ideologiske grundlag for de lovændringer på arbejdsløshedsområdet, som har været med til at sænke arbejdsløshedsydelserne og været med til at kontrollere de arbejdsløses bestræbelser på at få et arbejde i Danmark jævnfør \ref{teori_arbejdsloeshed} \parencite[19]{Andersen2003} \parencite[1679]{Atkinson1991}. De økonomiske modeller er orienteret mod at sikre en effektiv økonomi og et effektivt fungerende arbejdsmarked. Disse modeltyper som økonomerne benytter sig af svarer til \textit{covering law-modellen}\footnote{Carl Hempel udviklede covering law-modellen for at forstå naturvidenskabelige forklaringer \parencite[15]{Hedstroem2005}.}, hvor et foreliggende faktum forklares ud fra andre udsagn, herunder minimum en almen lov, som det pågældende faktum derfor er underordnet. Det vil sige, at for at forklare et faktum, bygger økonomerne oftest deres modeller på en eller flere antagelser.

Dette teoretiske afsnit om økonomiske forståelser af arbejdsløshed indeholder først en definition af arbejdsløshed. Herefter følger arbejdsløshed på lang sigt og modellerne trade-off mellem arbejde og fritid, basal jobsøgningsteori og principal-agent-modellen  Afslutningsvis følger arbejdsløshed på kort sigt samt modellen hysterese.


%%%%%%%%%%%%%%%%%%%%%%%%%%%%%%%%%%%%%%%%%%%%%%%%%%%%%%%%%%%
\subsubsection{Definition på arbejdsløshed} %%% Emil: vigtigt afsnit, men det er alt for indforstået

Arbejdsløshed defineres typisk inden for den økonomiske disciplin ud fra et anvendelsesorienteret sigte. Den ene af de to mest udbredte definitioner af arbejdsløshed kommer af, at det er muligt at måle antallet af individer på arbejdsløshedsunderstøttelse på et givet tidspunkt i løbet af året \parencite[594]{Mankiw2011}. Den anden definition anvendes i arbejdsmarkedsundersøgelser, stammer fra \textit{International Labour Organisation} og består af antallet af personer som står uden beskæftigelse samtidig med at være til rådighed for arbejdsmarkedet og aktivt arbejdssøgende \parencite{ILO1982}. Fordelen ved at anvende arbejdsmarkedsundersøgelser til at måle arbejdsløshed kontra at måle antallet af individer på arbejdsmarkedsunderstøttelse er, at lovændringer kan resultere i ændringer i, hvem som har ret til arbejdsløshedsunderstøttelse. %%% Emil: Uddybes

De arbejdsløse er sammen med beskæftigede per definition arbejdsstyrken, mens den resterende ikke økonomisk aktive del  befolkningen i den arbejdsdygtige alder per definition står uden for arbejdsstyrken\footnote{Denne opdeling anvendes til at udregne arbejdsløshedsraten, som er lig med procentdelen af arbejdsstyrken som er arbejdsløse, og arbejdsmarkedsdeltagelsesraten, som er lig med procentdelen af den befolkningen i den arbejdsdygtige alder som er en del af arbejdsstyrken. \parencite[595]{Mankiw2011}. Forholdet mellem arbejdsløshed og beskæftigelse er kompleks, fordi det er muligt for arbejdsløshedsraten at falde samtidig med at beskæftigelsen ikke stiger, hvis for eksempelvis arbejdsstyrken skrumper \parencite[449]{Cahuc2004}.}. Atkinson har kritiseret de fleste arbejdsmarkedsmodeller for at antage, at arbejdsløshedstilstande både starter og slutter med beskæftigelse uden at tage højde for at de i mange tilfælde både starter og slutter uden for arbejdsstyrken \parencite[1683]{Atkinson1991}. Atkinson er endvidere kritisk over for at behandle beskæftigede, arbejdsløse og personer uden for arbejdsstyrken som homogene kategorier \parencite[1683]{Atkinson1991}. Når man skelner mellen arbejdsløse og personer uden for arbejdsstyrken, kan man have dem som er aktivt jobsøgende, men som ikke modtager eller ikke har ret til arbejdsløshedsydelse. Omvendt kan man også have dem, som modtager arbejdsløshedsydelser, men som ikke er aktivt jobsøgende. Som et tredje eksempel er der også de såkaldte “modløse arbejdere”, som hverken søger job eller modtager arbejdsløshedsydelser\footnote{Beskæftigede er også heterogen kategori, hvor man kan komme i beskæftigelse efter at have været arbejdsløs som selvstændig eller lønmodtager, fuldtid eller deltid og hvad Atkinson kalder almindelige eller marginale jobs \parencite[1685]{Atkinson1991}, hvilket minder om Guy Standings definition af prækære jobs \parencite[168]{Standing2011}.}. %%% svært at forstå

Fuld beskæftigelse anvendes i forbindelse med et konkurrencedygtigt arbejdsmarked, hvor lønnen er i equilibirum samtidig med at være lig med værdien af arbejdskraftens marginalprodukt. Equilibrium modellen, som dette udsagn også kaldes, hænger sammen med de grundlæggende antagelser om \textit{udbud}, \textit{efterspørgsel} og \textit{arbejdskraftens marginalprodukt}. På arbejdsmarkedet er arbejdskraft, jord og kapital de inputs som anvendes til produktion af varer og ydelser. Efterspørgslen på arbejdskraft bygger på antagelserne om, at virksomheder er \textit{konkurrencedygtige} og \textit{profitmaksimerende} \parencite[383]{Mankiw2011}. Når den konkurrencedygtige virksomhed skal ansætte en person, skal der tages højde for antallet af ansattes påvirkning af produktionen af varer. Hvis arbejdskraftens marginalprodukt er profitabelt, kan det betale sig for virksomheden at ansætte en ny person \parencite[384]{Mankiw2011}. Når arbejdstagere er ansat til at arbejde til en løn i perfekt balance mellem udbud og efterspørgsel samtidig med, at lønnen er lig med værdien af arbejdskraftens marginalprodukt, så vil der være fuld beskæftigelse. Da fuld beskæftigelse reelt ikke eksisterer, skelner økonomiske arbejdsløshedsstudier mellem arbejdsløshed på lang sigt og arbejdsløshed på kort sigt, som vi henholdsvis vil beskæftige os med i det kommende to afsnit. %%% Emil: Forklar equilibrium + hvad betyder "Lønnen er lig med værdien af arbejdskraftens marginalprodukt"


%%%%%%%%%%%%%%%%%%%%%%%%%%%%%%%%%%%%%%%%%%%%%%%%%%%%%%%%%%%
\subsubsection{Arbejdsløshed på lang sigt}

Arbejdsløshed på lang sigt kaldes også den naturlige arbejdsløshedsrate, som er lig med den mængde arbejdsløshed en økonomi normalt er udsat for \parencite[592]{Mankiw2011}. På lang sigt skyldes arbejdsløshed friktionsledighed, strukturel lighed og sæsonledighed\footnote{I praksis er det vanskeligt at udskille friktions- og sæsonledighed fra strukturledighed, hvilket betyder, at de alle sammen oftest optræder som strukturarbejdsløshed \parencite{2015}.}. %%%% Emil: hvad betyder normalt
\textit{Friktionsledighed} som også kaldes skifteledighed opstår ofte i forbindelse med indtræden på arbejdsmarkedet eller ved jobskifte. Det tager tid at matche den rette arbejdstager til det rette job, fordi arbejdstagerne blandt andet har forskellige præferencer og færdigheder og jobs kræver forskellige færdigheder\footnote{Det bryder med equilibrium-modellen som antager, at alle arbejdstagere kan passe hvilket som helst jobs, fordi alle arbejdstagere og alle jobs er identiske. Hvis dette var sandt, og arbejdsmarkedet var i equilibrium, så ville et jobtab ikke medfører arbejdsløshed, fordi en fyret arbejdstager ville finde et nyt job til markedsløn \parencite[163]{Mankiw2007}.} \parencite[163]{Mankiw2007}.
\textit{Strukturel ledighed} opstår ved manglende faglig eller geografisk fleksibilitet på arbejdsmarkedet. Dette betyder, at lønnens balance mellem udbud og efterspørgsel ikke er i stand til at sikre fuld beskæftigelse\footnote{Dette bryder ligeledes med equilibrium-modellen, hvor de reelle lønninger tilpasser sig i perfekt balance med udbud og efterspørgsel. Men lønninger er ikke altid fleksible, fordi de reelle lønninger kan være fastsat over markedsniveau. Denne lønstivhed medfører strukturel arbejdsløshed, fordi den udbudte mængde af arbejdskraft er større end den efterspurgte mængde af arbejdskraft. Årsagerne her til er blandt andet minimumslønninger, fagforeningerne og effektivitetslønninger, som har til fælles at skabe lønninger over equilibrium \parencite[165]{Mankiw2007}.} \parencite[600]{Mankiw2011}.
\textit{Sæsonledighed} opstår på grund af sæsonmæssige svingninger i produktionen, hvor en stor del af sæsonarbejdsløsheden forekommer er midlertidig hjemsendelsesledighed, idet den arbejdsløse efter en kortere periode som arbejdsløs kommer tilbage til den samme arbejdsgiver \parencite{2015}.

Modellerne trade-off mellem arbejde og fritid, den basale jobsøgningsteori og principal-agent-modellen, som vil blive gennemgået i forbindelse med arbejdsløshed på lang sigt beskæftiger sig primært med arbejdsløshed som friktionsledighed.

%%%%%%%%%%%%%%%%%%%%%%%%%%%%%%%%%%%%%%%%%%%%%%%%%%%%%%%%%%%
\textbf{Trade-off'et mellem arbejde og fritid} bygger på en antagelse fra den neoklassiske økonomiske teori om, at for at have et arbejde skal man have valgt at tage et. Individet har begrænset tid, og står derfor over for et trade-off mellem arbejde og fritid\footnote{Implicit er dette også et trade-off mellem at forbruge goder og forbruge fritid \parencite[5]{Cahuc2004}.}. Det betyder, at når en handling vælges frem for andre mulige handlinger er der omkostninger herved \parencite[389]{Mankiw2011}. Eksempelvis, hvis hvis man vælger at holde en times fri frem for at arbejde den samme time til en timeløn på 100 kroner, så er omkostningerne herved altså 100 kroner\footnote{Trade-off'et mellem arbejde og fritid er senere hen blevet mere kompleks. Eksempelvis er den tid man ikke arbejder ikke nødvendigvis blot fritid, fordi denne tid kan bruges på produktion i husholdningen, som jo også er et supplement for ens lønindtægter \parencite[14]{Cahuc2004}.}. Ifølge Halvorsen hænger trade-offet mellem arbejde og fritid sammen med, at den økonomiske disciplin ser arbejdet som et onde samtidig med, at individet er rationelt og nyttemaksimerende \parencite[26]{Halvorsen1999}.

%%%%%%%%%%%%%%%%%%%%%%%%%%%%%%%%%%%%%%%%%%%%%%%%%%%%%%%%%%%
\textbf{Den basale jobsøgningsmodel} kaldes også for den partielle model og blev udviklet i 1970'erne af blandt andet McCall og Mortensen \parencite[109]{Cahuc2004}. Teorien forklarer, hvorfor friktionsledighed forekommer og bryder hermed med to væsentlige neoklassiske perspektiver. Den første perspektiv er, at der er fuld beskæftigelse, når arbejdsmarkedet er i equilibrium. Her vil en fyret arbejdstager automatisk vil finde et nyt job til markedsløn, fordi alle arbejdstagere kan varetage alle jobs \parencite[163]{Mankiw2007}. Det anden perspektiv er, at individer har \textit{fuldkommen information} om arbejdsmarkedet, og derfor ikke har brug for tid til at søge arbejde. Jobsøgningsteorien tager udgangspunkt i jobsøgning som en proces, hvor individer under ufuldkommen information har brug for tid til at finde de rette jobs i forhold til deres præferencer og færdigheder med henblik på at få den højeste løn som betaling for sin ydelse \parencite[108]{Cahuc2004}. Den optimale søgestrategi for en person som leder efter arbejde består i at vælge en \textit{reservationsløn}. En reservationsløn er den laveste løn en person er villig til at acceptere, hvilket betyder, at alle jobtilbud under dette beløb afvises \parencite[114]{McCall1970}. Og jo højere reservationslønnen er sat til, jo længere vil den gennemsnitlige arbejdsløshedsperiode være \parencite[848]{Mortensen1970}. Valget af reservationsløn kan betragtes som en bestræbelse på at maksimere egennytte. Det vil sige, at fordelene man kan opnå ved at acceptere et job dags dato må opvejes med fordelene i form af bedre job på længere sigt. Et bedre job på længere sigt kan både give en bedre løn, men kan også medfører indkomsttab mens søgningen foregår \parencite[1698]{Atkinson1991}.


%%%%%%%%%%%%%%%%%%%%%%%%%%%%%%%%%%%%%%%%%%%%%%%%%%%%%%%%%%%
\textbf{Principal-agent-modellen} udvikles fra slutningen af 1970'ernes af blandt andet Baily og Flemming på baggrund af den basale jobsøgningsmodel. Her tages der udgangspunkt i en kontrakt mellem en principal og en agent, som gør, at agenten kan tage flere risici, fordi principalen bærer byrden af disse risici. Adfærden hos principalen og agenten er afhængig af om jobsøgningsindsatsen er kontrollerbar eller ej, hvilket vil sige, om der er bevis for, at agenten virkelig har foretaget en indsats for at søge job \parencite[134]{Cahuc2004}. Når indsatsen \textit{er} kontrollerbar, kan principalen gøre udbetalingen af arbejdsløshedsydelsen afhængig af agenternes jobsøgningsindsats, hvilket betyder, at den optimale kontrakt mellem principal og agent giver agenten fuld kompensationsgrad for at søge arbejde, når agenten er arbejdsløs \parencite[138]{Cahuc2004}. Når indsatsen \textit{ikke} er kontrollerbar er der mulighed kan der være mulighed for, at agenten modtager en arbejdsløshedsydelse uden at gøre en indsats for at søge arbejde. Derfor er det nødvendigt for principalen, at den optimale kontakt \textit{ikke} giver fuld kompensationsgrad for arbejdsløshed med henblik på at give agenten større incitament for at søge arbejde\footnote{Baily definerer det optimale arbejdsløshedsforsikringsystem som en afvejning mellem jobsøgningsincitament og arbejdsløshedsforsikringen.
Den marginale gevinst af arbejdsløshedsforsikring er lig den marginale omkostning ved øget arbejdsløshed. Derved påviser Baily som en af de første sammenhængen mellem incitament for at søge jobs og størrelsen på arbejdsløshedsforsikring \parencite[379]{Baily1978}.} \parencite[379]{Baily1978}. %%% Baily er uforståelig

%%%%%%%%%%%%%%%%%%%%%%%%%%%%%%%%%%%%%%%%%%%%%%%%%%%%%%%%%%%
Ifølge Atkinson bygger politiske valg om at sænke ydelsesniveauerne for arbejdsløse på baggrund af en oversimplificering af arbejdsmarkedet, hvor økonomer typisk ser arbejdsløshedsydelser som havende en negativ effekt på arbejdsmarkedet med høje ydelser som forårsager, at arbejdsløse er mindre villige til at tage et arbejde \parencite[1680]{Atkinson1991}. Atkinson selv kritiserer modellerne for typisk at antage, at arbejdsløshedens effekt kan reduceres til beløbet på arbejdsløshedsydelsen uden at forholde sig til de institutionelle forhold i arbejdsløshedssystemet\footnote{I OECD-lande er betingelserne for modtagelse af arbejdsløshedsydelser eksempelvis typisk, at personen ikke må være frivillig arbejdsløs, der skal gøres en reel jobsøgningsindsats, man må ikke blive ved med at afvise jobtilbud og der er ydelsen dækker over en begrænset periode \parencite[1689]{Atkinson1991}.} \parencite[1688]{Atkinson1991}. De økonomiske modeller er altså ikke gode nok til at forklare institutionelle forhold som for eksempel kommunikation med jobcentret og a-kassen og deres påvirkning på den arbejdsløse på trods af forsøg på at tage højde for hvor længe der kan modtages ydelser, forskelle på arbejdsløshedsydelser og sociale ydelser og dem som ikke er berettiget til arbejdsløshedsydelser \parencite[1692]{Atkinson1991} \parencite[33-34]{Halvorsen1999} \parencite[114]{Cahuc2004}.

De økonomiske modeller forudsætter typisk en positiv sammenhæng mellem søgeintensitet og antallet af jobtilbud man modtager. Modellerne har dog typisk ikke fokus på, at jobsøgningen ikke er enstationær tilstand og jo længere arbejdsløsheden varer, jo færre jobtilbud vil der for det meste komme \parencite[119]{Cahuc2004}. Ifølge Halvorsen er et væsentligt problem med jobsøgningsmodellerne, at arbejdsløse næsten altid vil tage imod det første tilbud de modtager, hvilket betyder at variationer i arbejdsløshedslængden primært opstår ud fra sandsynligheden for at modtage et jobtilbud \parencite[28]{Halvorsen1999}.

Omvendt er en høj arbejdsløshedsydelsen også blevet sat i et mere positivt lys, fordi det kan bidrage til at have en positiv effeekt på søgeaktiviteten. Ifølge Tatsiramos er fordelene ved at modtage arbejdsløshedsforsikring større end omkostningerne. Selvom en høj arbejdsløshedsforsikring kan medføre en længere arbejdsløshedsperiode, viser et europæisk studie, at de arbejdsløse som modtager arbejdsløshedsydelse forbliver 2-4 måneder længere i det efterfølgende jobs end arbejdsløse som ikke modtager en arbejdsløshedsydelse \parencite[602]{Mankiw2011}. Dette perspektiv åbner op for, at et job ikke bare er et job, hvilket kommer af at beskæftigelse som tidligere nævnt ligesom arbejdsløshed og personer uden for arbejdsstyrken er en heterogen kategori. %%% Det ligger måske endu mere op til Bourdieu

 Afslutningsvis skal det nævnes, at de økonomiske modeller typisk ser også bort fra ikke-økonomiske incitamenter, hvor lønarbejde typisk foretrækkes frem for arbejdsløshed, fordi det giver selvrespekt, anderkendelse og man lever op til de sociale forventninger. Disse sociale og psykologiske forhold dominerer den sociologiske arbejdsløshedsforskning som vi som før nævnt vil gå i dybden med senere \parencite{Jahoda1971, Eisenberg1938, Ezzy1993, Halvorsen1999, Baum2001, Noerup2014}. %%% Emil: Senere - udnyt latex


%%%%%%%%%%%%%%%%%%%%%%%%%%%%%%%%%%%%%%%%%%%%%%%%%%%%%%%%%%%
\subsubsection{Arbejdsløshed på kort sigt} %%%% Emil: Bedre overgang måske

Arbejdsløshed på kort sigt kaldes også konjunkturledighed, og karakteriseres ved økonomiske udsving fra år til år \parencite[592]{Mankiw2011}. John Maynard Keynes forsøgte som den første at forklare disse udsving i forbindelse med den depressionen i 1930'erne.  Ifølge Keynes ville lønninger og priser ikke nødvendigvis tilpasse sig på kort sigt. På den måde ville økonomien være i en position, hvor efterspørgslen ikke var tilstrækkelig til at skabe en beskæftigelse svarende til fuld beskæftigelse. Når økonomien ikke ville kunne tilpasses på kort sigt, mente Keynes, at staten burde gribe ind og administrere efterspørgslen for at opnå det ønskede beskæftigelsesniveau \parencite[707]{Mankiw2011}. Det kan staten typisk gøre ved finanspolitisk at øge forbruget for at booste den økonomiske aktivitet\footnote{Et eksemepl herpå kunne være, når staten laver en kontrakt på 10 milliarder for at bygge tre nye atomkraftværker. Hermed skabes beskæftigelse og profit hos byggefirmaet, som resulterer i beskæftigelse og profit hos underleverandørerne. Alt i alt skaber der et forbrug, hvor hvert pund som er brugt får den samlede efterspørgsel på varer og ydelser til at stige for mere end et pund. Dette kaldes også multiplikatoreffekten som er defineret ved at være supplerende ændringer i den samlede efterspørgsel som finder sted, når  ekspansiv finanspolitik øger indkomst og dermed øger privatforbruget. \parencite[709]{Mankiw2011}.} eller ved pengepolitik, hvis centralbanken beslutter sig for at udvide mængden af penge \parencite[718]{Mankiw2011}. Udover Keynes insisteren på vigtigheden af arbejdsløshed på kort sigt spiller \textit{Phillipskurven} en stor rolle. Phillipskurven viser en negativ sammenhæng mellem arbejdsløshedsraten og inflationsraten, hvilket vil sige at år med lav arbejdsløshed har høj inflation, og år med høj arbejdsløshed har lav inflation \parencite[783]{Mankiw2011}. Finanspolitik og pengepolitik kan flytte økonomien langs Phillipskurven. Øgninger af pengemængden, øger det offentlige forbrug eller skattelettelser udvider den samlede efterspørgsel og flytter økonomien til et punkt i Phillipskruven, som giver et trade-off med lav arbejdsløshed for høj inflation \parencite[785]{Mankiw2011}.

%%%%%%%%%%%%%%%%%%%%%%%%%%%%%%%%%%%%%%%%%%%%%%%%%%%%%%%%%%%
\textbf{Hysterese} kendetegner midlertidige chok på økonomien, som har permanente eftervirkninger på arbejdsløshedsniveauet. Hermed er det muligt for, at arbejdsløshed på kort sigt har en effekt på arbejdsløsheden på lang sigt. Det kan ske ved, at visse arbejdsløse bliver ved med at være ekskluderet fra arbejdsmarkedet, fordi deres produktivitet er så lav, at det ikke er profitabelt at ansatte dem selv til en lav løn. Hvis der ikke er nogen måde at reintegrerer de arbejdsløse på, så vil de have en vedholdende effekt på arbejdsløshedsraten \parencite[477]{Cahuc2004}. Dette hænger sammen med den lave beskæftigelse af langtidsledige. De langtidsledige har svært ved at komme i beskæftigelse på grund af manglende påskønnelse af deres human kapital, manglende motivation i jobsøgning og kendsgerningen, at en lang arbejdsløshedsperiode kan fortolkes som en signal om at arbejdstagerens kvalitet ved ansættelsen kan forklare den dårlige præstation hos den langtidsledige \parencite[479]{Cahuc2004}.


%%%%%%%%%%%%%%%%%%%%%%%%%%%%%%%%%%%%%%%%%%%%%%%%%%%%%%%%%%%
\subsubsection{Opsummering}

Dette afsnit har behandlet arbejdsløshed i lyset af, at økonomisk teori har domineret arbejdsløshedsforskningen samtidig med at være det ideologiske grundlag for politiske beslutninger på arbejdsløshedsområdet. Inden for den økonomiske disciplin inddeles befolkningen typisk i arbejdsstyrken, som består af beskæftigede og arbejdsløse, og i den ikke-økonomisk aktive del af befolkningen, som står uden for arbejdsstyrken. I dette afsnit har vi beskæftiget os med en række modeller i relation til arbejdsløshed på kort og lang sigt. Arbejdsløshed på lang sigt kan opdeles i friktionsledighed, strukturel ledighed og sæsonledighed. Trade-off'et mellem arbejde og fritid, den basale jobsøgningsmodel og principal-agent-modellen er med til at forklare, hvorfor arbejdsløshed opstår, og hvordan man kan få arbejdsløse i beskæftigelse. Arbejdsløshed på kort sigt  opstår i forbindelse med konjunkturændringer, hvor hysterese er med til at forklare arbejdsløshed som konsekvens heraf. De økonomiske modeller er blevet kritiseret for at oversimplificere arbejdsløshedsproblemet ved at anvende for simple kategorier og for at reducere incitament til et økonomisk spørgsmål uden at forholde sig til institutionelle, sociale og psykologiske spørgsmål.







%%%%%%%%%%%%%%%%%%%%%%%%%%%%%%%%%%%%%%%%%%%%%%%%%%%%%%%%%%%
\newpage \subsection{\textsc{Sociologiske forståelser af arbejdsløshedsproblemet}}
%%%%%%%%%%%%%%%%%%%%%%%%%%%%%%%%%%%%%%%%%%%%%%%%%%%%%%%%%%%

Sociologien står i skarp kontrast til økonomien. Granovetter som er en af de mest førende sociologiske kritikere af den økonomiske diciplin definerer den økonomiske aktør som undersocialiseret: “Actors do not behave or decide as atoms outside a social context, nor do they adhere slavishly to a script written for them by the particular intersection of social categories that they happen to occupy. Their attempts at purposive action are instead embedded in concrete, ongoing systems of social relations.” \parencite[487]{Granovetter1985}. Hvor økonomien mangler psykologisk og social realisme, beskæftiger sociologien sig med identitet, social integration, normer og menneskelig adfærd såvel som et statistisk baseret styringsvidenskab efter et mere økonomisk mønster \parencite[36]{Halvorsen1999}. Sociologisk arbejdsløshedsforskning er desuden domineret af empirisk variabelanalyse, årsags-virkningsforhold og mekanismer som kan fører til dårligt mentalt helbred \parencite[38]{Halvorsen1999}. Størstedelen af arbejdsløshedsforskerne benytter sig af middle-range teorier \parencite[9]{Hedstroem2005}, som beskriver en begrænset del af den sociale virkelighed i modsætning grand theories. %%%% EmIL: "Sociologisk arbejdsløshedsforskning er desuden..." er en lidt uklar formulering

Dette teoretiske afsnit om sociologiske forståelser af arbejdsløshed indeholder først en strukturbaserede teorier som primært beskæftiger sig med sociale og psykologiske konsekvenser af arbejdsløshed herunder, stadiemodellen, Jahodas funktionelle deprivationsteori, Warrs vitaminmodel og marginaliseringsperspektivet. Herefter følger en række aktørbaserede teorier herunder rehabiliteringstilgangen, Fryers agency kritik, status passagemodellen og Halvorsens mestringsperpektiv. 


%%%%%%%%%%%%%%%%%%%%%%%%%%%%%%%%%%%%%%%%%%%%%%%%%%%%%%%%%%%
\subsubsection{Strukturbaserede teorier}

De negative konsekvenser af arbejdsløshed blev først beskrevet Marienthal-studiet af Jahoda, Lazarsfeld og Zeizel som foregår i den østrigske by Marienthal, hvor en voldsom arbejdsløshed havde ført til apati blandt de arbejdsløse \parencite[vii]{Lazarsfeld1971}. Sidenhen er der foretaget en række strukturbaserede teorier og modeller som er særlig relevante i forhold til sociale og psykologiske konsekvenser af arbejdsløshed herunder stadiemodellen, Jahodas funktionelle model, Warr's vitaminmodel og marginaliseringsperspektivet.

Lazarsfeld udvikler sammen med Eisbenberg en model som trækker erfaringerne fra Marienthal og en lang række andre arbejdsløshedsstudier. Jahoda, Lazarsfeld og Zeizel skelnede mellem det at være knækket, resigneret, fortvivlet og apatisk som reaktioner på arbejdsløshed i Marienthal \parencite[56]{Jahoda1971}. \textbf{Stadiemodellen} konkluderer, at arbejdsløse gennemgår tre forskellige stadier: Først oplever den arbejdsløse et chok opfulgt af en aktiv indsats for at søge job, hvor den arbejdsløse stadigvæk er optimistisk og ikke knækket. Dernæst, når al indsats er mislykket, bliver den arbejdsløse pessimistisk, fortvivlet og føler sig i nød. Til sidst bliver den arbejdsløse knækket \parencite[378]{Eisenberg1938}. Stadiemodellen er dog blevet kritiseret for at være metodisk problematisk, teorien modsætningsfuld og kategorierne hule. Ifølge Ezzy er modellen ikke så meget en teori som et deskriptivt framework, hvor den operationelle variabel er længden på arbejdsløshed \parencite[44]{Ezzy1993}.

Det centrale i \textbf{Jahodas funktionelle deprivationsteori} er, at arbejdsløshed medfører en række afsavn, som den arbejdsløse ville have fået ved at være i beskæftigelse \parencite[44]{Ezzy1993}. Udgangspunktet er Mertons opdeling af sociale funktioner som manifeste og latente. Arbejdets manifeste funktion består i at opfylde et økonomisk behov for indtægt, mens arbejdets latente funktion består i at opfylde et psykologisk behov: “First, employment imposes a time structure on the waking day; second, employment implies regularly shared experiences and contacts with people outside the nuclear family; third, employment links individuals to goals and purposes that transcend their own; fourth, employment defines aspects of personal status and identity; and finally, employment enforces activity.” \parencite[188]{Jahod1981}. Som en parallel til trade-offet mellem arbejde og fritid, kan “fritiden” hos den arbejdsløse ikke udfylde de samme funktioner som arbejdet har \parencite[189]{Jahod1981}. Ifølge Ezzy imødekommer lønarbejdet ikke automatisk individets grundlæggende psykologiske og sociale behov, Jahoda romantiserer derfor og ser bort fra, at lønarbejde for nogle kan være isolerende og ubehageligt \parencite[45]{Ezzy1993}.

I modsætning til Jahoda giver Warrs \textbf{vitaminmodel} mulighed for at skelne mellem at have et godt eller dårligt mentalt helbred både som beskæftiget og som arbejdsløs. Warr foreslår, at på samme måde som vitaminer har en effekt på det fysiske helbred, så har forskellige forhold i ens omgivelser en effekt på det mentale helbred. Disse forhold eller “vitaminer” består af muligheden for at have kontrol over ens tilværelse, muligheden for at anvende ens færdigheder, ydre mål, variation, klarhed i forhold til omgivelserne, penge, fysisk sikkerhed, social kontakt og værdsat social position \parencite[45]{Ezzy1993}. Det vil sige, at når vitaminniveauerne er lave i en utilfredsstillende jobtilværelse eller som arbejdsløs, går det ud over det mentale helbred. Samtidig er der også mulighed for at skelne mellem vitaminniveauet hos eksempelvis arbejdsløse middelalderende mænd med større problemer i forhold til penge, sikkerhed, personlig kontakt og en værdsat social position end hos arbejdsløse teenagere, som er mindre afhængige af arbejde for at have tilstrækkelige niveauer inden for de nævnte vitamintyper \parencite[46]{Ezzy1993}.

\textbf{Marginaliseringsperspektivet} anvendes både på arbejdsmarkedet, men også en række andre områder. Ifølge Elm larsen skal marginaliseringsperspektivet ses som en midterkategori mellem inklusion og eksklusion. Larsen definerer eksklusion som en ufrivillig ikke-deltagelse gennem forskellige typer af udelukkelsesmekanismer og -processer, som det ligger uden for indvidets og gruppens muligheder at få kontrol over \parencite[137]{Larsen2009}. Hermed er Larsen kritisk over for Luhmanns dikotomi inklusion/eksklusion, som Larsen i dens binære form ikke mener er særlig hensigtsmæssig i forhold til at beskrive virkeligheden \parencite[130]{Larsen2009}. Derfor argumenter han for, at marginalisering kan anvendes som en midtergruppe mellem de to, hvor individet bevæger sig i en proces mod inklusion eller eksklusion. På baggrund af dette har vi tegnet følgende model:\footnote{Modellen er også inspireret af lignende modeller benyttet af Lars Svedberg \parencite[44]{Svedberg1995} og Catharina Juul Kristensen \parencite[18]{Kristensen1999}.} som fremgår af tabel \ref{tab_marginaliseringsmodel_1}. 
% 
\begin{table}[H] \centering
\caption{Model over marginalisering}
\label{tab_marginaliseringsmodel_3}
\begin{tabular}{@{} m{3,4cm} c m{3,6cm} c m{3,6cm} @{}} \toprule
\textbf{Inkluderet} & & \textbf{Marginaliseret} & & \textbf{Ekskluderet} \\ \midrule
\end{tabular} \end{table} %%%%%
\begin{table}[H] \centering
\label{tab_marginaliseringsmodel}
\begin{tabular}{@{} m{5,9cm} m{5,9cm} @{}} 
  \textbf{Marginaliseringsproces} & \textbf{Eksklusionsproces} \\  
  --------------------------------------------> & --------------------------------------------> \\ 
\end{tabular} \end{table} %%%%%
\begin{table}[H] \centering
\label{tab_marginaliseringsmodel}
\begin{tabular}{@{} m{12,3cm} @{}} 
  \textbf{Inklusionsproces} \\  
  <--------------------------------------------------------------------------------------------- \\ \bottomrule
\end{tabular} \end{table}
%
Den første proces består af individer som går fra at være inkluderet til at indgå i en proces i retning mod marginalisering. Den anden proces består af individer som går fra at være marginaliseret til at indgå i en proces i retning mod inklusion. Marginaliseringsperspektivet giver hermed mulighed for at se de  bevægelser på arbejdsmarkedet inden for kategorier som ikke rigide.


%%%%%%%%%%%%%%%%%%%%%%%%%%%%%%%%%%%%%%%%%%%%%%%%%%%%%%%%%%%
\subsubsection{Aktørbaserede teorier}

De aktørbaserede teorier beskæftiger sig mere mindre med de negative sociale og psykologiske konsekvenser for den arbejdsløse og mere med, hvad agenten gør, ikke gør eller kan gøre for at komme ud af arbejdsløshedssituationen herunder rehabiliteringstilgangen, Fryers agency kritik, status passagemodellen og Halvorsens mestringsperpektiv.

\textbf{Fryers agency kritik} er ikke så meget en teori, men mere en direkte kritik af at beskæftige sig med passive aktører i arbejdsløshedsteorier såsom Jahoda og Warrs. Fryer foreslår, at det individet bringer til situationen er lige så vigtigt som konsekvenserne af arbejdsløshed. De proaktive arbejdsløse i Fryers studie oplever ikke sociale og psykologiske afsavn på trods af, at de lider af de materielle afsavn, som kommer af at miste indtægterne fra deres arbejde. Ifølge Fryer er disse proaktive arbejdsløse eksempler på aktive sociale agenter, som forsøger at få mening ud af deres situation and agerer ud fra målsætninger \parencite[47]{Ezzy1993}. Omvendt er Fryer blevet kritiseret for at lægge for meget vægt på kognitive processer og ignorerer de institutionelle begrænsninger som arbejdsløsheden oftest medfører \parencite[47]{Ezzy1993}.

Tiffany, Cowan og Tiffanys udvikler i 1970 \textbf{rehabiliteringstilgangen} på baggrund af et studie, som viser, at størstedelen af de arbejdsløse er arbejdsløse på grund af psykologiske årsager: “they show avoidance behavior pattern or what has been referred to as “work inhibition” which implies that they are physically capable of work but are prevented from working because of psychological disabilities” \parencite[43]{Ezzy1993}. Som konsekvens bør man ifølge Tiffany, Cowan og Tiffany rehabiliterer de arbejdsløse gennem terapi eller træning, så de kan vende tilbage på arbejdsmarkedet. Ifølge Ezzy ligner denne tilgang den historiske skelnen mellem dem, som var fysisk ude af stand til at arbejde og fortjente støtte, og dem, som var i fysisk stand til at arbejde, men som ikke arbejdede, fordi de var dovne, og derfor ikke fortjente støtte. Ezzy påpeger, at årsagen til, at tilgangen fokuser på individuelle forklaringer frem for strukturelle forklaringer er, at tilgangen har været mest toneangivende i perioder med relativ lav arbejdsløshed, mens de fleste andre sociologiske og psykologiske arbejdsløshedsstudier er foretaget i perioder med høj arbejdsløshed \parencite[43]{Ezzy1993}.
 
Ezzy anvender begrebet jobtab i stedet for arbejdsløshed\footnote{Denne problematik adskiller sig på den ene side fra andre veje til arbejdsløshed såsom dimittendledighed eller et vende tilbage til arbejdsstyrken igen og veje ud af beskæftigelse såsom pensionering, orlov eller at tage en uddannelse \parencite[48]{Ezzy1993}.} og sammenligner det at miste sit job med at gå i gennem et skilsmisseforløb, et sygdomsforløb eller opleve et dødsfald i familien. Disse overgange skal forstås i forlængelse af Glaser og Strauss' definition af en \textbf{status passage} som et individs ”movement into a different part of a social structure, or loss or gain of privilige, influence, or power, and changed behaviour” \parencite[48]{Ezzy1993}. En status passage er en del af individets biografi, hvilket involverer samtidige og tidligere erfaringer, som påvirker meningen der tillægges arbejdsløsheden. Ezzy skelner mellem integrative passager og afhændelsespassager\footnote{Denne skelnen er foretaget med udgangspunkt i Van Gennep skelnen mellem separation (for eksempel begravelse), transition (for eksempel jobskifte) og integration (for eksempel ægteskab) som kategorier for sociale passager\parencite[48]{Ezzy1993}.}. Integrative passager er oftest en overgangsperiode efterfulgt af integration i en ny status gennem en ceremonial proces som for eksempel bryllupper. Afhændelsespssager er en separation fra en status som oftest er en længerevarende overgangsfase med en usikker varighed for eksempel skilsmisser, sygdomsforløb, arbejdsløshed eller dødsfald \parencite[49]{Ezzy1993}. En afhændelsespassager fører ikke nødvendigvis i sig selv til mentale problemer, mistrivsel eller eksklusion, da det afhænger af den enkeltes identitet og selvopfattelse i relation til andre og samfundet og en lang række andre faktorer samt om afhændelsespassagen efterfølges af en reintegrativ passage, hvor den enkelte får en ny status eksempelvis et nyt job \parencite[32]{Noerup2014}.

Knut Halvorsen anskuer med \textbf{mestring}-perspektivet ligeledes de arbejdsløse som aktive aktører. De arbejdsløse kan dermed påvirke deres situation og være med til at forandre den i stedet for at være ofre for omstændighederne. For at kapere en negativ hændelse - som det at miste sit job er - indgår den arbejdsløse i forskellige psykiske, fysiske og sociale aktiviteter. Den problemorienterede mestring udgør konkrete strategier med formål at fjerne belastningen af den marginaliserede position for eksempel jobsøgning. Den emotionsorienterede mestring handler om, hvordan man ser verdenen for eksempel kan det at redefinere sig som hjemmegående ved at måde at søge ny basis for mening for at opretholde selvrespekten \parencite[47]{Halvorsen1999}. Ifølge Nørup ligger fokus i perspektivet, hvordan individet håndterer det, at stå midlertidigt uden for arbejdsmarkedet. På den måde forholder Halvorsen sig ikke til, hvad der sker med, når arbejdsløsheden bliver permanent eller langsigtet \parencite[30]{Noerup2014}. Nørup kritiserer endvidere Halvorsen for i sin ontologiske individualisme at fokuserer på, hvordan individet mestrer bestemte livssituationer under bestemte rammer frem for på hvordan de samfundsmæssige strukturer og sociale relationer påvirker eksklusionen \parencite[37]{Noerup2014}.


%%%%%%%%%%%%%%%%%%%%%%%%%%%%%%%%%%%%%%%%%%%%%%%%%%%%%%%%%%%
\subsubsection{Opsummering}
%%%%%%%%%%%%%%%%%%%%%%%%%%%%%%%%%%%%%%%%%%%%%%%%%%%%%%%%%%%

Vi har valgt at dele de sociologiske forståelser af arbejdsløshedsproblemet op i struktur- og aktørbaserede teorier. Fælles for de strukturbaserede teorier såsom stadiemodellen, Jahodas funktionelle deprivationsteori og Warrs vitaminmodel som i høj grad fokuserer på de sociale og psykologiske konsekvenser af arbejdsløshed. De aktørbaserede teorier såsom Fryers agency kritik, rehabiliteringstilgangen og Halvorsens mestringsperpektiv beskæftiger sig mere med, hvad agenten gør, ikke gør eller kan gøre for at håndtere arbejdsløsheden eller komme tilbage i beskæftigelse. Jahoda, Warr og andre strukturbaserede teoretikere er blevet kritiseret for at gøre de arbejdsløse til passive individer, mens Fryer, Halvorsen og andre agentbaserede teoretikere omvendt er blevet kritiseret for at se bort fra de strukturelle og institutionelle begrænsninger i arbejdsløshedsperspektivet. Jahoda er i særdeleshed blevet kritiseret for at fremhæve  lønarbejdet som et ubetinget gode, hvilket hænger sammen med det førnævnte elendighedsperspektiv på arbejdsløsheden. Denne beskrivelse af lønarbejdet står i modsætning til den økonomisk teori, der som nævnt så lønarbejdet som en byrde\footnote{Dette perspektiv har imidlertidigt være udfordret af den marxistiske tradition inden for sociologien, hvor lønarbejdet anset for at være fremmedgørende \parencite[48]{Halvorsen1999}.}. Status massagemodellen og marginaliseringsperspektivet giver mulighed for at anskue arbejdsløshed i bevægelse og dermed også mere dynamisk end flere af de sociologiske og økonomiske teorier.





%%%%%%%%%%%%%%%%%%%%%%%%%%%%%%%%%%%%%%%%%%%%%%%%%%%%%%%%%%%
\newpage \subsection{SFI - en case}
%%%%%%%%%%%%%%%%%%%%%%%%%%%%%%%%%%%%%%%%%%%%%%%%%%%%%%%%%%%

Det Nationale Forskningscenter for Velfærd (SFI) er storproducent af policy-studier på arbejdsmarkeds- og beskæftigelsesområdet med rapporter som typisk er bestilt af Beskæftigelesministeriet eller forskellige ministerier og kommuner. Rapporterne tager typisk udgangspunkt i en grupper af personer. Hvis man kigger på SFIs rapporter de sidste 20 år er målgrupperne hovedsageligt ledige\footnote{Også kaldet udsatte ledige, arbejdsmarkedsparate ledige, ikke-arbejdsmarkedsparate ledige, langtidsledige og forsikrede ledige}, dagpengemodtagere\footnote{Også kaldet sygedagpengemodtagere og aktiverede dagpengemodtagere.}, kontanthjælpsmodtagere\footnote{Også kaldet ikke arbejdsmarkedsparatemodtagere, de svageste kontanthjælpsmodtagere og aktiverede kontanthjælpsmodtagere.}, sygemeldte og arbejdsskadede\footnote{Også kaldet skadeslidte beskæftigede og personer som har nedsat arbejdsevne efter en ulykke i fritiden.} samt pensionister og efterlønsmodtagere\footnote{Der er også foretaget en hel del undersøgelse om handicappede, indvandrere, efterkommere, mænd, kvinder, ældre og højtuddannede.}. Fokus handler hovedsageligt om at få dem i beskæftigelse eksempelvis ved at bringe de langtidsledige tættere på arbejdsmarkedet i \textit{Tættere på arbejdsmarkedet} (2011), ved at måle beskæftigelseseffekten af dagpengeophør i \textit{Dagpengemodtagers situation omkring dagpengeophør} (2014), ved at kigge på indsatser over for ikke arbejdsmarkedsparate kontanthjælpsmotagere i \textit{Veje til beskæftigelse} (2010), ved at måle effekten af den beskæftigelesrettede indsats for sygemeldte i \textit{Effekten af den beskæftigelsesrettede indsats for sygemeldte} (2012) eller ved at kigge på pensionisters og efterlønsmodtageres genindtræden op arbejdsmarkedet i \textit{Pensionisters og efterlønsmodtageres arbejdskraftspotentiale} (2012). Det som er kendetegnede ved denne typer rapporter er et grundlæggende fokus på at få de pågældende personer tilbage i beskæftigelse hvad man kan gøre og ikke hvad deres situation egentlig betyder for deres liv\footnote{Der skal ikke menes med, at disse rapporter slet ikke forholder sig til de pågældende personers liv. I \textit{Veje til Beskæftigelse} (2010) fortælles der igennem 30 kvalitative interviews med sagsbehandlere, at de oplever de ikke-arbejdsmarkedsparate kontanthjælpsmodtagere som værende en heterogen gruppe som har gavn af forskellige typer indsatser alt efter, hvilke udfordringer de har. Nogle har for eksempel helbredsproblemer, mens andre har brug for hjælp til daglige gøremål.}.





%%%%%%%%%%%%%%%%%%%%%%%%%%%%%%%%%%%%%%%%%%%%%%%%%%%%%%%%%%%
\newpage \subsection{\textsc{Afsluttende opsamling}}
%%%%%%%%%%%%%%%%%%%%%%%%%%%%%%%%%%%%%%%%%%%%%%%%%%%%%%%%%%%

\textbf{Marginaliseringsperspektivet i stedet for arbejdsløshed:} Arbejdsløse fremtræder forskelligt og et ikke et problem. Arbejdsløse fremtræder i dagligdagen, den offentlige debat og blandt forskerne på forskellige måder. Det kan være som ledige, dagpengemodtagere, kontanthjælpsmodtagere. Dertil kommer der en masse tillægsord som udsatte, arbejdsmarkedsparate, ikke-arbejdsmarkedsparate, langtids, forsikrede, ikke-forsikrede, sæson/friktion/førstegang, syge, aktiverede, svage, sygemeldte, arbejdsskadede, skadeslidte. Dertil kommer der undergrupper som handikappede, indvandrere, efterkommere, mænd, kvinder, ældre, højtuddannede\footnote{Denne lange beskrivelse af forskellige typer arbejdsløse kommer fra en gennemgang af Det Nationale Forskningscenter for Velfærd (SFI) rapporter fra de sidste 20 år.}. Dertil kommer der en masse forskellige definitioner af arbejdsløse fra forskellige forskere og institutioner. En af de mest gængse definitioner kommer fra International Labour Organization som definerer arbejdsløse som det antal personer som står uden beskæftigelse samtidig med at være til rådighed for arbejdsmarkedet og aktivt arbejdssøgende \parencite{ILO1982}. Men hvad har de arbejdsløse egentlig tilfælles ud over at de står uden lønnet arbejde. Det er netop, hvad Halvorsen spørger sig selv og konkluderer, at arbejdsløse ikke er en ensartet gruppe, men en kategori sammensat af forskellige mennesker med forskellige udfordringer.  Arbejdsløshed er en institutionel konstruktion, som påvirkers af sociale ordninger og arbejdsmarkedets organisering: “The unemployed are not a group of people, but an economic and adminsitrative category” \parencite{Kelvin1985} \parencite[18]{Halvorsen1999}. Overordnet kan man sige, at arbejdsløshed ikke bare et socialt problem, men som flere til dels adskilte sociale problemer. Samtidig er arbejdsløse ikke bare en ensartet gruppe, men forskellige mennesker med forskellige problemstillinger . Til sidst fremstår arbejdsløse - om man kalder dem ledige eller noget tredje som en kategori forbundet med negative beskrivelser \parencite[12]{Halvorsen1999}. Definition af arbejdsløshed. Det centrale at have med - inddeling af arbejdsmarkedet - kategorierne beskæftiget, arbejdsløs og uden for arbejdsstyrken - jobsøgning - aktiv indsats/handling for at søge jobs - incitament - motivation for at handle. Atkinson, Jørgen Elm, Ezzy mv. Kernen i vores teoretiske og empiriske arbejde er arbejdsløshed. Arbejdsløshed defineres i \textit{Den Store Danske} som den manglende overensstemmelse på arbejdsmarkedet mellem udbud af arbejdskraft og efterspørgsel efter arbejdskraft \parencite{2015}. I de almindelige statistiske definitioner opdeles den danske befolkning i folk der er inden for og uden for arbejdsstyrken. Dem der er inden for arbejdsstyrken er enten i beskæftigede eller arbejdsløse, mens alle andre per definition betragtes som værende uden for arbejdsstyrken \parencite{2015a}. Problemet med denne opdeling er, at personer der klassificeres som uden for arbejdsstyrken ofte kommer fra beskæftigelse og bevæger sig tilbage i beskæftigelse. De følger måske ikke standarddefinitionen på arbejdsløshed\footnote{\textit{International Labour Organization} definerer arbejdsløse som det antal personer som står uden beskæftigelse samtidig med at være til rådighed for arbejdsmarkedet og aktivt arbejdssøgende \parencite{ILO1982}. Denne definition fremgår både af \textit{Den Store Danske}, Danmarks Statistik, Beskæftigelsesministeriet med flere.}, men kan i et lidt bredere perspektiv godt betragtes som arbejdsløse. Eksempelvis kan personer på revalideringsydelse, kontanthjælp og førtidspension godt vende tilbage i beskæftigelse igen. For at få en bedre forståelse af arbejdsløses sociale mobilitet på arbejdsmarkedet må vi derfor bryde med de almindelige definitioner af arbejdsløshed\footnote{Vi er vel opmærksomme på den symbolske kamp, der ligger i at gøre dette, hvilket fremgår af de politiske diskussioner om arbejdsløse og kontanthjælpsmodtagere, der er ’skjult’ i aktiveringsforløb...}. For at bryde med standarddefinitionerne af arbejdsløshed vil først og fremmest redegøre for sociologisk-videnskabelige og økonomisk-videnskabelige tilgang til arbejdsløse og arbejdsløshed for så til sidst at anvende Bourdieu og marginaliseringsbegrebet til at lave en teoretisk operationalisering af arbejdsløshed. Når en stor gruppe får sværere ved at konkurrere om de ledige jobs, vil arbejdsløsheden i mindre grad lægge pres på løndannelsesprocessen, og løntilpasningen vil således ikke kunne sikre en tilbagevenden til høj beskæftigelse. Jo længere tid arbejdsmarkedet er præget af høj arbejdsløshed, jo flere af de arbejdsløse vil blive marginaliserede, og jo større bliver den strukturelle ledighed \parencite{2015}.  Marginaliseringbegrebet åbner op for arbejdsløshed i forhold til at inkluderer dem som står uden for arbejdsstyrken. Flere måde at anskue de arbejdsløse på som helhed, det vil sige alle eller som dele, hvilket både kan være i grupper (for eksempel ledige, langtidsledige og kontanthjælpsmodtagere) eller som årsag (friktionsledighed, konjunkturledighed, strukturledighed og sæsonledighed). Blik for uden for arbejdsstyrken, hvilket vil sige studerende, efterløn, pensionister mv. Vores definition af arbejdsløse er så bred som overhovedet muligt. Med arbejdsløs har vi som udgangspunkt den bredeste definition overhovedet, hvilket er det at stå uden arbejde. Den teoretiske pointe er at arbejdsløshed defineres og behandles forskelligt alt efter om det er økonomer, sociologer mv.. Vores fokus ligger i forlængelse af marginaliseringsbegrebet \parencite{Larsen2009} samt Bourdieus perspektiver om at være placeret et specifikt sted i det sociale rum og at være på kanten af arbejdsmarkedet. Arbejdsløse versus ledige: Fokus på arbejdsmarkedsparate arbejdsløse, men værd opmærksom på, at arbejdsmarkedsparathed kan have en mening (økonomiske incitamenter) vi ikke ønsker. Arbejdsløse versus ledige... Arbejdsløshed er en person uden arbejde, mens ledig er en person som står til rådig på arbejdsmarkedet. Spørgsmål om skyld.

\textbf{Psykiske plager og selvrespekt:} Fælles for de strukturbaserede teorier er, at de behandle de sociale og psykologiske konsekvenser af arbejdsløshed, når man er arbejdsløs. Her er der tale om Rehabiliteringstilgangen repræsenteret ved Tiffany, Cowan og Tiffany, Stadiemodellen repræsenteret ved Lazarsfed og Eisenberg, Jahodas funktionelle deprivationsteori og Warrs vitaminmodel. Hysterese - det at være arbejdsløs - De aktørbaserede teorier beskæftiger sig mere med, at arbejdsløshed ikke nødvendigvis er negativt i sig selv og beskæftiger sig mere med den proces som får individer til at opleve psykologiske og materielle afsavn samtidig med, hvad agenten gør, ikke gør eller kan gøre for at komme videre. Her er der tale om Fryers agency kritik, Ezzys status passagemodel og Halvorsens mestringsperpektiv.

\textbf{Incitamenter.} - det at bevæge sig mod beskæftigelse Arbejdsløshed på lang sigt: friktionsledighed, strukturel lighed og sæsonledighed. Trade-off mellem arbejde og fritid. Basal jobsøgningsteori. Trade-off mellem forsikring og incitament.Incitamenter. - det at bevæge sig mod beskæftigelse Arbejdsløshed på lang sigt: friktionsledighed, strukturel lighed og sæsonledighed. Trade-off mellem arbejde og fritid. Basal jobsøgningsteori. Trade-off mellem forsikring og incitament - Økonomiske vanskeligheder og det offentlige og private sikkerhedsnet 

Den økonomiske gennemgang bidrager med at vise den dominerende perspektiv på arbejdsløse. Fokus ligger på at få folk i beskæftigelse (fra marginalisering til inklusion). Den sociologiske og socialpsykologiske gennemgang bidrager med at få et indblik i de arbejdsløses vilkår. Fokus på arbejdsløshed (marginalisering og eksklusion). Deprivation (Jahoda, Lazarsfeld og Zeizel 1971; Eisenberg og Lazarsfeld 1938; Tiffany, Cowan og Tiffany 1970; Warrs (987) er delvist brugbart i forståelsen af arbejdsløshed som havende en social og psykologisk påvirkvning på arbejdsløse. Halvorsen er delvis brugbar i forståelsen af at individerne bliver aktører som kan handle og har ressourcer. Social passage (Glaser og Strauss 1971)er relevant i vores definition af arbejdsløshed som midlertidigt.

% 
\begin{table}[H]
\centering
\caption{Oversigt over økonomiske og sociologiske teoriperspektiver}
\label{tab_spellrun}
\resizebox{0.6\textwidth}{!}{%
\begin{tabular}{@{}|l|l|l|@{}} \toprule
Tematik & Økonomi & Sociologi \\ \midrule
Begrebet &  &  \\ \midrule
Incitamenter & Søgeteori & Mestring \\ \midrule
Mentalt helbred & Hysterese & Deprivationsteori \\ \midrule
Det offentlige system & Principal-agent-model & \\ \bottomrule
\end{tabular} }
\end{table}
% 

Litteratur som gerne må læses ifm. med dette afsnit: 
%
 \begin{enumerate} [topsep=6pt,itemsep=-1ex]
   \item \sout{\parencite{Halvorsen1999}}
   \item (Dencker Larsen 2014, s. 17-38)
   \item \parencite{Baum2001, Baum2006}
   \item Goul Andersen
   \item ny forskning (web of science)
 \end{enumerate}




%%%%%%%%%%%%%%%%%%%%%%%%%%%%%%%%%%%%%%%%%%%%%%%%%%%%%%%%%%%

%Local Variables: 
%mode: latex
%TeX-master: "report"
%End: