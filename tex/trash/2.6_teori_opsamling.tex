% -*- coding: utf-8 -*-
% !TeX encoding = UTF-8
% !TeX root = ../report.tex


%%%%%%%%%%%%%%%%%%%%%%%%%%%%%%%%%%%%%%%%%%%%%%%%%%%%%%%%%%%
\newpage \section{\textsc{Opsamling \label{teori_operationalisering}}}
%%%%%%%%%%%%%%%%%%%%%%%%%%%%%%%%%%%%%%%%%%%%%%%%%%%%%%%%%%%

Hypoteser og operationalisering

% Beskæftiget, arbejdsløs og ikke deltagelse -> pointen her kædes sammen med information om grænsen mellem ikkedeltagelse og jobsøgning \parencite[117-118]{Cahub2004}

% Den økonomiske gennemgang bidrager med at vise den dominerende perspektiv på arbejdsløse. Fokus ligger på at få folk i beskæftigelse (fra marginalisering til inklusion). Den sociologiske og socialpsykologiske gennemgang bidrager med at få et indblik i de arbejdsløses vilkår. Fokus på arbejdsløshed (marginalisering og eksklusion). Deprivation (Jahoda, Lazarsfeld og Zeizel 1971; Eisenberg og Lazarsfeld 1938; Tiffany, Cowan og Tiffany 1970; Warrs (987) er delvist brugbart i forståelsen af arbejdsløshed som havende en social og psykologisk påvirkvning på arbejdsløse. Halvorsen er delvis brugbar i forståelsen af at individerne bliver aktører som kan handle og har ressourcer. Social passage (Glaser og Strauss 1971)er relevant i vores definition af arbejdsløshed som midlertidigt.


% Beskæftigelsesmobilitet ved arbejdsløshed
% %
%  \begin{enumerate} [topsep=6pt,itemsep=-1ex]
%    \item Hypotese 1: Arbejdsløse er mere fleksible i deres jobsøgning end beskæftigede
%    \item Hypotese 2: Beskæftigede er mere fleksible i deres jobsøgning end arbejdsløse
%    \item Hypotese 3: Ingen forskel
%  \end{enumerate}

% Forklaring på hypotese 1:
%  \begin{enumerate} [topsep=6pt,itemsep=-1ex]
%    \item DBO -> se eksamensopgave
%    \item Reservationsløn falder (søgeteori)
%    \item Søge væk fra fagområder
%    \item Nogen som ikke søger bredt akademikerne
%  \end{enumerate}





%Local Variables: 
%mode: latex
%TeX-master: "report"
%End: