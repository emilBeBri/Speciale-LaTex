% -*- coding: utf-8 -*-
% !TeX encoding = UTF-8
% !TeX root = ../report.tex

\chapter{appendiks} \label{appendiks}


\section{\texttt{SOCSTIL\_KODE} og {\texttt{SOCIO}  \label{appendiks_SOCSTIL_SOCIO}}

Følgende er en fyldestgørende beskrivelse af \texttt{SOCSTIL\_KODE} og\texttt{SOCIO}.


\subsection{\texttt{SOCSTIL\_KODE} \label{}}

\texttt{SOCSTIL\_KODE} er variablen “socioøkonomisk status”, som angiver befolkningens tilknytning til arbejdsmarkedet ultimo november. Tællingsenheden i statistikken er personen. Den primære tilknytning til arbejdsmarkedet bestemmes ved først at identificere de forskellige bruttobestande (tilknytninger til arbejdsmarkedet), den enkelte person indgår i ultimo november. Hvis en person indgår i mere end en bruttobestand, bestemmes den primære tilknytning til arbejdsmarkedet ud fra et sæt prioriteringsregler. Prioriteringsreglerne er fastlagt, således at de i videst muligt omfang følger ILO-retningslinierne. ILO-retningslinierne foreskriver, at beskæftigelse skal vægtes højere end ledighed. Prioriteringsrækkefølgen er kun ændret en smule over årene og fremgår af Statistiske Efterretninger (serien Arbejdsmarked), der udkommer årligt. I det følgende relaterer årstalsangivelserne sig til referencetidspunktet for tilknytningen til arbejdsmarkedet ultimo november. Den registerbaserede arbejdsstyrkestatistik pr. 1.1.2009 benævnes således, i denne sammenhæng, den registerbaserede arbejdsstyrkestatistik 2008, fordi tilknytningen til arbejdsmarkedet opgøres ultimo november 2008.

Variablen til opgørelse af den primære arbejdsmarkedstilknytning kaldes \texttt{ARBSTIL} i perioden 1980-1993, \texttt{NYARB} i perioden 1994-1995 og \texttt{SOCSTIL} i perioden 1996-2008. Variablen ændrer navn fra \texttt{ARBSTIL} til \texttt{NYARB}, fordi der indføres en mere detaljeret underopdeling af personerne uden for arbejdsstyrken i 1994. I 1996 ændrer variablen navn til SOCSTIL, fordi underopdelingen af lønmodtagerne ændres.

Beskæftigede og arbejdsløse udgør arbejdsstyrken. Den øvrige del af befolkningen er uden for arbejdsstyrken.
Variablen muliggør en opdeling af befolkningens primære status i tre overordnede kategorier (når variablen NOVPRIO=1 (novemberprioritering)):
\begin{enumerate} [topsep=6pt,itemsep=-1ex]
  \item Beskæftigede (\texttt{SOCSTIL}=115-135)
  \item Arbejdsløse (\texttt{SOCSTIL}=200-201)
  \item Personer uden for arbejdsstyrken (\texttt{SOCSTIL}=310-400)
\end{enumerate}

Beskæftigede og arbejdsløse udgør arbejdsstyrken. Den øvrige del af befolkningen er uden for arbejdsstyrken. Gruppen uden for arbejdsstyrken kan overordnet opdeles i fire kategorier i perioden fra 1994-2008. Følgende koder bliver hen-ført til de fire kategorier:
\begin{enumerate} [topsep=6pt,itemsep=-1ex]
  \item Midlertidigt uden for arbejdsstyrken: (\texttt{SOCSTIL}=316-323,327,333-335)
  \item Tilbagetrækning fra arbejdsstyrken: (\texttt{SOCSTIL}=315,324-325)
  \item Pensionister: (\texttt{SOCSTIL}=328,329,331)
  \item Andre uden for arbejdsstyrken: (\texttt{SOCSTIL}=310,326,330,332,400)
\end{enumerate}


\subsubsection{\texttt{SOCSTIL\_KODE}: Dannelse og databrud \label{}}

Befolkningen er henført til en af de tre hovedgrupper efter samme metode i perioden 1980-2001. I 2002 er der et databrud i statistikken, der bevirker, at beskæftigelsen falder med ca. 17.500 som følge af databruddet. Databruddet var nødvendigt, fordi forandringer på arbejdsmarkedet samt nye datakilder gjorde det hensigtsmæssigt at foretage ændringer i statistikken.

De fire væsentligste ændringer i 2002 er:
\begin{itemize} [topsep=6pt,itemsep=-1ex]
  \item Der bliver strammet op på kravene til aktivitet (målt ved overskud/underskud) for at tælle med som selvstændig
  \item For lønmodtagere i den offentlige sektor bliver lønstatistikken introduceret som supplerende kilde til opgørelse af arbejdsmar-kedstilknytning ultimo november.
  \item Efterlønsmodtagere der samtidig er beskæftigede bliver klassificeret som beskæftigede.
  \item En person, der ikke med sikkerhed er lønmodtager på referencetidspunktet, bliver ikke længere opgjort som lønmodtager i til-fælde, hvor personen ikke er fuldt arbejdsløs og samtidig er aktiveret i en foranstaltning uden løn.
 \end{itemize}

Databruddet i 2002 påvirker den relative størrelse af de tre grupper og betyder, at antallet af selvstændige falder med ca. 10.000, hvor nogle af de selvstændige i stedet bliver klassificeret som lønmodtager samt at antallet af lønmodtagere falder med ca. 7.500, hvor det især er beskæftigelsen i offentlig forvaltning og service, der falder.

I 2008 er der et større brud i statistikken som følge af, at statistikken overgår til at anvende eIndkomst som datagrundlag for lønmodtagerbeskæftigelsen. Det hidtidige datagrundlag har været det centrale oplysningsseddelregister. Det centrale oplysningsseddelregister var dannet på baggrund af årlige indberetninger fra arbejdsgiverne til SKAT. På oplysningssedlen kunne arbejdsgiveren angive, at lønmodtageren var beskæftiget hele året, eller hvis dette ikke var tilfældet angive fra- og til-datoer for ansættelsesforholdets varighed. Danmarks Statistik er dog af den formodning, at helårsmarkeringen nok er blevet brugt i for mange tilfælde. EIndkomstregistret baserer sig på månedlige indberetninger fra arbejdsgiveren. Det betyder, at oplysningen om, hvornår et ansættelsesforhold rent faktisk er gældende, har fået en større grad af sikkerhed. Dette påvirker niveauet for beskæftigelsen i nedadgående retning. Hvis der sammenlignes med udviklingen i det kvartalsvise arbejdstidsregnskab fra 4. kvt. 2007 til 4. kvt. 2008, betyder overgangen til eIndkomst at beskæftigelsen falder med 60.000-70.000 alene som følge af over-gangen til eIndkomst. I 2008 er der anvendt en årsversion af eIndkomstregistret, der er dannet af SKAT ved en summation af de månedlige indberetninger for året. Ved de fremtidige offentliggørelser vil den registerbaserede arbejdsstyrkestatistik blive direkte baseret på de månedlige indberetninger til eIndkomst.

Til opgørelsen af personer uden for arbejdsstyrken blev der før 2008 anvendt oplysninger fra mange forskellige statistikregistre. En del af disse statistikregistre blev efterfølgende samlet i statistikken om personer uden ordinær beskæftigelse. Inden for rammerne af dette statistiksystem foregår der en tværgående databehandling, hvilket betyder, at kvaliteten af disse oplysninger er bedre end hidtil.

Tidligere har det udelukkende været lønmodtagere, der havde en summeret årsløn svarende til ca. 10.000 kr., der kunne blive klassificeret som lønmodtagere. Dette krav var indført, fordi datagrundlaget for lønmodtagerbeskæftigelsen var årsbaseret, og periodeangivelserne var usikre. Som følge af de mere sikre periodeangivelser i det nye datagrundlag er kravet til årslønnen reduceret. Fra 2008 skal en lønmodtager blot som minimum have en løn, der svarer til fire timers beskæftigelse til garantiløn i løbet af året for at kunne blive klassificeret som lønmodtager ultimo november.

Personer der har en årsløn der svarer til fire timers beskæftigelse bliver fra 2008 opgjort som beskæftigede ultimo november, såfremt de er beskæftigede ifølge periodeangivelserne i eIndkomstregistret, eller såfremt der er indbetalt ATP i året, der svarer til maksimum beløbet for den givne ATP-sats. Når en person har flere job som lønmodtager ultimo november, er der foretaget en prioritering mellem jobbene ud fra enten årlige oplysninger (fra SKAT's årlige eIndkomstregister) eller de månedlige oplysninger fra det månedlige eIndkomstregister.

Ændringerne i datagrundlaget for lønmodtagerbeskæftigelsen bevirker, at beskæftigelsesniveauet for lønmodtagere bliver lavere. Strukturelt ses påvirkningen af databruddet især for de yngre aldersgrupper, dvs. de 18-29-årige, som har faldende beskæftigelsesfrekvenser. Personer i denne aldersgruppe har ofte job, der ligger drypvis hen over året. Netop denne gruppe må antages at have været særligt udsat for tendensen til, at arbejdsgiveren har indberettet, at personen var ansat hele året, selvom det ikke nødvendigvis var tilfældet. For de helt unge aldersgrupper er der derimod en stigning i beskæftigelsesfrekvenserne. Årsagen hertil er, at kravet om en summeret løn på minimum ca. 10.000 kr. ikke længere er gældende. Herudover falder beskæfti-gelsen i vikarbranchen markant. Dette kan ligeledes forklares med, at de månedlige indberetninger giver et øget kvalitet m.h.t. periodiseringen af de enkelte ansættelsesforhold.


\subsubsection{\texttt{SOCSTIL\_KODE}: Selvstændige \label{}}

De beskæftigede kan underopdeles i grupperne:
\begin{itemize} [topsep=6pt,itemsep=-1ex]
  \item Selvstændige (\texttt{SOCSTIL}=115-118)
  \item Medarbejdende ægtefæller (\texttt{SOCSTIL}=120)
  \item Lønmodtagere (\texttt{SOCSTIL}=130-135)
\end{itemize}

Fra 1999 indhentes der oplysninger fra a-kassen ASE, der anvendes til en bedre brancheplacering af arbejdsløshedsforsikrede selvstændige. Fra 2008 udgår gruppen af arbejdsløshedsforsikrede selvstændige af statistikken. Antallet af arbejdsløshedsforsikrede selvstændige var i forvejen beskedent. I 2007 var der således kun 2.897 arbejdsløshedsforsikrede selvstændige. Analyser viste, at de arbejdsløshedsforsikrede selvstændige i mange tilfælde var tidligere momsbetalere, der ikke længere var selvstændige. Det var derfor mere retvisende at lade dem udgå af statistikken.


\subsubsection{\texttt{SOCSTIL\_KODE}: Lønmodtagere \label{}}

Beskæftigelsesopgørelsen for lønmodtagere baserer sig i perioden 1980 til 2007 på de oplysninger om ansættelsesperioder, som arbejdsgiverne angiver på oplysningssedlen. Disse oplysninger anvendes ikke til administrative formål af SKAT, og derfor er kvaliteten ikke altid den bedste. Det har især betydning for de befolkningsgrupper, der ikke udgør den egentlige kernearbejdsstyrke, fx yngre og ældre personer som ofte har en løsere tilknytning til arbejdsmarkedet. Fra 2008 overgår statistikken til at anvende de månedlige eIndkomst indberetninger som datagrundlag og dermed øges kvaliteten af oplysningerne om ansættelsesperioderne. Fra 2008 anvendes lønstatistikkens inddata i stedet for lønstatistikkens slutregister. Valideringsarbejdet i lønstatistikken betyder i praksis bl.a., at en del indberetninger kasseres. I mange tilfælde har disse indberetninger dog gyldige disco-koder og kan der-for anvendes i beskæftigelsesstatistikken. Overgangen til at anvende lønstatistikkens inddata betyder, at antallet af lønmodtagere uden nærmere angivelse (dvs. med uoplyst arbejdsfunktion) falder markant. Før 2008 betød valideringsarbejdet i lønstatistikken kombineret med ændringer i dannelsen af arbejdsfunktionen i arbejdsklassifikationsmodulet over tid, at antallet af indberetninger i den private sektor kunne svinge fra år til år. Derfor tilrådes det at anvende informationen om socioøkonomisk status for lønmodtagere med forsigtighed. Analyser af udviklingen i antallet af lønmodta-gere på et bestemt færdighedsniveau bør således altid ses i sammenhæng med udviklingen i antallet af lønmodtagere uden nærmere angivelse.


\subsubsection{\texttt{SOCSTIL\_KODE}: Personer uden for arbejdsmarkedet \label{}}

Fra 1994 og frem sker der en yderligere underopdeling af personerne uden for arbejdsstyrken. Ændringen skyldes øgede brugerbehov for mere detaljerede oplysninger om personer uden for arbejdsstyrken, bl.a. foranlediget af den aktive arbejdsmarkedspolitik. Overordnet kan grupper uden for arbejdsstyrken aggregeres til følgende fire grupper i hele perioden:
\begin{itemize} [topsep=6pt,itemsep=-1ex]
  \item Midlertidigt uden for arbejdsstyrken: (\texttt{SOCSTIL} =316-323,327,333-335)
  \item Tilbagetrækning fra arbejdsstyrken: (\texttt{SOCSTIL}=315,324-325)
  \item Pensionister: (\texttt{SOCSTIL}=328,329,331)
  \item Andre uden for arbejdsstyrken: (\texttt{SOCSTIL}=310,326,330,332,400)
\end{itemize}

Personer midlertidigt uden for arbejdsstyrken kan opdeles i følgende undergrupper:
\begin{enumerate} [topsep=6pt,itemsep=-1ex]
  \item Beskæftigelse uden løn
  \item Uddannelsesforanstaltninger/ fra 2007 vejledning og opkvalificering
  \item Uoplyst aktivering (udgår i 2000)
  \item Særlig/anden aktivering (udgår i 2007)
  \item Integrationsuddannelse (tilkommer i 2001)
  \item Arbejdsmarkedsorlov
  \item Barselsdagpenge
  \item Sygedagpenge
  \item Revalidering
  \item Modtagere af ledighedsydelse (tilkommer i 2003)
  \item Aktivering ifølge kontanthjælpsstatistikken (tilkommer i 2003 og udgår i 2008)
  \item Delvist ledige (tilkommer i 2008)
\end{enumerate}

\textbf{Undergruppe 1-6} er dannet på baggrund af samkøring af oplysninger fra statistikken over arbejdsmarkedspolitiske foranstalt-ninger og består af personer uden for arbejdsstyrken (både forsikrede og ikke-forsikrede), der er i en arbejdsmarkedspolitisk foranstaltning eller på arbejdsmarkedsorlov ultimo november. Kategoriseringen af de arbejdsmarkedspolitiske foranstaltninger følger generelt kategoriseringen i statistikken over arbejdsmarkedspolitiske foranstaltninger. I øvrigt henvises til excel-arket socstil_foranst.xls, der indeholder en oversigt over, hvilke foranstaltninger der indgår i de seks grupper. Indtil 2001 var det et krav at personen skulle komme fra en tilstand som ledig eller fra en arbejdsmarkedspolitisk foranstaltning uden løn for at blive klassificeret i kategori 6. Fra 2002 er det ikke længere et krav. \textbf{Undergrupperne 7-8} er dannet på baggrund af samkøring af oplysninger fra sygedagpengeudbetalingsregistret. Indtil 2001 var det et krav at personen skulle komme fra en tilstand som ledig eller fra en arbejdsmarkedspolitisk foranstaltning uden løn for at blive klassificeret i kategori 7-8. Fra 2002 er det ikke længere et krav. \textbf{Undergruppe 9} består af personer, der modtager en revalideringsydelse i november. Revalideringsydelse kan være en almindelig revalideringsydelse, revalideringsydelse i forbindelse med virksomhedsrevalidering eller forrevalidering. \textbf{Undergruppe 10} består af modtagere af ledighedsydelse i november. Det blev allerede muligt at modtage ledighedsydelse i 2000, men gruppen er først indført i statistikken fra 2003. \textbf{Undergruppe 11} består af personer, der er aktiverede ifølge kontanthjælpstatistikken. Personerne indgår ikke i statistikken over arbejdsmarkedspolitiske foranstaltninger på referencetidspunktet ultimo november. Statistikken over arbejdsmarkedspolitiske foranstaltninger og kontanthjælpsstatistikken er ikke harmoniserede, hvilket betyder, at der ikke nødvendigvis er overensstemmelse mellem, hvorledes en given person optræder i de to statistikker. Gruppe 11 er indført fra 2003. Grupperingen af undergrupperne 9-11 følger grupperingen i statistikken “Hjælp efter lov om aktiv socialpolitik”. Fra 2008 er oplysninger for \textbf{undergrupperne 1-11} udelukkende baseret på statistikken om personer uden ordinær beskæftigelse. \textbf{Undergruppe 12} består af personer, der er delvist ledige i referenceugen ifølge statistikken for personer uden ordinær beskæfti-gelse.

Personer i tilbagetrækning fra arbejdsstyrken kan opdeles i følgende undergrupper;
1. Modtagere af efterløn
2. Modtagere af overgangsydelse (udgår i 2007)
3. Modtagere af flexydelse (tilkommer i 2008)

Indtil 2002 blev modtagere af efterløn og overgangsydelse klassificeret som værende uden for arbejdsstyrken, uanset om de arbejdede ved siden af. Årsagen hertil var, at antallet af beskæftigede ellers ville blive for højt grundet de lidt usikre dateringer af ansættelsesperioderne. Fra 2002 anvendes der oplysninger fra Arbejdsmarkedsstyrelsens RAM-register, der muliggør en mere nøjagtig vurdering af, om personen er beskæftiget. Derfor vil efterlønsmodtagere der arbejder ved siden af blive opgjort som beskæftigede fra 2002 og frem.

Før 1997 anvendtes oplysninger fra arbejdsklassifikationsmodulet til at afgrænse personer, der modtager efterløn eller overgangsydelse. Fra 1997 og frem anvendes oplysninger fra statistikken over arbejdsmarkedspolitiske foranstaltninger, hvilket betyder, at der sker mere præcis bestemmelse af tilbagetrækning fra arbejdsstyrken på referencetidspunktet.

Kilden til gruppen der modtagere flexydelse er statistikken om personer uden ordinær beskæftigelse.

Gruppen af pensionister kan opdeles i følgende undergrupper:
\begin{enumerate} [topsep=6pt,itemsep=-1ex]
  \item Alderspensionister
  \item Førtidspensionister
  \item Modtagere af tjenestemandspension/modtagere af anden pension
\end{enumerate}

\textbf{Alders- og førtidspensionister} består af personer, der modtager folke- hhv. førtidspension (inkl. pensioner fra ATP). Aldersgrænsen mellem førtids- og alderspension går indtil 2004 ved 67 år. Pr. 1. juli 2004 nedsættes pensionsalderen til 65 år for personer der fylder 65 år efter denne dato. Derfor kan 65-årige både være førtids- og alderspensionister i 2004. I 2005 gælder det samme for 66-årige. Fra 2007 er alle fra 65 år opgjort som alderspensionister givet at de ikke er i beskæftigelse. \textbf{Modtagere af anden pension/tjenestemandspension} består fra og med 2004 af personer, som ifølge indkomststatistikregistret modtog tjenestemandspension, pensionsudbetalinger fra pensionskasse, pengeinstitut, forsikringsselskab, privat pensionsord-ning inkl. arbejdsgiveradministrerede, pension fra udlandet og pension fra en tidligere arbejdsgiver i årets løb. Indtil 2003 bestod gruppen kun af modtagere af tjenestemandspension. Det betyder, at antallet stiger fra 2003 til 2004.

Gruppen “andre uden for arbejdsstyrken” kan opdeles i følgende undergrupper;
\begin{enumerate} [topsep=6pt,itemsep=-1ex]
  \item Kontanthjælpsmodtagere
  \item Modtagere af introduktionsydelse (tilkommer i 2001)
  \item Børn og unge
  \item Personer under uddannelse
  \item Øvrige uden for arbejdsstyrken
\end{enumerate}

Før 2008 er datakilden til (ikke arbejdsmarkedsparate) kontanthjælpsmodtagere og modtagere af introduktionsydelse kontanthjælpsstatistikregistret. Fra 2008 bliver datakilden statikken om personer uden ordinær beskæftigelse. Børn og unge er dannet på baggrund af alderen ultimo november. Personer under uddannelse består af personer, der er i gang med en ordinær uddannelse ifølge uddannelsesstatistikregistret. Gruppen “øvrige uden for arbejdsstyrken” er en restgruppe, som det ikke er muligt at finde arbejdsmarkedsrelevante informationer om. Den sekundære og tertiære/ikke-novemberrelaterede tilknytning til arbejdsmarkedet De beskæftigedes eventuelle sekundære tilknytning til arbejdsmarkedet som lønmodtager kan identificeres via NOVPRIO=2 (novemberpriotering) og SOCSTIL=139. Job som lønmodtager, der ikke relaterer sig til ultimo november, eller som er det tertiæ-re job ultimo november har NOVPRIO=3 eller NOVPRIO=9 (novemberprioritering) og SOCSTIL=138.




%Local Variables: 
%mode: latex
%TeX-master: "report"
%End: 