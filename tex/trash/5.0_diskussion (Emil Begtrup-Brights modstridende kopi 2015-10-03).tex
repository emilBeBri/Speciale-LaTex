% -*- coding: utf-8 -*-
% !TeX encoding = UTF-8
% !TeX root = ../report.tex


%%%%%%%%%%%%%%%%%%%%%%%%%%%%%%%%%%%%%%%%%%%%%%%%%%%%%%%%%%%
\chapter{\textsc{Diskussion} \label{analyse}}
%%%%%%%%%%%%%%%%%%%%%%%%%%%%%%%%%%%%%%%%%%%%%%%%%%%%%%%%%%%

% Den symbolske kamp om de arbejdsløse: Vores kortlægning af arbejdsløses strategi set i lyset af den symbolske kamp om at definere de arbejdsløse som afgrænset området - politiske kampe

% Gisninger om et eller andet

% Økonomi versus sociologi -> konkret/abstrukt, induktivt/deduktivt, forudsigelse/forklaring, kvantitativt/kvalitativt

% Økonomernes søgeteori: Arbejdsløshed ses ikke som en udelukkende negativ ting, fordi jobsøgning kan ses som en investering med sigte om at få et bedre job \parencite[27-28]{Halvorsen1999}.

% Moneca versus økonomerne: Sammenligne vores kortlægninger farvet med indkomst med økonomperspektiver om reservations - hos bevæger man sig et socialt rum (fleksibilitet), hvor hos økonomerne er lønudvikling den primære driftkraft... Inden for segmenterne vil vi se forskellige lønudviklinger - der kan være ens løn i segmenterne og der kan være stor forskel på løn i segmenterne.

% Korttidsledig versus langtidsledig: Vi har 1 års ledighed som absolut minimum, hvor man kan se at man strækker sig langt i søgnming er jobs. Er det strukket til sit yderste. Det er et godt spørgsmål. Vi kan meget grovt sige det om over et år. Spørgsmålet er om vi kan dissekere det mere fint ved at fokusere på forskellige ledighedsperioder.

% Arbejdsløshed i DK - vores periode er de gyldne år fra 1996-2009 - der er ikke krise

% Kritik af det fleksible arbejdsmarked - det fleksible arbejdsmarked hverken muliggør eller tillader et engagement i eller en begejstring ved nogen aktuel beskæftigelse \parencite[58]{Bauman2002}. Ifølge Bauman går økonomisk vækst hånd i hånd med udskiftningen af overenskomstmæssige jobs med fleksibelt arbejde, tab af jobsikkerhed i kraft af løbende kontrakter, tidsbegrænsede job og lejlighedsvise ansættelser, med nedsækæringer, omstruktureringer og rationalsiiseringer,d er alt sammen fører til lavere beskæftigelse \parencite[67]{Bauman2002}.




%%%%%%%%%%%%%%%%%%%%%%%%%%%%%%%%%%%%%%%%%%%%%%%%%%%%%%%%%%%
% Trash
%%%%%%%%%%%%%%%%%%%%%%%%%%%%%%%%%%%%%%%%%%%%%%%%%%%%%%%%%%%


%Local Variables: 
%mode: latex
%TeX-master: "report"
%End: 