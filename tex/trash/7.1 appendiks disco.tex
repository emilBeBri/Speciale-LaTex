% -*- coding: utf-8 -*-
% !TeX encoding = UTF-8
% !TeX root = ../report.tex

\chapter{appendiks} \label{appendiks}


\section{disco-88 \label{}}

DISCO-88 er den officielle danske version af den internationale fagklassifikation, International Standard Classification of Occupations (ISCO-88), som er blevet udviklet af International Labour Organisation (ILO). DISCO-88 afløser i 1996 fagklassifikationen fra 1981 baseret på en dansk fagklassifikation. I 2010 blev ISCO-88 udskiftet af ISCO-08.

De tre hovedformål med ISCO-88 er for det første at lette international kommunikation angående
beskæftigelsesforhold ved at forsyne nationale statistikere med et redskab, som kan
gøre nationale beskæftigelsesdata internationalt sammenlignelige, for det andet ønsket at åbne mulighed for at producere internationale beskæftigelsesdata på en sådan måde, at data kan være nyttige i såvel forsknings- og beslutningsprocesser som ved konkret iværksættelse af aktiviteter, og for det tredje er det intentionen at tilvejebringe en model, der kan fungere som en
velegnet inspirationskilde i de lande, hvor man arbejder med at udvikle eller revidere
fagklassifikationer.

Opbygningen af fagklassifikationen baserer sig på arbejdsfunktioner og færdigheder. Med arbejdsfunktioner menes typen af det faktisk udførte arbejde, hvilket vil sige summen af de udførte opgaver, der knytter sig til den enkelte person beskæftigelse. 
Med færdigheder menes evnen til at varetage opgaverne inden for en givens arbejdsfunktion, hvilket for det første afhænger af niveauet af færdighederne hos den enkelte person (kompleksiteten og omfanget af de opgaver der indeholdes i en given arbejdsfunktion), og for det andet afhænger specialiseringen af færdighederne hos den enkelte person såvel af påkrævet viden, anvendte maskiner, værktøjer og materialer som af typen af producerede varer og tjenesteydelser.

Færdighederne er opdelt i en niveauer med udgangspunkt i den internationale uddannelsesklassifikation, International Standard Classification of Education (ISCED) som i store træk kan sammenlignes med Dansk Uddannelses-Nomenklatur (DUN), som er udarbejdet efter danske forhold af Danmarks Statistik og Undervisningsministeriet. ISCED placeres på de 10 uddannelsesniveauer:
  \item[0.] Førskoleniveau (børnehaveklasse)
  \item[1.] Grundskoleniveau I (1.-6. klasse)
  \item[2.] Grundskoleniveau II (7.-10. klasse/årgang)
  \item[3.] Gymnasialt niveau I (10. uddannelsesår)
  \item[4.] Gymnasialt niveau II (11.-12. uddannelsesår)
  \item[5.] Korte videregående uddannelser (13.-14. uddannelsesår)
  \item[6.] Mellemlange videregående uddannelser (15.-16. uddannelsesår)
  \item[7.] Lange videregående uddannelser (17.-18. uddannelsesår)
  \item[8.] Forskerniveau (19.- uddannelsesår)
  \item Udenfor niveauplacering
\end{description}
Det bør ligeledes understreges, at færdighederne kan erhverves gennem uformel oplæring og erfaring.

ISCO-88 og DISCO-88 opererer med fire færdighedsniveauer. Første færdighedsniveau er defineret med henvisning til DUN uddannelsesniveau 0-1, som man i Danmark i praksis uden uddannelse. Andet færdighedsniveau er defineret med henvisning til DUN uddannelsesniveau 2-4, som er personer i besiddelse af færdigheder på grundniveau. Tredje færdighedsniveau er defineret med henvisning til DUN uddannelsesniveau, som er personer i besiddelse af færdigheder på mellemniveau. Fjerde færdighedsniveau er defineret med henvisning til DUN uddannelsesniveau 7-8, som er personer i besiddelse af færdigheder på højeste niveau.

Stillingsbeskrivelsenserne samler et sæt af arbejdsfunktioner, der i indhold og opgaver er karakteriseret ved en høj grad af ensartethed. Stillingsbeskrivelserne er opbygget som en hierarkisk struktur med fire niveauer: 10 hovedgrupper, 27 overgrupper, 111 mellemgrupper og 372 undergrupper. Detaljeringsgrad øges, jo lavere niveau i klassifikationen man betragter. De 10 hovedgrupper fremgår således:
\begin{description}
  \item[0.] Militært arbejde
  \item[1.] Ledelse på øverste plan i virksomheder, organisationer og den offentlige
sektor
  \item[2.] Arbejde, der forudsætter færdigheder på højeste niveau inden for pågældende område
  \item[3.] Arbejde, der forudsætter færdigheder på mellemniveau
  \item[4.] Kontorarbejde
  \item[5.] Salgs-, service- og omsorgsarbejde
  \item[6.] Arbejde inden for landbrug, gartneri, skovbrug, jagt og fiskeri, der forudsætter færdigheder på grundniveau
  \item[7.] Håndværkspræget arbejde
  \item[8.] Proces- og maskinoperatørarbejde samt transport- og anlægsarbejde
  \item[9.] Andet arbejde
\end{description}

I DISCO-88 henføres hovedgruppe 2 til fjerde færdighedsniveau og hovedgruppe 3 til tredje færdighedsniveau. Hovedgrupperne 4-8 henføres til andet færdighedsniveau og hovedgruppe 9 til første færdighedsniveau. Arbejdet, som indeholdes i såvel hovedgruppe 1 som hovedgruppe 0, er så heterogent, at det med hensyn til disse hovedgrupper ikke er muligt at foretage en entydig henføring til et færdighedsniveau.

Principielt skelner DISCO-88 ikke mellem selvstændigt erhvervsdrivende og lønmodtagere, hvis de udfører samme arbejdsfunktioner. Tilsvarende skelnes ikke mellem arbejdsgivere og arbejdstagere, hvis de faktisk udfører samme art/type
arbejde. 

I Danmarks Statistik klassificeres efter to principper: Arbejdsklassifikationsmodulet (AKM) og Den registerbaserede arbejdsstyrkestatistik (RAS). AKM klassificerer den enkelte person efter væsentligste beskæftigelse (beskæftigelsesstatus, stilling og branche for vigtigste arbejdssted) samt omfanget af beskæftigelse og ledighed i løbet af et kalenderår. RAS klassificerer den enkelte personer efter dennes tilknytning til arbejdsmarkedet på et givet tidspunkt i løbet af året (slutningen af november) svarende til oplysningerne om ansættelsesperiode fra den oplysningsseddel, som arbejdsgiverne hvert år skal aflevere til Told- og Skattestyrelsen for hver enkelt ansat.


\section{discoalle\_indk \label{}}

Discoalle\_indk anvender vi i perioden 1996 til 2009, fordi den klassificerer den væsentligste beskæftigelse i året på baggrund af DISCO-88. Discoalle\_indk går fra 1991 til 2009 og udskiftes i 2010 af variablen DISCO08ALLE\_INDK.

DISCOALLE\_INDK er for lønmodtagere lig med fagklassifikationskoden for det
arbejdssted, hvor de har fået størst løn i løbet af året (1991-2008) og fra 2009 det
arbejdssted hvor der er arbejdet flest timer i løbet af året. Hovedkilden til DISCO-
koderne er indberetninger til Danmarks Statistiks lønstatistikregister. For selvstændigt erhvervsdrivende og medarbejdende ægtefælle er DISCOALLE\_INDK lig med fagklassifikationskoden for det arbejde, de udfører i deres virksomhed.

Fra 1996 til 2002 er DISCOALLE\_INDK fircifret, og fra 2003 sekscifret. De sidste 2 cifre
fra 2003 er en underopdeling af den fircifrede kode. Nogle af DISCO-værdierne har et- to- eller trecifrede, hvilket skyldes, at det ikke har været muligt at fastlægge koden på et mere detaljeret niveau - hovedsageligt for imputerede værdier (kodeværdi sat ud fra oplysninger fra andre registeroplysninger end fagklassifikationsoplysninger fra Danmarks Statistiks lønstatistik). 

Populationen i datasættet omfatter alle personer som er mindst 15 år ved årets udgang eller har indkomst i løbet af året eller formue den 31. december. For at få konsistens over tid i vedhæftede tabel og graf er populationen lig med alle fuldt skattepligtige personer, som har været i Danmark både primo og ultimo året, og som ved årets udgang er mindst 15 år.


\section{problemer med discoalle\_indk \label{}}

Sammenligninger over tid på mere end første eller to første cifre i DISCOALLE\_INDK skal foretages med varsomhed, da dannelsen af DISCOALLE\_INDK ændrer sig over tid, hvilket fremgår af følgende:
\begin{description}
  \item Fra og med 1. januar 1997 flytter frisørerne til Maritim A-kasse - dette er først blevet indarbejdet i modellen for dannelse af DISCOALLE\_INDK fra og med 1999, hvilket betyder, at lidt over 1.000 personer i 1997 og 1998 har fået DISCO-kode 6152 i stedet for 5141.
  \item Fra og med 2000 udgår stillingsoplysninger fra CPR-registeret, da kvaliteten efterhånden er blevet for dårlig, hvilket giver 470.000 ekstra manglende DISCO-koder fra år 2000.
  \item Fra år 2003 ændres opgørelsesmetoden væsentligt. Den største ændring er en følge af, at en række arbejdsløshedskasser lægges sammen. Det bevirker at de ikke kan bruges som grundlag for imputering af DISCO-koden. Totalt stiger antallet af personer med manglende DISCO-kode med 535.000. Samtidig ændres DISCO-koden for selvstændige med under 5 ansatte. De skifter fra ledelse til at være håndværker eller lignende (den relevante DISCO-kode for virksomhedens område).
  \item Før år 2003 fik personer uden for arbejdsmarkedet en imputeret DISCO-kode, hvis der var oplysning om medlemskab af a-kasse for tidligere år, og før 2000 tillige, hvis der var oplysning om stilling fra CPR-registeret. Det omfatter for årene før år 2000 ca. 600.000 personer og for årene 2000-2002 ca. 300.000 personer. Det betyder, at opgørelser af DISCOALLE\_INDK før år 2003 kan indeholde DISCO-kode for arbejde i tidligere år for fx pensionister.
\end{description}
 

\section{dannelse af discoalle\_indk \label{}}

Variablen DISCOALLE\_INDK dannes ud fra DISCOLO-EN\_INDK (kode for lønmodtager) og DISCOSEL\_INDK (kode for selvstændige) samt BESKST/BESKST02 (beskæftigelses-status). DISCOALLE\_INDK dannes ud fra formlen: DISCOALLE\_INDK=DISCOSEL\_INDK for BESKST=1 eller 2 (selvstændige, medarbejdende ægtefælle), DISCOALLE\_INDK=DISCOLOEN\_INDK for BESKST>2 (lønmodtagere, pensionister mv.).

Sammenligninger over tid på mere end de to første cifre i DISCOALLE\_INDK skal foretages med varsomhed, da:
\begin{description}
  \item der er forskellig metode til opgørelse af DISCO-koden for selvstændige og lønmodtagere
  \item der er forskellige opgørelsesmetoder for lønmodtagere afhængig af sektor (privat, stat, kommunal)
  \item der hvert år mangler koder for en del lønmodtagere, som så søges imputeret ved forskellige metoder
  \item opgørelsesmetoden for statsligt ansatte er ændret i 1998
  \item opgørelsesmetoden er ændret for selvstændige i år 2003
  \item imputeringsmetoden for lønmodtagere er ændret i 2000, 2003 og 2004
\end{description}

DISCO-selvstændige (personer med BESKST/BESKST02 =1,2) sættes ud fra deres virksomheds BRANCHE. Før 2003 bliver 30 pct. af de selvstændige karakteriseret som ledere med under 10 ansatte (to første cifre i DISCOALLE\_INDK lig 13). Fra 2003 ændres metoden hvorefter selvstændige får tildelt DISCO-kode. Derved falder andelen af selvstændige, der karakteriseres som ledere med under 10 ansatte fra 30 pct. af de selvstændige i 2002 til tre pct. i 2003. Hovedreglen fra 2003 er, at selvstændige med under fem ansatte antages at arbejde med i produktionen, og dermed får DISCO-kode som håndværker, butiksansat og lign (kodeværdi større end 200000). Selvstændige med fem til ni ansatte antages at være ledere og får DISCO-kode mellem 130000 og 131900 (ledelse af virksomheder med færre end 10 ansatte). Selvstændige med mindst 10 ansatte får DISCO-kode mellem 122000 og 123900 (ledelse i virksomheder med 10 eller flere ansatte).

Hovedkilden til DISCO-løn i indkomststatistikken er Danmarks Statistiks Lønstatistiks DISCO-koder. Først findes den sektor (privat, stat og kommune) hvorfra personen har fået størst lønudbetaling i løbet af året, og personens DISCO-kode fra denne sektor i lønstatistikken vælges. Hvis der ikke findes en DISCO-kode i den valgte sektor, tages DISCO kode fra en af de andre sektorer, hvis en sådan findes. Hvis en lønmodtager ikke har nogen DISCO-kode i lønstatistikken, søges koden imputeret ud fra forskellige oplysninger.

1991 til 2002 foregår imputeringen ved følgende prioritering af kilderne: 
\begin{description}
  \item[1.] For personer under visse uddannelser (typisk med lærlingeperiode) sættes en DISCO-kode.
  \item[2.] DISCO-kode sættes for visse stillingstyper i centrale personregister (anvendes ikke fra og med 2000).
  \item[3.] DISCO-kode sættes for medlemskab af visse A-kasser og for enkelte ud fra kombination med højeste fuldførte kompetencegivende uddannelse og arbejdsstedsbranche.
  \item[4.] DISCO-kode sættes for visse kombinationer af højeste fuldførte kompetencegivende uddannelse og arbejdsstedsbranche.
\end{description}

Fra år 2003 foregår imputeringen ved følgende prioritering af kilderne:
\begin{description}
  \item[1.] Hvis arbejdsstedet er det samme i året forud tages DISCO-koden fra det foregående års lønstatistik (bruges fra og med 2004).
  \item[2.] DISCO-kode sættes for visse kombinationer af højeste fuldførte kompetencegivende uddannelse og arbejdsstedsbranche.
  \item[3.] For personer under visse uddannelser (typisk med lærlingeperiode) sættes en DISCO-kode.
  \item[4.] DISCO-kode sættes for medlemskab af visse A-kasser og for enkelte ud fra kombination med højeste fuldførte kompetencegivende uddannelse og arbejdsstedsbranche.
\end{description}

Variablen DISCOTYP angiver, hvilken af de ovenfor nævn-te kilder DISCO-kode for lønmodtagere har i indkomst-statistikken. 

For de privatansattes vedkommende drejer det sig om op-lysninger fra alle virksomheder med 10 eller flere ansatte, som indberetter sekscifrede DISCOLØN koder sammen med lønbeløbet for den pågældende ansættelse. Enkelte mindre virksomheder indberetter frivilligt eller gennem Dansk Arbejdsgiverforening. Men de har altså ikke indberetningpligt. For de offentligt ansattes vedkommende hentes stillingsoplysninger i forbindelse med offentlig lønoplysninger og konverteres til DISCO-koder. For kommunerne konverteres kommunale stillingskoder til DISCO-koder. For statsansatte anvendtes samme metode som for kommunaltansatte før år 1998. Fra 1998 sættes koderne i de statslige lønkontorer ved, at de godkender en tidligere sat kode eller selv sætter en ny.


\section{dannelse af vores disco-koder \label{}}

%---------------------------------------------------------------------------------------
%		DISCOALLE\_INDK 			-> 		DISCO\_4 		->		DISCO\_MONECA
%---------------------------------------------------------------------------------------
% sekscifret kode, 2003-2009		fircifret kode
% fircifret kode, 1996-2002
%---------------------------------------------------------------------------------------
%		400/800 udfald					400 udfald				167 udfald
%---------------------------------------------------------------------------------------


Vi anvender DISCOALLE\_INDK i perioden 1996 til 2009 til en variabel på 167 udfald.

Årsagen til at vi har valgt perioden 1996 til 2009 er, at det er en forholdsvis stabil periode at bruge fagklassifikationen disco88, da den før 1996 tager udgangspunkt i en dansk fagklassifikation og den efter 2009 har nogle væsentlige skifte til DISCO08. De mest væsentlige skift sker i 2003.

Fra 1996 til 2002 er DISCOALLE\_INDK fircifret og har mellem 475 og 483 udfald. Fra 2004 til 2009 er DISCOALLE\_INDK sekscifret og mellem 732 og 805 udfald. I 2003 er DISCOALLE\_INDK sekscifret og har 1776 udfald.

\section{første skridt \label{}}

Det første vi gør er, at gøre DISCOALLE\_INDK fircifret for alle år. Det gør vi ved at fjerne de to sidste cifre. I praksis betyder det som nedenfor, at kategorierne bliver reduceret:

% 1000 ->	100000	Ledelse på øverste plan i virksomheder, organisationer og den offentlige sektor
% 1100 ->	110000	Lovgivningsarbejde samt ledelse i offentlig administration og interesseorganisationer
% 1110 ->	111000	Lovgivningsarbejde og overordnet administration af lovgivning
% 1120 ->	112000	Overordnet offentlig ledelse
% 1140 ->	114000	Overordnet ledelse af interesseorganisationer og humanitære organisationer
% 1141 ->	114100	Ledelse af politiske partiorganisationer
% 1142 ->	114200	Ledelse af økonomiske interesseorganisationer
% 		114210	Politisk ansvarlige/politisk valgte ledere i økonomiske interesseorganisationer
% 		114220	Ansatte ledere i økonomiske interesseorganisationer
% 1143 ->	114300	Ledelsesarbejde i humanitære eller andre interesseorganisationer
% 1200 ->	120000	Øverste ledelse i virksomheden
% 1210 ->	121000	Ledelse omfattende virksomheden som helhed
% 		121010	Øverste ledelsesniveau, administrerende direktører eller tilsvarende
% 		121020	Tværgående direktører

Efter at reduceret alle årene i DISCOALLE\_INDK til fircifrede, så er der 515 forskellige udfald bortset fra år 2003 som har 784 udfald pga. væsentlig mange fejlkodninger som ikke er blevet fjernet og ultimo bliver til ubekendt/missing.


\section{andet skridt \label{}}

Næste skridt er at lave vores nye variabel på 167 udfald. Vores opdeling bygger videre på Anton Grau Larsen og Jonas Toubøl disco-opdelinger (2015).

Den første forskel er, at vi har inkluderet de imputerede koder på færdighedsniveau 1, 2 og 3, hvor det har givet mening. F.eks. hos ingeniører og arkitekter (2141) er alle undergrupperne inkluderet både i vores og Larsen og Toubøls:
\begin{description}
  \item Arkitektarbejde og planlaegning af anlaegsarbejder (2141)
  \item Ingenioerarbejde vedroerende bygninger og anlaeg (2142)
  \item Ingenioerarbejde vedroerende staerkstroem (2143)
  \item Ingenioerarbejde vedroerende svagstroem (2144)
  \item Ingenioerarbejde vedroerende ikke-elektriske motorer og maskinanlaeg (2145) 
  \item Ingenioerarbejde vedroerende kemiske processer ved industriel produktion (2146)
  \item Mineingeniørarbejde (2147), Landinspektørarbejde (2148)
  \item Andet arkitekt- og ingenioerarbejde med videre (2149) .
\end{description}
Forskellen er så, at vi også inkluderer mellemgrupppen: Arkitekt- og ingenioerarbejde med videre (214).

Den anden forskel er, at de kategorier som Grau Larsen og Toubøl ikke har inkluderet, fordi kategoriseringen ikke passer ind samt de imputerede koder på færdighedsniveau 1, 2 og 3 har vi samlet under ni kategorier inden for hovedgrupperne (bortset fra hovedgruppen militært arbejde, som kun har et enkelt niveau) F.eks. ubekendt inden for landbrug, gartneri, skovbrug, jagt og fiskeri, der forudsaetter faerdigheder paa grundniveau (6999), som er de som er de amputerede koder:
\begin{description}
  \item Arbejde inden for landbrug, gartneri, skovbrug, jagt og fiskeri, der forudsaetter faerdigheder paa grundniveau (6000)
  \item Arbejde inden for landbrug, gartneri, skovbrug, jagt og fiskeri, der forudsaetter faerdigheder paa grundniveau (6100)
\end{description}

% Vores koder ser således ud:
% \begin{description}
%   \item 110: Militaert arbejde	110
%   \item 1110: Lovgivere	1110
%   \item 1120: Ledelse i lovgivende myndigheder og interesseorganisationer	1120 	1140 	1141 	1142	1143
%   \item 1200: Ledelse indenfor administration og erhverv	1200 	1220	1221	1222	1223	1224	1225	1226	1227	1228	1229	1230	1231	1232	1233	1234	1235	1236	1237	1239	1300	1310	1312	1313	1314	1315	1316	1317	1318	1319
%   \item 1210: Direktoerer	1210
%   \item 1999: Ubekendt ledelse paa oeverste plan	1000	1100
%   \item 2111: Naturvidenskabeligt arbejde	2110	2111	2112	2113	2114	2224
%   \item 2120: Arbejde med matematik, aktuariske og statistiske metoder	2120	2121	2122
%   \item 2131: Udvikling og analyse af software og applikationer	2130	2131	2132	2139
%   \item 2141: Ingenioerer og arkitekter	2140	2141	2142	2143	2144	2145	2146		2147	2148	2149
%   \item 2213: Raadgivning inden for landbrug, skovbrug og fiskeri	2213	3213
%   \item 2221: Laeger, medicinsk og life-science arbejde	2211	2212	2221	2223
%   \item 2222: Tandlaege	2222
%   \item 2230: Sygerplejer og jordmoder	2230	3230	3231
%   \item 2310: Undervisning paa erhvervsskole- og videregaaende skoleniveau 	2300	2310	2320
%   \item 2331: Folkeskolelaerer	2331	3310
%   \item 2332: Paedagogisk arbejde	2332	3320
%   \item 2340: Specialpaedagogisk arbbejde	2340	3330
%   \item 2351: Undervisningsmetode specialister	2351	2352
%   \item 2359: Andet undervisningsarbejde	2350	2359	3340
%   \item 2411: Arbejde inden for finans, marketing og oekonomi	2411	2419
%   \item 2421: Advokat, dommer og andet juridisk arbejde	2420	2421	2422	2429
%   \item 2431: Bibliotekararbejde og andet informationsarbejde	2430	2431	2432
%   \item 2441: Oekonom	2441
%   \item 2442: Samfundsvidenskabeligt arbejde og historie	2442	2443
%   \item 2444: Translatoer og andet sprogvidenskabeligt arbejde	2444
%   \item 2445: Psykolog	2445
%   \item 2446: Socialraadgivningsarbejde	2446
%   \item 2451: Journalist, forfattere, reklame og marketing	2451	3472
%   \item 2452: Kunstnere	2452	2453	2454	2455	3473	3474
%   \item 2460: Religioest arbejde	2460	3480
%   \item 2999: Ubekendt arbejde med faerdigheder paa hoejeste niveau	2000	2100	2200	2210	2220	2330	2400	2410	2440	2450	2470
%   \item 3111: Teknikerarbejde inden for ingenioervirksomhed	3111	3112	3116	3119	3151
%   \item 3113: Teknikerarbejde inden for videnskab og elektronik 	3113	3114	3115	3117	3132	3139	3152
%   \item 3118: Teknisk tegnearbejde	3118
%   \item 3121: Teknikerarbejde inden for IT	3121	3122
%   \item 3131: Fotografarbejde, TV og AV-medie	3131
% 3133: Teknikerarbejde indenfor roentgen- og behandlingsudstyr	3133
% 3141: Skibsingenioerer, officerer og foerere	3141	3142
% 3143: Flypilotsarbejde og lign. arbejde	3143
% 3144: Flyveleder og tekniker indenfor luftfartssikkerhed	3144	3145
% 3211: Teknikarbejde inden for biovidenskab, medicin og patologi	3211	3212	3227
% 3221: Sundhedsassistent	3221	3223
% 3225: Tandlaegeassistent	3225
% 3228: Arbejde inden for farmaci	3228
% 3229: Andet arbejde indenfor helse og sundhed	2229	2412	3222	3224	3226	3229
% 3411: Arbejde med omsaetning af vaerdipapirer og valuta	3411
% 3412: Forsikringsarbejde	3412
% 3413: Ejendomsmaegler 	3413
% 3415: Salgsarbejde (agenter)	3415
% 3416: Indkoebsarbejde	3416
% 3419: Arbejde med kredit- og laangivning	3419
% 3421: Erhvervsmaeglerarbejde og lign. arbejde 	3417	3421	3429
% 3422: Speditoerarbejde	3422
% 3423: Jobformidling	3423
% 3433: Arbejde med regnskab, statistik og matematik	3433	3434
% 3439: Sekretaerarbejde og lign. arbejde	3431	3439	3449	4115
% 3441: Told-, skat- og udstedningsarbejde	3441	3442	3444
% 3443: Formidling af offentlige ydelser	3443
% 3450: Politi, detektiv og jurdisk arbejde paa mellemniveau	3432	3450
% 3460: Socialraadgivningsarbejde paa mellemniveau	3460
% 3471: Design- og dekoratoerarbejde	3471
% 3475: Sportsudoevere, traenere, instruktoerer og dommere	3475
% 3999: Ubekendt arbejde med faerdigheder paa mellemste niveau	3000	3100	3110	3120	3130	3140	3150	3200	3210	3220	3300	3400	3410	3420	3430	3440	3470
% 4111: Tekstbehandlings- og indtastningsarbejde	4111	4112	4113	4114	4143
% 4121: Alm. regnskabs- og bogfoeringsarbejde	4121
% 4122: Kontorarbejde vedr. statistik og finans	4122
% 4131: Lagerekspeditionsarbejde	4131
% 4132: Registreringsarbejde inden for produktion	4132
% 4133: Registreringsarbejde inden for transport- og trafikstyring	4133
% 4141: Arbejde med arkivering, kopiering og bibliotek	4141
% 4142: Postbud- og betjentarbejde samt postsorteringsarbejde	4142
% 4190: Interviewarbejde og andet kontorarbejde	4190
% 4211: Kundebetjening og lign. arbejde ved kasse	4211	4212	4213
% 4215: Pantelaane- og inkassatorarbejde og lign. arbejde	4214	4215
% 4221: Rejsebureauarbejde	3414	4221
% 4222: Receptionist og kundeinformationsarbejde	4222	4223
% 4999: Ubekendt kontorarbejde 	4000	4100	4110	4120	4130	4140	4200	4210	4220
% 5111: Servicearbejde af passagerer	5111
% 5112: Kontrolarbejde under rejser	5112
% 5113: Turist- og rejselederarbejde	5113
% 5121: Rengoeringsinspektoer- og husbestyrerarbejde	5121
% 5122: Kokkearbejde og lign. arbejde i koekken	5122
% 5123: Tjenere og bartendere	5123
% 5131: Boerneomsorgsarbejde	5131
% 5133: Social- og sundhedsarbejde i private hjem (inkl. plejehjem)	5133
% 5139: Ambulance- og omsorgsarbejde inden for sundhed	5132	5139
% 5141: Frisoer- og kosmetologarbejde	5141
% 5142: Servicearbejde ikke klassificeret andetsteds	5142
% 5143: Undertakers and embalmers	5143	5149
% 5161: Brandslukningsarbejde	5161
% 5162: Politiarbejde	5162
% 5163: Faengselsbetjentarbejde	5163
% 5210: Modelarbejde	5210
% 5220: Salgsarbejde i butik, foedevarer og lign.	5220
% 5999: Ubekendt salgs-, service- og omsorgsarbejde	5000	5100	5110	5120	5130	5140	5160	5200
% 6112: Arbejde med plantevaekst og dyreopdraet	1311	6110	6111	6112	6120	6121	6122	6129	6130	9211
% 6141: Skovbrugsarbejde	6140	6141	6142
% 6152: Arbejde med fiskeri i kyst- og soeomraader samt dykning	6150	6151	6152	6153
% 	6154	7216
% 6999: Ubekendt arbejde inden for landbrug, gartneri, skovbrug, jagt og fiskeri, d	6000	6100
% 7121: Alm. bygningsarbejde	7112	7121	7129	7113
% 7122: Murer-, brolaegnings- og stenhuggerarbejde	7122
% 7123: Arbejde med opfoerelse af betonkonstruktioner	7123
% 7124: Toemrer- og bygningssnedkerarbejde	7124
% 7131: Arbejde med tagdaekning	7131
% 7132: Arbejde med gulv- og klinkelaegning	7132
% 7133: Stukkatoer- og isoleringsarbejde	7133	7134
% 7135: Glarmesterarbejde	7135
% 7136: VVS-arbejde	7136
% 7137: Elektrikerarbejde	7137	7139
% 7141: Malerarbejde	7141
% 7142: Sproejtelakererarbejde	7142
% 7143: Rensning, desinfektion og skadedyrsbekaempelse	7143
% 7210: Pladearbejde, svejsning og lign. arbejde	7200	7210	7211	7212	7213	7214	7215
% 7220: Smede og vaerktoejsmagerarbejde	7220	7221	7222	7223	7224	8211
% 7231: Mekanikerarbejde, motorkoererktoejer og cykler	7231
% 7232: Flymekanikerarbejde	7232
% 7233: Mekanikerarbejde, landbrug samt koele-montoer	7233
% 7241: Elektromekanikerarbejde	7241
% 7242: Elektronikmekaniker og IT-installatoer/service	7242	7243	7244	7245
% 7311: Finmekanikerarbejde og fremstilling af proteser	7310	7311	7312	7313
% 7321: Praecisionshaandvaerk	7320	7321	7322	7323	7324	7330	7331	7332	7424
% 	7431	7432	8262
% 7341: Grafisk og trykkerarbejde	7341	7342	7343	7345	7346	8251	8252
% 7411: Slagterarbejde, fiskehandel og lign. arbejde	7411
% 7412: Bager- og konditoriarbejde, eksl. industri	7412
% 7413: Fremstilling af mejeriprodukter	7413
% 7414 Arbejde med konservering af frugt og groent samt fremstilling af juice	7414
% 7421: Haandvaerkspraeget arbejde i traeindustri	7421	7422	7423	8140	8240
% 7433: Tekstil, beklaedning, sko og lign. arbejde	7433	7434	7435	7436	7437	7440	7441	7442
% 7999: Ubekendt haandvaerkspraeget arbejde	7000	7100	7110	7120	7130	7140	7230
% 	7240	7300	7340	7400	7410	7415	7416	7420	7430
% 8111: Operatoerarbejde i mine og raastofudvinding	7111	8110	8111	8112	8113
% 8120: Operatoerarbejde i metalproduktion	8120	8121	8122	8123	8124
% 8131: Operatoerarbejde i glas og keramik	8130	8131	8139
% 8141: Operatoerarbejde i trae	8141
% 8142: Operatoerarbejde i papir, pap og lign.	3123	8142	8143
% 8150: Operatoerarbejde i kemi ekskl. medicinal	8150	8151	8152	8153	8154	8155	8159	8221	8222	8229
% 8161: Energi- og vandforsyningsarbejde	8161	8162	8163
% 8212: Operatoerarbejde i cement, sten o.a.	8212
% 8223: Operatoerarbejde i metaloverflader	8223
% 8224: Operatoerarbejde i fotografiske produkter	7344	8224
% 8231: Operatoerarbejde i plast og gummi	8230	8231	8232
% 8253: Operatoerarbejde i papirsprodukter	8253
% 8261: Operatoerarbejde i tekstil, skind og laeder	8261	8263	8264	8265	8266	8269
% 8270: Operatoer- og fremstillingsarbejde i naering og nydelse	8270	8271	8272	8273	8274	8275	8276	8277	8278	8279
% 8290: Andet operatoerarbejde	8280	8281	8282	8283	8284	8285	8286	8287	8290
% 8311: Lokomotivfoerer	8311
% 8312: Tograngering	8312
% 8321: Chauffoerer af biler, taxier og varevogne	8321	8322
% 8323: Buschauffoerer	8323
% 8324: Lastbilchauffoerer	8324
% 8331: Arbejde med andre mobile maskiner	8331	8332	8333	8334
% 8340: Daeksarbejde og medhjaelp paa skibe	8340
% 8999: Ubekendt proces- og maskinoperatoerarbejde samt transport- og anlaegsarbejde	8000	8100
% 	8160	8170	8200	8210	8220	8250	8260	8300	8310	8320	8330
% 9113: Doer-, gade- og telefonsaelgere og lign.	9110	9113	9120
% 9131: Rengoeringsarbejde i private hjem	9131
% 9132: Hjaelpere og rengoering, ekskl. private hjem	9132
% 9133: Manuel vask og presning af toej	9133
% 9141: Ejendomsinspektoer og  andet rengoering	9140	9141	9142
% 9151: Sikkerhedsvagt, kurér og forefaldende	5169	9150	9151	9152	9153
% 9162: Gadefejning og forefaldende	9162
% 9212: Manuelt skov-, fiskeri og jagtarbejde	9212	9213	9311
% 9312: Manuelt arbejde i anlaegssektor	9312
% 9313: Manuelt arbejde i byggesektor	9313
% 9322: Pakkeriarbejde	9161	9320	9322
% 9333: Arbejde med lastning og opfyldning	9330
% 9998: Ubekendt Andet arbejde	9000	9100	9130	9160	9200	9210	9300	9310
% 9999: Ubekend	0	1329	2414	3661	4191	9262	9800	9820	9826	9831	9860
% 	9900	9913	9919	9920	9930	993	9933	9960	9990	9999


KILDER
http://www.dst.dk/da/Statistik/dokumentation/Times/personindkomst/discoalle-indk.aspx
http://www.dst.dk/da/Statistik/dokumentation/Nomenklaturer/DISCO-88.aspx
http://www.dst.dk/da/Statistik/dokumentation/Nomenklaturer/discoloen.aspx



%Local Variables: 
%mode: latex
%TeX-master: "report"
%End: 