% -*- coding: utf-8 -*-
% !TeX encoding = UTF-8
% !TeX root = ../report.tex

\chapter{baggrund} \label{baggrund}


1 CENTRALE AKTØRER OG TAL

LEDIGHESDSTAL
I perioden 1996 til 2009 er der mellem to og otte procent nettoledige. Ind til da tog det lang tid at nedbringe ledigheden fra de 10-12 procent, som den var vokset til efter de to oliekriser i 1970’erne og de syv magre år fra 1987-1993. Fra 1994 begynder ledigheden at falde. i 1996 ligger ledigheden således på 8 procent. Efter 1996 falder ledigheden forholdsvist stabilt bortset omkring årtusindeskiftet, hvor ledigheden først er forholdsvis stabil, hvorefter den falder stiger meget lidt. I 2008 er ledigheden således faldet til to procent. Fra 2009 begynder den så at stige kraftigt (AE-rådet 2012).

CENTRALE AKTØRER
Arbejdsløshedskasserne, staten og kommunen er centrale aktører som har spillet centrale roller i  danske arbejdsløshedsforsikringssystem siden etableringen i 1907 og er kendetegnet som at følge den såkaldte Gent-model, hvor staten anerkender og yder tilskud til arbejdsløshedskasser organiseret af forsikringstagere (i praksis fagbevægelsen), og at det for det enkelte individ er frivilligt om denne vil forsikre sig mod arbejdsløshed (Jensen 2007: 33f). Et væsentligt skifte siden 1907 skete med Wechselmann-udvalget, hvor kommunale fik mindre ansvar og staten overtog den marginale risiko ved ledighedsstigninger fra arbejdsløshedskasserne, hvilket vil sige, at staten fuldt ud betalte medudgifter ved stigende ledighed. *Arbejdsformidlingen* blev på samme tid etableret for at formidle ledige jobs og ledig arbejdskraft (Pedersen 2007:83) og eksisterede ind til den blev nedlagt med Kommunalreformen i 2007 og erstattet af de kommunale jobcentre, hvor staten og kommunen i fællesskabet samarbejder om beskæftigelsesindsatsen på lokalniveau.




3 ARBEJDSLØSHED SOM ØKONOMISK PROBLEM
Jørgen Goul Andersen identificerer et skifte inden for en bred strømning af økonomisk teori fra slutningen af 1970'erne, hvor velfærdsstaten blev anskuet som et middel at afbøde *markedsfejl* til i stigende grad at fokusere på *politikfejl* og *forvridninger* på markedet, hvilket kan karakteriseres som et skifte fra efterspørgsel til udbud og et skifte fra makro til mikro (Goul Andersen 2003: 19).

Søgeteorien
Søgeteorien blev grundlagt i 1960'erne og kendte teoretikere er John J. McCall, George Stigler, Peter Diamon, Dalte T. Mortensen og Christopher A. Pissarides og den basale søgemodel handler om arbejdsløse som søger beskæftigelse. Forestil en arbejdsløs som af og til får jobtilbud. Den arbejdsløse kender først lønnen på jobtilbuddet, når det modtages. Efter at have modtaget tilbuddet, skal den arbejdsløse beslutte sig for, om tilbuddet accepteres eller afslås. Dette gør den arbejdsløse på baggrund af en reservationsløn, som er fastsat af forventningerne til lønnen og viden om lønfordelingerne på arbejdsmarkedet. Hvis det modtagne tilbud er større end reservationslønnen, accepteres tilbuddet, og ellers afslås det (Rosholm 2009: 159f). Marginalisering kan i denne sammenhæng forklares med, at den arbejdsløses kvalifikationer bliver mindre værd efterhånden som ledighedsperioden bliver længere, at den arbejdsløse gradvist mister forbindelsen til gamle kolleger eller at den arbejdsløse dømmes på baggrund af sin langtidsledighed (Rosholm 2009: 160f, Goul Andersen 2003: 20).

Matching-teorien
Matching-teorien deler arbejdsmarkedet op i to typer agenter: arbejdsgivere og lønmodtagere og er kendt for teoretikere som Dale T. Mortensen, Christopher A. Pissarides Peter A. Diamond, Alvin E. Roth and Lloyd Shapley. Her leder lønmodtagerne efter job i de perioder, hvor de er arbejdsløse, mens arbejdsgivere slår stillinger op, så længe de vurderer, at det kan betale sig. Ledige lønmodtagere og job mødes på arbejdsmarkedet i en matching-proces. Når der skabes kontakt mellem en arbejdsgiver og lønmodtagere, opstår der en forhandling mellem arbejdsgiver og lønmodtager om fordelingen af overskuddet i en eventuel ansættelse. Arbejdsgiveren kan på den ene side presse lønmodtageren til at acceptere en løn, som ligger under arbejdskraftens marginalprodukt, fordi lønmodtageren ikke uden videre kan finde nyt job på anden vis end ved at vente på det næste jobtilbud. Lønmodtageren kalkulerer på baggrund af forskellen mellem den tilbudte løn og værdien af at være ledig (*outside option*). Matching-teorien kan bruges til at analyse den umiddelbare effekt af at ændre i ledighedsydelser som eksempelvis dagpenge eller kontakthjælp (Rosholm 2009: 162f).

Insider-outsider-teorien
Insider-outsider-terien er udviklet af Assar Lindbeck og Dennis Snower i 1980'erne og fokuserer på omkostningerne forbundet med ansættelser og afskedigelser af arbejdskraft. Omkostningerne kan være i forbindelse med søge- og optræning, afskedigelse, uproduktiv konkurrence mellem to grupper af lønmodtagere. Insidere/outsidere kan blandt andet defineres som beskæftige/arbejdsløse, fagforeningsmedlemmer/ikke-medlemmer, ansatte i gode jobs/dårlige jobs. Insidere vil forsøge at forhandle sig til så høje lønninger som muligt og afholde andre for at underbyde dem på markedet. Insidernes magt består blandt andet i, at arbejdsgiverne har omkostninger vil at afskedige dem og kan skabe omkostninger ved strejke, aktioner og mobning af nyansatte. Langtidsledige, i modsætning til beskæftige og korttidsledige, kan tilbyde sin arbejdskraft til reduceret løn og forsøge at overbevise en arbejdsgiver om at blive ansat så længere arbejdsgiverens omkostninger ved at ansætte en outsider ikke stiger gevinsten ved at gøre det (Rosholm 2009:164, Goul Andersen 2003:20).



4 ARBEJDSLØSHEDENS SOCIALE KONSEKVENSER

Jahoda, Lazarsfeld og Zeizels Marienthal-studie (Nørup:23)
Marienthal er et klassisk studie af de sociale konsekvenser af arbejdsløshed i et lille samfund gennemført af Marie Jahoda i samarbejde med Paul Lazarsfeld og Hans Zeizel (1971). Marienthal var et industriby som led af høj arbejdsløshed i 1920'erne, og studiet undersøger hvad der sker med arbejderne i den østrigske by Marienthal, når de oplever arbejdsløshed. Med Marienthal udvikler Jahoda deprivationsperspektivet, som er det mest udbredte perspektiv i de teoretiske diskussioner af arbejdsløshed og eksklusion fra arbejdsmarkedet (Creed & Macintyre:2001). Hovedargumentet er, at arbejdsløshed medfører social eksklusion og isolation, tab af struktur i hverdagen og selvtillid og en betydelig øget risiko for psykiske problemer (Jahoda, 1981, Jahoda m.fl. 1997). Marienthal baserer sig på en grundlæggende antagelse om arbejde deltagelse på arbejdsmarkedet opfylder både et psykologisk behov for individet og et økonomisk behov for indtægt. Jahoda opstiller ikke knækket vilje, resignation, fortvivlelse og apati som fire stadier eller reaktioner den arbejdsløse gennemgår (Jahoda 1979, Jahoda m.fl. 1997). Arbejdsdeltagelsen har ifølge Jahoda fem funktioner: tidsmæssig struktur i dagligdagen, sociale kontakter, deltagelse i kollektive formål, status og identitet og regelmæssig aktivitet (Jahoda, 1981, Jahoda m.fl. 1997).
Nørup kritiserer brugen af deprivationsperspektivet i dansk regi, fordi det danske samfund i dag er en moderne velfærdstat med relativt højtuddannet arbejdskraft, og Marienthal er en mindre industriby med lavt uddannet arbejdskraft i et 1930'ernes Østrig som ikke er i nærheden af et velfærdssamfund (Nørup2012:34).

Eisenberg og Lazarsfeld (Nørup:24)
I *The psychological effects of unemployment* (1938) konkluderer Philip Eisenberg og Paul Lazarfeld, at arbejdsløse gennemlever tre stadier: “First there is shock, which is followed by an active hunt for a job, during which the individual is still optimistic and unresigned; he still maintains an unbroken attitude. Second, when all efforts fail, the individual becomes pessimistic, anxious, and suffers active distress; this is the most crucial state of all. And third, the individual becomes fatalistic and adapts himself to his new state but with a narrower scope He now has a broken attitude.” (Eisenberg & Lazarsfeld 1938:378).
Studiet har vundet udbredelse inden for socialpsykologien (Boyd 2014, Wang & Greenwood 2014, Kahn 2013, Ezzy 1993, Ragland-Sulivan & Barglow 1981, Finley & Lee 1981, Hayes & Nutman 1981, Hill 1978, Briar 1977, Harison 1976). Men studiet er også blevet kritiseret på baggrund af en problematisk og modsætningsfuld metode (Fryer 1985, Ezzy 1993) og på baggrund af det empiriske fundament i særdeleshed vedrørende psykologiske faktorer som eksempelvis mentalt helbred og selværd (Fryer 1985, Hartley 2011, Shamir 1986). (Nørup kalder det for *Stadie Model*)

Tiffany, Cowan og Tiffany (Nørup:25)
Hovedargumentet i studiet *The unemployed: A social-psychological portrait*  af Donald Tiffany, James Cowan og Phyllis Tiffany (1970) er, at majoriteten af arbejdsløse og ekskluderede fra arbejdsmarkedet står uden for arbejdsmarkedet på grund af psykologiske problemer. Sammenhængen mellem arbejdsløshed og psykologiske problemer går derfor begge veje, hvilket betyder, at psykologiske kan være årsagen til arbejdsløshed på samme tid med, at arbejdsløshed i sig selv også medfører psykologiske problemer: ”They show avoidance behaviour patterns or what has been referred to as ”work inhibition” which implies that they are physically capable of work but prevented from work because of psychological disabilities” (Tiffany, Cowan and Tiffany, 1970). Ifølge Tiffany, Cowan og Tiffany er løsningen, at staten rehabiliterer disse arbejdsløse, så de kan komme tilbage på arbejdsmarkedet gennem træning eller terapi (Tiffany, Cowan and Tiffany: 1970). Douglas Ezzy peger på, at denne tilgang har ligheden mellem den historiske distinktion mellem *deserving poor*, som fysisk var ude af stand til at arbejde og fortjente støtte og *non-deserving poor*, som ikke arbejde selvom de var i fysisk stand til at arbejde (Ezzy 1993). Ezzy påpeger ligeledes på, at tilgangen har været mest toneangivende i perioder med højkonjunktur og relativ lav ledighed til sammenligning med perioder med lavkonjunktur og lav ledig (Ezzy 1993).
Perspektivet kritiseres for at have lighedstræk med den neoklassiske økonomiske betragtning af arbejdsløshed som frivilligt og derfor ved at skyde skylden på ofret (Miles:1987) (Nørup kalder det for *Rehabiliteringstilgangen*)

Warr (Nørup:25)
I *Work, Unemployment and Mental Health* opstiller Peter Warr ni faktorer i omgivelser som har betydning for det mentale helbred i forbindelse med arbejdsløshed: mulighed for kontrol, mulighed for at benytte erhvervede færdigheder, eksternt genererede mål, variation, klarhed i forhold til omgivelserne, penge/ indtjening, fysisk sikkerhed, mulighed for social kontakt og social position (Warr:1987). Individets mentale helbred afspejler det akkumulerede niveau af faktorerne, så det at miste et arbejde eller det at have et dårligt arbejde i omgivelserne afspejler individets mentale heldbred (Ezzy:1993). (Nørup kalder det for *vitaminmodellen*)

Halvorsens teori om mestring (Nørup:28)
Knut Halvorsen udvikler i sit forfatterskab en modpol til Jahoda, Lazarsfeld og Zeizel (1971), Eisenberg og Lazarsfeld (1938), Tiffany, Cowan og Tiffany (1970) og Warrs (1987) passive og ensartede individperspektiv ved at betragte arbejdsløse som forskelligartede og handlende. De arbejdsløse anskues derfor som aktive aktører, der kan påvirke og forandre deres situation i stedet for at være ofre for omstændighederne (Halvorsen 1994, 1999). Den arbejdsløse vil indgå i forskellige fysiske, psykiske og sociale aktiviteter for at afbøde effekter af arbejdsløshed og minimere stress og mental belastning (Halvorsen: 1999, Fryer & Fagan:1993, Fryer:1986, Fryer & Payne:1984, O’Brien:1985). Halvorsen skelner mellem den problemorienterede mestring, som udgøres af konkrete strategi med formål at fjerne belastningen af den marginaliserede position (fx jobsøgning) og den emotionsorienterede mestring, som handler om hvordan man ser verdenen (Halvorsen 1999)
Nørup kritiserer Halvorsen for i sin ontologiske individualisme at fokuserer på, hvordan individet mestrer bestemte livssituationer under bestemte rammer (graden af eksklusion forklares som et resultat af individuelle handlinger og ressourcer) frem for på hvordan de samfundsmæssige strukturer og sociale relationer påvirker eksklusionen (Nørup 2012:37).

Glaser & Strauss’ teori om sociale passager
Barney Glaser og Anselm Strauss definerer en status passage som et individs ”movement into a different part of a social structure, or loss or gain of privilige, influence, or power, and changed behaviour”. Ezzy beskriver anvendelsen af teorien på arbejdsløshed og exit fra arbejdsmarkedet, som processer frem for enten-eller tilstande. Hermed kan exit fra arbejdsmarkedet sammenlignes med andre status passage som eksempelvis skilsmisse, sygdom eller dødsfald i familien (Ezzy:1993). Van Gennep benytter separation (begravelse), transition (overgangsfase mellem to jobs) og integration (ægteskab) som kategorier for sociale passager (Ezzy:1993, Van Gennep 1977). Ezzy identificerer exit fra arbejdsmarkedet som tab af job som en afhændelespassage i modsætning til exit fra arbejdsmarkedet som indtræden i uddannelsessystemet som noget helt andet (Ezzy 1993). En afhændelsespassager fører ikke nødvendigvis sig selv til mentale problemer, mistrivsel eller eksklusion, da det afhænger af den enkeltes identitet og selvopfattelse i relation til andre og samfundet og en lang række andre faktorer samt om afhændelsespassagen efterfølges af en reintegrativ passage, hvor den enkelte får en ny status eksempelvis et nyt job (Ezzy 1993).

Kommentarer
Deprivation (Jahoda, Lazarsfeld & Zeizel 1971; Eisenberg & Lazarsfeld 1938; Tiffany, Cowan & Tiffany 1970; Warrs (987) er delvist brugbart i forståelsen af arbejdsløshed som havende en social og psykologisk påvirkvning på arbejdsløse. Halvorsen er delvis brugbar i forståelsen af at individerne bliver aktører som kan handle og har ressourcer. Social passage (Glaser & Strauss 1971)er relevant i vores definition af arbejdsløshed som midlertidigt.


---------------------------------------------------------------------------------------------------


5 SKITSERING AF RELEVANT TEORI OG LITTERATUR
Vi ønsker at bruge Bourdieu i vores primære teoriapparat til at fortælle en historie om hvilke muligheder man har for at ernære sig, når man oplever arbejdsløshed, med særligt fokus på hvilken beskæftigelse man får efter en periode med ledighed. Vender man tilbage til arbejde i samme felt, eller bevæger man sig ind på et nyt? Derved vil vi diskutere, hvilken praksis der hænger sammen med hvilke felter, og hvad det siger om hvilke felter der ligger nær hinanden. Eller måske siger noget om hvor desperat man skal være, for at bevæge sig ud over det felt man er trænet ind i. Vi vil gerne diskutere hvad der strukturerer folk, der oplever arbejdsløsheds opfattelse af handlingsrum, som vi ser det komme til udtryk igennem deres praksis mellem forskellige typer af jobs efter perioder med ledighed.

Her inddrager vi centrale økonomiske og sociologiske teorier om arbejdsløshed samt hvordan den danske arbejdsløshedsmodel historisk har udviklet sig til det den er i dag, og hvordan den ser ud i dag. Relevante forskere inden for arbejdsmarkedsforskning historisk, sociologisk og økonomisk er eksempelvis Jesper Due, Jørgen Steen Madsen, Bent Jensen, Per H. Jensen, Aage Huulgaard og Hans-Carl Jørgensen. Relevante økonomiske teorier er fx søgeteorien, matching-teorien og insider-outsider-teorien udarbejdet af George Stigler, Peter Diamon, Dalte T. Mortensen, Christopher A. Pissarides, Assar Lindbeck, Dennis Snower, mv. Relevante sociologiske teorier og teoretikere kunne fx være Marie Jahoda  Marienthal-studie, Philip Eisenberg og Paul Lazarfeld, Donald Tiffany, James Cowan og Phyllis Tiffany, Peter Warr, Knut Halvorsen, Barney Glaser og Anselm Strauss, Catharina Juul Kristensen og Jørgen Elm Larsen.

Vi kan endnu ikke endnu sige i hvilken grad den nævnte teori blive anvendt. Vi vil kæde disse retninger sammen med vores primære inspirationskilde Bourdieu, eller bruge det som en baggrund for at forstå den eksisterende litteratur om arbejdsløshed til så at vise i hvilken tradition vi skriver os ind på.


---------------------------------------------------------------------------------------------------



%Local Variables: 
%mode: latex
%TeX-master: "report"
%End: 