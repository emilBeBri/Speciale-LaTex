% -*- coding: utf-8 -*-
% !TeX encoding = UTF-8
% !TeX root = ../report.tex


\chapter{BAGGRUND} \label{baggrund}


%%%%%%%%%%%%%%%%%%%%%%%%%%%%%%%%%%%%%%%%%%%%%%%%%%%%%%%%%%%%%%%%%%%%%%%%%%%%%%%%%%%%%%%%%%%%%%%%
%%%%%%%%%%%%%%%%%%%%%%%%%%%%%%%%%%%%%%%%%%%%%%%%%%%%%%%%%%%%%%%%%%%%%%%%%%%%%%%%%%%%%%%%%%%%%%%%
%%%%%%%%%%%%%%%%%%%%%%%%%%%%%%%%%%%%%%%%%%%%%%%%%%%%%%%%%%%%%%%%%%%%%%%%%%%%%%%%%%%%%%%%%%%%%%%%


\section{Arbejdsløshed i en historisk rammer \label{}}


%%%%%%%%%%%%%%%%%%%%%%%%%%%%%%%%%%%%%%%%%%%%%%%%%%%%%%%%%%%%%%%%%%%%%%%%%%%%%%%%%%%%%%%%%%%%%%%%
%%%%%%%%%%%%%%%%%%%%%%%%%%%%%%%%%%%%%%%%%%%%%%%%%%%%%%%%%%%%%%%%%%%%%%%%%%%%%%%%%%%%%%%%%%%%%%%%
%%%%%%%%%%%%%%%%%%%%%%%%%%%%%%%%%%%%%%%%%%%%%%%%%%%%%%%%%%%%%%%%%%%%%%%%%%%%%%%%%%%%%%%%%%%%%%%%


\section{Centrale aktører og tal på arbejdsløshedsområdet \label{}}

% \subsection{Arbejdsløshedstal \label{}}
% I perioden 1996 til 2009 er der mellem to og otte procent nettoledige. Ind til da tog det lang tid at nedbringe ledigheden fra de 10-12 procent, som den var vokset til efter de to oliekriser i 1970’erne og de syv magre år fra 1987-1993. Fra 1994 begynder ledigheden at falde. i 1996 ligger ledigheden således på 8 procent. Efter 1996 falder ledigheden forholdsvist stabilt bortset omkring årtusindeskiftet, hvor ledigheden først er forholdsvis stabil, hvorefter den falder stiger meget lidt. I 2008 er ledigheden således faldet til to procent. Fra 2009 begynder den så at stige kraftigt (AE-rådet 2012).

% \subsection{Centrale aktører \label{}}
% Arbejdsløshedskasserne, staten og kommunen er centrale aktører som har spillet centrale roller i  danske arbejdsløshedsforsikringssystem siden etableringen i 1907 og er kendetegnet som at følge den såkaldte Gent-model, hvor staten anerkender og yder tilskud til arbejdsløshedskasser organiseret af forsikringstagere (i praksis fagbevægelsen), og at det for det enkelte individ er frivilligt om denne vil forsikre sig mod arbejdsløshed (Jensen 2007: 33f). Et væsentligt skifte siden 1907 skete med Wechselmann-udvalget, hvor kommunale fik mindre ansvar og staten overtog den marginale risiko ved ledighedsstigninger fra arbejdsløshedskasserne, hvilket vil sige, at staten fuldt ud betalte medudgifter ved stigende ledighed. *Arbejdsformidlingen* blev på samme tid etableret for at formidle ledige jobs og ledig arbejdskraft (Pedersen 2007:83) og eksisterede ind til den blev nedlagt med Kommunalreformen i 2007 og erstattet af de kommunale jobcentre, hvor staten og kommunen i fællesskabet samarbejder om beskæftigelsesindsatsen på lokalniveau.





%Local Variables: 
%mode: latex
%TeX-master: "report"
%End: