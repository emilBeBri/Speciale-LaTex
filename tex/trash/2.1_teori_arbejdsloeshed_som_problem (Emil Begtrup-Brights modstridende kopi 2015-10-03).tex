% -*- coding: utf-8 -*-
% !TeX encoding = UTF-8
% !TeX root = ../report.tex


%%%%%%%%%%%%%%%%%%%%%%%%%%%%%%%%%%%%%%%%%%%%%%%%%%%%%%%%%%%
\newpage \section{\textsc{Arbejdsløshed som problemstilling} \label{}}
%%%%%%%%%%%%%%%%%%%%%%%%%%%%%%%%%%%%%%%%%%%%%%%%%%%%%%%%%%%

C. Wright Mills konstaterer: “No problem can be adequately formulated unless the values involved and the apparent threat to them are stated.” \textbf{\parencite[129]{Mills1959}}. Med inspiration fra Mills vil vi gribe arbejdsløshed som problemstilling ved at kontekstualisere kontekstualisere arbejdsløshed som et socialt fænomen med fokus på hvilke værdier, som er involverede og truslerne herimod. Først vil vi beskrive en kort historik af arbejdsløshed som et socialt fænomen. Derefter vil vi beskrive dagens syn på arbejdsløshed, herunder beskrive de dominerende diskurser forbundet med arbejde og arbejdsløshed. %%%% Emil: Uddyb citatet. Søren: Det kunne være bedre med et Bourdieu-citat


%%%%%%%%%%%%%%%%%%%%%%%%%%%%%%%%%%%%%%%%%%%%%%%%%%%%%%%%%%%
\subsection{Kort historik over arbejdsløshed i et dansk perspektiv} 
%%%%%%%%%%%%%%%%%%%%%%%%%%%%%%%%%%%%%%%%%%%%%%%%%%%%%%%%%%%
%%%%% Jens: Kort, præcist og meget informativt afsnit... Super godt

Arbejdsløshed anskues i dag som et socialt problem. Som begreb kom arbejdsløshed dog først til verden i løbet af det 19. århundrede. I dette århundrede blev arbejdsløshed ligeledes også et adskilt fænomen fra fattigdom \parencite[3]{Halvorsen1999}. I Danmark blev det op igennem 1800-tallet og omkring århundredeskiftet en dominerende tanke at anskue arbejdsløshed som et kollektivt og statsligt anliggende, som delvist var forårsaget af forhold arbejdstagerne ikke havde kontrol over. Dette bliver slået fast i 1907 med den første danske arbejdsløshedsforsikringslov\footnote{Hermed blev arbejdsløshedskasserne, staten og kommunerne de centrale aktører i  danske arbejdsløshedsforsikringssystem. Dette kendetegner den såkaldte Gent-model, hvor staten anerkender og yder tilskud til arbejdsløshedskasser organiseret af forsikringstagere (i praksis fagbevægelsen), og at det for det enkelte individ er frivilligt om denne vil forsikre sig mod arbejdsløshed \parencite{Jensen2007a}.}, som blev vedtaget med et bredt flertal i Landstinget og Folketinget. Baggrunden herfor var en kommission, som anbefalede, at samfundet måtte træde ind over for arbejdsløshed, fordi kommissionen kunne konstatere, at arbejdsløshed var et socialt onde, som ramte arbejdstagerne uden, at de havde skyld heri \parencite[69]{Pedersen2007}. Med indførelsen af arbejdsløshedsforsikringen får den danske velfærdsstat som rolle at administrere arbejdsløshed som et socialt problem. Det var dog først i 1970, at staten overtog den marginale risiko og arbejdsløshedsforsikringen gik fra at være en privat forsikringsordning til at være en statsfinansieret velfærdsordning \parencite[83]{Pedersen2007}. %%%% Emil: Kan fyldes en smule mere på. Måske Foucault hevisnng til dannelse af befolkning. Kan suppleres med Hornemann Møller fra DDS som har gode inddelinger.

Fra 1907 til 1970 blev de arbejdsløses vilkår løbende styrket, men beskæftigelseskrisen fra 1970'erne og frem til midten af 1990'erne medfører en mere aktiv arbejdsmarkedspolitik end tidligere, der blandt andet indebar løbende besparelser. Som det fremgår af figur \ref{fig_udvikl.arbejdsloeshed}, så er Oliekrisen i 1973 et vendepunkt for arbejdsløshedsprocenten, som på fem år går fra under to procent til over otte procent. Under Anker Jørgensen gennemføres blandt andet efterlønsordningen i 1977 som et af flere initiativer til at vende udviklingen \parencite[86]{Pedersen2007}. I 1982, da Poul Schlüter overtager statsministerposten ligger arbejdsløshedsprocenten på over 10 procent. Som konsekvens heraf fastfryses dagpengesatsen fra 1982 til 1986 som en af flere initiativer. Omkring 1987 er arbejdsløsheden falder til under otte procent, men herefter stiger arbejdsløsheden igen til omkring 10 procent i 1993, da Poul Nyrup Rasmussen overtager statsministerposten \parencite[88]{Pedersen2007}. Under Nuyrup Rasmussen øges vægten på ret, pligt og individuel behovsorientering \parencite[92]{Pedersen2007}, og samtidig falder arbejdsløsheden stødt. Ved Anders Fogh Rasmussens indtrædelse på statsministerposten i 2001 ligger arbejdsløsheden på seks procent. Under Fogh Rasmussen bliver det muligt for tværfaglige a-kasser at oprettes for at øge konkurrencen på området \parencite[97]{Pedersen2007}. På trods af en kort stigning i 2003, så falder arbejdsløshedsprocenten, mens Fogh Rasmussen sidder ved magten. I 2008 indtræder finanskrisen og arbejdsløsheden begynder at stige. Som konsekvens heraf indføres dagpengereformen i 2010 under Lars Løkke Rasmussen, hvor dagpengeperioden blev halveret fra 4 til 2 år og genoptjeningspligten fordobles fra 26 til 52 uger \parencite{lov_dagpenge}. %%%% Jens: Jeg ved ikke om det var en konsekvens af krisen
% 
\begin{figure}[H]
\begin{centering}
	\caption{Udviklingen i arbejdsløshedsprocenten: Kilde AE-Rådet}
	\includegraphics[width=0.8\textwidth]{fig/teori/historiskudvikling.png}

	\footnotesize{Bruttoledigheden opgør Danmarks Statistik kun tilbage til 2007. Finansministeriet har dog foretaget en tilbageføring til 1996, som er anvendt til figuren. Kilde: AE på baggrund af Danmarks Statistik, Finansministeriet og OECD \parencite[2]{Bjoersted2012}.}
	\label{fig_udvikl.arbejdsloeshed}
\end{centering}
\end{figure}
% 
Ifølge Keane og Owens er udviklingen af velfærdsstaten i Danmark og andre lande bygget på et normativt grundsyn om at alle som udgangspunkt skal forsørge sig selv gennem et arbejde \textbf{\parencite[18]{Keane1986}}. Lønarbejdet bidrager til at sikre social integration blandt velfærdsstatens medlemmer. Og velfærdsstatens aktive arbejdsmarkedspolitik er med til at opretholde en vis levestandard for dem, som ikke har et arbejde samtidig med at benytte en “gulerod” til at få folk i arbejde gennem økonomiske incitamenter \textbf{\parencite[7]{Halvorsen1999}}. %%% Henvisningerne til Keane og Halvorsen er gode nok, men man kunne godt overveje at anvende en andre henvisninger fx. nogle fra FAOS


%%%%%%%%%%%%%%%%%%%%%%%%%%%%%%%%%%%%%%%%%%%%%%%%%%%%%%%%%%%
\subsection{Dagens syn på arbejdsløshed i et dansk perspektiv} 
%%%%%%%%%%%%%%%%%%%%%%%%%%%%%%%%%%%%%%%%%%%%%%%%%%%%%%%%%%%
%%%% Jens: Lidt i tvivl om hvad dette afsnit skal. Det er godt og godt skrevet, men vil dette blive taget op igen?
%%%% Jens: Måske skulle man vare springe arbejdsdiskurserne over og gå direkte til arbejdsløshedsdiskurserne
%%%% Emil: kort analyse hvem/hvad diskurserne tjener. Der mangler en Enhedslisten/venstrefløjdiskurs. Skrives mere ud. Kædes sammen med Bourdieu afsnit.

Arbejdsløshed regnes ifølge Halvorsen for en af nutidens største udfordringer for velfærdsstaten både nationalt og internationalt \textbf{\parencite[8]{Halvorsen1999}}. \textbf{\parencite[98]{Bauman1999}}. %%%% Søren: Henvisningen til Halvorsen er god nok, men man kunne godt overveje at anvende en anden eller flere andre henvisninger - Bauman henviser til survey over europæeres bekymringer om arbejdsløshed

Som socialt problem indgår arbejdsløshed i diskurser i forbindelse med arbejdets betydning og i den forbindelse også betydning af \textit{fravær} af arbejde. De diskurser, som er knyttet til arbejdsløshedsfænomenet er både med til at påvirke, hvordan arbejdsløse klassificeres, og hvordan arbejdsløse forstår dem selv og deres situation \parencite[12]{Halvorsen1999}. %%% Jens: Omformuler

Halvorsen skelner mellem tre arbejdsløshedsdiskurser \parencite[13]{Halvorsen1999}.

\textit{Elendighedsdiskursen} handler om, at arbejdsløshed er lig med social død. Det vil sige arbejdsløshed har en negativ påvirkning på den enkelte arbejdsløses mentale helbred og sociale anerkendelse. Utallige historier i medierne knytter sig til denne diskurs med overskrifter som for eksempel: “Knæk. Arbejdsløshed rammer hele familien” (Politiken, 19.04.2013), “Arbejdsløse rammes af stress” (Politiken, 18.07.2010), “Ekspert: Unge arbejdsløse risikerer ar mange år frem” (Berlingske Tidende, 26.11.2010) og “Arbejdsløse frygter aldrig at finde job igen” (Politiken, 01.01.2011).
Den første diskurs knytter sig til retten til arbejde, hvor lønarbejdet både er lig med selvrealisering og er en forudsætning for, at man kan fungere som en god samfundsborger\footnote{Lars Svendsen skelner inden for den europæiske idéhistorie mellem to grundlæggende forskellige arbejdsopfattelser. Indtil reformationen blev arbejdet anset som en \textit{meningsløs forbandelse}, og efter reformationen blev arbejdet anset som et \textit{meningsfyldt kald} \parencite[13]{Svendsen2008}.
I det moderne samfund beskriver Bauman, at arbejdet bliver påvirket af de æstetiske kriteriers fremmearch, og dets værdi bedømmes ifølge Bauman ud fra dets evne til at skabe behagelige oplevelser. Det vil sige, at det arbejde som ikke rummer en tifledsstillelse i sig selg er uden værdi \textbf{\parencite[169-215]{Baum2006}}.}. %%%% Søren: Ifølge Bauman er den moderne arbejdsetik forbundet med to underforståede forudsætninger. Den første forudsætning går på, at man for at skaffe sig det, som skal til for at holde sig i live og være lykkelig, er nødt til at gøre noget der anses afor at væhhave en værdii og være værd at betale for - det vil sige en form for quid pro quo - noget for noget. Den anden forudsætning er, at det er moralsk forkasteligt at stille sig tilfreds med det, som man allerede har, og på den måde slå sig til tådls med mindre og ikke mere. På den måde er arbejdet en værdi i sig selv \parencite[15]{Bauman2002}. Arbejdsetikken havde to formål nemlig at løse den hastigt voksende industris problemmer med tilstrækkeligt med udbud af arbejdskraft og udrydde nødvendigheden af at sørge for dem, der af den ene eller den anden grund ikke kunne sørge for dem selv parencite[24]{Bauman2002}. 

\textit{Beskæftigelsesdiskursen} handler om, at arbejdsløshed er lig med sløseri med ressourcer. Det vil sige arbejdsløshed har en negativ påvirkning på de statslige finanser og det private erhvervsliv. Dette fylder også en del i finansnyhederne, som eksempelvis “Arbejdsløse koster kassen” (Ekstra Bladet, 14.11.2008), “Ledighed sender folk i sygesengen” (Jyllands-Posten, 18.03.2013) og “Høj ledighed truer EUs økonomi” (Berlingske, 03.07.2014).
Den anden diskurs knytter sig til arbejdspligt, hvor lønarbejde er lig med den grundlæggende værdiskabende aktivitet i samfundet\footnote{Efter anden verdenskrig gik velfærdsstaterne ind i en ny historisk fase, hvor regeringerne forsøgte at skaffe fuldtidsjobs til alle voksne igennem en politik som havde til formål for det første at stimulere privat og offentlig vækst \parencite[17]{Keane1986}.}. %%% evt. parencite[34]{Bauman2002}

\textit{Moraldiskursen} handler om, at arbejdsløshed skyldes dovenskab og manglende arbejdsmotivation. Det vil sige den enkeltes dovenskab og manglende arbejdsmotivation har en påvirkning på arbejdsløsheden. Denne diskurs har fyldt meget i mediedebatten med overskrifter som “Joachim B. til arbejdsløs: Du er for slap” (Politiken.dk, 19.04.2012), “Dovne Robert på kontanthjælp i 11 år: Hellere kontanthjælp end et lortejob” (Ekstra Bladet, 11.09.2012) og “Vi arbejdsløse bliver opfattet som dumme, dovne og dårlige mødre” (Politiken, 06.11.2014).
Den tredje diskurs knytter sig ligeledes til arbejdspligt, hvor lønarbejdet er lig med et nødvendigt onde, som er nødvendigt for at få samfundet til at fungere og et onde, fordi det enkelte individ er tvunget til at arbejde\footnote{Den dominerende opfattelse af arbejdet som et \textit{meningsløs forbandelse} \parencite[13]{Svendsen2008} kan stadigvæk siges at være gældende i dag (\textbf{henvisning mangler}).}.
%%% Søren: Arbejdets sidste tjeneste i forbrugersamfundet er at skylde skyulden for de fattiges elendighed på deres manglende vilje til at arbejde og dermed anklage dem for moralsk depravation og fremstille fattigdommen som straffen for en synd \parencite[61]{Bauman2002}


Alle tre diskurser er fælles om, at arbejdsløshed anskues som et onde. Den danske realpolitik er domineret af de to sidstnævnte diskurser ved, at arbejdsløse mødes med de “økonomiske realiteter” eller “nødvendighedens politik” (\textbf{henvisning mangler}) med den føromtalte dagpengereform fra 2010 samtidig med, at denne reform og kommende reformer bakkes op af udsagn som “Det skal kunne betale sig at arbejde” \parencite{Stoejberg2015}. %%% Evt. \parencite[46]{Bauman2002} om økonomisk vækst



%%%%%%%%%%%%%%%%%%%%%%%%%%%%%%%%%%%%%%%%%%%%%%%%%%%%%%%%%%%
\subsection{Definition på arbejdsløshed}
%%%%%%%%%%%%%%%%%%%%%%%%%%%%%%%%%%%%%%%%%%%%%%%%%%%%%%%%%%%

Arbejdsløshed defineres typisk inden for den økonomiske disciplin ud fra, hvordan det er muligt at måle arbejdsløshed. Som vi senere skal komme tilbage til er der rigtig mange måder at måle arbejdsløshed på. Opgørelsen er langt fra uskyldig, og har en lang række underliggende antagelser om hvad arbejdsløshed er for et fænomen. Foreløbig skal der konstateres, at de to mest udbredte definitioner er den \textit{registerdata}baserede arbejdsløshedsopgørelse og den \textit{survey}baserede arbejdsløshedsopgørelse. Den registerbaserede arbejdsløshedsopgørelse måler antallet af personer på arbejdsløshedsunderstøttelse på et givet tidspunkt i løbet af året i hele populationen \parencite[594]{Mankiw2011}. Den surveybaserede arbejdsløshedsopgørelser er en stikprøvebaseret metode som anvendes i de såkaldte arbejdskraftsundersøgelser. Her defineres arbejdsløse ud fra \textit{International Labour Organisation}s definition, som er antallet af personer som står uden beskæftigelse samtidig med at være til rådighed for arbejdsmarkedet og aktivt arbejdssøgende \parencite{ILO1982}. I Danmark kaldes førstnævnte registerarbejdsløsheden og den anden kaldes for AKU-ledigheden \parencite{DST2014a}. Ulempen ved den registerbaserede arbejdsløshedsopgørelse er, at lovændringer kan resultere i ændringer i, hvem som har ret til arbejdsløshedsunderstøttelse for eksempel er dem som har ret til dagpenge ikke den samme type før og efter dagpengereformen i 2010\footnote{Fra 1994 og flere år frem var der strid om ledighedstallene fremsat af daværende finansminister Mogens Lykketoft. I 1993 blev der indført en ordning, som gav lønmodtagere mulighed for at tage orlov til at uddanne sig med henblik på at forkorte arbejdsløshedskøen med 15.000. Året startede kritikken som gik på, at den reelle arbejdsløshed var på vej op og regeringen sminkede statistikken ved at flytte titusinder af danskere fra arbejdsløshed til orlovsordninger, beskæftigelsesprojekter og andet. Bag striden lå de officielle månedstal fra Danmarks Statistik over dagpengemodtagere. På det tidspunkt var der mange der mente, at arbejdsløsheden lå langt højere end de officielle tal viste, fordi de officielle tal ikke omfattede førtidspensionister, efterlønsmodtagere over 60 år, ledige kursusdeltagere, personer i beskæftigelsesordninger, personer på ungdomsydelse og personer på revalideringsydelse \parencite{henvisning mangler}. Det der før var en strid er dog nu historie, så selvom ledighedstatikkerne har ændret sig, så viser tallene i figur \label{fig_udvikl.arbejdsloeshed}, at beskæftigelseskrisen vendte i 1993}. %%%%% Jens: Lovgivningsmæssige ændringer kan altså ændre på definition af hvem der skal registreres som ledige og går dermed sammenligninger over tid problematiske

De arbejdsløse er sammen med beskæftigede per definition arbejdsstyrken, mens den resterende ikke økonomisk aktive del befolkningen i den arbejdsdygtige alder per definition står uden for arbejdsstyrken\footnote{Denne opdeling anvendes til at udregne arbejdsløshedsraten, som er lig med procentdelen af arbejdsstyrken som er arbejdsløse, og arbejdsmarkedsdeltagelsesraten, som er lig med procentdelen af den befolkningen i den arbejdsdygtige alder som er en del af arbejdsstyrken. \parencite[595]{Mankiw2011}. Forholdet mellem arbejdsløshed og beskæftigelse er kompleks, fordi det er muligt for arbejdsløshedsraten at falde samtidig med at beskæftigelsen ikke stiger, hvis for eksempelvis arbejdsstyrken skrumper \parencite[449]{Cahuc2004}.}. Atkinson har kritiseret de fleste arbejdsmarkedsmodeller for at antage, at arbejdsløshedstilstande både starter og slutter med beskæftigelse uden at tage højde for at de i mange tilfælde både starter og slutter uden for arbejdsstyrken \parencite[1683]{Atkinson1991}. Atkinson er endvidere kritisk over for at behandle beskæftigede, arbejdsløse og personer uden for arbejdsstyrken som homogene kategorier \parencite[1683]{Atkinson1991}. Det kan være svært at skelne mellem en arbejdsløs og en person uden for arbejdsstyrken. Hvis vi tager udgangspunkt i de fire nedenstående eksempler på personer som ikke er i beskæftigelse:
%
 \begin{enumerate} [topsep=6pt,itemsep=-1ex]
   \item person søger job og modtager dagpenge
   \item person søger job, men modtager ikke dagpenge
   \item person søger ikke job, men modtager dagpenge
   \item person søger hverken job eller modtager dagpenge
 \end{enumerate}
% 
Det 1. person er tydeligvis arbejdsløs, fordi vedkommende søger job og modtager dagpenge. Men hvad med de andre tre? Den 2. person ville i den registerbaserede arbejdsløshedsopgørelse sandsynligvis blive karakteriseret som arbejdsløs, men ikke i den surveybaserede arbejdsløshedsopgørelse. Den 3. person ville aldrig blive karakteriseret som arbejdsløs i den registerbaserede opgørelse, men ville sandsynligvis være arbejdsløs i den surveybaserede opgørelse. %%%% Jens: Hov har I ikke byttet rundt på 2. og 3. person?
Den 4. person ville hverken blive karakteriseret som arbejdsløs i den ene eller den anden opgørelsen, men egentlig kunne vedkommende karakteriseret som en “modløs arbejdstager”, som er røget ud af dagpengesystemet og har mistet modet i forhold til at finde arbejde\footnote{Beskæftigede er også heterogen kategori, hvor man kan komme i beskæftigelse efter at have været arbejdsløs som selvstændig eller lønmodtager, fuldtid eller deltid og hvad Atkinson kalder almindelige eller marginale jobs \parencite[1685]{Atkinson1991}, hvilket minder om Guy Standings definition af prækære jobs \parencite[168]{Standing2011}.}.

I økonomisk teori er fuld beskæftigelse anvendes i forbindelse med et konkurrencedygtigt arbejdsmarked. I det konkurrencedygtige arbejdsmarked er lønnen i equilibrium, hvilket vil sige i perfekt balance mellem udbud og efterspørgsel. Samtidig er lønnen lig med værdien af arbejdskraftens marginalprodukt, hvilket vil sige profit per ansat. Equilibrium modellen, som dette udsagn også kaldes, hænger sammen med de grundlæggende antagelser om \textit{udbud}, \textit{efterspørgsel} og \textit{arbejdskraftens marginalprodukt}. På arbejdsmarkedet er arbejdskraft, jord og kapital de inputs som anvendes til produktion af varer og ydelser. Efterspørgslen på arbejdskraft bygger på antagelserne om, at virksomheder er \textit{konkurrencedygtige} og \textit{profitmaksimerende} \parencite[383]{Mankiw2011}. Når den konkurrencedygtige virksomhed skal ansætte en person, skal der tages højde for antallet af ansattes påvirkning af produktionen af varer. Hvis arbejdskraftens marginalprodukt er profitabelt, kan det betale sig for virksomheden at ansætte en ny person \parencite[384]{Mankiw2011}. Når arbejdstagere er ansat til at arbejde til en løn i perfekt balance mellem udbud og efterspørgsel samtidig med, at lønnen er lig med værdien af arbejdskraftens marginalprodukt, så vil der være fuld beskæftigelse. %%% Emil: Vi kunne godt bruge en halv sides Marx som kritik af det konkurrencedygtige arbejdsmarked
%%% Bauman: I producentsamfundet indtog fuld beskæftigelse en lidt tvetydig position, det den på en og samme gang var ret og pligt. Afhængigt at hvilken af ansættelseskontraktens parter der gjorde brug af princippet, rykkede den ene eller anden af detrs modaliteter i centrum, men som det fgælder for normer, skulle begge aspekter være til stede for at sikre princippets overordnende kraft. Tanken om fuld beskæftigelse som et uudværligt kendetegn ved et normalt samfund indebar både en pligt, som alle frivilligt påtog sig, og en fælles vilde, der blev ophøjet til en universel rettighed \parencite[61]{Bauman2002}. Fuld beskæftigelse bør også ses i lyset af, at industrisamfundets første årtier var plaget af en konstant mangel på arbejdskraft. Derfro blev den industrilelle æras fattige omdfineret til arbejdskraftens reservehær. Beskæftigelse, uafbrugt beskæftigelse, beskæftigelse, der ikke levende tid til udskejelser, var blevet normen, mens fattigdom var blevet gjort identisk med arbejdsløshed \parencite[132]{Bauman2002}.
Da fuld beskæftigelse reelt ikke eksisterer, skelner økonomiske arbejdsløshedsstudier mellem arbejdsløshed på lang sigt og arbejdsløshed på kort sigt, som vi henholdsvis vil beskæftige os med i det kommende to afsnit. 

% \subsubsection{SFI - en case}
% Det Nationale Forskningscenter for Velfærd (SFI) er storproducent af policy-studier på arbejdsmarkeds- og beskæftigelsesområdet med rapporter som typisk er bestilt af Beskæftigelesministeriet eller forskellige ministerier og kommuner. Rapporterne tager typisk udgangspunkt i en grupper af personer. Hvis man kigger på SFIs rapporter de sidste 20 år er målgrupperne hovedsageligt ledige\footnote{Også kaldet udsatte ledige, arbejdsmarkedsparate ledige, ikke-arbejdsmarkedsparate ledige, langtidsledige og forsikrede ledige}, dagpengemodtagere\footnote{Også kaldet sygedagpengemodtagere og aktiverede dagpengemodtagere.}, kontanthjælpsmodtagere\footnote{Også kaldet ikke arbejdsmarkedsparatemodtagere, de svageste kontanthjælpsmodtagere og aktiverede kontanthjælpsmodtagere.}, sygemeldte og arbejdsskadede\footnote{Også kaldet skadeslidte beskæftigede og personer som har nedsat arbejdsevne efter en ulykke i fritiden.} samt pensionister og efterlønsmodtagere\footnote{Der er også foretaget en hel del undersøgelse om handicappede, indvandrere, efterkommere, mænd, kvinder, ældre og højtuddannede.}. Fokus handler hovedsageligt om at få dem i beskæftigelse eksempelvis ved at bringe de langtidsledige tættere på arbejdsmarkedet i \textit{Tættere på arbejdsmarkedet} (2011), ved at måle beskæftigelseseffekten af dagpengeophør i \textit{Dagpengemodtagers situation omkring dagpengeophør} (2014), ved at kigge på indsatser over for ikke arbejdsmarkedsparate kontanthjælpsmotagere i \textit{Veje til beskæftigelse} (2010), ved at måle effekten af den beskæftigelesrettede indsats for sygemeldte i \textit{Effekten af den beskæftigelsesrettede indsats for sygemeldte} (2012) eller ved at kigge på pensionisters og efterlønsmodtageres genindtræden op arbejdsmarkedet i \textit{Pensionisters og efterlønsmodtageres arbejdskraftspotentiale} (2012). Det som er kendetegnede ved denne typer rapporter er et grundlæggende fokus på at få de pågældende personer tilbage i beskæftigelse hvad man kan gøre og ikke hvad deres situation egentlig betyder for deres liv\footnote{Der skal ikke menes med, at disse rapporter slet ikke forholder sig til de pågældende personers liv. I \textit{Veje til Beskæftigelse} (2010) fortælles der igennem 30 kvalitative interviews med sagsbehandlere, at de oplever de ikke-arbejdsmarkedsparate kontanthjælpsmodtagere som værende en heterogen gruppe som har gavn af forskellige typer indsatser alt efter, hvilke udfordringer de har. Nogle har for eksempel helbredsproblemer, mens andre har brug for hjælp til daglige gøremål.}.


%%%%%%%%%%%%%%%%%%%%%%%%%%%%%%%%%%%%%%%%%%%%%%%%%%%%%%%%%%%
\subsection{Årsager til arbejdsløshed - lang sigt og kort sigt}
%%%%%%%%%%%%%%%%%%%%%%%%%%%%%%%%%%%%%%%%%%%%%%%%%%%%%%%%%%%

Arbejdsløshed på lang sigt kaldes også den naturlige arbejdsløshedsrate og er en samlet betegnelse for den mængde arbejdsløshed en økonomi normalt er udsat for \parencite[592]{Mankiw2011}. På lang sigt skyldes arbejdsløshed friktionsledighed, strukturel lighed og sæsonledighed\footnote{I praksis er det vanskeligt at udskille friktions- og sæsonledighed fra strukturledighed, hvilket betyder, at de alle sammen oftest optræder som strukturarbejdsløshed \parencite{2015}.}. %%%% Emil: hvad betyder normalt - altså uden for kriser
\textit{Friktionsledighed} som også kaldes skifteledighed opstår ofte i forbindelse med indtræden på arbejdsmarkedet eller ved jobskifte. Det tager tid at matche den rette arbejdstager til det rette job, fordi arbejdstagerne blandt andet har forskellige præferencer og færdigheder og jobs kræver forskellige færdigheder\footnote{Det bryder med equilibrium-modellen som antager, at alle arbejdstagere kan passe hvilket som helst jobs, fordi alle arbejdstagere og alle jobs er identiske. Hvis dette var sandt, og arbejdsmarkedet var i equilibrium, så ville et jobtab ikke medfører arbejdsløshed, fordi en fyret arbejdstager ville finde et nyt job til markedsløn \parencite[163]{Mankiw2007}.} \parencite[163]{Mankiw2007}. %%%% Emil: Det skal ikke være en fodnote

\textit{Strukturel ledighed} opstår ved manglende faglig eller geografisk fleksibilitet på arbejdsmarkedet. Dette betyder, at lønnens balance mellem udbud og efterspørgsel ikke er i stand til at sikre fuld beskæftigelse \parencite[600]{Mankiw2011}. %%%%% Emil: Hvorfor er det ikke muligt at sikre fuld beskæftigelse 
Dette bryder med equilibrium-modellen, hvor de reelle lønninger tilpasser sig i perfekt balance med udbud og efterspørgsel. Men lønninger er ikke altid fleksible, fordi de reelle lønninger kan være fastsat over markedsniveau. Denne lønstivhed medfører strukturel arbejdsløshed, fordi den udbudte mængde af arbejdskraft er større end den efterspurgte mængde af arbejdskraft. Årsagerne her til er blandt andet minimumslønninger, fagforeningerne og effektivitetslønninger, som har til fælles at skabe lønninger over equilibrium \parencite[165]{Mankiw2007}.
\textit{Sæsonledighed} opstår på grund af sæsonmæssige svingninger i produktionen, hvor en stor del af sæsonarbejdsløsheden forekommer er midlertidig hjemsendelsesledighed, idet den arbejdsløse efter en kortere periode som arbejdsløs kommer tilbage til den samme arbejdsgiver \parencite{2015}.

Arbejdsløshed på kort sigt kaldes også konjunkturledighed, og karakteriseres ved økonomiske udsving fra år til år \parencite[592]{Mankiw2011}. John Maynard Keynes forsøgte som den første at forklare disse udsving i forbindelse med den depressionen i 1930'erne.  Ifølge Keynes ville lønninger og priser ikke nødvendigvis tilpasse sig på kort sigt. På den måde ville økonomien være i en position, hvor efterspørgslen ikke var tilstrækkelig til at skabe en beskæftigelse svarende til fuld beskæftigelse. Når økonomien ikke ville kunne tilpasses på kort sigt, mente Keynes, at staten burde gribe ind og administrere efterspørgslen for at opnå det ønskede beskæftigelsesniveau \parencite[707]{Mankiw2011}. Det kan staten typisk gøre ved finanspolitisk at øge forbruget for at booste den økonomiske aktivitet\footnote{Et eksempel herpå kunne være, når staten laver en kontrakt på 10 milliarder for at bygge tre nye atomkraftværker. Hermed skabes beskæftigelse og profit hos byggefirmaet, som resulterer i beskæftigelse og profit hos underleverandørerne. Alt i alt skaber der et forbrug, hvor hvert pund som er brugt får den samlede efterspørgsel på varer og ydelser til at stige for mere end et pund. Dette kaldes også multiplikatoreffekten som er defineret ved at være supplerende ændringer i den samlede efterspørgsel som finder sted, når  ekspansiv finanspolitik øger indkomst og dermed øger privatforbruget. \parencite[709]{Mankiw2011}.} eller ved pengepolitik, hvis centralbanken beslutter sig for at udvide mængden af penge \parencite[718]{Mankiw2011}. Udover Keynes insisteren på vigtigheden af arbejdsløshed på kort sigt spiller \textit{Phillipskurven} en stor rolle. Phillipskurven viser en negativ sammenhæng mellem arbejdsløshedsraten og inflationsraten, hvilket vil sige at år med lav arbejdsløshed har høj inflation, og år med høj arbejdsløshed har lav inflation \parencite[783]{Mankiw2011}. Finanspolitik og pengepolitik kan flytte økonomien langs Phillipskurven. Øgninger af pengemængden, øger det offentlige forbrug eller skattelettelser udvider den samlede efterspørgsel og flytter økonomien til et punkt i Phillipskruven, som giver et trade-off med lav arbejdsløshed for høj inflation \parencite[785]{Mankiw2011}.

\textbf{Hysterese} kendetegner midlertidige chok på økonomien, som har permanente eftervirkninger på arbejdsløshedsniveauet. Hermed er det muligt for, at arbejdsløshed på kort sigt har en effekt på arbejdsløsheden på lang sigt. Det kan ske ved, at visse arbejdsløse bliver ved med at være ekskluderet fra arbejdsmarkedet, fordi deres produktivitet er så lav, at det ikke er profitabelt at ansatte dem selv til en lav løn. Hvis der ikke er nogen måde at reintegrerer de arbejdsløse på, så vil de have en vedholdende effekt på arbejdsløshedsraten \parencite[477]{Cahuc2004}. Dette hænger sammen med den lave beskæftigelse af langtidsledige. De langtidsledige har svært ved at komme i beskæftigelse på grund af manglende påskønnelse af deres human kapital, manglende motivation i jobsøgning og kendsgerningen, at en lang arbejdsløshedsperiode kan fortolkes som en signal om at arbejdstagerens kvalitet ved ansættelsen kan forklare den dårlige præstation hos den langtidsledige \parencite[479]{Cahuc2004}.


%%%%%%%%%%%%%%%%%%%%%%%%%%%%%%%%%%%%%%%%%%%%%%%%%%%%%%%%%%%
\subsection{Opsummering} 
%%%%%%%%%%%%%%%%%%%%%%%%%%%%%%%%%%%%%%%%%%%%%%%%%%%%%%%%%%%
%%%%% Jens: Overkill

I dette afsnit har vi kontekstualiseret arbejdsløshed som problemstilling ved kort at skitsere arbejdsløshed i en historisk og en nutidig kontekst i et dansk perspektiv. Historisk er danske arbejdsløses forhold løbende blevet styrket i perioden fra den første lov om arbejdsløse i 1907 til 1970'erne. Siden har arbejdsløses forhold i højere grad været til debat og under under pres. Ifølge Halvorsen dominerer diskurserne “arbejdsløshed som social død” (elendighedsdiskursen), “arbejdsløshed som sløseri med ressourcer” (beskæftigelsesdiskursen) og “arbejdsløshed som dovenskab” (moraldiskursen) i større eller mindre grad synet på arbejdsløshed i hverdagen og den offentlige debat. Disse diskurser knytter sig ligeledes til de forskellige videnskabsdiscipliner, hvilket vi vil komme ind på i afsnittene om henholdsvis økonomiske og sociologiske forståelser af arbejdsløshed.

%%%% Jens: Det er ikke nødvendigt at gentage hvad I har gjort. Men jeg vil gerne vide hvorfor I har gjort det og hvad I har fået ud af det, som I vil tage med videre. Det står mig nemlig ikke helt klart. Det er et rigtig godt afsnit, men hvorfor er det der og hvad skal jeg tage med videre som læser?

Dette afsnit har behandlet arbejdsløshed i lyset af, at økonomisk teori har domineret arbejdsløshedsforskningen samtidig med at være det ideologiske grundlag for politiske beslutninger på arbejdsløshedsområdet. Inden for den økonomiske disciplin inddeles befolkningen typisk i arbejdsstyrken, som består af beskæftigede og arbejdsløse, og i den ikke-økonomisk aktive del af befolkningen, som står uden for arbejdsstyrken. I dette afsnit har vi beskæftiget os med en række modeller i relation til arbejdsløshed på kort og lang sigt. Arbejdsløshed på lang sigt kan opdeles i friktionsledighed, strukturel ledighed og sæsonledighed. Trade-off'et mellem arbejde og fritid, den basale jobsøgningsmodel og principal-agent-modellen er med til at forklare, hvorfor arbejdsløshed opstår, og hvordan man kan få arbejdsløse i beskæftigelse. Arbejdsløshed på kort sigt  opstår i forbindelse med konjunkturændringer, hvor hysterese er med til at forklare arbejdsløshed som konsekvens heraf. De økonomiske modeller er blevet kritiseret for at oversimplificere arbejdsløshedsproblemet ved at anvende for simple kategorier og for at reducere incitament til et økonomisk spørgsmål uden at forholde sig til institutionelle, sociale og psykologiske spørgsmål.

Efter denne gennemgang af økonomiske forståelser af arbejdsløshedsproblemet, vil vi gå over til at beskrive de sociologiske forståelser af arbejdsløshedsproblemet.


%Local Variables: 
%mode: latex
%TeX-master: "report"
%End: