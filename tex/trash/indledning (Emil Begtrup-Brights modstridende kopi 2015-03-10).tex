% -*- coding: utf-8 -*-
% !TeX encoding = UTF-8
% !TeX root = ../report.tex

\chapter{Indledning} \label{intro}


INTRODUKTION
Den danske beskæftigelsespolitik beskæftiger sig med at få ledige i job. I 2013 nedsatte regeringen et ekspertudvalg med tidligere skatteminister Carsten Koch i spidsen, som havde til opgave at udarbejde to rapporter om indsatsen over for ledige. Udvalgets anbefalinger i den første rapport var blandt andet en ny, individuel og jobrettet indsats for den enkelte, målrettet brug af opkvalificering i beskæftigelsesindsatsen, styrket fokus på virksomhedernes behov, styrkede økonomiske incitamenter for beskæftigelsessystemet og mindre bureaukrati.

Beskæftigelsesanbefalingerne beskriver primært hvilke midler man kan tage i brug for at få de ledige i arbejde. I dette speciale vil vi beskæftige os med, hvad det er for et arbejde de ledige får, når de kommer i beskæftigelse igen. Dette vil vi gøre ved at fokusere på, hvad den ledige kommer fra og går til før og efter en ledighedsperiode. Vi er interesseret i hvorledes såkaldte styrkede økonomiske incitamenter for beskæftigelsessystemet og deprivation som sociale konsekvenser af arbejdsløsheden spiller ind i forhold til at få ledige i job. Et eksempel kunne være en ledig sociolog, der har været ledig over længere tid eller er på vej ud af dagpengesystemet, tager et job som tjener. 
Samtidig er vi interesseret i sammenhængen mellem hvilken beskæftigelse man kommer fra og hvilke ressourcer man har til rådighed. Et eksempel kunne være, at forskellen mellem to ingeniører, hvor den ene kommer i job som ingeniør efter sin uddannelse og den anden kommer i job som pædagogmedhjælper. Den historie vi gerne vil fortælle handler om hvilke muligheder man har for at ernære sig, når man oplever arbejdsløshed, med særligt fokus på hvilken beskæftigelse, der så er alternativ til den, man oprindeligt havde. Derved vil vi diskutere hvilken praksis der hænger sammen med hvilke felter, og hvad det siger om hvilke felter der ligger nær hinanden, eller måske siger noget om hvor desperat man skal være, for at bevæge sig ud over det felt man er trænet ind i. Vi vil gerne diskutere hvad der strukturerer folk der oplever arbejdsløsheds opfattelse af handlingsrum, som vi ser det komme til udtryk igennem deres praksis mellem forskellige typer af jobs.

PROBLEMFORMULERING
Hvad betyder kortere og længere perioder med ledighed for de involverede personers sociale mobilitet på det danske arbejdsmarked?

UNDERSØGELSESSPØRGSMÅL
- Hvilke former for beskæftigelse skifter personer mellem efter en ledighedsperiode?
- Hvordan er lediges mobilitet på arbejdsmarkedet afhængig af hvilke ressourcer de ledige har?
- Hvilke konsekvenser har ledighed for aktørernes mobilitet på arbejdsmarkedet?
- Hvordan åbner den sociale mobilitet på arbejdsmarkedet op for andre måde at anskue ledighedsbegrebet på end i den gængse historiske, sociologiske og økonomiske forskning?

METODE
Vi ønsker at bruge MONECA, som er et netværksanalytisk redskab til at analysere mobilitet mellem forskellige typer af beskæftigelse for på den måde at lave en datadrevet (=*praksis*-drevet) skitse af hvilke beskæftigelsestyper der ligger tæt på hinanden frem for en teoretisk og institutionel. Det vil sige når en person med et arbejde er ledig en periode, hvilken beskæftigelse vender vedkommende tilbage til? Hvilke kanaler eksisterer for forskellige jobtyper efter perioder med arbejdsløshed? 
Den basale ide om at se på sammenhængene mellem job - ledig - job skal varieres med forskellige længder af ledighed, såvel som vigtige sociale karakteristika såsom køn, indkomst, familietype, geografi. Det er naturligvis også overordnet, men vi vil se på om der skulle gemme sig nogle interessante historier ved at fokusere på specifikke grupper. Vi er meget optagede af at rette det her så vi kan bruge Bourdieu fremfor diverse økonometrikere, der sidder og spytter tal ud omkring incitamenter for at få job i arbejde. 
En slags 2. second stage på analysen, som nok kun er relevant hvis tiden tillader det, er at udnytte ESS/EVS data til at sige noget om de værdier og sociale forhold, som disse surveys giver mulighed for. Det vil vi bruge at se på de clusters vi får ud af SNA (social netværksanalyse)-modellen, og se på hvad vi kan sige om deres værdier, muligvis ud fra en MCA, muligvis bare krydstabeller, hvis tiden er knap.


DISPOSITION

1. Indledning
- problemformulering

2. Baggrund
- historisk tilgang til ledighed, beskæftigelse, arbejdsmarked
- økonomisk tilgang til ledighed
- social tilgang til ledighed

3. Teori
- Bourdieu
- andre teoretikere

4. Metode
- Videnskabsteori
- Registerdata fra Danmarks Statistik
- MONECA
- evt. European Value Survey og European Social Survey
- evt. Multipel Korrespondance Analyse
- evt. Kvalitative interviews

5. Analyse
- kortlægning af perioden 1995-2009
- udvalgte typer af ledige

6. Diskussion

7. Konklusion



------------------------------------------------------------------------------------------

UGEPLAN
- læsegruppemøde (Søren)
- skabe discovariable, krydstabellen 
- tjekke ledighed er pålidelig 

------------------------------------------------------------------------------------------

TIDSPLAN

September
- korrektur både af os selv, andre sociologer og mødre, venner og bekendte
- læse specialet høj for hinanden
- specialekontrakt og foreslå mødedag til Jens

August
 - Konklusion, diskussion og analyse færdiggøres
 - Emil en uge sommerskole
 - evt. interviews

Juli 
- ferie, sol sand og strand 

Juni
- Analyse hoveddel færdig
- Statistisk data helt færdig
- Teori stort set udarbejdet

Maj
- Metodeafsnit færdig
- moneca-kort færdige (ikke færdiglayoutede men indholdsmæssigt)
- krydstabeller næsten færdige, tages stilling til MCA
- evs/ess-data færdig

April
- MONECA second run færdig, tages stilling til om flere skal laves. historier udarbejdes
- DST databehandling færdig
- læsning af bøger og skitser af teoriafsnit
- evs/ess-data forsøges kobles på dst-data

Marts
- MONECA first run færdig, historier udarbejdes
- DST databehandling i gang
- indledende krydstabeller
- læsning af bøger og skitser af teoriafsnit
- baggrundsafsnit færdigt

Februar
- problemformulering, disposition og tidsplan på plads
- læsning
- dataeyeballing (DST/MONECA)
- test run af Antons kode
- merge datasæt 

------------------------------------------------------------------------------------------


%Local Variables: 
%mode: latex
%TeX-master: "report"
%End: 