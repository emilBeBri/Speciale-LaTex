% -*- coding: utf-8 -*-
% !TeX encoding = UTF-8
% !TeX root = ../report.tex

%%%%%%%%%%%%%%%%%%%%%%%%%%%%%%%%%%%%%%%%%%%%%%%%%%%%%%%%%%%
\newpage \section{\textsc{Mobilitet \label{}}}
%%%%%%%%%%%%%%%%%%%%%%%%%%%%%%%%%%%%%%%%%%%%%%%%%%%%%%%%%%%


%%%%%%%%%%%%%%%%%%%%%%%%%%%%%%%%%%%%%%%%%%%%%%%%%%%%%%%%%%%
\subsection{\textsc{Mobilitet som problemstilling}}
%%%%%%%%%%%%%%%%%%%%%%%%%%%%%%%%%%%%%%%%%%%%%%%%%%%%%%%%%%%

Formål i sig selv. 

Kobling med arbejdsløshed.


%%%%%%%%%%%%%%%%%%%%%%%%%%%%%%%%%%%%%%%%%%%%%%%%%%%%%%%%%%%
\subsection{\textsc{Beskæftigelsesmobilitet}}
%%%%%%%%%%%%%%%%%%%%%%%%%%%%%%%%%%%%%%%%%%%%%%%%%%%%%%%%%%%

Økonomiske teorier og studier. %%% Italienske økonomer 

Sociologiske teorier og studier. %%% White og andre 



%%%%%%%%%%%%%%%%%%%%%%%%%%%%%%%%%%%%%%%%%%%%%%%%%%%%%%%%%%%
\subsection{\textsc{Arbejdsmarkedssegmenter}}
%%%%%%%%%%%%%%%%%%%%%%%%%%%%%%%%%%%%%%%%%%%%%%%%%%%%%%%%%%%

Formål.

Anvendelse. %%% Toubøl og Larsen 



%%%%%%%%%%%%%%%%%%%%%%%%%%%%%%%%%%%%%%%%%%%%%%%%%%%%%%%%%%%
\subsection{\textsc{Opsummering}}
%%%%%%%%%%%%%%%%%%%%%%%%%%%%%%%%%%%%%%%%%%%%%%%%%%%%%%%%%%%


%%%%%%%%%%%%%%%%%%%%%%%%%%%%%%%%%%%%%%%%%%%%%%%%%%%%%%%%%%%
% Trash
%%%%%%%%%%%%%%%%%%%%%%%%%%%%%%%%%%%%%%%%%%%%%%%%%%%%%%%%%%%

% \section{Kategorisering af arbejdsmarkedet i segmenter \label{kategori_teori}} %label var tidligere bare disco


% fra Jonas Toubøls slide om Moneca til netværksfag i efteråret

% Den typiske teoridrevne tilgang:

% Opstil en teori om arbejdsmarkedets segmentering
% Dual labour market theory, mikroklasser, fag, stillinger osv.

% Opdel arbejdsmarkedets brancher ud fra din teori om segmentering

% Beskriv empirisk de forskelle der er på arbejdskraften de forskellige dele af arbejdsmarkedet imellem

% Analyser om der kan siges at være tale om en segmentering der bekræfter teorien
% F.eks. hvis der er et sammenfald mellem køn, etnicitet og de dele af arbejdsmarkedet der er præget af usikre ansættelsesvilkår og dårlige løn- og arbejdsforhold

% Konkludér

% Problemer ved den typiske tilgang:

% Man observerer ikke direkte mobiliteten på arbejdsmarkedet (man ser på ‘snapshots’, der ikke er dynamiske)
% - Ihukom definition af segmenter

% Segmenterne er derfor ikke empirisk overserverede størrelser der er defineret ud fra tilstedeværelsen af arbejdskraftsstrømme og disses intensitet

% Mit forsøg på at overkomme problemet:

% Netværksanalysens værktøjer kan jeg bruge til at kortlægge mobiliteten på arbejdsmarkedet og dermed identificere segmenterne ud fra strømme af arbejdskraft
% NB! Jeg er interesseret i at betragte stillingskift – ikke individer

% Resultat: En metode til empirisk at identificere de givne segmenter som supplement til den teoridrevne tilgang





% Social netværksanalyse er en relationel analysemetode. Vi benytter en udgave af denne til at kortlægge lediges beskæftigelsesmobilitet, som Toubøl og Larsen har udviklet til at inddele sociale grupper i større segmenter \parencite{Touboel2013, Touboel2015}. Med deres tilgang ligger vi i forlængelse af arbejdsmarkedssegmenteringsteorien udviklet i 1960'erne og 1970'erne (Becker 1993; Hall 1970; Phelps Brown 1977), hvor flere studier viste, at der eksisterede barrierer mellem de gode og de dårlige jobs i forhold til løn og arbejdsvilkår med videre (Bluestone 1971; Doeringer og Piore 1971 og 1975; Gordon 1972; Reich et al. 1973). Det nye i Toubøl og Larsens tilgang er, at metoden generer segmenter a posteriori i stedet for a priori at lave segmenter ud fra et teoretisk perspektiv (fx Osterman 1975; Fichtenbaum et al 1994; Stier og Grusky 1990) eller på baggrund af løn, færdigheder og arbejdsvilkår (fx Boston 1990; Hudson 2007; Daw og Hardie 2012). Men givet at sociale grupper og deres relationer til andre grupper er komplekse, samt at mange grupper kan have relationer til mange andre grupper, hvordan kan man så overhovedet inddele disse grupper i meningsfulde segmenter? Formuleringens forførende enkelhed bør ikke narre nogen. Det er et af sociologiens grundlæggende spørgsmål: Hvad konstituerer en social gruppe, efter hvilke principper finder en sådan konstituering sted, og hvilken betydning har den? %%% Jens: start med dette, frem for det tekniske. Vis at det er dette teoretiske spørgsmål i forsøger at angribe på en bestemt måde ved at bruge netværksanalysen

% Det er i sin essens spørgsmålet om hvad klasse er, hvordan klasse virker og hvad klasse gør. En række nye bidrag til forskning i klasse, der betoner forskellige traditioner eller forsøger at sammenfatte forskellige indsigter fra disse traditioner, har i løbet af 00'erne og 10'erne vundet indpas, se for eksempel \textcite{Lareau2008}, \textcite{Resnick2006}, \textcite{Savage2013} og \textcite{Andrade2014}. Det er ikke dette speciales hovedformål at klargøre hvad social klasse \emph{er}, eller give et fyldestgørende overblik over diskussionerne omkring det. Det er dog et centralt og helt nødvendigt element at forholde os til forskellige definitioner af klasse, når specialets primære metode i sin essens handler om at skabe klynger baseret på mobilitet mellem arbejdsstillinger. Det rejser en række spørgsmål om hvorledes man bør inddele i klasser, og hvad styrker og begrænsninger er i vores måde at klassificere forskellige subgrupper af ledige på. Man kunne sige at et spøgelse vil gå gennem specialet - klasssespørgsmålets spøgelse. Vi vil med jævne mellemrum fremmane dette spøgelse for læseren, således at denne med egne øjne kan se, hvordan det animerer vores arbejde.


% %%%%%%%%%%%%%%%%%%%%%%%%%%%%%%%%%%%%%%%%%%%%%%%%%%%%%%%%%%%%%%%%%%%%%%%%%%%%%%%%%%%%%%%%%%%%%%%%%%%

% \section{At snuble over sin egen position i rummet}

% Det første vi rent praktisk har stået overfor, er vores inddeling af beskæftigelseskategorier fra \texttt{DISCO}, som vil blive uddybet på side \pageref{disco_omkodninger}. Her stod vi overfor at skulle sammenlægge en række beskæftigelseskategorier, da det ikke er muligt at benytte alle de knap ottehundrede forskellige \texttt{DISCO}-kategorier, af åbenlyse årsager, relateret til overskuelighed. Bourdieu beskriver, hvordan der eksisterer en homologi mellem de sociale og mentale strukturer, mellem de objektive adskillelser i den sociale verden og de perspektiver, agenterne selv ser denne verden igennem \parencite[12]{Bourdieu1992}. Deraf følger, at ethvert blik på disse strukturer er svært adskilleligt fra éns egen position i det sociale rum. Vi vil have tendens til at kende de felter, vi selv er i berøring med, bedre end de felter, vi ikke kender til, da vores kategorier til at forstå verden igennem ikke er individuelle, men kollektive repræsentationer, og de er struktureret ud fra den eller de sociale grupper, vi befinder (eller har befundet) os i \parencite[12]{Bourdieu1992}. Der opstår en forvrængning:
% % %
% \begin{quote} \small %\raggedright %(bloktekst on/off)
% alene af den grund, at for at kunne strukturere, beskrive og fremstille det, må man i samme bevægelse i videst muligt omfang lægge afstand til det. Her sker der typisk en forvrængning, fordi man i den teoretiske model, der konstrueres af det sociale, glemmer at den er et produkt af en teoretisk og distancerende holdning. En refleksiv sociologi i ordets egentlige betydning må konstant være på vagt over for den form for “akademisk etnocentrisme”, der overser alt det, forskeren indlæser i undersøgelsesgenstanden, i kraft af at han eller hun befinder sig udenfor den og studerer den på afstand og oppefra. \sourceatright{\emph{\parencite[62]{Bourdieu1996}}}
% \end{quote}
% %
% Det findes næppe en måde mere “på afstand og oppefra” end at se på registerdata gennem en VPN-forbindelse til Danmarks Statistiks servere. Der er en voldsom afstand til den praksis vi forsøger at beskrive, og det er helt centralt, at vi og læseren hele tiden forholder sig til dette vilkår for analysen%%% Jens: fine overvejelser nedenfor, men skal det legitimeres med Bourdieu?
% %
% \footnote{Under dataarbejdet sad vi og skulle tjekke om min kode for inddeling i ledighedsperioder tog højde for en række scenarier. Vi plejer at vælge 20-30 paneler ud, som vi så kigger på for at se om koden gør det den skal. Her sad vi og kiggede på om vores ledighedsvariabel opførte sig som ønsket. Det er derfor interessant at se på dem der ryger ind og ud af ledighed flere gange i perioden, for at se om koden er fleksibel nok til at kategorisere dem korrekt. vi sad derfor og kiggede på et panel, der havde 3 års beskæftigelse fra 1996 til 1999, efterfulgt af revalideringsydelse, derefter arbejde et enkelt år, og diverse overgangsydelser frem til 2009. De fleste af panelets udfald i \texttt{SOCSTILL} var koderne \texttt{325}, \texttt{326} og \texttt{322}. De koder stod for revalideringsydelse, kontanthjælp og dagpenge. Det går pludselig op for mig, at det her ikke bare er et panel, det er konkret menneske, der mellem 1996 og 2009 oplevede en voldsom deroute. At det overhovedet er muligt at opleve så abstrakt en relation til et “element” i sit genstandsfelt, at man får oplevelsen af at vågne ud af abstraktionen, og tænke “gud, det er et menneske”, det siger meget om farerne ved den metode vi har valgt at gå til arbejdsløshed på. Bourdieu skriver i indledningen til \emph{The Weight of the World}, “the discussion must provide all the elements necessary to analyze the interviewees' positions objectively and to understand their point of view, and it must accomplish this without reducing the individual to a specimen in a display case” \parencite[2]{Bourdieu1999}. Der er en klar væsensforskel i metode som gør den metodemæssigt kvalitativt orienterede \emph{The Weight of the World} har nemmere ved at forholde sig direkte til dem, der undersøges, men idealet er centralt og bør nok særlig forfølges af dem, der - som os - benytter redskaber, der fordrer eller hvis grundlag måske ligefrem \emph{er} denne \emph{specimen}-tilgang.}. 
% %
% Et eksempel på denne effekt opstod tidligt i kategoriseringsprocessen. Her skulle Emil vurdere hvilke grupper indenfor den overordnede kategori, \emph{Assistentarbejde indenfor sundhedssektoren} (\texttt{3220}), der kunne lægges sammen. Emil havde enormt svært ved at lægge gruppen \emph{Arbejde med emner inden for fysioterapi, kiropraktik mv.} (\texttt{3226}) sammen med gruppen \emph{Arbejde med emner inden for ergoterapi, zoneterapi, yoga med videre} (\texttt{3229}). Han kunne jo se, at det jo er to helt forskellige ting! Fysioterapeuter og kiropraktikere er accepterede og institutionaliserede dele af sundhedssektoren, mens den anden kategori indeholder sundhedsrelaterede behandlere, der befinder sig i randen af disse institutioner. Emil har gået til yoga, og han har en fornemmelse af, at de ikke er samme typer som fysioterapeuter og kiropraktorer. Dem kunne han da umuligt lægge sammen, da det ville være at gøre vold på den sociale virkelighed. Så gik det op for Emil, at han en halv time forinden havde slået alle undergrupper, der arbejder med teknikerarbejde, sammen til en enkelt gruppe, og dét uden at blinke. Vi taler om mennesker, der laver så forskelligartede ting som at arbejde med elektroniske anlæg\footnote{Såsom byggetekniker og landmaalingstekniker.}, bygningsrelateret anlægsarbejde\footnote{Såsom elektroniktekniker, køletekniker, teletekniker.} og arbejde vedrørende maskiner og røranlæg \footnote{Såsom gastekniker, konstruktør, maskintekniker, VVS-tekniker, værktøjstekniker. Dette er “\emph{eksklusiv vedligeholdelse af maskiner ombord på skibe}”.}. Der er tale om en gruppe arbejdsfunktioner, der i gennemsnit beskæftiger 40.507 mennesker om året. Til sammenligning beskæftigede gruppen af fysioterapeuter og kiropraktikere 12.499 personer, mens yogalærerne og zoneterapueterne beskæftigede 2.382, i alt 14.811 personer, under halvdelen af antallet med teknikeruddannelserne. Emil kender ingen køleteknikere, vejrobservatøre, gasteknikere eller laboranter, men han kender en fysioterapeut og flere der har taget diverse yogalæreruddannelser. Det er en kæmpeudfordring i forbindelse med dette projekt. Det er helt nødvendigt og udemærket at benytte sin faglige og personlige viden om sociale relationer, vilkår og status tilknyttet forskellige beskæftigelser til at skabe nogle meningsfyldte grundkategorier. Om man “bør” gøre det giver efter vores opfattelse et spørgsmål med omtrent som meget mening som “bør man rette sig efter tyngekraften?”. Det er i bedste fald naivt. Der findes ikke en taksonomi, der står udenfor de sociale kampe, disse taksonomier er et produkt af. Særligt med sammensatte og til en vis grad abstrakte inddelinger som “beskæftigelseskategorier” må vi gå til opgaven med omtanke. Som Bourdieu skriver i \emph{Practical Reason}:
% %
%  \begin{quote} \small %\raggedright %(bloktekst on/off)
% 	To endavour to think the state is to take the risk of taking over (or being taken over by) a thought of the state, that is, of applying to the state categories of thought produced and 	guaranteed by the state and hence to misrecognize its most profound truth. \sourceatright{\emph{\parencite[35]{Bourdieu1998}}}
%  \end{quote}
%  % 
% Her er inddelingen af teknikere farlig, da den har en tilsyneladende og \emph{troskyldig} objektivitet bag sig. I en vis forstand er den rimelig nok, da disse opdelinger i eksempelvis tekniker-kategorierne ud fra arbejdsfunktion- og færdighedsskøn sandsynligvis godtgør nogle delte sociale vilkår, der også er omsat til kognitive strukturer for den enkelte. I moderne samfund er homologien mellem sociale strukturer og kognitive strukturer i høj grad formet af uddannelsessystemet som en, hvis ikke den, afgørende smeltedigel for de habituelle dispositioner hos agenterne \parencite[12]{Bourdieu1992}. Det er klart, at i et samfund der i så høj grad kræver institutionaliseret kulturel kapital gennem uddannelsesystemet, vil der være tilbøjelighed til at de lignende vilkår skabt igennem dette system også kommer til udtryk i delt socialtet. Der er derfor en vis fornuft i benytte sig af et klassifikationsystem, der bruger dette som udgangspunkt for inddelingen. Omvendt vil en for stor tiltro til denne inddeling vanskeliggøre netop den opgave, vi har stillet os selv, nemlig at fremvise et systematiske forskelle og ligheder mellem ledige baseret på de mønstre, som deres jobmuligheder giver dem, og se hvordan det hænger sammen med fordelingen af andre slags ressourcer. %%% Jens: hvorfor og hvordan?

% \texttt{DISCO} har nogle ganske fornuftige egenskaber, i og med inddelingen hos Danmarks Statistik det er baseret på hvilken arbejdsfunktion, der udføres, hvilket defineres som “\emph{et sæt af arbejdsopgaver, der i indhold og opgaver er karakteriseret ved en høj grad af ensartethed}”, samt en vurdering af det færdighedsniveau, de er udført på. Det defineres som “\emph{en funktion af kompleksiteten og omfanget af de opgaver, der er indeholdt i en given arbejdsfunktion}” \parencite[7]{Ploug2011}. Dette skulle gerne godtgøre en vis grundlæggende enshed i de førnævnte kognitive og sociale strukturer, så længe at vi befinder os på et tilstrækkeligt detaljeret niveau. Som Gitte Sommer Harris pointerer om empirisk arbejde med klassebegreber, har \texttt{DISCO}-variablene, grundet deres detaljeniveau, et ganske særligt potentiale for at kunne operationaliseres til en række vidt forskellige definitioner af klasse, lige fra Gruskys mikroklasser, over Goldthorpes EGP-klasseskema, til Olin Wrights inddeling baseret på uddannelses- og lederressourcer \parencite[172]{Harrits2014}. 

% Det gennemgribende problem vi ser er, at en højere-ordensinddeling på et (stærkt) teoretisk eller administrativt funderet grundlag, ikke har evnen til at indfange de feltspecifikke logikker, funderet i praksis. Der er en pendant til når Bourdieu kritiserer  teoretisk reduktionisme, som det kommer til udtryk ved at direkte relationer formodes givne mellem et kulturelt udtryk og de sociale klasser, de formodes at være stillet til eller udspringer af. Det reducerer disse udtryk til en funktioner, der overser den interne logik hos dem, der har skabt dette produkt:
% %
% \begin{quote} \small %\raggedright %(bloktekst on/off)
% One cannot understand what is going on without reconstructing the laws specific to this particular universe, which, with its lines of force tied to a particular distribution of specific kinds of capital (economic, symbolic, cultural, and so on), provides the principle for the strategies adopted by different producers, the alliances they make, the schools they found, and the art they defend. \sourceatright{\emph{\parencite[544]{Bourdieu1988}}}
% \end{quote}
% %
% I vores tilfælde kan denne tilgang til feltlogikker benyttes både til producenterne af statens statistikker hos Danmarks Statistik, samt den perlerække af felter, som udgør det danske arbejdsmarked, som de arbejdsløse i vores undersøgelse pendulerer ind og ud af. Samt den specifikke logik, der fungerer indenfor hvert subfelt af arbejdsmarkedet, og som gælder de arbejdsløse, der oplever tibagevendende eller længerevarende arbejdsløshed, hvorigennem der opstår helt særlige overlevelsesstrategier. Hos producenterne hos Danmarks Statistik kan en opdeling ud fra “\emph{ensheden i arbejdsfunktionen}”, samt “\emph{kompleksiteten og omfanget af en given arbejdsfunktion}” netop, grundet feltlogikken hos DST selv, være slørende for de feltlogikker, der får arbejdsløse til at søge hen i jobs, der sandsynligvis har en vis “enshed i arbejdsfunktionen” og indeholder samme \emph{type} “kompleksitet”, som man er vant til at håndtere. Men der er stor sandsynlighed for at vi ikke kan se den form for enshed og den form for kompleksitet, fordi vi står udenfor feltets logik og de love, der udgør dette felt, de sociale kræfter, der fungerer i netop denne sammensætning af symbolske og materielle ressourcer. %%% Jens: hvor er i på vej hen?
% % indsæt eventuelt kritik af Bourdieus feltbegreb fra forbrugskulturfaget her.  

% Vi kan helt sikkert ikke afdække eller rekonstruere de komplekse love %%% Jens: dårligt ord
% , der gælder for forskellige felter indenfor det danske arbejdsmarked %%% Jens: er der felter indenfor arbejdsmarkedet?
% , samt de særlige forhold, der gør sig gældende for arbejdsløse indenfor disse felter. Det er vi gennem vores VPN-forbindelse alt for langt væk fra praksis til. Men det vi kan gøre, er at beskrive nogle helt centrale mulighedsstrukturer for arbejdsløse. Som Bourdieu beskriver i relation til praksis i \emph{Pascalian Meditations}, er de praktiske nødvendigheder udtryk for dels de strukturer af håb og forventninger, der er indlejret i habitus, og dels de strukturelle muligheder, som er konstituerende for et socialt rum \parencite[211]{Bourdieu2000}. \textbf{Vores bidrag ligger i at beskrive disse objektive sandsynligheder på arbejdsmarkedet i forbindelse med arbejdsløshed for den enkelte, og dermed synliggøre en dimension af det sociale rum, der strukturerer disse håb og forventninger.} %%% Jens har highlightet det her uden kommentar
%  Det lader sig kun gøre hvis vi har så praksisfølsom en model som overhovedet muligt, hvilket vi mener moneca giver gode muligheder for, så vi kan beskrive disse objektive strukturer så fyldestgørende som muligt, givet at de er skrevet ud fra den føromtalte “\emph{specimen}”-tilgang. Her giver overblikket gennem registerdata netop unikke muligheder for at afdække disse mulighedsstrukturer. \textbf{Udfordringen er ikke benytte disse kategorier, som staten har stillet til rådighed for os, på en sådan måde, at præcisionen i den sociale inddeling er størst indenfor de felter, vi selv har en relation til, og stort set ikke-eksisterende på de felter, vi ikke kender til, hvor vi bliver nødt til at forlade os på statens højere kategori-niveauer.} %%% Jens har highlightet det uden kommentar
%  Det har vi forsøgt at undgå ved at benytte så lavt et niveau som er praktisk muligt i vores \texttt{DISCO}-inddeling, da det ikke altid er muligt at benytte sig af kategorierne på det laveste niveau, samt være sensitive i vores tilgang til disse inddelinger.


%%%%%%%%%%%%%%%%%%%%%%%%%%%%%%%%%%%%%%%%%%%%%%%%%%%%%%%%%%%


%Local Variables: 
%mode: latex
%TeX-master: "report"
%End: