% -*- coding: utf-8 -*-
% !TeX encoding = UTF-8
% !TeX root = ../report.tex



%%%%%%%%%%%%%%%%%%%%%%%%%%%%%%%%%%%%%%%%%%%%%%%%%%%%%%%%%%%%%%%%%%%%%%%%%%%%%%%%%%%%%%%%%%%%%%%%%%%
%%%%%%%%%%%%%%%%%%%%%%%%%%%%%%%%%%%%%%%%%%%%%%%%%%%%%%%%%%%%%%%%%%%%%%%%%%%%%%%%%%%%%%%%%%%%%%%%%%%
%%%%%%%%%%%%%%%%%%%%%%%%%%%%%%%%%%%%%%%%%%%%%%%%%%%%%%%%%%%%%%%%%%%%%%%%%%%%%%%%%%%%%%%%%%%%%%%%%%%


\section{KORT OVER DE LEDIGE BEVÆGELSER PÅ ARBEJDSMARKEDET \label{}}

Kortet viser et netværk af forskellige arbejdsstillinger indelt i 150 \texttt{DISCO}-kategorier og 32 segmenter. I netværket bevæger individer sig mellem forskellige typer af arbejdsstillinger. Det sker når en person går fra at være beskæftiget i en arbejdsstilling til at være beskæftiget i en anden arbejdsstilling efter en mellemliggende periode med ledighed eller uden beskæftigelse. Arbejdsstillingerne kommer til udtryk som de 150 \texttt{DISCO}-kategorier  tager form som noder i netværket, og personernes bevægelser mellem forskellige arbejdsstillinger er det som frembringer  i netværket. Nedenstående kort giver os mulighed for at overskue beskæftigelsesmønstre for lediges bevægelser ind og ud af arbejdsmarkedet.
% 
\begin{figure}[h]
\begin{centering}
	% \caption{Mobilitetsmønstre hos ledige, 1996-2009}
	\includegraphics[width=\textwidth]{fig/metode/hovedkort.pdf}
	\label{fig_hist_beskaeftigede_allekategorier}
\end{centering}
\end{figure}

Som det fremgår af kortet findes der nogle meget store segmenter som omfatter meget forskelligt type arbejde. Her er det eksempelvis relevant at se på det store faglige segment (5.3), omsorgs-segmentet (3.14), kontor-segmentet (5.2), magister-segmentet (5.1) og salgs-segmentet (5.4), som repræsenterer de fem største segmenter i forhold til antal unikke personer, antal skift ud-og-ind på arbejdsmarkedet og antal noder (\texttt{DISCO-kategorier}). 

Det faglige segmenter er det absolut største og repræsenterer 37 noder (og 37 \texttt{DISCO}-kategorier) og indeholder 24 \% af alle disco-kategorierne. Eftersom størrelsen på kategorierne varierer ganske betragteligt, som beskrevet i afsnit \ref{fig_hist_beskaeftigede_allekategorier}, er et mere sigende mål hvor mange beskæftigede, der i gennemsnit er tale om over perioden. Her har segmentet en andel på 36 \%,hvilket svarer til 237.411 personer. De to største \texttt{DISCO}-kategorier befinder sig i dette segment, og de står sammen for 12 \% af arbejdsmarkedet for ledige. Taget i betragtning af at det næststørste segment kun har en andel på 16,5 \%, må man sige at segmentet fylder ganske meget på det danske arbejdsmarked. 

Det fremgår også af kortet, at det ikke er alle noder som samler sig med andre til større segmenter. Der er i alt 13 noder, som ligger sig for sig selv. Når en node er for sig selv, betyder det, at der ikke er en signifikant bevægelse til andre noder. For at bruge et eksempel med lægerne. Lægerne er både en node og et segment, hvilket egentlig giver meget god mening, da lægearbejdet er så specialiseret, at ingen andre faggrupper kan varetager en læges job hverken uden eller med en periode uden for beskæftigelse. Det er hvad man inden for arbejdsmarkedssegmenteringsteorien ville kalde for enten funktionel eller institutionel form for beskæftigelsesmønster. På den ene side skal man nemlig have de rette færdigheder for at have mulighed for at indtage en bestemt arbejdsstilling, og for det andet skal man egentlig også have det rette certifikat for at komme i betragtning til en bestemt arbejdsstilling \parencite[3]{TouboelLarsenJensen2013} \parencite[4]{TouboelLarsen2015}. Dette er dog ikke ensbetydende med, at lægerne ikke kunne bevæge sig mod andre type arbejdsstillinger (noder/\texttt{DISCO}kategorier) især efter en ledighedsperiode. Vi kan se, at der har været skift fra lægearbejde mod sygeplejearbejde i 43, mod undervisning paa universiteter og andre hoejere laereanstalter 15 tilfælde, mod ledelse af virksomhed faerre end 10 ansatte i 13, mod militaert arbejde i seks tilfælde, mod alment kontorarbejde i seks tilfælde og mod rengoerings- og koekkenhjaelpsarbejde i fem tilfælde. Dette skal selvfølgelig sammenlignes med de 1917 tilfælde af skift fra lægearbejde til lægearbejde. Der er altså meget få længer som vælge at skifte væk fra lægearbejdet efter en ledighedsperiode. Hovedårsagen er mest sandsynligt, at det har at gøre med at manglen på læger altid er stor og derfor ledighedsperioden stor (henvisning), hvilket betyder at selvom man er ledig i en periode vender man typisk tilbage i arbejde inden for en periode, hvilket fremgår summen af ledighedsperioder gennem hele arbejdslivet, som er særlig lav for læger se afsnit \ref{?}. Hvis det modsatte var tilfældet nemlig, at der er for mange læger i arbejdsstyrken, kunne man forestiller sig, at lægerne ville søge nye jobs og at gruppen som bevæger sig mod regnørings-- og køkkenhjælpsarbejde ville være større. Hvis vi kortlage beskæftigelsesmønstre i for eksempel Cuba, som flest antal læger pr. indbyggere i verden (henvisning), kunne det eksempelvis være tilfældet, hvilket muligvis kunne resultere i at lægerne ikke lå i et segment for sig selv.

For at illustrere kortet vil vi tage fat i en case med akademikernes fordeling på kortet.


%%%%%%%%%%%%%%%%%%%%%%%%%%%%%%%%%%%%%%%%%%%%%%%%%%%%%%%%%%%%%%%%%%%%%%%%%%%%%%%%%%%%%%%%%%%%%%%%%%%

\subsection{Case: Akademikernes fordeling på det store kort \label{}}

Når man taler om uddannelsesgrupper opdeles de typisk i faglærte, ufaglærte og videregående uddannelser, de videregående uddannelser opdeles i de korte, de mellemlange og de lange, de lange videregående uddannelser i hmuanistiske, naturvidenskabelig, sundhedsvidenskabelige, samfundsvidenskabelige, tekniske, farmaceutiske, teologiske og så videre \parencite{Groes2014}. 

For at skelne mellem på den ene side de forskellige udddannelsesgrupper og på den anden side opdelingen af de lange videregående uddannelser i faggrupper, kan man sige at førstnævnte tager udgangspunkt i den internationale uddannelsesklassifikation\footnote{International Standard Classification of Education (\texttt{ISCED}) placerer uddannelse på ti niveauer: førskoleniveau (børnehaveklasse), grundskoleniveau I (1.-6. klasse), grundskoleniveau II (7.-10. klasse/årgang), gymnasialt niveau I (10. uddannelsesår), gymnasialt niveau II (11.-12. uddannelsesår), korte videregående uddannelser (13.-14. uddannelsesår), mellemlange videregående uddannelser (15.-16. uddannelsesår), lange videregående uddannelser (17.-18. uddannelsesår) og forskerniveau (19.- uddannelsesår) (henvisning).} og sidstnævnte følger de danske universiteters inddeling på baggrund af hvilket fakultet, man er på. Sidstnævnte er problematisk, fordi psykologi på Aarhus Universitet for eksempel ligger under det sundhedsvidenskabelige fakultet, fordi det har rødder i sundhedsvidenskaben, mens det på Københavns Universitet ligger under det samfundsvidenskabelige fakultet. Vore indelinger af akademikre på arbejdsmarkedet bryder med begge selninger. Som det fremgår af kortet, kan man se alle de mørkeblå noder er akademikerarbejdsstillinger (eller viden på højeste niveau). Her er der 12 segmenter ud af de 32 forskellige som indeholder personer med viden på højeste niveau, som fremadrettet vil blive kaldt for akademisk arbejdskraft. De tre segmenter, vi ønsker at fremhæve er karakteriseret ved, at det er tre clusters med flest akademikere og som vi kalder for magisterclusteret, djøfclusteret og kreaclusteret\footnote{De ni andre segmenter er: 1) \emph{Udvikling og analyse af software og applikationer} (\texttt{2130}) og \emph{Edb teknisk arbejde, primaert programmoer} (\texttt{3120}), 2) \emph{Ingenioerer og arkitekter} (\texttt{2141}) og Fire forskllige typer af teknikerarbejde (\texttt{3112, 3115, 3118, 3181}), 3) \emph{Laege} (\texttt{2221}), 4) \emph{Tandlaege} (\texttt{2222}), 5) \emph{Jordemoder, overordnet sygepleje mv} (\texttt{2230}) og \emph{Sygeplejearbejde} (\texttt{3230}), 6) \emph{Kulturformidling og informationsarbejde, primaert bibliotekar} (\texttt{2430}), 7) \emph{Overordnet revisions og regnskabsarbejde, herunder registeret revisor og statsautoriseret revisor} (\texttt{2411}) og otte andre \texttt{DISCO}-kategorier med kontor-, administrations- og revisionsarbejde mv., 8) \emph{Religioest arbejde} (\texttt{2482}) og 9) \emph{Overordnet socialraadgivningsarbejde} (\texttt{2446}) og \emph{Administrativt arbejde vedr. offentlige ydelser og afgifter} (\emph{3440}).}

I perioden 1996 til 2009 har de lange videregående uddannelser haft den største  procentvise tilvækst, men de talte alligevel mindre en halvdelen af antallet med en mellemlang mellemlang videregående uddannelse. %%%% Den samlede arbejdsstyrke har været nogenlunde konstant. 
Nettoledigheden for de lange videregående uddannelser bevæger sig parallelt med de andre faggrupper i perioden 1996 til 2009 er nettoledighedsgennemsnittet 6000 personer. 
% 
\begin{table}[H] \centering
\caption{Arbejdsstyrken fordelt på uddannelsesgrupper. Kilde: DST}
\label{tab_uddannelse}
\begin{tabular}{lrrrrrrr} \toprule
	& \multicolumn{1}{c}{Grundskole} & \multicolumn{1}{c}{GYM} & \multicolumn{1}{c}{EUD} & \multicolumn{1}{c}{KVU}	& \multicolumn{1}{c}{MVU} & \multicolumn{1}{c}{LVU} & Alle	\\ \midrule
1996	&	1.572.425	&	310.367	&	1.294.603	&	118.561	&	419.763	&	162.520	&	\\
1997	&	1.556.567	&	317.815	&	1.314.644	&	123.732	&	434.072	&	170.017	&	\\
1998	&	1.536.264	&	323.421	&	1.335.701	&	130.000	&	449.752	&	177.948	&	\\
1999	&	1.524.624	&	325.966	&	1.348.020	&	135.721	&	465.883	&	185.261	&	\\
2000	&	1.510.944	&	325.431	&	1.364.746	&	140.052	&	482.040	&	192.667	&	\\
2001	&	1.499.835	&	324.567	&	1.379.370	&	145.075	&	498.671	&	201.119	&	\\
2002	&	1.488.688	&	322.904	&	1.391.768	&	150.452	&	515.013	&	210.416	&	\\
2003	&	1.480.263	&	320.959	&	1.399.158	&	156.772	&	530.979	&	220.133	&	\\
2004	&	1.473.175	&	321.088	&	1.406.980	&	159.774	&	545.612	&	230.323	&	\\
2005	&	1.466.360	&	321.403	&	1.411.090	&	162.922	&	559.638	&	239.798	&	\\
2006	&	1.455.210	&	322.732	&	1.414.756	&	165.711	&	572.289	&	249.919	&	\\
2007	&	1.439.702	&	324.260	&	1.419.023	&	168.987	&	584.270	&	261.475	&	\\
2008	&	1.517.184	&	329.613	&	1.428.161	&	173.227	&	598.317	&	273.095	&	\\
2009	&	1.460.590	&	329.557	&	1.439.554	&	177.572	&	613.044	&	285.460	&	\\  \bottomrule
\end{tabular} \end{table}
%





%%%%%%%%%%%%%%%%%%%%%%%%%%%%%%%%%%%%%%%%%%%%%%%%%%%%%%%%%%%%%%%%%%%%%%%%%%%%%%%%%%%%%%%%%%%%%%%%%%%

\subsubsection{Magisterclusteren \label{}}
% 
Magisterclusteren består primært af meget forskelligt arbejde. Clusteren indeholder
 \begin{enumerate} [topsep=6pt,itemsep=-1ex]
   \item \emph{Arbejde med emner inden for fysik, kemi, astronomi, meteorologi, geologi og geofysik} (\texttt{2110})
   \item \emph{Arbejde med emner inden for de biologiske grene af naturvidenskab} (\texttt{2210})
   \item \emph{Dyrlaege} (\texttt{2223})
   \item \emph{Farmaceut} (\texttt{2224})
   \item \emph{Arbejde med emner inden for medicin, odontologi, veterinaervidenskab og farmaci i oevrigt} (\texttt{2229})
   \item \emph{Undervisning paa universiteter og andre hoejere laereanstalter} (\texttt{2311})
   \item \emph{Undervisning paa gymnasier, erhvervsskoler mv} (\texttt{2321})
   \item \emph{Folkeskolelaerer} (\texttt{2331})
   \item \emph{Undervisning af handicappede mennesker} (\texttt{2341})
   \item \emph{Arbejde vedr. undervisning i oevrigt, primaert kursusvirksomhed} (\texttt{2350})
   \item \emph{Samfundsvidenskabeligt arbejde og historie} (\texttt{2442})
   \item \emph{Sprogvidenskabeligt arbejde} (\texttt{2444})
   \item \emph{Psykolog} (\texttt{2445})
   \item \emph{Arbejde med administration af lovgivningen inden for den offentlige sektor} (\texttt{2470})
   \item \emph{Blandet undervisning i folkeskoler, erhvervsskoler, gymnasier og hoejere laereanstalter samt forskningstilrettelaeggelse og kontrol af undervisningsarbejde} (\texttt{2930}) 
 \end{enumerate}
% 


%%%%%%%%%%%%%%%%%%%%%%%%%%%%%%%%%%%%%%%%%%%%%%%%%%%%%%%%%%%%%%%%%%%%%%%%%%%%%%%%%%%%%%%%%%%%%%%%%%%

% \subsubsection{Kreaclusteren \label{}}
% % 
% Kreaclusteren består af forskelligt kreativt arbejde. Clusteren indeholder:
%  \begin{enumerate} [topsep=6pt,itemsep=-1ex]
%    \item \emph{Ledelse af virksomhed faerre end 10 ansatte} (\texttt{1300})
%    \item \emph{Alment journalistisk arbejde og skribentarbejde} (\texttt{2451}) 
%    \item \emph{Illustrationsgrafisk arbejde vedr. formidling og kunstnerisk arbejde vedr. billedkunst og formgivning} (\texttt{2452}) 
%    \item \emph{Kunsterisk arbejde indenfor dans, musik, koreografi, skuespil eller film} (\texttt{2481}) 
%    \item \emph{Blandet journalist, kunst og skribentarbejde} (\texttt{2945}) 
%    \item \emph{Arbejde med lyd, lys og billeder ved film og teaterforestillinger mv samt betjening af medicinsk udstyr} (\texttt{3130}) 
%    \item \emph{Arbejde inden for kunst, underholdning og sport} (\texttt{3470})
%  \end{enumerate}
% % 


%%%%%%%%%%%%%%%%%%%%%%%%%%%%%%%%%%%%%%%%%%%%%%%%%%%%%%%%%%%%%%%%%%%%%%%%%%%%%%%%%%%%%%%%%%%%%%%%%%%

% \subsubsection{Djøfferclusteren \label{}}
% % 
% Djøfferclusteren består primært af forskelligt økonomi- og juridisk arbejde. Clusteren indeholder
%  \begin{enumerate} [topsep=6pt,itemsep=-1ex]
%    \item \emph{Lovgivningsarbejde samt ledelse i offentlig administration og interesseorganisationer} (\texttt{1100}) 
%    \item \emph{Arbejde med matematik, aktuariske og statistiske metoder} (\texttt{2120}) 
%    \item \emph{Udvikling og planlaegning af personalespoergsmaal} (\texttt{2412}) 
%    \item \emph{Ledelsesraadgivning og andre specialfunktioner indenfor organisation} (\texttt{2419}) 
%    \item \emph{Advokat, dommer og andet juridisk arbejde} (\texttt{2420}) 
%    \item \emph{Oekonomi} (\texttt{2441}) 
%  \end{enumerate}




%%%%%%%%%%%%%%%%%%%%%%%%%%%%%%%%%%%%%%%%%%%%%%%%%%%%%%%%%%%%%%%%%%%%%%%%%%%%%%%%%%%%%%%%%%%%%%%%%%%


% % 
% \begin{table}[H] \centering
% \caption{Arbejdsløshed fordelt på uddannelsesgrupper. Kilde: DST}
% \label{tab_uddannelse_arbejdsloeshed}
% \begin{tabular}{lrrrrrr} \toprule
% 	& \multicolumn{1}{c}{Grundskole} & \multicolumn{1}{c}{STX} & \multicolumn{1}{c}{EUD} & \multicolumn{1}{c}{KVU}	& \multicolumn{1}{c}{MVU} & \multicolumn{1}{c}{LVU} & Alle	\\ \midrule
% 1996	&	81.151	&	13.622	&	60.666	&	6.145	&	11.871	&	6.418	&	\\
% 1997	&	68.797	&	11.806	&	58.219	&	5.610	&	11.304	&	6.641	&	\\
% 1998	&	55.107	&	8.872	&	45.082	&	4.204	&	8.799	&	5.221	&	\\
% 1999	&	47.700	&	7.363	&	41.245	&	3.868	&	8.517	&	5.090	&	\\
% 2000	&	47.203	&	7.031	&	42.168	&	4.250	&	8.964	&	5.039	&	\\
% 2001	&	43.330	&	6.396	&	40.099	&	4.143	&	8.315	&	4.906	&	\\
% 2002	&	43.826	&	7.018	&	43.365	&	5.310	&	9.853	&	6.620	&	\\
% 2003	&	52.604	&	8.981	&	53.668	&	6.801	&	13.079	&	8.639	&	\\
% 2004	&	48.517	&	8.307	&	47.637	&	5.850	&	12.629	&	8.075	&	\\
% 2005	&	39.159	&	7.106	&	35.808	&	4.706	&	10.888	&	6.975	&	\\
% 2006	&	30.184	&	5.715	&	24.019	&	3.430	&	8.451	&	5.776	&	\\
% 2007	&	23.436	&	4.133	&	16.806	&	2.459	&	5.888	&	4.813	&	\\
% 2008	&	15.707	&	2.669	&	15.353	&	1.916	&	3.824	&	3.297	&	\\
% 2009	&	29.452	&	5.283	&	39.438	&	4.581	&	8.227	&	6.459	&	\\ \bottomrule
% \end{tabular} \end{table}
% %







%Local Variables: 
%mode: latex
%TeX-master: "report"
%End: