% -*- coding: utf-8 -*-
% !TeX encoding = UTF-8
% !TeX root = ../report.tex


%%%%%%%%%%%%%%%%%%%%%%%%%%%%%%%%%%%%%%%%%%%%%%%%%%%%%%%%%%%
\newpage \section{\textsc{Sociologisk teori} \label{}}
%%%%%%%%%%%%%%%%%%%%%%%%%%%%%%%%%%%%%%%%%%%%%%%%%%%%%%%%%%%
%%%%% Jens: I skal overveje om, I nogensinde kommer til at bruge de sociologiske teorier. Mange sociologiske teorier har et helt andet fokus end jer, fordi de fokusere på sociale og psykologiske forhold, men det er ikke det I laver. Måske det skulle kortes ned til noget à la: Der er de her perspektiver inden for sociologien. På det seneste er der dog kommet mere aktørbaserede modeller, som i højere grad er mere i dialog med de økonomiske teorier, hvilket I skal signalerer uden at redegørei så høj grad for alle tudesociologerne. Dermed bliver vi den type sociologer som gerne vil i dialog med økonomerne.

Dette teoretiske afsnit om sociologiske forståelser af arbejdsløshed indeholder først en strukturbaserede teorier med fokus på sociale og psykologiske konsekvenser af arbejdsløshed herunder, stadiemodellen, Jahodas funktionelle deprivationsteori og Warrs vitaminmodel. Herefter følger en række aktørbaserede teorier herunder rehabiliteringstilgangen, Fryers agency kritik og Halvorsens mestringsperpektiv. Afslutningsvis følger status passagemodellen og marginaliseringsperpektivet.


%%%%%%%%%%%%%%%%%%%%%%%%%%%%%%%%%%%%%%%%%%%%%%%%%%%%%%%%%%%
\subsection{Økonomikritik}
%%%%%%%%%%%%%%%%%%%%%%%%%%%%%%%%%%%%%%%%%%%%%%%%%%%%%%%%%%%

Sociologien står i skarp kontrast til økonomien. Granovetter som er en af de mest førende sociologiske kritikere af den økonomiske diciplin definerer den økonomiske aktør som undersocialiseret: “Actors do not behave or decide as atoms outside a social context, nor do they adhere slavishly to a script written for them by the particular intersection of social categories that they happen to occupy. Their attempts at purposive action are instead embedded in concrete, ongoing systems of social relations.” \parencite[487]{Granovetter1985}. %%%% Jens: Er Granovetter en af vores helte? Bourdieus anke mod Granovetter var, at han var netværksanalytiker.
Hvor økonomien mangler psykologisk og social realisme, beskæftiger sociologien sig med identitet, social integration, normer og menneskelig adfærd såvel som et statistisk baseret styringsvidenskab efter et mere økonomisk mønster \parencite[36]{Halvorsen1999}. Sociologisk arbejdsløshedsforskning er desuden domineret af empirisk variabelanalyse, årsags-virkningsforhold og mekanismer som kan fører til dårligt mentalt helbred \parencite[38]{Halvorsen1999}. Størstedelen af arbejdsløshedsforskerne benytter sig af middle-range teorier \parencite[9]{Hedstroem2005}, som beskriver en begrænset del af den sociale virkelighed i modsætning grand theories. %%%% Emil: "Sociologisk arbejdsløshedsforskning er desuden..." er en lidt uklar formulering

\parencite{Granovetter1995}


%%%%%%%%%%%%%%%%%%%%%%%%%%%%%%%%%%%%%%%%%%%%%%%%%%%%%%%%%%%
\subsection{Strukturbaserede teorier om sociale og psykologiske konsekvenser}
%%%%%%%%%%%%%%%%%%%%%%%%%%%%%%%%%%%%%%%%%%%%%%%%%%%%%%%%%%%

De negative konsekvenser af arbejdsløshed blev først beskrevet Marienthal-studiet af Jahoda, Lazarsfeld og Zeizel som foregår i den østrigske by Marienthal, hvor en voldsom arbejdsløshed havde ført til apati blandt de arbejdsløse \parencite[vii]{Lazarsfeld1971}. Sidenhen er der foretaget en række strukturbaserede teorier og modeller som er særlig relevante i forhold til sociale og psykologiske konsekvenser af arbejdsløshed herunder stadiemodellen, Jahodas funktionelle model, Warr's vitaminmodel og marginaliseringsperspektivet.

Lazarsfeld udvikler sammen med Eisbenberg en model som trækker erfaringerne fra Marienthal og en lang række andre arbejdsløshedsstudier. Jahoda, Lazarsfeld og Zeizel skelnede mellem det at være knækket, resigneret, fortvivlet og apatisk som reaktioner på arbejdsløshed i Marienthal \parencite[56]{Jahoda1971}. \textbf{Stadiemodellen} konkluderer, at arbejdsløse gennemgår tre forskellige stadier: Først oplever den arbejdsløse et chok opfulgt af en aktiv indsats for at søge job, hvor den arbejdsløse stadigvæk er optimistisk og ikke knækket. Dernæst, når al indsats er mislykket, bliver den arbejdsløse pessimistisk, fortvivlet og føler sig i nød. Til sidst bliver den arbejdsløse knækket \parencite[378]{Eisenberg1938}. Stadiemodellen er dog blevet kritiseret for at være metodisk problematisk, teorien modsætningsfuld og kategorierne hule. Ifølge Ezzy er modellen ikke så meget en teori som et deskriptivt framework, hvor den operationelle variabel er længden på arbejdsløshed \parencite[44]{Ezzy1993}.

Det centrale i \textbf{Jahodas funktionelle deprivationsteori} er, at arbejdsløshed medfører en række afsavn, som den arbejdsløse ville have fået ved at være i beskæftigelse \parencite[44]{Ezzy1993} %%%% Jens: Sætningen er sprogligt ukorrekt
. Udgangspunktet er Mertons opdeling af sociale funktioner som manifeste og latente. Arbejdets manifeste funktion består i at opfylde et økonomisk behov for indtægt, mens arbejdets latente funktion består i at opfylde et psykologisk behov: “First, employment imposes a time structure on the waking day; second, employment implies regularly shared experiences and contacts with people outside the nuclear family; third, employment links individuals to goals and purposes that transcend their own; fourth, employment defines aspects of personal status and identity; and finally, employment enforces activity.” \parencite[188]{Jahod1981}. Som en parallel til trade-offet mellem arbejde og fritid, kan “fritiden” hos den arbejdsløse ikke udfylde de samme funktioner som arbejdet har \parencite[189]{Jahod1981}. Ifølge Ezzy imødekommer lønarbejdet ikke automatisk individets grundlæggende psykologiske og sociale behov, Jahoda romantiserer derfor og ser bort fra, at lønarbejde for nogle kan være isolerende og ubehageligt \parencite[45]{Ezzy1993}.

I modsætning til Jahoda giver Warrs \textbf{vitaminmodel} mulighed for at skelne mellem at have et godt eller dårligt mentalt helbred både som beskæftiget og som arbejdsløs. Warr foreslår, at på samme måde som vitaminer har en effekt på det fysiske helbred, så har forskellige forhold i ens omgivelser en effekt på det mentale helbred. Disse forhold eller “vitaminer” består af muligheden for at have kontrol over ens tilværelse, muligheden for at anvende ens færdigheder, ydre mål, variation, klarhed i forhold til omgivelserne, penge, fysisk sikkerhed, social kontakt og værdsat social position \parencite[45]{Ezzy1993}. Det vil sige, at når vitaminniveauerne er lave i en utilfredsstillende jobtilværelse eller som arbejdsløs, går det ud over det mentale helbred. Samtidig er der også mulighed for at skelne mellem vitaminniveauet hos eksempelvis arbejdsløse middelalderende mænd med større problemer i forhold til penge, sikkerhed, personlig kontakt og en værdsat social position end hos arbejdsløse teenagere, som er mindre afhængige af arbejde for at have tilstrækkelige niveauer inden for de nævnte vitamintyper \parencite[46]{Ezzy1993}. %%%% Søren: suppler konklusionen fra Warr bogen

Efter at have beskrevet en række strukturbaserede teorier med fokus på sociale og psykologiske konsekvenser af arbejdsløshed, vil vi gå over til at beskrive en række aktørbaserede teorier.


%%%%%%%%%%%%%%%%%%%%%%%%%%%%%%%%%%%%%%%%%%%%%%%%%%%%%%%%%%%
\subsection{Aktørbaserede teorier}
%%%%%%%%%%%%%%%%%%%%%%%%%%%%%%%%%%%%%%%%%%%%%%%%%%%%%%%%%%%

De aktørbaserede teorier beskæftiger sig mere eller mindre med de negative sociale og psykologiske konsekvenser for den arbejdsløse og mere med, hvad agenten gør, ikke gør eller kan gøre for at komme ud af arbejdsløshedssituationen herunder rehabiliteringstilgangen, Fryers agency kritik, status passagemodellen og Halvorsens mestringsperpektiv. %%%%% MARGINALISERING / STATUS PASSAGE

\textbf{Fryers agency kritik} er ikke så meget en teori, men mere en direkte kritik af at beskæftige sig med passive aktører i arbejdsløshedsteorier såsom Jahoda og Warrs. Fryer foreslår, at det individet bringer til situationen er lige så vigtigt som konsekvenserne af arbejdsløshed. De proaktive arbejdsløse i Fryers studie oplever ikke sociale og psykologiske afsavn på trods af, at de lider af de materielle afsavn, som kommer af at miste indtægterne fra deres arbejde. Ifølge Fryer er disse proaktive arbejdsløse eksempler på aktive sociale agenter, som forsøger at få mening ud af deres situation and agerer ud fra målsætninger \parencite[47]{Ezzy1993}. Omvendt er Fryer blevet kritiseret for at lægge for meget vægt på kognitive processer og ignorerer de institutionelle begrænsninger som arbejdsløsheden oftest medfører \parencite[47]{Ezzy1993}.

Tiffany, Cowan og Tiffanys udvikler i 1970 \textbf{rehabiliteringstilgangen} på baggrund af et studie, som viser, at størstedelen af de arbejdsløse er arbejdsløse på grund af psykologiske årsager: “they show avoidance behavior pattern or what has been referred to as “work inhibition” which implies that they are physically capable of work but are prevented from working because of psychological disabilities” \parencite[43]{Ezzy1993}. Som konsekvens bør man ifølge Tiffany, Cowan og Tiffany rehabiliterer de arbejdsløse gennem terapi eller træning, så de kan vende tilbage på arbejdsmarkedet. Ifølge Ezzy ligner denne tilgang den historiske skelnen mellem dem, som var fysisk ude af stand til at arbejde og fortjente støtte, og dem, som var i fysisk stand til at arbejde, men som ikke arbejdede, fordi de var dovne, og derfor ikke fortjente støtte. %%% Minder om skelnen mellem “de værdigt trængende” og “de ikke-trængende” \parencite[27]{Bauman2002}
Ezzy påpeger, at årsagen til, at tilgangen fokuser på individuelle forklaringer frem for strukturelle forklaringer er, at tilgangen har været mest toneangivende i perioder med relativ lav arbejdsløshed, mens de fleste andre sociologiske og psykologiske arbejdsløshedsstudier er foretaget i perioder med høj arbejdsløshed \parencite[43]{Ezzy1993}.
 
Knut Halvorsen anskuer med \textbf{mestring}-perspektivet ligeledes de arbejdsløse som aktive aktører. De arbejdsløse kan dermed påvirke deres situation og være med til at forandre den i stedet for at være ofre for omstændighederne. For at kapere en negativ hændelse - som det at miste sit job er - indgår den arbejdsløse i forskellige psykiske, fysiske og sociale aktiviteter. Den problemorienterede mestring udgør konkrete strategier med formål at fjerne belastningen af den marginaliserede position for eksempel jobsøgning. Den emotionsorienterede mestring handler om, hvordan man ser verdenen for eksempel kan det at redefinere sig som hjemmegående ved at måde at søge ny basis for mening for at opretholde selvrespekten \parencite[47]{Halvorsen1999}. Ifølge Nørup ligger fokus i perspektivet, hvordan individet håndterer det, at stå midlertidigt uden for arbejdsmarkedet. På den måde forholder Halvorsen sig ikke til, hvad der sker med, når arbejdsløsheden bliver permanent eller langsigtet \parencite[30]{Noerup2014}. Nørup kritiserer endvidere Halvorsen for i sin ontologiske individualisme at fokuserer på, hvordan individet mestrer bestemte livssituationer under bestemte rammer frem for på hvordan de samfundsmæssige strukturer og sociale relationer påvirker eksklusionen \parencite[37]{Noerup2014}.

Efter at have beskrevet en række aktørbaserede teorier, vil vi afslutningsvis beskrive status passagemodellen og marginaliseringsperpektivet.


%%%%%%%%%%%%%%%%%%%%%%%%%%%%%%%%%%%%%%%%%%%%%%%%%%%%%%%%%%%
\subsection{Perspektiver på arbejdsløshed som overgange og processer} 
%%%%%%%%%%%%%%%%%%%%%%%%%%%%%%%%%%%%%%%%%%%%%%%%%%%%%%%%%%%
%%%% Jens: Gør mere ud af status passage og marginalisering. Måske mere Van Gennep - Bourdieu har skrevet en lille tekst som hedder Rites Of Institutions, som er et modspil til Van Gennep i forhold til hvem som overhovedet har mulighed for at lave disse overgange - eliten er bedre til at være studerende, og man er bedre til kvinde i forhold til....

Status passagemodellen og marginaliseringsperpektivet er to perspektiver som er mere generelle end de struktur- og aktørbaserede teorier, fordi de i højere grad beskæftiger sig med perspektiver som kan overføres på arbejdsmarkedet såvel som på andre området. Vi mener, at perspektiver er særlig anvendelige til at bløde kategorier som arbejdsløshed, beskæftigelse og uden for arbejdsstyrken op, så de mere bliver overgange eller processer frem for bestemte faser. %%%%% Jens: Endelig nærmer vi os noget vi skal bruge til noget...

Ezzy anvender begrebet jobtab i stedet for arbejdsløshed\footnote{Denne problematik adskiller sig på den ene side fra andre veje til arbejdsløshed såsom dimittendledighed eller et vende tilbage til arbejdsstyrken igen og veje ud af beskæftigelse såsom pensionering, orlov eller at tage en uddannelse \parencite[48]{Ezzy1993}.} og sammenligner det at miste sit job med at gå i gennem et skilsmisseforløb, et sygdomsforløb eller opleve et dødsfald i familien. Disse overgange skal forstås i forlængelse af Glaser og Strauss' definition af en \textbf{status passage} som et individs ”movement into a different part of a social structure, or loss or gain of privilige, influence, or power, and changed behaviour” \parencite[48]{Ezzy1993}. En status passage er en del af individets biografi, hvilket involverer samtidige og tidligere erfaringer, som påvirker meningen der tillægges arbejdsløsheden. Ezzy skelner mellem integrative passager og afhændelsespassager\footnote{Denne skelnen er foretaget med udgangspunkt i Van Gennep skelnen mellem separation (for eksempel begravelse), transition (for eksempel jobskifte) og integration (for eksempel ægteskab) som kategorier for sociale passager\parencite[48]{Ezzy1993}.}. Integrative passager er oftest en overgangsperiode efterfulgt af integration i en ny status gennem en ceremonial proces som for eksempel bryllupper. Afhændelsespssager er en separation fra en status som oftest er en længerevarende overgangsfase med en usikker varighed for eksempel skilsmisser, sygdomsforløb, arbejdsløshed eller dødsfald \parencite[49]{Ezzy1993}. En afhændelsespassager fører ikke nødvendigvis i sig selv til mentale problemer, mistrivsel eller eksklusion, da det afhænger af den enkeltes identitet og selvopfattelse i relation til andre og samfundet og en lang række andre faktorer samt om afhændelsespassagen efterfølges af en reintegrativ passage, hvor den enkelte får en ny status eksempelvis et nyt job \parencite[32]{Noerup2014}. %%%%% Jens: Interessant. Der burde stå noget à la, at vi mener overgange og faser er særlig relevante. Sætte det i spil over for fx den økonomiske teori 

\textbf{Marginaliseringsperspektivet} anvendes både på arbejdsmarkedet, men også en række andre områder. Ifølge Elm larsen skal marginaliseringsperspektivet ses som en midterkategori mellem inklusion og eksklusion. Larsen definerer eksklusion som en ufrivillig ikke-deltagelse gennem forskellige typer af udelukkelsesmekanismer og -processer, som det ligger uden for indvidets og gruppens muligheder at få kontrol over \parencite[137]{Larsen2009}. Hermed er Larsen kritisk over for Luhmanns dikotomi inklusion/eksklusion, som Larsen i dens binære form ikke mener er særlig hensigtsmæssig i forhold til at beskrive virkeligheden \parencite[130]{Larsen2009}. Derfor argumenter han for, at marginalisering kan anvendes som en midtergruppe mellem de to, hvor individet bevæger sig i en proces mod inklusion eller eksklusion. På baggrund af dette har vi tegnet følgende model:\footnote{Modellen er også inspireret af lignende modeller benyttet af Lars Svedberg \parencite[44]{Svedberg1995} og Catharina Juul Kristensen \parencite[18]{Kristensen1999}.} som fremgår af tabel \ref{tab_marginaliseringsmodel_1}. 
% 
\begin{table}[H] \centering
\caption{Model over marginalisering}
\label{tab_marginaliseringsmodel_3}
\begin{tabular}{@{} m{3,4cm} c m{3,6cm} c m{3,6cm} @{}} \toprule
\textbf{Inkluderet} & & \textbf{Marginaliseret} & & \textbf{Ekskluderet} \\ \midrule
\end{tabular} \end{table} %%%%%
\begin{table}[H] \centering
\label{tab_marginaliseringsmodel}
\begin{tabular}{@{} m{5,9cm} m{5,9cm} @{}} 
  \textbf{Marginaliseringsproces} & \textbf{Eksklusionsproces} \\  
  --------------------------------------------> & --------------------------------------------> \\ 
\end{tabular} \end{table} %%%%%
\begin{table}[H] \centering
\label{tab_marginaliseringsmodel}
\begin{tabular}{@{} m{12,3cm} @{}} 
  \textbf{Inklusionsproces} \\  
  <--------------------------------------------------------------------------------------------- \\ \bottomrule
\end{tabular} \end{table}
%
Den første proces består af individer som går fra at være inkluderet til at indgå i en proces i retning mod marginalisering. Den anden proces består af individer som går fra at være marginaliseret til at indgå i en proces i retning mod inklusion. Marginaliseringsperspektivet giver hermed mulighed for at se de  bevægelser på arbejdsmarkedet inden for kategorier som ikke rigide.

%%% Søren: Bauman. Bauman skriver også om eksklusion - eksklusion som kan kædes sammen med diskuserne \parencite[128-129]{Bauman2002}

%%%%%%%%%%%%%%%%%%%%%%%%%%%%%%%%%%%%%%%%%%%%%%%%%%%%%%%%%%%
\subsubsection{Opsummering}
%%%%%%%%%%%%%%%%%%%%%%%%%%%%%%%%%%%%%%%%%%%%%%%%%%%%%%%%%%%

Vi har valgt at dele de sociologiske forståelser af arbejdsløshedsproblemet op i strukturbaserede teorier og aktørbaserede teorier samt beskrive status passagemodellen og marginaliseringsperspektivet sammen. Fælles for de strukturbaserede teorier såsom stadiemodellen, Jahodas funktionelle deprivationsteori og Warrs vitaminmodel som i høj grad fokuserer på de sociale og psykologiske konsekvenser af arbejdsløshed. De aktørbaserede teorier såsom Fryers agency kritik, rehabiliteringstilgangen og Halvorsens mestringsperpektiv beskæftiger sig mere med, hvad agenten gør, ikke gør eller kan gøre for at håndtere arbejdsløsheden eller komme tilbage i beskæftigelse. Jahoda, Warr og andre strukturbaserede teoretikere er blevet kritiseret for at gøre de arbejdsløse til passive individer, mens Fryer, Halvorsen og andre agentbaserede teoretikere omvendt er blevet kritiseret for at se bort fra de strukturelle og institutionelle begrænsninger i arbejdsløshedsperspektivet. Jahoda er i særdeleshed blevet kritiseret for at fremhæve lønarbejdet som et ubetinget gode, hvilket hænger sammen med det førnævnte elendighedsdiskursen, hvor arbejdsløshed er lig med social død. Status massagemodellen og marginaliseringsperspektivet giver mulighed for at anskue arbejdsløshed som overgange og processer i bevægelse og dermed også mere dynamisk end flere af de andre sociologiske.

%%%%%%%%%%%%%%%%%%%%%%%%%%%%%%%%%%%%%%%%%%%%%%%%%%%%%%%%%%%
I dette teoretiske afsnit om arbejdsløshed har vi gennemgået arbejdsløshed som problemstilling samt beskrevet økonomiske og sociologiske forståelser af arbejdsløshed.

Overordnet kan man sige, at når økonomer beskæftiger sig med arbejdsløshed, benytter de sig typisk af register- og surveybaserede arbejdsløshedsopgørelser. Sociologer er mindre glade for definitioner, hvilket fremgår af Kelvin og Jarrett: “The unemployed are not a group of people, but an economic and adminsitrative category” \parencite{Kelvin1985}. Den “sociologiske” pointe er netop, at arbejdsløshed ikke er en ensartet gruppe, men en kategori sammensat af forskellige mennesker med forskellige udfordringer. Dette fremgår tydeligt, når arbejdsløse, ledige, dagpengemodtagere og kontanthjælpsmodtagere i Danmark ofte beskrives i kombination med tillægsord såsom udsatte, (ikke)-arbejdsmarkedsparate, langtids-, (ikke-)forsikrede, sæson, friktions-, førstegangs-, syge, aktiverede, svage og arbejdsskadede samt i undergrupper såsom handikappede, indvandrere/efterkommere, mænd/kvinder, ung/ældre og lavt/højtuddannede\footnote{Denne lange beskrivelse af forskellige typer arbejdsløse kommer fra en gennemgang af Det Nationale Forskningscenter for Velfærd (SFI) rapporter fra de sidste 20 år.}. Derfor bør man passe på med at beskrive arbejdsløse som en homogen gruppe, men når man laver kvantitative arbejdsløshedsstudier, kan man dog ikke komme uden om de register- og surveybaserede arbejdsløshedsopgørelser. Derfor mener vi, at status passagemodellen, marginaliseringsperspektivet og Atkinsons økonomiske kritik er anvendelige. I det fjerde teoretiske hovedafsnit, vil vi med vores teoretiske operationalisering anvende disse perspektiver med henblik på at nedbryde de økonomiske kategorier \textit{arbejdsløshed}, \textit{beskæftigelse} og \textit{uden for arbejdsstyrken} ved at se bevægelser ind og ud af beskæftigelse som marginaliserings-, inklusions- og eksklusionsprocesser.

I afsnittet arbejdsløshed som problemstilling skelnede vi mellem tre dominerende arbejdsløshedsdiskurser: elendighedsdiskursen, beskæftigelsesdiskursen og moraldiskursen. Vi vurderer, at de strukturbaserede arbejdsløshedsteorier inden for sociologien ligger i forlængelse af elendighedsdiskursen ved at fokusere på de psykologiske og sociale konsekvenser af arbejdsløshed. Hvor arbejdsløshed anses for en social og psykologisk elendighed, giver lønarbejdet social anerkendelse og fylder de sociale og psykologiske behov, som arbejdsløsheden giver afsavn for\footnote{Dette perspektiv har i midlertidigt være udfordret af den marxistiske tradition inden for sociologien, hvor lønarbejdet anset for at være fremmedgørende \textbf{\parencite[48]{Halvorsen1999}.}}. Vi vurderer, at de økonomiske teorier og modeller, som vi har beskrevet, ligger i forlængelse af beskæftigelsesdiskursen og moraldiskursen. Arbejdsløshed eksisterer, fordi mennesket er dovent af natur og der skal være en økonomisk gevinst i sigte for at arbejde. Arbejdet er dog samtidig et nødvendigt onde. På den ene side agerer mennesket for at maksimere sin egen nytte økonomisk, og på den anden side er lønarbejdet en nødvendighed for at stimulere samfundet, staten og det private erhvervsliv.

På den måde bliver den sociologiske og økonomiske måde at beskæftige sig med arbejdsløshed på også på mange måder en skelnen mellem økonomiske incitamenter \parencite{Mankiw2011, Cahuc2004, Mankiw2007, McCall1970, Mortensen1970, Baily1978} og de sociale og psykologiske konsekvenser af arbejdsløshed \parencite{Jahoda1971, Eisenberg1938}. Men der er også økonomer, som stiller sig kritisk over for at reducere arbejdsløshed til incitamenter \parencite{Atkinson1991}, og sociologer som enten stiller sig kritisk over for at reducere arbejdsløse til passive ofre \parencite{Ezzy1993, Halvorsen1999} eller beskæftiger sig med arbejdsløs på et andet plan uden at være decideret kritiske over for tankegangen \parencite{Larsen2009}.

Efter dette første teoretiske hovedafsnit, hvor vi har beskrevet arbejdsløshedsområdet i forhold til at kontekstualisere arbejdsløshed som problemstilling og som forskningsfelt, vil vi gå videre til kapitlets andet hovedafsnit, som beskriver mobilitet som et forskningsfelt i sig selv og i kombination med den viden vi har om arbejdsløshedsområdet.



%Local Variables: 
%mode: latex
%TeX-master: "report"
%End: