% -*- coding: utf-8 -*-
% !TeX encoding = UTF-8
% !TeX root = ../report.tex

%%%%%%%%%%%%%%%%%%%%%%%%%%%%%%%%%%%%%%%%%%%%%%%%%%%%%%%%%%%
\newpage \section{\textsc{Arbejdsløshed \label{}}}
%%%%%%%%%%%%%%%%%%%%%%%%%%%%%%%%%%%%%%%%%%%%%%%%%%%%%%%%%%%

Dette teoretiske afsnit om arbejdsløshed indleder med at beskrive arbejdsløshed som problemstilling. Efterfølgende beskrives økonomiske forståelser af arbejdsløshedsproblemet. Bagefter følger en beskrivelse af sociologiske forståelser af arbejdsløshedsproblemet. Til sidst foreligger en opsummering. 


%%%%%%%%%%%%%%%%%%%%%%%%%%%%%%%%%%%%%%%%%%%%%%%%%%%%%%%%%%%
\subsection{\textsc{Arbejdsløshed som problemstilling}}
%%%%%%%%%%%%%%%%%%%%%%%%%%%%%%%%%%%%%%%%%%%%%%%%%%%%%%%%%%%

C. Wright Mills konstaterer: “No problem can be adequately formulated unless the values involved and the apparent threat to them are stated.” \parencite[129]{Mills1959}. Derfor vil vil indlede dette teoretiske afsnit om arbejdsløshed med kort at kontekstualisere arbejdsløshed som et socialt problem med fokus på hvilke værdier, som er involverede og truslerne herimod. (2.2.1.1) Først vil vi beskrive en kort historik af arbejdsløshed som et socialt problem. (2.2.1.2) Derefter vil vi beskrive dagens syn på arbejdsløshed herunder beskrive de dominerende diskurser forbundet med arbejde og arbejdsløshed. %%%% Det kunne være bedre med et Bourdieu-citat


%%%%%%%%%%%%%%%%%%%%%%%%%%%%%%%%%%%%%%%%%%%%%%%%%%%%%%%%%%%
\subsubsection{Kort historik over arbejdsløshed i et dansk perspektiv}

Arbejdsløshed anskues i dag som et socialt problem. Som begreb kom arbejdsløshed dog først til verden i løbet af det 19. århundrede. I dette århundrede blev arbejdsløshed ligeledes også et adskilt fænomen fra fattigdom \textbf{\parencite[3]{Halvorsen1999}}. I Danmark blev det op igennem 1800-tallet og omkring århundredeskiftet en dominerende tanke at anskue arbejdsløshed som et socialt anliggende, som delvist var forårsaget af forhold arbejderne ikke havde kontrol over. Dette bliver slået fast i 1907 med den første danske arbejdsløshedsforsikringslov\footnote{Hermed blev arbejdsløshedskasserne, staten og kommunerne de centrale aktører i  danske arbejdsløshedsforsikringssystem. Dette kendetegner den såkaldte Gent-model, hvor staten anerkender og yder tilskud til arbejdsløshedskasser organiseret af forsikringstagere (i praksis fagbevægelsen), og at det for det enkelte individ er frivilligt om denne vil forsikre sig mod arbejdsløshed \parencite{Jensen2007a}.}, som blev vedtaget med et bredt flertal i Landstinget og Folketinget. Baggrunden herfor var en kommission, som anbefalede, at samfundet måtte træde ind over for arbejdsløshed, fordi kommissionen kunne konstaterer, at arbejdsløshed var et socialt onde, som ramte arbejderne uden, at de var skyld heri \parencite[69]{Pedersen2007}. Med indførelsen af arbejdsløshedsforsikringen får den danske velfærdsstat som rolle at administrere arbejdsløshed som et socialt problem. Det var dog først i 1970, at staten overtog den marginale risiko og arbejdsløshedsforsikringen gik fra at være en privat forsikringsordning til at være en statsfinansieret velfærdsordning \parencite[83]{Pedersen2007}. Fra 1907 til 1970 blev de arbejdsløses vilkår løbende styrket, men beskæftigelseskrisen fra 1970'erne og frem til midten af 1990'erne medfører en mere aktiv arbejdsmarkedspolitik end tidligere, der blandt andet indebar løbende besparelser. Eksempelvis indføres efterlønsordningen i 1977 under Anker Jørgensens socialdemokratiske regering \parencite[86]{Pedersen2007}, dagpengesatsen fastfryses fra 1982 til 1986 under Poul Schlüters firkløverregering \parencite[88]{Pedersen2007}, vægten på ret, pligt og individuel behovsorientering øges i 1993 under Poul Nyrup Rasmussens socialdemokratisk ledede regering \parencite[92]{Pedersen2007}, muligheden for tværfaglige a-kasser oprettes for at øge konkurrencen under Anders Fogh Rasmussens VKO-regering \parencite[97]{Pedersen2007} og senest dagpengereformen fra 2010, hvor dagpengeperioden blev halveret fra 4 til 2 år og genoptjeningspligten fordobles fra 26 til 52 uger \parencite{lov_dagpenge}. %%% Henvisningen til Halvorsen er god nok, men man kunne godt overveje at anvende en anden eller flere andre henvisninger fx. Bauman

Ifølge Keane og Owens er udviklingen af velfærdsstaten i Danmark og andre lande bygget på et normativt grundsyn om at alle som udgangspunkt skal forsørge sig selv gennem et arbejde \textbf{\parencite[18]{Keane1986}}. Lønarbejdet bidrager til at sikre social integration blandt velfærdsstatens medlemmer. Og velfærdsstatens aktive arbejdsmarkedspolitik er med til at opretholde en vis levestandard for dem, som ikke har et arbejde samtidig med at benytte en “gulerod” til at få folk i arbejde gennem økonomiske incitamenter \textbf{\parencite[7]{Halvorsen1999}}. %%% Henvisningerne til Keane og Halvorsen er gode nok, men man kunne godt overveje at anvende en andre henvisninger fx. nogle fra FAOS


%%%%%%%%%%%%%%%%%%%%%%%%%%%%%%%%%%%%%%%%%%%%%%%%%%%%%%%%%%%
\subsubsection{Dagens syn på arbejdsløshed i et dansk perspektiv} 

Arbejdsløshed regnes ifølge Halvorsen for en af nutidens største udfordringer for velfærdsstaten både nationalt og internationalt \textbf{\parencite[8]{Halvorsen1999}}. Som socialt problem indgår arbejdsløshed i diskurser i forbindelse med arbejdets betydning og i den forbindelse også betydning af \textit{fravær} af arbejde. De diskurser, som er knyttet til arbejdsløshedsfænomenet er både med til at påvirke, hvordan arbejdsløse klassificeres, og hvordan arbejdsløse forstår dem selv og deres situation \textbf{\parencite[12]{Halvorsen1999}}. %%% Henvisningen til Halvorsen er god nok, men man kunne godt overveje at anvende en anden eller flere andre henvisninger fx. Bauman

Halvorsen skelner mellem tre forskellige diskurser om lønarbejde \parencite[13]{Halvorsen1999}. Den første diskurs knytter sig til retten til arbejde, hvor lønarbejdet både er lig med selvrealisering og er en forudsætning for, at man kan fungere som en god samfundsborger\footnote{Lars Svendsen skelner inden for den europæiske idéhistorie mellem to grundlæggende forskellige arbejdsopfattelser. Indtil reformationen blev arbejdet anset som en \textit{meningsløs forbandelse}, og efter reformationen blev arbejdet anset som et \textit{meningsfyldt kald} \parencite[13]{Svendsen2008}. I det moderne samfund beskriver Bauman, at arbejdet bliver æstetisk, fordi den enkelte eksempelvis skal kunne identificere sig med sit arbejde eller at ens arbejde skal være autentisk \textbf{\parencite[?]{Baum2006}}.}. Den anden diskurs knytter sig til arbejdspligt, hvor lønarbejde er lig med den grundlæggende værdiskabende aktivitet i samfundet\footnote{Efter anden verdenskrig gik velfærdsstaterne ind i en ny historisk fase, hvor regeringerne forsøgte at skaffe fuldtidsjobs til alle voksne igennem en politik som havde til formål for det første at stimulere privat og offentlig vækst \parencite[17]{Keane1986}.}. Den tredje diskurs knytter sig ligeledes til arbejdspligt, hvor lønarbejdet er lig med et nødvendigt onde, som er nødvendigt for at få samfundet til at fungere og et onde, fordi det enkelte individ er tvunget til at arbejde\footnote{Den dominerende opfattelse af arbejdet som et \textit{meningsløs forbandelse} \parencite[13]{Svendsen2008} kan stadigvæk siges at være gældende i dag (\textbf{henvisning}).}. Halvorsen skelner mellem tre ækvivalente arbejdsløshedsdiskurser \parencite[13]{Halvorsen1999}. \textit{Elendighedsdiskursen} handler om, at arbejdsløshed er lig med social død. Utallige historier i medierne knytter sig til denne diskurs med overskrifter som for eksempel: “Knæk. Arbejdsløshed rammer hele familien” (Politiken, 19.04.2013), “Arbejdsløse rammes af stress” (Politiken, 18.07.2010), “Ekspert: Unge arbejdsløse risikerer ar mange år frem” (Berlingske Tidende, 26.11.2010) og “Arbejdsløse frygter aldrig at finde job igen” (Politiken, 01.01.2011). \textit{Beskæftigelsesdiskursen} handler om, at arbejdsløshed er lige med sløseri med ressourcer. Dette fylder også en del i finansnyhederne, som eksempelvis “Arbejdsløse koster kassen” (Ekstra Bladet, 14.11.2008), “Ledighed sender folk i sygesengen” (Jyllands-Posten, 18.03.2013) og “Høj ledighed truer EUs økonomi” (Berlingske, 03.07.2014). \textit{Moraldiskursen} handler om, at arbejdsløshed skyldes dovenskab og manglende arbejdsmotivation. Denne diskurs har fyldt meget i mediedebatten med overskrifter som “Joachim B. til arbejdsløs: Du er for slap” (Politiken.dk, 19.04.2012), “Dovne Robert på kontanthjælp i 11 år: Hellere kontanthjælp end et lortejob” (Ekstra Bladet, 11.09.2012) og “Vi arbejdsløse bliver opfattet som dumme, dovne og dårlige mødre” (Politiken, 06.11.2014). Alle tre diskurser er fælles om, at arbejdsløshed anskues som et onde. Den danske realpolitik er domineret af de to sidstnævnte diskurser ved, at arbejdsløse mødes med de “økonomiske realiteter” eller “nødvendighedens politik” (\textbf{henvisning}) med den føromtalte dagpengereform fra 2010 samtidig med, at denne reform og kommende reformer bakkes op af udsagn som “Det skal kunne betale sig at arbejde” (Venstre, 2015).

De forskellige videnskabsdiscipliner knytter sig ligeledes til disse diskurser, hvilket vi vil komme ind på i afsnittene om økonomiske og sociologiske forståelser af arbejdsløshed. Overordnet kan man dog sige, at de økonomiske og sociologiske discipliner beskæftiger sig med forskellige forskningsområder \textbf{\parencite[10-11]{Halvorsen1999}}. Inden for den økonomiske disciplin knytter arbejdsløshedsproblemet sig på makroplan på de gevinsterne ved fuld beskæftigelse kontra spild af ressourcer blandt i form af velfærdsydelser (\textbf{henvisning}), og på et mikroplan på mangel på økonomisk selvstændighed (\textbf{henvisning}). På begge planer beskæftiger økonomer sig med, hvordan arbejdsløse kommer i beskæftigelse (\textbf{henvisning}). Inden for den sociologiske disciplin knytter arbejdsløshedsproblemet sig på makroplan til det økonomiske og idémæssige fundament for velfærdsstatens ideologi (\textbf{henvisning}), og på mikroplan til mangel på selvrespekt (\textbf{henvisning}), anerkendelse (\textbf{henvisning}) og socialt kontakt (\textbf{henvisning}) samt en trussel mod individets individets socialisering (\textbf{henvisning}) \textbf{\parencite[10-11]{Halvorsen1999}.}  %%% Henvisningerne til Halvorsen er gode nok, men man kunne godt overveje at anvende andre henvisninger med referencer fra de to afsnit om økonomiske og sociologiske forståelser af arbejdsløshed


%%%%%%%%%%%%%%%%%%%%%%%%%%%%%%%%%%%%%%%%%%%%%%%%%%%%%%%%%%%
\subsubsection{Opsummering}

I dette afsnit har vi kontekstualiseret arbejdsløshed som problemstilling ved kort at skitsere arbejdsløshed i en historisk og en nutidig kontekst i et dansk perspektiv. Historisk er danske arbejdsløses forhold overordnet løbende blevet styrket i perioden fra den første lov om arbejdsløse i 1907 til 1970'erne. Siden har arbejdsløses forhold i højere grad været til debat og under under pres. Ifølge Halvorsen dominerer diskurserne “arbejdsløshed som social død”, “arbejdsløshed som sløseri med ressourcer” og “arbejdsløshed som dovenskab” i større eller mindre grad synet på arbejdsløshed i hverdagen og den offentlige debat samt inden for politik og videnskabelig forskning. Efter at have kontekstualiseret arbejdsløshed som problemstilling, vil vi dykke ned i den videnskabelige forskning på arbejdsløshedsområdet. Inden vi går til den sociologiske disciplin, vil vi begynde med den økonomiske disciplin.






%%%%%%%%%%%%%%%%%%%%%%%%%%%%%%%%%%%%%%%%%%%%%%%%%%%%%%%%%%%
\newpage \subsection{\textsc{Økonomiske forståelser af arbejdsløshedsproblemet}}
%%%%%%%%%%%%%%%%%%%%%%%%%%%%%%%%%%%%%%%%%%%%%%%%%%%%%%%%%%%

%%% Læseplan
	% Makiw og Taylor: Economics
	% Makiw: Macroeconomics
	% Birch Sørensen: Introducing Advanced Macroeconomics
	% Pierre Cahuc og André Zylbergberg: Labor economics
	% Specialer: Anne Nylev, Mille Bjørk, Morten Stig Hansen
	% Halvorsen

Den arbejdsmarkedsøkonomiske disciplin er orienteret mod at sikre en effektiv økonomi og et effektivt fungerende arbejdsmarked \parencite[26]{Halvorsen1999}. Fra 1970'erne og fremefter har økonomisk teori haft en betydelig gennemslagskraft i arbejdsløshedsforskning samtidig med at være det ideologiske grundlag for lovændringer på arbejdsløshedsområdet i Danmark \parencite[19]{Andersen2003}. Den danske velfærdsstat påvirker den økonomisk adfærd, hvilket også er tilfældet for arbejdsløses handlinger i forhold til at tage eller varetage et arbejde \parencite[26]{Halvorsen1999}. Dette teoretiske afsnit om økonomiske forståelser af arbejdsløshed indeholder (2.2.2.1) en kort introduktion til relevante økonomiske begreber i forhold til arbejdsmarkedet og arbejdsløshed. Efterfølgende beskrives centrale arbejdsløshedsteorier, hvilket er (2.2.2.2) arbejde-fritidsmodellen, (2.2.2.3) søgeteori, (2.2.2.4) principal-agent teori (moral hazard) og (2.2.2.5) hysterese. Til sidst følger en kritik af de økonomiske forståelser af arbejdsløshedsproblemet.


%%%%%%%%%%%%%%%%%%%%%%%%%%%%%%%%%%%%%%%%%%%%%%%%%%%%%%%%%%%
\subsubsection{Arbejdsmarkedsøkonomi 101 \parencite[24-26]{Halvorsen1999}}

%%%%%%%%%%%%%%%%%%%%%%%%%%%%%%%%%%%%%%%%%%%%%%%%%%%%%%%%%%%
\noindent \textbf{(1) Incitamenter:} \parencite[1-20]{Mankiw2011}
% (1.1) Ten Principles of Economics. How people make decisions: (1) trade-offs: efficiency/equity, (2) costs, (3) marginal changes: marginal benefits/marginal costs, (4) incentives ; How people interact: (5) trade, (6) market economy, (7) governments: promoting efficiency/equity, market failure, externality, market power ; How the economy as a whole works: (8) economic growth, standard of living/productivity, (9), inflation, (10), Phillips curve: inflation/unemployment, business cycle


%%%%%%%%%%%%%%%%%%%%%%%%%%%%%%%%%%%%%%%%%%%%%%%%%%%%%%%%%%%
\noindent \textbf{(2) Udbud, efterspørgsel og equilibrium:} 
På arbejdsmarkedet er arbejdskraft, jord og kapital de centrale produktionsfaktorer. I dette henseende er udbud og efterspørgsel af arbejdskraft helt centralt. Efterspørgslen på arbejdsmarkraft bygger på antagelserne om, at virksomheder er konkurerende, og virksomheder er profit-maksimerende \parencite[383]{Mankiw2011}. I den henseende er forholdet det input som anvendes til at lave et produkt og outputtet af det produkt... \parencite[383-388]{Mankiw2011}
% 	(A) The demand for labour: 
% 		- The production function and the marginal product of labour: the production function = the relationship between the quantity of inputs used to make a good and the quantity of output of that good, the marginal product of labour = the increase in the amount of output from an additional unit of labour ; 
% 		- The value of the marginal product and the demand for labour: the value of the marginal product = the marginal product of an input x the price of the output
% 		- What cause the labour demand curve to shift: the output price, technological change, the supply of other factors

Udbud \parencite[388-390]{Mankiw2011}
% 	(B) The supply of labour
% 		- The trade-off between work and leisure
% 		- What causes the labour supply curve to shift: changes in tastes (increase in female labour), changes in alternative opportunities, immigration

Equilibrium \parencite[390-395]{Mankiw2011}
% 	(C) Equilibrium in the labour market: (1) the wage adjusts to balance the supply and demand for labour and (2) the wage equals the value of the marginal product of labour
% 		- Shifts in labour supply (fewer or more workers leads to higher or lower wages)
% 		- Shifts in labour demand (more or less demand leads to more or less workers / higher or lower wages)

Indkomst [403-410]{Mankiw2011}
% (6.19) (A) Some Determinants of Equilibrium Wages
% 		- Compensating differentials = a difference in wages that arises to offset the non-monetary characteristics of different jobs (fx night shift, dangerous job, boring job etc.)
% 		- Human capital = the accumulation of investments in people such as education and on-the-job training
% 		- Ability, effort and chance
% 		- An alternative view of education - signalling: 
% 		- Above-equilibrium wages: minimum wage laws, unions and efficiency wages: 


%%%%%%%%%%%%%%%%%%%%%%%%%%%%%%%%%%%%%%%%%%%%%%%%%%%%%%%%%%%
\noindent \textbf{(3) Ideologisk grundlag:} Inden for den økonomiske disciplin ses lønarbejdet som et onde/en byrde, hvor det ligger i den menneskelige natur undvige at arbejde - med andre ord at være doven jævnfør Malthus og andre. Derfor den utilitaristiske tradition, hvor rationelle aktører handler nyttemaksimerende. Arbejdsløshed er et arbejdsmarkedsproblem og beskæftigelse er succeskriteriet \parencite[26]{Halvorsen1999}.
% (6.20) % 	(B) The Political Philosophy of Redistributing Income
% 		- Utilitarianism (Bentham, Mill) = the government should choose policies to maximize the total utility of everyone in society - diminishing marginal utility
% 		- Liberalism (Rawls) = the government should aim to maximize the well-being of the worst-off person in society - maximim criterion
% 		- Libertarianism (Nozick) = the government should punish crimes and enforce voluntary agreements but not redistribute income
% 		- Libertarian paternalism (Thaler/Sunstein) = the government should nudge in the direction of improving their own and society's welfare
% 		- Policies to reduce poverty
% 		- Minimum wage laws
% 		- Social security
% 		- Negative income tax = high-income families would pay a tax based on their incomes, low-income families would receive a subsidy
% 		- In-kind transfers = transfers to the poor given in the form of goods and services rather than cash
% 		- Anti-poverty policies and work incentives


%%%%%%%%%%%%%%%%%%%%%%%%%%%%%%%%%%%%%%%%%%%%%%%%%%%%%%%%%%%
\noindent \textbf{(4) Definition af arbejdsløse:} Inden for og uden for arbejdsstyrken \parencite[592-601]{Mankiw2011}
% (9) The Real Economy in the Long Run
% (9.28) Unemployment (s. 592-614): the long-run problem and the short-run problem, natural rate of unemployment, cyclical unemployment 
% 	(A) Identifying Unemployment
% 		- What is unemployment? someone who does not have a job, someone who does not have a job and who is available for work someone who is at the working age who are able and available for work at current wage rates and who do not have a job
% 		- How is unemployment measured?
% 			-the claimant count: novembertal, problem: governmental changes in who are entitled to unemployment benefits
% 			-labour force surveys: ILO definition, employed/unemployed/not in the labour force (or economically active), labour force = the total number of workers including the employed and the unemployed, unemployment rate = the percentage of the labour force that is unemployed, labour force participation rate (or economic active rate) = the percentage of the adult population that is in the labour force
% 		- How long are the unemployed without work? short-term/long-term
% 		- Why are there always som people unemployed? frictional unemployment = unemployment that results because it takes time for workers to search for the jobs that best suit their tastes and skills, structural unemployment = unemployment that results because the number of jobs available in some labour markets is insufficient to provide a job for everyone who wants one.


%%%%%%%%%%%%%%%%%%%%%%%%%%%%%%%%%%%%%%%%%%%%%%%%%%%%%%%%%%%
\noindent \textbf{(5) The long run og the short run:} Typer af arbejdsløshed: friktionsledighed/skifteledighed, sæsonledighed, konjunkturledighed og strukturledighed \parencite[707-725, 782-815]{Mankiw2011}
% (9.28) Unemployment (s. 592-614): the long-run problem and the short-run problem, natural rate of unemployment, cyclical unemployment 
% (12) Short-run Economic Fluctuations
% (12.33) Keynes and IS-LM Analysis (s. ): Keynes attempt to eplain short-run economic fluctuation in general and the Great Depression in particular, and he critizes classical economic theory for only eplaing the long-run effect of policies.
% 	(A) The Keynesian Cross
% 		- Planned spending, saving or investment = the desired or intened actions of households and firms
% 		- Actual spending, saving or investment = the realized or ex port outcome resulting from actions of households and firms
% 		- Deflationary and inflationary gaps = gaps between equilibrium and full employment
% 	(B) The Multiplier Effect = the additional shifts in aggregate demand that result when expansionary fiscal policy increases income and thereby increases consumer spending (the government makes a contract for £10 billion to bud three new nuclea power plants -> raise employment and profits in the construction company -> the construction company buy ressources from other contractors -> raise employment and profits among contracters -> spending in comsumer goods -> each pound spend can raise the aggreate demand for goods and services by more than a pound.
% 	(C) The IS and LM curves: IS-LM describes equilibrium in two markets and together determines general equilibrium in the economy. General equlibrium in the economy occurs at the point where the goods market and money market are both in equlibrium at a particular interst rate and level of income. IS stands for Investment and Saving; LM stands for Liquidity and Money. The thing linking these two markets is the rate of interst (i). The IS vcurve shows the relationsship between the interst rate and the level of income (Y) n the goods market.
% 	(D) General Equilibrium Using the IS-LM Model
% 		- Fiscal policy: increase spending to boost economic activity
% 		- Monetary policy: the central bank decied to expand the money supply
% 	(E) From IS-LM to Aggregate Demand

% (12.36) The Short-run Trade-off Between Inflation and Unemployment (s. 782-815)
% 	(A) The Phillips Curve: negative corralation between the rate unemployment and the rate of inflation - low unemployment tend to have high inflation, and high unemployment tend to have low inflation. A curve that shows the short-run trade-off between inflation and unemployment


%%%%%%%%%%%%%%%%%%%%%%%%%%%%%%%%%%%%%%%%%%%%%%%%%%%%%%%%%%%
\noindent \textbf{(6)Mikro- og makroøkonomi:} Makroøkonomisk teori beskæftiger sig med at studere årsagerne til arbejdsløshed. Mikroøkonomisk teori beskæftiger sig med individernes valg som årsag til ledighed med fokus på incitamenter og ydelser \parencite[25]{Halvorsen1999}.



%%%%%%%%%%%%%%%%%%%%%%%%%%%%%%%%%%%%%%%%%%%%%%%%%%%%%%%%%%%
\subsubsection{\textbf{Arbejde-fritidsmodellen} \parencite[27]{Halvorsen1999}}

(1) Arbejde-fritidsmodellen antager at arbejdskarft bestemmes af individers afvejninger mellem arbejde og fritid under ulige lønsatser. Afvejningen er en funktion af to forhold - indtægt og fritid. Fritid skal forstås som et konsumgode, hvor prises er implicit givet af lønnen. Denne model kan også indholde andre værdier end de rent økonomiske. Becker har udvidet denne model fra traditionel økonomisk teori til også at inddrage omsorg.  Arbejdstilbuddet vil derfor under forudsætning at arktørerne vil maksimere deres totalnytte være bestemet af præferencer, indtægt som ompås på markedet og produktivitetet(nytten) af den aktivitet som sker uden for markedet. Nytten af arbejdsløshed er således den øgede fritid. \parencite[27]{Halvorsen1999} \parencite[388-390]{Mankiw2011}


%%%%%%%%%%%%%%%%%%%%%%%%%%%%%%%%%%%%%%%%%%%%%%%%%%%%%%%%%%%
\subsubsection{\textbf{Søgeteori} \parencite[27-29]{Halvorsen1999}}

% 	(B) Job Search = the process by which workers find appropriate jobs given their tastes and skills
% 		- Why some frictional unemployment is inevitable
% 		- Public policy and job search: government policies try to facilitate job search in various ways (jobcenter, arbejdsformidlingen)
% 		- Unemployment insurance: can increase unemployment (lack of incentives) - Tatsiramos the benefits to workers searching for jobs and receiving unemployment insurrance is greater than the cost
% 	(C) Minimum Wage Laws: if the wage is kept above the equilibrium level for any reason, the result is unemployment
% 	(D) Unions and Collective Bargaining
% 		- The economics of unions: collective bargaining = the process by which uions and firms agree on the terms of employment - when a union raises the wage above the quilibrium level, it rases the quantity of labour supplied and reduces the quantitaty of labour demanded, resulting in unemployment (insiders/outsiders)
% 		- Are unions good or bad for the economy? 
% 	(E) The Theory of Efficiency Wages: firms operate more efficently if wages are above the quilibrium level - Works health - Worker turnover (workers wont leave) - Worker effort (incentive to work hard) - Worker quality (icentive to seek jobs - reservations wage) \parencite[592-601]{Mankiw2011}

\textbf{(1) Introduktion:} I søgeteorien betragtes søgeprocessen som en bestræbelse på at maksimere egennyte - det vil sige fordelene man opnår i form af bedre job på længere sigt må afvejes mod omkostninger i form af indkomsttab og omkostninger ved at indhente informationer om lediges jobs med videre mens søgningen foregår. Arbejdsløshed ses ikke som en udelukkende negativ ting, fordi jobsøgning kan ses som en ivestering med sigte om at få et bedre job. Men teorien implicerer, at jo bedre mulighed for alternativ forsøgelse, desto længere bliver søgeperioden. I teorien forudsættes en positiv sammenhæng mellem søgeintensitet og antal jobjobstilbud man modtager \parencite[27-28]{Halvorsen1999}

\textbf{(2) Reservationsløn:} Om man rent faktisk accepterer arbejdstilbuddet er hængigt af den enkelte jobsøgere reservationsløn - det vil sige den laveste løn en arbejdsløs er villig til at accepterer for at tage et job. Denne påvirkes blandet andet af arbejdsløshedsforsikringens niveau., fordi alternativomkostninger ved jobsøgning er lavere end den ellers ville have være uden arbejdsløshedsforsikring. Arbejdsløshedsforsikring subsiderer jobsøgningen. Men reservationslønnen påvirkes også af arbejdsløshedsniveauet - jo højere arbejdsløshed, jo mindre er sandsynligheden for at få et jobtilbud og desto lavere vil reservationslønnen være. Engelske og amerikanske studier viser, at høj kompensationsgrad giver høj reservationsløn. Svensk studier viser, at dagpenge har en stærk effekt på reservationslønnen. Reservationslønnen antager også at falde jo nærmere man kommer slutningen af dagpengeperioden. Positiv varighedsafhængighed betyder at jo længere tid man er arbejdsløshed, jo mere vil reservationslønnen falde og sandsynligheden for at komme i arbejde øges over tid. Negativ varighedsafhængighed betyder jobtilbudene vil blive færre og sandsynligheden for at få et arbejde vil falde. \parencite[28]{Halvorsen1999}

\textbf{(3) Kritik 1:} Problemet med jobsøgningsteori er at arbejdsløse næsten altid modtager et tilbud, så snart tilbudet er modtaget, hvilket betyder at variationer i varighedsperioden opstår primært ud fra variation i sandsynligheden for at modtagere et jobtilbud \parencite[28-29]{Halvorsen1999}. \textbf{Kritik 2:} Et andet problem er ifølge Atkinson at lønarbejde foretrækkes frem for ledighed fordi man derved lever op til sociale forventninger og får selvrespekt - det vil sige ikke-økonomiske aspekter af arbejdsmotivation er af stor betydning for arbejdsmarkedsadfærden. \parencite[29]{Halvorsen1999}. \textbf{Kritik 3:} Et tredje problem er, at det er urealistisk at fremstille varigheden af arbejdsløshed som en strategisk tilpasning med sigte på nyttemaksimering, hvor jobsøgning og afgørelse om accept af jobtilbud ses som et resultatet af frivillig valg - øget søgeintensitet koster endvidere penge fo, da for lav ydelser kan bidrage til lav søgeaktivitet - for at søge jobs har man brug for et minimum af økonomiske resultater, så dagpengeydelserne kan således virke som en positiv effekt på søgeaktivetet - noget som giver en anden hypotese. \parencite[29]{Halvorsen1999}.


%%%%%%%%%%%%%%%%%%%%%%%%%%%%%%%%%%%%%%%%%%%%%%%%%%%%%%%%%%%
\subsubsection{\textbf{Principal-agent teori / Moral Hazard} \parencite[29-31]{Halvorsen1999}}

(1) Denne teori er en forsikringsvariant af principal-agentteori kalder “moral hazard” og som henviser til den effekt forsikring har på menneskelig adfærd. Moralsk risiko opstår, når en forsikringstager (agent) uden bekostninger kan påvirker sandsynligheden for en hændelse uden at forsikringsselskabet (principalen) ved noget om det - det vil sige at informationen er assymetrisk mellem den forsikrede og forsikringsselskabet, fordi forsikringsselskabet ikke kan observere den forsikredes handlinger eller kende hans information, men bare kan observere resultatet af handlingerne. Den forsikrede kan således påvirke sandsynligheden for en hændelse. Tilstedeværelsen af en forsikringsordning bevirker en øgning i forventet tab som følge af forsikringstagers risikoadfærd. For arbejdsløse betyder det, at de frivillig kan sige et job op eller være seriøs om at søge nyt arbejde og afslå jobtilbud eller tilbud om oplæring samtidig med at denne information kan holdes skjult for arbejdsformidlingen. Risiko som med høj sandsynlighed indeholder en moralsk risikokomponent, bør derfor ikke have fuld kompensationsgrad. For at løse moralsk risko problemet vil den forsikrede må bære en egenrisiko, hvilket vil sige kompensationsgraden må være lavere en 100 procent - alternativ at de udøves kontrol med hvorvidt der søges forsikring for opfyldens lovens krav. Dermed kan principalen (staten) påvirker agentens (den arbejdsløses) handlinger ved at udøve kontrol (jobformidlingen) eller gennem kompensationsgraden. Dagpengeordningen åbner op for moralsk risiko særlig i forhold til hvor læge an går ledig. Dette hænger sammen med at principalen ikke direkte har tiltang til agentens handlingsmotiver og handlingsbetingelse. Således begrænset og assymetrisk information giver derfor rum for opportunistisk adfærd fra agentens side for eksempel ved at holde information tilbage \parencite[29-30]{Halvorsen1999}.

(2) Man kan begrænse den opportunistiske adfærd. For det første ved at gi ufuldstændig dækning af tab, det vil sige operere med en egenrisiko. For det andet kan man gennem etbalisering af kontrolordninger ved for ekskempel at stille krav om at den ledige er reel arbejdssøgerende og således villig til at tage arbejde eller deltage i arbejdsmarkedstiltag \parencite[29-30]{Halvorsen1999}. 

(3) Svært at efterprøve empirisk \parencite[33]{Halvorsen1999}.



%%%%%%%%%%%%%%%%%%%%%%%%%%%%%%%%%%%%%%%%%%%%%%%%%%%%%%%%%%%
\subsubsection{\textbf{Hysterese} \parencite[32-33]{Halvorsen1999}}

(1) Hysterese henviser til processer som gør det vanskeligt at reducere arbejdsløshed, når den først er kommet op på et højt niveau. Dette er primært en maktroøkonomisk forklaring, men noget af forklaringer kan også knyttes til aktørerne på arbejdsmarkedet - både arbejdsmarkedskraftskøbere og arbejdskraftssælgere. Teorier om hysterese er knyttet til permanente eftervirkninger på øget arbejdsløshed. Dette knytter sig blandt andet til, at langtidsledige risikerer at blive forbigået ved nyansættelser \parencite[32-33]{Halvorsen1999}.


%%%%%%%%%%%%%%%%%%%%%%%%%%%%%%%%%%%%%%%%%%%%%%%%%%%%%%%%%%%
\subsubsection{Kritik af økonomiske forståelser af arbejdsløshed \parencite[33-36]{Halvorsen1999}}

\noindent \textbf{(1) Sociologernes kritik:} Granovetter, Hedstroem, Halvorsen

\noindent \textbf{(2) Økonomernes kritik:} Goul

\noindent \textbf{(3) Halvorsens nøgleord}: Incitamenter. Dagpengeordningens kompensationsgrad. Analyse af dagpengeordningen isoleret (vi tager alt med). Matching mellem arbejdssøgers kvalifikationer og jobkrav (meget relevant for os). Rationelle aktører forudsætte at man har god information (sjældent tilfældet). Sociologer. Discouraged worker effect. At modtage understøttelse er ikke money-for-nothing. . \parencite[33-36]{Halvorsen1999}


%%%%%%%%%%%%%%%%%%%%%%%%%%%%%%%%%%%%%%%%%%%%%%%%%%%%%%%%%%%
\subsubsection{Opsummering}

\noindent Effektivitet = effektiv tildeling af tilgængelige ressourcer - trade-off mellem lighed (demokrati) og effektivitet (marked) \parencite[24-25]{Halvorsen1999}. Inden for den økonomiske disciplin ses lønarbejdet som et onde/en byrde, hvor det ligger i den menneskelige natur undvige at arbejde - med andre ord at være doven jævnfør Malthus og andre. Derfor den utilitaristiske tradition, hvor rationelle aktører handler nyttemaksimerende. Arbejdsløshed er et arbejdsmarkedsproblem og beskæftigelse er succeskriteriet. Økonomien er et statistisk baseret styringsvidenskab som har forudsigelse som den primære målsætning. Dette bliver især tydeligt i den samfundsøkonomiske forskning om den negative virkning af sociale sikringsordninger over for den arbejdsløse, hvor ydelserne bliver holdt lave for at afskrække frivillig arbejdsløshed \parencite[26]{Halvorsen1999}. 



















%%% Litteratur
	% økonomisk pensum:
		% “Makiw og Taylor: Economics”
		% “Makiw: Macroeconomics”
		% “Birch Sørensen: Introducing Advanced Macroeconomics”
		% “Pierre Cahuc og André Zylbergberg: Labor economics”
	% specialer:
		% “Anne Nylev: Konsekvenser af arbejdsløshed i krisetider - Teoretisk analyse af konjunkturafhængig arbejdsløshedsforsikring”
		% “Mille Bjørk: Gør arbejdsløshed efterløn til et attraktivt valg? - En empirisk analyse af arbejdsløshedseffekten på tilbagetrækning”
		% “Morten Stig Hansen: Optimal konjunkturafhængig - aktivering og arbejdsløshedsunderstøttelse”
	% Sociologer og bløde økonomer: Grannovetter, Hedstroem, Halvorsen, Rosholm, Jørgen Goul
	% Økonomer: Andersen og Svarer, Bailey, Chetty, Chei og Karni, Feldstein, Fredriksen og Holmlund, Geerdsen, Lalive, Landait og Michaeillat, Mortensen, Schmieder, Tatsiramos, Tranæs og Kreiner
	% Andet: Den store danske






























% \subsubsection{Økonomi}

% Dette fokus vurderer vi hænger sammen med hvad Jørgen Goul Andersen kalder for et skifte inden for en bred strømning af økonomisk teori fra slutningen af 1970'erne, hvor velfærdsstaten blev anskuet som et middel at afbøde “markedsfejl” til i stigende grad at fokusere på “politikfejl” og “forvridninger” på markedet, hvilket kan karakteriseres som et skifte fra efterspørgsel til udbud og et skifte fra makro til mikro \parencite[19]{Andersen2003}. Vi vil i det følgende fremhæve den generelle søgeteori som i forskellige afarter dominerer den økomiske-videnskabelige debat og har stor indflydelse på policy-studier og den danske beskæftigelsespolitik.

% \subsection{Søgeteoriens forskellige søgemodeller\label{}}

% % I lærebøgerne fra økonomisk institut ved KU fremhæves xx, xx og xx's teorier om arbejdsløshed/forsikring/økonomiske incitamenter.

% Forskning inden for søgeteorien har vist, at jobsøgningsintensiteten og den løn som de er villige til at accepterer er centrale for længden af en arbejdsløshedsperiode. Det betyder, at arbejdsløshedsunderstøttelse har indflydelse på arbejdsløses adfærd og længden på af en arbejdsløshedsperiode. Martin Feldstein er en kritiker af arbejdsløshedsforsikring, fordi den får arbejdsløsheden til at stige og understøttelsen er for høj. 

% (Bailey 1978: 379).

% \subsubsection{Bailey-Chetty-betingelsen (teoretisk matchingmodel)\label{}}

% Martin Neil Bailey (1978) definerer det optimale arbejdsløshedsforsikringsystem som en afvejning mellem jobsøgningsincitament og arbejdsløshedsforsikringen.

% Den marginale gevinst af arbejdsløshedsforsikring er lig den marginale omkostning ved øget arbejdsløshed.

% konkluderer, at der er en afvejning mellem incitament til jobsøgning og arbejdsmarkedsforsikring.

% % Intro - Forsikring - Moralfare

% \subsubsection{Søgemodeller \label{}}
% Søgeteorien blev grundlagt i 1960'erne og kendte teoretikere er John J. McCall, George Stigler, Peter Diamon, Dale T. Mortensen og Christopher A. Pissarides og den basale søgemodel handler om arbejdsløse som søger beskæftigelse. Forestil en arbejdsløs som af og til får jobtilbud. Den arbejdsløse kender først lønnen på jobtilbuddet, når det modtages. Efter at have modtaget tilbuddet, skal den arbejdsløse beslutte sig for, om tilbuddet accepteres eller afslås. Dette gør den arbejdsløse på baggrund af en reservationsløn, som er fastsat af forventningerne til lønnen og viden om lønfordelingerne på arbejdsmarkedet. Hvis det modtagne tilbud er større end reservationslønnen, accepteres tilbuddet, og ellers afslås det \parencite[159f]{Rosholm2009}. Marginalisering kan i denne sammenhæng forklares med, at den arbejdsløses kvalifikationer bliver mindre værd efterhånden som ledighedsperioden bliver længere, at den arbejdsløse gradvist mister forbindelsen til gamle kolleger eller at den arbejdsløse dømmes på baggrund af sin langtidsledighed \parencite[160f]{Rosholm2009} \parencite[20]{Andersen2003}.

% \subsubsection{Partielle søgemodeller \label{}}

% \subsubsection{Matchingmodeller \label{}}
% Matching-teorien deler arbejdsmarkedet op i to typer agenter: arbejdsgivere og lønmodtagere og er kendt for teoretikere som Dale T. Mortensen, Christopher A. Pissarides Peter A. Diamond, Alvin E. Roth and Lloyd Shapley. Her leder lønmodtagerne efter job i de perioder, hvor de er arbejdsløse, mens arbejdsgivere slår stillinger op, så længe de vurderer, at det kan betale sig. Ledige lønmodtagere og job mødes på arbejdsmarkedet i en matching-proces. Når der skabes kontakt mellem en arbejdsgiver og lønmodtagere, opstår der en forhandling mellem arbejdsgiver og lønmodtager om fordelingen af overskuddet i en eventuel ansættelse. Arbejdsgiveren kan på den ene side presse lønmodtageren til at acceptere en løn, som ligger under arbejdskraftens marginalprodukt, fordi lønmodtageren ikke uden videre kan finde nyt job på anden vis end ved at vente på det næste jobtilbud. Lønmodtageren kalkulerer på baggrund af forskellen mellem den tilbudte løn og værdien af at være ledig (“outside option”). Matching-teorien kan bruges til at analyse den umiddelbare effekt af at ændre i ledighedsydelser som eksempelvis dagpenge eller kontakthjælp \parencite[162f]{Rosholm2009}

% \subsection{Insider-outsider-teorien \label{}}
% Insider-outsider-terien er udviklet af Assar Lindbeck og Dennis Snower i 1980'erne og fokuserer på omkostningerne forbundet med ansættelser og afskedigelser af arbejdskraft. Omkostningerne kan være i forbindelse med søge- og optræning, afskedigelse, uproduktiv konkurrence mellem to grupper af lønmodtagere. Insidere og outsidere kan blandt andet defineres som beskæftige og arbejdsløse, fagforeningsmedlemmer og ikke-medlemmer, ansatte i gode jobs/dårlige jobs. Insidere vil forsøge at forhandle sig til så høje lønninger som muligt og afholde andre for at underbyde dem på markedet. Insidernes magt består blandt andet i, at arbejdsgiverne har omkostninger vil at afskedige dem og kan skabe omkostninger ved strejke, aktioner og mobning af nyansatte. Langtidsledige, i modsætning til beskæftige og korttidsledige, kan tilbyde sin arbejdskraft til reduceret løn og forsøge at overbevise en arbejdsgiver om at blive ansat så længere arbejdsgiverens omkostninger ved at ansætte en outsider ikke stiger gevinsten ved at gøre det \parencite[164]{Rosholm2009} \parencite[20]{Andersen2003}.



%%%%%%%%%%%%%%%%%%%%%%%%%%%%%%%%%%%%%%%%%%%%%%%%%%%%%%%%%%%
% \subsubsection{Den Store Danske Encyclopædi}
% Kernen i vores teoretiske og empiriske arbejde er arbejdsløshed. Arbejdsløshed defineres i \textit{Den Store Danske} som den manglende overensstemmelse på arbejdsmarkedet mellem udbud af arbejdskraft og efterspørgsel efter arbejdskraft \parencite{2015}. I de almindelige statistiske definitioner opdeles den danske befolkning i folk der er inden for og uden for arbejdsstyrken. Dem der er inden for arbejdsstyrken er enten i beskæftigede eller arbejdsløse, mens alle andre per definition betragtes som værende uden for arbejdsstyrken \parencite{2015a}. Problemet med denne opdeling er, at personer der klassificeres som uden for arbejdsstyrken ofte kommer fra beskæftigelse og bevæger sig tilbage i beskæftigelse. De følger måske ikke standarddefinitionen på arbejdsløshed\footnote{\textit{International Labour Organization} definerer arbejdsløse som det antal personer som står uden beskæftigelse samtidig med at være til rådighed for arbejdsmarkedet og aktivt arbejdssøgende \parencite{ILO1982}. Denne definition fremgår både af \textit{Den Store Danske}, Danmarks Statistik, Beskæftigelsesministeriet med flere.}, men kan i et lidt bredere perspektiv godt betragtes som arbejdsløse. Eksempelvis kan personer på revalideringsydelse, kontanthjælp og førtidspension godt vende tilbage i beskæftigelse igen. For at få en bedre forståelse af arbejdsløses sociale mobilitet på arbejdsmarkedet må vi derfor bryde med de almindelige definitioner af arbejdsløshed\footnote{Vi er vel opmærksomme på den symbolske kamp, der ligger i at gøre dette, hvilket fremgår af de politiske diskussioner om arbejdsløse og kontanthjælpsmodtagere, der er ’skjult’ i aktiveringsforløb...}. For at bryde med standarddefinitionerne af arbejdsløshed vil først og fremmest redegøre for sociologisk-videnskabelige og økonomisk-videnskabelige tilgang til arbejdsløse og arbejdsløshed for så til sidst at anvende Bourdieu og marginaliseringsbegrebet til at lave en teoretisk operationalisering af arbejdsløshed.
% Når en stor gruppe får sværere ved at konkurrere om de ledige jobs, vil arbejdsløsheden i mindre grad lægge pres på løndannelsesprocessen, og løntilpasningen vil således ikke kunne sikre en tilbagevenden til høj beskæftigelse. Jo længere tid arbejdsmarkedet er præget af høj arbejdsløshed, jo flere af de arbejdsløse vil blive marginaliserede, og jo større bliver den strukturelle ledighed \parencite{2015}.
% Arbejdsløshed forårsages og inddeles typisk i friktionsledighed\footnote{Friktionsledighed (også kaldet skifteledighed) opstår ofte i forbindelse med indtræden på arbejdsmarkedet eller ved jobskift, hvor en mellemliggende ledighedsperiode kan forekomme selv i perioder med høj økonomisk aktivitet \parencite{2015}.}, sæsonledighed\footnote{Sæsonledighed opstår på grund af sæsonmæssige svingninger i produktionen, som for eksempel kan skyldes vejret. En stor del af sæsonarbejdsløsheden forekommer som midlertidig hjemsendelsesledighed, idet den arbejdsløse efter en kortere periode som arbejdsløs kommer tilbage til den samme arbejdsgiver. Denne type ledighed rammer for eksempel bygge- og anlægsaktiviteten og turisterhvervene \parencite{2015}.}, konjunkturledighed\footnote{Konjunkturledighed skyldes, at den samlede efterspørgsel i samfundet efter varer og tjenester ikke er tilstrækkelig til at skabe en beskæftigelse svarende til fuld beskæftigelse \parencite{2015}.} og strukturledighed\footnote{Strukturarbejdsløshed forårsages af manglende faglig eller geografisk fleksibilitet på arbejdsmarkedet, hvorved markedsmekanismen ved hjælp af tilpasning af lønnen ikke er i stand til at sikre fuld beskæftigelse \parencite{2015}.}. I praksis er det ifølge \textit{Den Store Danske} vanskeligt at udskille friktions- og sæsonledighed fra strukturledighed, hvilket betyder, at der oftest i den økonomisk-politiske debat kun sondres mellem konjunkturarbejdsløshed og strukturarbejdsløshed \parencite{2015}.
% I praksis er det ifølge \textit{Den Store Danske} vanskeligt at udskille friktions- og sæsonledighed fra strukturledighed, hvilket betyder, at der oftest i den økonomisk-politiske debat kun sondres mellem konjunkturarbejdsløshed og strukturarbejdsløshed \parencite{2015}.






















%%%%%%%%%%%%%%%%%%%%%%%%%%%%%%%%%%%%%%%%%%%%%%%%%%%%%%%%%%%
\newpage \subsection{\textsc{Sociologiske forståelser af arbejdsløshedsproblemet}}
%%%%%%%%%%%%%%%%%%%%%%%%%%%%%%%%%%%%%%%%%%%%%%%%%%%%%%%%%%%

Sociale omgivelsers tilnærmelser - funktion og normativ deprivationsteori. %%% \parencite{Halvorsen1999} 

Institutionel teori. %%% \parencite{Halvorsen1999} 

Agent Teori. %%% \parencite{Halvorsen1999} 

Handlingsteori. %%% \parencite{Halvorsen1999} 

Teorier om afhængighed. %%% \parencite{Halvorsen1999} 

Marginalisering. %%% \parencite{Halvorsen1999} 

Kritik. %%% \parencite{Halvorsen1999}, Hedström


% Det Nationale Forskningscenter for Velfærd (SFI) er storproducent af policy-studier på arbejdsmarkeds- og beskæftigelsesområdet med rapporter som typisk er bestilt af Beskæftigelesministeriet eller forskellige ministerier og kommuner. Rapporterne tager typisk udgangspunkt i en grupper af personer. Hvis man kigger på SFIs rapporter de sidste 20 år er målgrupperne hovedsageligt ledige\footnote{Også kaldet udsatte ledige, arbejdsmarkedsparate ledige, ikke-arbejdsmarkedsparate ledige, langtidsledige og forsikrede ledige}, dagpengemodtagere\footnote{Også kaldet sygedagpengemodtagere og aktiverede dagpengemodtagere.}, kontanthjælpsmodtagere\footnote{Også kaldet ikke arbejdsmarkedsparatemodtagere, de svageste kontanthjælpsmodtagere og aktiverede kontanthjælpsmodtagere.}, sygemeldte og arbejdsskadede\footnote{Også kaldet skadeslidte beskæftigede og personer som har nedsat arbejdsevne efter en ulykke i fritiden.} samt pensionister og efterlønsmodtagere\footnote{Der er også lavet en hel del undersøgelse om handicappede, indvandrere, efterkommere, mænd, kvinder, ældre og højtuddannede.}. Fokus handler hovedsageligt om at få dem i beskæftigelse eksempelvis ved at bringe de langtidsledige tættere på arbejdsmarkedet i \textit{Tættere på arbejdsmarkedet} (2011), ved at måle beskæftigelseseffekten af dagpengeophør i \textit{Dagpengemodtagers situation omkring dagpengeophør} (2014), ved at kigge på indsatser over for ikke arbejdsmarkedsparate kontanthjælpsmotagere i \textit{Veje til beskæftigelse} (2010), ved at måle effekten af den beskæftigelesrettede indsats for sygemeldte i \textit{Effekten af den beskæftigelsesrettede indsats for sygemeldte} (2012) eller ved at kigge på pensionisters og efterlønsmodtageres genindtræden op arbejdsmarkedet i \textit{Pensionisters og efterlønsmodtageres arbejdskraftspotentiale} (2012).

% Det som er kendetegnede ved denne typer rapporter er et grundlæggende fokus på at få de pågældende personer tilbage i beskæftigelse hvad man kan gøre og ikke hvad deres situation egentlig betyder for deres liv\footnote{Der skal ikke menes med, at disse rapporter slet ikke forholder sig til de pågældende personers liv. I \textit{Veje til Beskæftigelse} (2010) fortælles der igennem 30 kvalitative interviews med sagsbehandlere, at de oplever de ikke-arbejdsmarkedsparate kontanthjælpsmodtagere som værende en heterogen gruppe som har gavn af forskellige typer indsatser alt efter, hvilke udfordringer de har. Nogle har for eksempel helbredsproblemer, mens andre har brug for hjælp til daglige gøremål.}.


% \subsubsection{Jahoda, Lazarsfeld og Zeizels Marienthal-studie} %%% (Nørup:23)
% Marienthal er et klassisk studie af de sociale konsekvenser af arbejdsløshed i et lille samfund gennemført af Marie Jahoda i samarbejde med Paul Lazarsfeld og Hans Zeizel (1971). Marienthal var et industriby som led af høj arbejdsløshed i 1920'erne, og studiet undersøger hvad der sker med arbejderne i den østrigske by Marienthal, når de oplever arbejdsløshed. Med Marienthal udvikler Jahoda deprivationsperspektivet, som er det mest udbredte perspektiv i de teoretiske diskussioner af arbejdsløshed og eksklusion fra arbejdsmarkedet (Creed og Macintyre:2001). Hovedargumentet er, at arbejdsløshed medfører social eksklusion og isolation, tab af struktur i hverdagen og selvtillid og en betydelig øget risiko for psykiske problemer (Jahoda, 1981, Jahoda m.fl. 1997). Marienthal baserer sig på en grundlæggende antagelse om arbejde deltagelse på arbejdsmarkedet opfylder både et psykologisk behov for individet og et økonomisk behov for indtægt. Jahoda opstiller ikke knækket vilje, resignation, fortvivlelse og apati som fire stadier eller reaktioner den arbejdsløse gennemgår (Jahoda 1979, Jahoda m.fl. 1997). Arbejdsdeltagelsen har ifølge Jahoda fem funktioner: tidsmæssig struktur i dagligdagen, sociale kontakter, deltagelse i kollektive formål, status og identitet og regelmæssig aktivitet (Jahoda, 1981, Jahoda m.fl. 1997).
% Nørup kritiserer brugen af deprivationsperspektivet i dansk regi, fordi det danske samfund i dag er en moderne velfærdstat med relativt højtuddannet arbejdskraft, og Marienthal er en mindre industriby med lavt uddannet arbejdskraft i et 1930'ernes Østrig som ikke er i nærheden af et velfærdssamfund (Nørup2012:34).

% \subsubsection{Eisenberg og Lazarsfeld} %%% (Nørup:24)
% I *The psychological effects of unemployment* (1938) konkluderer Philip Eisenberg og Paul Lazarfeld, at arbejdsløse gennemlever tre stadier: “First there is shock, which is followed by an active hunt for a job, during which the individual is still optimistic and unresigned; he still maintains an unbroken attitude. Second, when all efforts fail, the individual becomes pessimistic, anxious, and suffers active distress; this is the most crucial state of all. And third, the individual becomes fatalistic and adapts himself to his new state but with a narrower scope He now has a broken attitude.” (Eisenberg og Lazarsfeld 1938:378).
% Studiet har vundet udbredelse inden for socialpsykologien (Boyd 2014, Wang og Greenwood 2014, Kahn 2013, Ezzy 1993, Ragland-Sulivan og Barglow 1981, Finley og Lee 1981, Hayes og Nutman 1981, Hill 1978, Briar 1977, Harison 1976). Men studiet er også blevet kritiseret på baggrund af en problematisk og modsætningsfuld metode (Fryer 1985, Ezzy 1993) og på baggrund af det empiriske fundament i særdeleshed vedrørende psykologiske faktorer som eksempelvis mentalt helbred og selværd (Fryer 1985, Hartley 2011, Shamir 1986). (Nørup kalder det for *Stadie Model*)

% \subsubsection{Tiffany, Cowan og Tiffany} %%% (Nørup:25)
% Hovedargumentet i studiet *The unemployed: A social-psychological portrait*  af Donald Tiffany, James Cowan og Phyllis Tiffany (1970) er, at majoriteten af arbejdsløse og ekskluderede fra arbejdsmarkedet står uden for arbejdsmarkedet på grund af psykologiske problemer. Sammenhængen mellem arbejdsløshed og psykologiske problemer går derfor begge veje, hvilket betyder, at psykologiske kan være årsagen til arbejdsløshed på samme tid med, at arbejdsløshed i sig selv også medfører psykologiske problemer: ”They show avoidance behaviour patterns or what has been referred to as ”work inhibition” which implies that they are physically capable of work but prevented from work because of psychological disabilities” (Tiffany, Cowan and Tiffany, 1970). Ifølge Tiffany, Cowan og Tiffany er løsningen, at staten rehabiliterer disse arbejdsløse, så de kan komme tilbage på arbejdsmarkedet gennem træning eller terapi (Tiffany, Cowan and Tiffany: 1970). Douglas Ezzy peger på, at denne tilgang har ligheden mellem den historiske distinktion mellem *deserving poor*, som fysisk var ude af stand til at arbejde og fortjente støtte og *non-deserving poor*, som ikke arbejde selvom de var i fysisk stand til at arbejde (Ezzy 1993). Ezzy påpeger ligeledes på, at tilgangen har været mest toneangivende i perioder med højkonjunktur og relativ lav ledighed til sammenligning med perioder med lavkonjunktur og lav ledig (Ezzy 1993).
% Perspektivet kritiseres for at have lighedstræk med den neoklassiske økonomiske betragtning af arbejdsløshed som frivilligt og derfor ved at skyde skylden på ofret (Miles:1987) (Nørup kalder det for *Rehabiliteringstilgangen*)

% \subsubsection{Warr} %%% (Nørup:25) 
% I *Work, Unemployment and Mental Health* opstiller Peter Warr ni faktorer i omgivelser som har betydning for det mentale helbred i forbindelse med arbejdsløshed: mulighed for kontrol, mulighed for at benytte erhvervede færdigheder, eksternt genererede mål, variation, klarhed i forhold til omgivelserne, penge og  indtjening, fysisk sikkerhed, mulighed for social kontakt og social position (Warr:1987). Individets mentale helbred afspejler det akkumulerede niveau af faktorerne, så det at miste et arbejde eller det at have et dårligt arbejde i omgivelserne afspejler individets mentale heldbred (Ezzy:1993). (Nørup kalder det for *vitaminmodellen*)

% \subsubsection{Halvorsens teori om mestring} %%% (Nørup:28)
% Knut Halvorsen udvikler i sit forfatterskab en modpol til Jahoda, Lazarsfeld og Zeizel (1971), Eisenberg og Lazarsfeld (1938), Tiffany, Cowan og Tiffany (1970) og Warrs (1987) passive og ensartede individperspektiv ved at betragte arbejdsløse som forskelligartede og handlende. De arbejdsløse anskues derfor som aktive aktører, der kan påvirke og forandre deres situation i stedet for at være ofre for omstændighederne (Halvorsen 1994, 1999). Den arbejdsløse vil indgå i forskellige fysiske, psykiske og sociale aktiviteter for at afbøde effekter af arbejdsløshed og minimere stress og mental belastning (Halvorsen: 1999, Fryer og Fagan:1993, Fryer:1986, Fryer og Payne:1984, O’Brien:1985). Halvorsen skelner mellem den problemorienterede mestring, som udgøres af konkrete strategi med formål at fjerne belastningen af den marginaliserede position (for eksempel jobsøgning) og den emotionsorienterede mestring, som handler om hvordan man ser verdenen (Halvorsen 1999)

% Nørup kritiserer Halvorsen for i sin ontologiske individualisme at fokuserer på, hvordan individet mestrer bestemte livssituationer under bestemte rammer (graden af eksklusion forklares som et resultat af individuelle handlinger og ressourcer) frem for på hvordan de samfundsmæssige strukturer og sociale relationer påvirker eksklusionen (Nørup 2012:37).

% \subsubsection{Glaser og Strauss’ teori om sociale passager \label{}}
% Barney Glaser og Anselm Strauss definerer en status passage som et individs ”movement into a different part of a social structure, or loss or gain of privilige, influence, or power, and changed behaviour”. Ezzy beskriver anvendelsen af teorien på arbejdsløshed og exit fra arbejdsmarkedet, som processer frem for enten-eller tilstande. Hermed kan exit fra arbejdsmarkedet sammenlignes med andre status passage som eksempelvis skilsmisse, sygdom eller dødsfald i familien (Ezzy:1993). Van Gennep benytter separation (begravelse), transition (overgangsfase mellem to jobs) og integration (ægteskab) som kategorier for sociale passager (Ezzy:1993, Van Gennep 1977). Ezzy identificerer exit fra arbejdsmarkedet som tab af job som en afhændelespassage i modsætning til exit fra arbejdsmarkedet som indtræden i uddannelsessystemet som noget helt andet (Ezzy 1993). En afhændelsespassager fører ikke nødvendigvis sig selv til mentale problemer, mistrivsel eller eksklusion, da det afhænger af den enkeltes identitet og selvopfattelse i relation til andre og samfundet og en lang række andre faktorer samt om afhændelsespassagen efterfølges af en reintegrativ passage, hvor den enkelte får en ny status eksempelvis et nyt job (Ezzy 1993).

% Kommentarer
% Deprivation (Jahoda, Lazarsfeld og Zeizel 1971; Eisenberg og Lazarsfeld 1938; Tiffany, Cowan og Tiffany 1970; Warrs (987) er delvist brugbart i forståelsen af arbejdsløshed som havende en social og psykologisk påvirkvning på arbejdsløse. Halvorsen er delvis brugbar i forståelsen af at individerne bliver aktører som kan handle og har ressourcer. Social passage (Glaser og Strauss 1971)er relevant i vores definition af arbejdsløshed som midlertidigt.






% \section{Marginalisering \label{}}

% Marginalisering: dynamisk fænomen \parencite[23-24]{Halvorsen1999}

% For at forstå de arbejdsløses bevægelser i tid og rum, anvender vi marginaliseringsbegrebet, hvor vi trækker på Jørgen Elms Larsens perspektiver om marginalisering som en midterkategori mellem inklusion og eksklusion. Larsen definerer eksklusion som en ufrivillig ikke-deltagelse gennem forskellige typer af udelukkelsesmekanismer og -processer, som det ligger uden for indvidets og gruppens muligheder at få kontrol over \parencite[137]{Larsen2009}. Larsen er kritisk over for Luhmanns inddeling inklusion og eksklusion, som han i sin binære form ikke mener er særlig hensigtsmæssig i forhold til at beskrive virkeligheden \parencite[130]{Larsen2009}. Derfor argumenter han for, at marginalisering kan anvendes som en midtergruppe mellem de to, hvor individet bevæger sig i en proces mod inklusion eller eksklusion.

% Hvis vi benytter os af de førnævnte statistiske definitioner, som fortæller, at beskæftigede og arbejdsløse er en del af arbejdsstyrken og resten af befolkningen står uden for arbejdsstyrken, ser vi to processer, hvilket fremgår af den model\footnote{Modellen er også inspireret af lignende modeller benyttet af Lars Svedberg \parencite[44]{Svedberg1995} og Catharina Juul Kristensen \parencite[18]{Kristensen1999}.}, vi har udviklet i tabel \ref{tab_marginaliseringsmodel_1}. 

% Den \textbf{første} proces består af individer som går fra at være inkluderet til at indgå i en proces i retning mod marginalisering. Et eksempel her på kunne for eksempel være en person som går fra at være beskæftiget som sygeplejerske til at være arbejdsløs. 

% Den \textbf{anden} proces består af individer som går fra at være marginaliseret til at indgå i en proces i retnng mod inklusion. Et eksempel her på kunne for eksempel være den samme person fra foregående eksempel som går fra at være arbejdsløs til at blive beskæftiget som sygeplejerske igen.
% %
% \begin{table}[H] \centering
% \caption{Model over marginalisering 1}
% \label{tab_marginaliseringsmodel_1}
% \begin{tabular}{@{} m{5,9cm} m{5,9cm} @{}} \toprule
% \textbf{Inkluderet} & \textbf{Marginaliseret} \\ \midrule
%   beskæftiget & arbejdsløs \\  
% \end{tabular} \end{table} %%%
% \begin{table}[H] \centering
% \begin{tabular}{@{} m{12,3cm} @{}} 
%   \textbf{Marginaliseringsproces} \\  
%   <--------------------------------------------------------------------------------------------- \\
% \end{tabular} \end{table} %%%
% \begin{table}[H] \centering
% \begin{tabular}{@{} m{12,3cm} @{}} 
%   \textbf{Inklusionsproces} \\  
%   <--------------------------------------------------------------------------------------------- \\ \bottomrule
% \end{tabular} \end{table}
% %
% Modellen, som fremgår af tabel \ref{tab_marginaliseringsmodel_1}, ser bort fra dem som står uden for arbejdsstyrken, selvom der reelt er udveksling mellem arbejdsstyrken og dem uden for arbejdsstyrken. For eksemepl kan en sygeplejestuderende gå fra at stå uden for arbejdsstyrken til at være en del af arbejdsstyrken enten som arbejdsløs eller i beskæftigelse som sygeplejerske. Et andet eksempel er en forhenværende sygeplejerske som er folkepensionist og derved står uden for arbejdsstyrken, men som genindtræder i beskæftigelse som sygeplejeske. Hermed opstår to nye processer som fremgår af den nye model i tabel \ref{tab_marginaliseringsmodel_2}, som indeholder alle fire processer. Den \textbf{tredje} proces består af individer som går fra at være ekskluderet, i kraft af at stå uden for arbejdsstyrken, til at indgå i en proces i retning mod retning mod marginalisering ved at blive arbejdsløs. Den \textbf{fjerde} proces består ligeledes af individer som går fra at være ekskluderet, i kraft af at stå uden for arbejdsstyrken, men som kommer til at indgå i en proces i retning mod retning mod inklusion ved at komme i beskæftigelse.
% % 
% \begin{table}[H] \centering
% \caption{Model over marginalisering 2}
% \label{tab_marginaliseringsmodel_2}
% \begin{tabular}{@{} m{3,2cm} c m{3,5cm} c m{4cm} @{}} \toprule
% \textbf{Inkluderet} & & \textbf{Marginaliseret} & & \textbf{Ekskluderet} \\ \midrule
%   beskæftiget  & & arbejdsløs & & udenfor arbejdsstyrken \\  
% \end{tabular} \end{table} %%%
% \begin{table}[H] \centering
% \begin{tabular}{@{} m{5,9cm} m{5,9cm} @{}} 
%   \textbf{Marginaliseringsproces} & \textbf{Eksklusionsproces} \\  
%   --------------------------------------------> & --------------------------------------------> \\ 
% \end{tabular} \end{table} %%%
% \begin{table}[H] \centering
% \begin{tabular}{@{} m{12,3cm} @{}} 
%   \textbf{Inklusionsproces} \\  
%   <--------------------------------------------------------------------------------------------- \\ \bottomrule
% \end{tabular} \end{table}
% %

%   % beskæftiget  & & “midlertidigt” uden beskæftigelse & & vender ikke tilbage i beskæftigelse \\  

% Vores model viser inklusion, marginalisering og eksklusion på arbejdsmarkedet i et spektrum mellem inkluderet og ekskluderet. 

% Formålet med modellen er ikke at komme med en ny model for arbejdsmarkedet som alternativ til andre måde at anskue arbejdsmarkedet på, men at anvende en model som åbner op for forskellige måde at anskue arbejdsløse på og have en model som indeholder alle i den danske befolkning. Med inkluderet på arbejdsmarkedet forstår vi ingen eller kort afstand til arbejdsmarkedet og med ekskluderet forstår vi en stor afstand til arbejdsmarkedet. Idealtypen på de inkluderede er fuldtidsansatte i faste og sikre stillinger\footnote{Guy Standing fremhæver i \textit{The Precariat, The New Dangerous Class} prekariatet som atypisk beskæftigelse som er mindre sikkert end fastansættelser \parencite{Standing2011}.}. De ekskluderede som forbliver ekskluderet er dem som aldrig bliver en del af eller vender tilbage på arbejdsmarkedet. I et spektrum mellem det at være inkluderet og ekskluderet ligger forskellige grader af marginalisering. En nyuddannet farmaceut som er i løntilskud i en farmaceutstilling, hvor vedkommende er blevet lovet beskæftigelse efter løntilskudsstillingens ophør er derfor tættere på at være inkluderet end en nyuddannet farmaceut som efter halvandet år på dagpenge endnu ikke er kommet til samtale til nogle af de farmaceutstillinger vedkommende har søgt.

% Den sidstnævnte farmaceut som ikke har held med ansøgningerne, vil efter dagpengereformen i 2010 miste retten til dagpenge efter to år, hvilket betyder, at vedkommende har et halvt års dagpenge igen. I den sammenhæng har vi en antagelse om at længden på arbejdsløshedsperioden (eller med andre ord processen mod inklusion eller eksklusion alt efter hvordan det går) spiller en rolle for, hvilke strategier vedkommende har for at vende tilbage i beskæftigelse\footnote{Denne antagelse bakkes især op økonomiske teorier og policy-studier som vil blive behandlet i dette kapitel}. Vedkommende kan eksempelvis vælge at søge videre i farmaceutstillinger, søge bredere i ud stillinger som ikke er farmaceutstillinger, men ligger tæt på farmaceut eller som kræver akademiske kompetencer eller søge stillinger som intet har at gøre med farmaceutarbejde eller akademisk arbejde. Her bliver det et spørgsmål om vedkommende skal vente til at få det “rette arbejde” (hvad end det vil sige) eller “bare” få et eller andet arbejde. Dette bliver i høj grad et spørgsmål mellem arbejdet som en nødvendighed eller arbejdsmarkedets doxa om at arbejde ikke bare er noget man varetager af nødvendighed, men fordi man synes, at det er meningsfyldt\footnote{Ifølge Lars Svendsen findes der inden for europæisk idéhistorie to grundlæggende forskellige opfattelser af hvad arbejde er for en størrelse. Op til reformationen blev arbejde anset som en “meningsløs forbandelse”, og efter reformationen blev arbejdet anset som et “meningsfuldt kald” parencite[13]{Svendsen2010}, hvilket Bourdieu ville kalde et udtryk for arbejdslivets illusio, hvilket vil sige, at troen på at arbejdet er en vigtig del af livet.}.

% Arbejdets værdi og individets strategier for at komme i beskæftigelse er centralt i de to efterfølgende afsnit om den sociologisk-videnskabelige og økonomisk-videnskabelige tilgang til arbejdsløse og arbejdsløshed. Det at komme i beskæftigelse igen er ikke bare at komme i beskæftigelse igen. For farmaceuten er det noget andet at komme i beskæftigelse som farmaceut, som vedkommende har taget en lang videregående uddannelse for at have kompetencer til end at tage et arbeje som kassemedarbejder. Bourdieu skelner mellem “den store elendighed” og “den lille elendighed” \parencite[4]{Bourdieu1999}. Det at komme i beskæftigelse er et udtryk for at komme ud af  “den store elendighed” som arbejdsløsheden og alle de problemer som følger det at være arbejdsløs, men det at komme i beskæftigelse kan også være et udtryk for at vælge den “den lille elendighed”, som for består af forringede arbejdsvilkår og det at blive tvunget til at ofre sig lidt før man bliver ofret og samtidig være taknemmelig over, at man ikke hører til blandt de allersvageste. Hermed tilpasser farmaceuten sig arbejdsmarkedets umiddelbare behov ved i kraft af at være arbejdsløs accepterer samfundets objektive strukturer og derved får nogle realistiske forventninger i forhold til vedkommendes position i den sociale verden, hvor vedkommende som arbejdsløs i sidste ende må tage det arbejde hvad vedkommende kan få.






%%%%%%%%%%%%%%%%%%%%%%%%%%%%%%%%%%%%%%%%%%%%%%%%%%%%%%%%%%%
\subsection{\textsc{Opsummering}}
%%%%%%%%%%%%%%%%%%%%%%%%%%%%%%%%%%%%%%%%%%%%%%%%%%%%%%%%%%%

Arbejdsløse fremtræder forskelligt og et ikke et problem %%% Halvorsenb - der er allerede skrevet et afsnit om det her hvor der tales om ledige, dagpengemodtagere mv.

Marginalisering. %%% Halvorsen 

Incitamenter og arbejdsvilje. %%% Halvorsen 

Psykiske plager og selvrespekt. %%% Halvorsen 

Økonomiske vanskeligheder og det offentlige og private sikkerhedsnet. %%% Halvorsen 



%%%%%%%%%%%%%%%%%%%%%%%%%%%%%%%%%%%%%%%%%%%%%%%%%%%%%%%%%%%
% Fra problemstillingsafsnittet - det passede ikke rigtig ind og måske det passer bedre i en konklusion i forbindelse med en blanding af arbejdsløse ikke bare er arbejdsløse kombineret med marginalisering
% Arbejdsløse fremtræder i dagligdagen, den offentlige debat og blandt forskerne på forskellige måder. Det kan være som ledige, dagpengemodtagere, kontanthjælpsmodtagere. Dertil kommer der en masse tillægsord som udsatte, arbejdsmarkedsparate, ikke-arbejdsmarkedsparate, langtids, forsikrede, ikke-forsikrede, sæson/friktion/førstegang, syge, aktiverede, svage, sygemeldte, arbejdsskadede, skadeslidte. Dertil kommer der undergrupper som handikappede, indvandrere, efterkommere, mænd, kvinder, ældre, højtuddannede\footnote{Denne lange beskrivelse af forskellige typer arbejdsløse kommer fra en gennemgang af Det Nationale Forskningscenter for Velfærd (SFI) rapporter fra de sidste 20 år.}. Dertil kommer der en masse forskellige definitioner af arbejdsløse fra forskellige forskere og institutioner. En af de mest gængse definitioner kommer fra International Labour Organization som definerer arbejdsløse som det antal personer som står uden beskæftigelse samtidig med at være til rådighed for arbejdsmarkedet og aktivt arbejdssøgende \parencite{ILO1982}. Men hvad har de arbejdsløse egentlig tilfælles ud over at de står uden lønnet arbejde. Det er netop, hvad Halvorsen spørger sig selv og konkluderer, at arbejdsløse ikke er en ensartet gruppe, men en kategori sammensat af forskellige mennesker med forskellige udfordringer.  Arbejdsløshed er en institutionel konstruktion, som påvirkers af sociale ordninger og arbejdsmarkedets organisering: “The unemployed are not a group of people, but an economic and adminsitrative category” \parencite{Kelvin1985} \parencite[18]{Halvorsen1999}.
% Overordnet kan man sige, at arbejdsløshed ikke bare et socialt problem, men som flere til dels adskilte sociale problemer. Samtidig er arbejdsløse ikke bare en ensartet gruppe, men forskellige mennesker med forskellige problemstillinger . Til sidst fremstår arbejdsløse - om man kalder dem ledige eller noget tredje som en kategori forbundet med negative beskrivelser \parencite[12]{Halvorsen1999}.


%%%%%%%%%%%%%%%%%%%%%%%%%%%%%%%%%%%%%%%%%%%%%%%%%%%%%%%%%%%
% Konklusioner:
% %
%  \begin{itemize} [topsep=6pt,itemsep=-1ex]
%    \item Marginaliseringbegrebet åbner op for arbejdsløshed i forhold til at inkluderer dem som står uden for arbejdsstyrken.
%    \item Den økonomiske gennemgang bidrager med at vise den dominerende perspektiv på arbejdsløse. Fokus ligger på at få folk i beskæftigelse (fra marginalisering til inklusion).
%    \item Den sociologiske og socialpsykologiske gennemgang bidrager med at få et indblik i de arbejdsløses vilkår. Fokus på arbejdsløshed (marginalisering og eksklusion).
%    \item Flere måde at anskue de arbejdsløse på som helhed, det vil sige alle eller som dele, hvilket både kan være i grupper (for eksempel ledige, langtidsledige og kontanthjælpsmodtagere) eller som årsag (friktionsledighed, konjunkturledighed, strukturledighed og sæsonledighed).
%    \item Blik for uden for arbejdsstyrken, hvilket vil sige studerende, efterløn, pensionister mv.
%    \item Vores definition af arbejdsløse er så bred som overhovedet muligt. Med arbejdsløs har vi som udgangspunkt den bredeste definition overhovedet, hvilket er det at stå uden arbejde. 
%    \item Den teoretiske pointe er at arbejdsløshed defineres og behandles forskelligt alt efter om det er økonomer, sociologer mv.. Vores fokus ligger i forlængelse af marginaliseringsbegrebet \parencite{Larsen2009} samt Bourdieus perspektiver om at være placeret et specifikt sted i det sociale rum og at være på kanten af arbejdsmarkedet.
%    \item Vi mangler noget litteratur om arbejdsstyrken - måske det skal ind over sociologerne.
%   \item Arbejdsløse versus ledige: Fokus på arbejdsmarkedsparate arbejdsløse, men værd opmærksom på, at arbejdsmarkedsparathed kan have en mening (økonomiske incitamenter) vi ikke ønsker. Arbejdsløse versus ledige... Arbejdsløshed er en person uden arbejde, mens ledig er en person som står til rådig på arbejdsmarkedet. Spørgsmål om skyld
%  \end{itemize}



%%%%%%%%%%%%%%%%%%%%%%%%%%%%%%%%%%%%%%%%%%%%%%%%%%%%%%%%%%%
% Trash
%%%%%%%%%%%%%%%%%%%%%%%%%%%%%%%%%%%%%%%%%%%%%%%%%%%%%%%%%%%


%%%%%%%%%%%%%%%%%%%%%%%%%%%%%%%%%%%%%%%%%%%%%%%%%%%%%%%%%%%
% \subsection{Teoriapparat}
% Vi ønsker at bruge Bourdieu i vores primære teoriapparat til at fortælle en historie om hvilke muligheder man har for at ernære sig, når man oplever arbejdsløshed, med særligt fokus på hvilken beskæftigelse man får efter en periode med ledighed. Vender man tilbage til arbejde i samme felt, eller bevæger man sig ind på et nyt? Derved vil vi diskutere, hvilken praksis der hænger sammen med hvilke felter, og hvad det siger om hvilke felter der ligger nær hinanden. Eller måske siger noget om hvor desperat man skal være, for at bevæge sig ud over det felt man er trænet ind i. Vi vil gerne diskutere hvad der strukturerer folk, der oplever arbejdsløsheds opfattelse af handlingsrum, som vi ser det komme til udtryk igennem deres praksis mellem forskellige typer af jobs efter perioder med ledighed.
% Her inddrager vi centrale økonomiske og sociologiske teorier om arbejdsløshed samt hvordan den danske arbejdsløshedsmodel historisk har udviklet sig til det den er i dag, og hvordan den ser ud i dag. Relevante forskere inden for arbejdsmarkedsforskning historisk, sociologisk og økonomisk er eksempelvis Jesper Due, Jørgen Steen Madsen, Bent Jensen, Per H. Jensen, Aage Huulgaard og Hans-Carl Jørgensen. Relevante økonomiske teorier er for eksempel søgeteorien, matching-teorien og insider-outsider-teorien udarbejdet af George Stigler, Peter Diamon, Dalte T. Mortensen, Christopher A. Pissarides, Assar Lindbeck, Dennis Snower, mv. Relevante sociologiske teorier og teoretikere kunne for eksempel være Marie Jahoda  Marienthal-studie, Philip Eisenberg og Paul Lazarfeld, Donald Tiffany, James Cowan og Phyllis Tiffany, Peter Warr, Knut Halvorsen, Barney Glaser og Anselm Strauss, Catharina Juul Kristensen og Jørgen Elm Larsen.
% Vi kan endnu ikke endnu sige i hvilken grad den nævnte teori blive anvendt. Vi vil kæde disse retninger sammen med vores primære inspirationskilde Bourdieu, eller bruge det som en baggrund for at forstå den eksisterende litteratur om arbejdsløshed til så at vise i hvilken tradition vi skriver os ind på.


%%%%%%%%%%%%%%%%%%%%%%%%%%%%%%%%%%%%%%%%%%%%%%%%%%%%%%%%%%%
% \subsection{Arbejdsløshedstal}
% I perioden 1996 til 2009 er der mellem to og otte procent nettoledige. Ind til da tog det lang tid at nedbringe ledigheden fra de 10-12 procent, som den var vokset til efter de to oliekriser i 1970’erne og de syv magre år fra 1987-1993. Fra 1994 begynder ledigheden at falde. i 1996 ligger ledigheden således på 8 procent. Efter 1996 falder ledigheden forholdsvist stabilt bortset omkring årtusindeskiftet, hvor ledigheden først er forholdsvis stabil, hvorefter den falder stiger meget lidt. I 2008 er ledigheden således faldet til to procent. Fra 2009 begynder den så at stige kraftigt (AE-rådet 2012).





% %%%%%%%%%%%%%%%%%%%%%%%%%%%%%%%%%%%%%%%%%%%%%%%%%%%%%%%%%%%
% Halvorsen - 
% (2.1) Introduktion: 
%  \begin{itemize} [topsep=6pt,itemsep=-1ex]
%    \item Dansk, europæisk og internationalt plan - de skandinaviske lande og velfærdsstaten \parencite[8-9]{Halvorsen1999}. 
%    \item Diskurser i relation til arbejde og fravær af arbejde - Lønnet arbejde har både materielt og immaterielt afkast- Arbejdsløs er en ensartet kategori og en negativt ladet kategori \parencite[12-13]{Halvorsen1999}.
%  \end{itemize}
% %%% Litteratur: Halvorsen

% \noindent (2.2) Diskurser 
%  \begin{itemize} [topsep=6pt,itemsep=-1ex]
%    \item Diskurser om arbejdsløshed: (1) Elendighedsdiskurs: arbejdsløshed som social død, (2) Beskæftigelsesdiskurs: arbejdsløshed som sløseri med ressourcer, (3) Moraldiskurs: arbejdsløshed skyldes dovenskab og manglende arbejdsmotivation - - i alle tre diskurser er arbejdsløshed et onde og lønarbejdet godt \parencite[13]{Halvorsen1999}
%    \item Diskurser om lønarbejde: (1) Integrationsargumentet: lønarbejde som selvrealisering og forudsætning for at kunne fungere som fuldstændig samfundsborger, (2) Værdiskabelsesargumentet: lønarbejde er den grundlæggende værdiskabende aktivitet, (3) Moralargumentet: lønarbejde er et nødvendigt onde, men afgørende for samfundet \parencite[13]{Halvorsen1999} - (1) og (2) kører på ret og (3) kører på pligt \parencite[13]{Halvorsen1999}.
%    \item Eksempler på diskurser: Medier \parencite[14-15]{Halvorsen1999}, Politik \parencite[15-17]{Halvorsen1999}, Befolkningen: meningsmålinger \parencite[9, 16-17]{Halvorsen1999}, Videnskabsdiscipliner \parencite[14, 17]{Halvorsen1999}.
%  \end{itemize}
% %%% Litteratur: Halvorsen, Lars Paludan-Müller, Lars Svendsen, nyhedsartikler, måske man kan smide nogle referencer ind på noget fra økonomiske og sociologiske forståelser af arbejdsløshed når det er færdigt (husk at det skal være kort) -> skriv referencer som du er i tvivl om med bold, så du kan vende tilbage til dem

% \noindent (2.3) Arbejdsløshedstematikker på statsligt og individuel plan og herinden for økonomiske og sociale spørgsmål
% %
%  \begin{itemize} [topsep=6pt,itemsep=-1ex]
%    \item Statsligt plan: (1) Økonomisk spørgsmål: beskæftigelse som produktiv kraft \parencite[11]{Halvorsen1999} - Gevinsten ved fuld beskæftigelse \parencite[10]{Halvorsen1999} kontra arbejdsløshed som spild af ressourcer \parencite[10]{Halvorsen1999}. (2) Socialt spørgsmål: Brud med princippet om ret og pligt \parencite[10]{Halvorsen1999}  (6) økonomisk og idémæssig fundament for velfærdsstatens ideologi \parencite[11]{Halvorsen1999}.
%    \item Individuel plan: (1) Økonomisk spørgsmål: økonomisk selvstændighed. (2) Socialt spørgsmål: selvrespekt og anerkendelse, socialt kontakt med andre, central og institutionaliseret basis for individets socialisering \parencite[11]{Halvorsen1999}.
%  \end{itemize} 
% %%% Litteratur: Halvorsen, Lars Paludan-Müller, måske man kan smide nogle referencer ind på noget fra økonomiske og sociologiske forståelser af arbejdsløshed når det er færdigt (husk at det skal være kort) -> skriv referencer som du er i tvivl om med bold, så du kan vende tilbage til dem

% \noindent (2.4) Arbejdsløshedens fremtrædelsesformer:
%  \begin{itemize} [topsep=6pt,itemsep=-1ex]
%    \item Forskellige definitioner af arbejdsløse fx ILO og forskellige forskere \parencite[18-19]{Halvorsen1999}
%    \item Arbejdsløse fremtræder på forskellige måder: sæsonledige, afskedige, underbeskæftigede, langtidsledige, hyppighed, vandreledighed, førstegangsledige, selvopfattelse, dagpengemodtager,  \parencite[19-20, 22-23]{Halvorsen1999}
%    \item Hvad har de arbejdsløse egentlig tilfælles ud over at de står uden lønnet arbejde. Forskellige perspektiver på det \parencite[20-21]{Halvorsen1999}
%    \item Arbejdsløse er ikke en ensartet gruppe, men en kategori sammensat af forskellige mennesker med forskellige udfordringer.  Arbejdsløshed er en institutionel konstruktion, som påvirkers af sociale ordninger og arbejdsmarkedets organisering: “The unemployed are not a group of people, but an economic and adminsitrative category” \parencite{Kelvin1985}. Det er den store udfordring i arbejdsløshedsforskningen \parencite[18]{Halvorsen1999}.
%  \end{itemize}
% %%% Litteratur: Halvorsen, måske man kan smide nogle referencer ind på noget fra økonomiske og sociologiske forståelser af arbejdsløshed når det er færdigt (husk at det skal være kort) -> skriv referencer som du er i tvivl om med bold, så du kan vende tilbage til dem

% \noindent (2.5) Konklusion:
%  \begin{itemize} [topsep=6pt,itemsep=-1ex]
%    \item Ikke bare et socialt problem, men som flere til dels adskilte sociale problemer \parencite[12]{Halvorsen1999}.
%    \item Ikke bare en ensartet gruppe, men forskellige mennesker med forskellige problemstillinger \parencite[12]{Halvorsen1999}. 
%  \end{itemize}
% %%% Litteratur: 




%%%%%%%%%%%%%%%%%%%%%%%%%%%%%%%%%%%%%%%%%%%%%%%%%%%%%%%%%%%

 % \begin{enumerate} [topsep=6pt,itemsep=-1ex]
 %   \item 
 % \end{enumerate}

% “”

% \noindent 


%Local Variables: 
%mode: latex
%TeX-master: "report"
%End: