% -*- coding: utf-8 -*-
% !TeX encoding = UTF-8
% !TeX root = ../report.tex

% socio/socstil segmentering

1.11 og 1.12
2221: Læger og 2222: tandlæger voldsomt høj grad af intern mobilitet. Klassisk

#klynger med høj densitet og intern mobilitet:

3.1, skibsfart og fiskeri. Tydelig ensartethed i genstandsfelt, godtgør nok en feltbeskrivelse af en art.

2.29: jordemoder og overordnet sygepleje mv samt sygeplejerske. Vi kan se at disse to er sammen i en klynge for sig og ikke i den store omsorgsklynge (LA's kvindefængsel). Vi har at gøre med arbejde på disco-niveau 3 og 2, det vil sige at dette arbejde trods ensheden i genstandsfelt, differentierer sig ud på grund af sit højere færdighedsniveau i form af diplomer, samt muligvis udskilnen hvor dem der har jobbet ligger nærmere grundet den type færdigheder der kræves, i modsætning til kvindefængsel-klustren. udgør 2,6 \% af alle beskæftigede.

3.11 omsorgsarbejdeklyngen meget høj intern mobilitet 65 \%, samt en densitet på 79 \%. Det vil sige at nærmest alle klynger er forbundet med nærmest alle andre klynger, således at ikke bare den interne mobilitet er meget høj, men skiftene internt kan gå på kryds og tværs. Dette udgør derfor i høj grad en form for klasse. samtidig udgør den knap 20 \% af de ledige og 13 \% af alle beskæftigede. I form af mobilitetsmønstre må man sige, at hvis man arbejder med en af disse stillinger, er man voldsomt tilbøjelig til at skifte til en af de andre indenfor klyngen. En høj grad af ensartethed indenfor socialitet bør derfor kunne findes i denne gruppe. (ja - indtil videre: kvindefag!) 

2.29 og 3.11 er sandsynligvis samme felt, men viser en tydelig stratificering indenfor feltet i form af kvalicerende uddannelsesniveauer. Jordemoder: Danmarks højeste snit. Der kommer man ikke ind bare sådan. Egentligt overraskende at sammenhængen ikke er til akademikerklyngen (jvf min ekskæreste Louise der skifter fra antropologi til jordemoder. 12-tals pigerne). 

3.5 Landbrugsklyngen. Igen: voldsomt høj densitet, udgør omtrent 2 \% af alle. vi kan se at genstandsfeltet er enormt vigtigt, kendskabet til området. I den forbindelse skal man også huske de geografiske dimensioner, den konkrete lokalisering som vi skal se på senere (R Danmarkskort)

3.13
interessant klynge. Selvom den interne mobilitet er høj er densiteten noget lavere. Stadig 60 \%, hvilket er ok godt, men noget lavere end de allerbedste klynger hvor over 3/4 af noderne er forbundet med hinanden. Ved første øjekast indeholder den også grupper, der umiddelbart må forventes at indeholde meget forskellige genstandsfelter såvel som arbejdsfunktioner. Ved at zoome ind kan vi se denne klynge er slået sammen gennem to klynger skabt på 2. niveau. Her ser vi teknisk betonet arbejde indenfor biologisk produktion, medicin og forskning, samt teknisk arbejde med kemi, fysik, astrononomi og geologi. Disse to udgør en klynge på 2. niveau sammen med  kvalitetskontrol af fastsatte standarder indenfor diverse produktionsfærer, hvilket giver fin mening. Man må forvente at kompetencer opnået i teknikerarbejde indenfor kemi, fysisik, biologi etc kan benyttes netop inden kvalitetskontrol og sikkerhed af eksempelvis kemiske standarder.

Den anden klynge på niveau to består af assitenstentarbejde indenfor en række sundshedsrelateret arbejde, samt fysioterapi, yoga, ergoterapi og række andre jobtyper, med fokus på behandling af menneskekroppen. Denne sammenlægning er ikke overraskende. Det overraskende sker i sammenlægningen af disse to klynger. Det ses at 3282 rent faktisk har stærk forbindelse til 3150, men også en nogenlunde stærk forbindelse til 3111, og en ganske svag forbindelse til 3211 (læseren bør huskes på hvad definitionen er af en forbindelse her - at der ikke eksisterer nogle barrierer i form af den forventede mulighed for at tage job når man kommer fra XXX og går til XXX). Den har på samme tid en intern mobilitet på cirka 50 \% 








ide: lav grafer ud komponenter. Hvilke komponenter indeholder dette netværk? Ret centralt. Hvor er der ihvertfald *slet* ikke forbindelser? Tjek adelsdata igennem.









vi kan også sige noget om at hvis man i perioder med arbjedsløshed ikke engang skifter, så er der virkelig tale om stærke barrierer for sammenfald.









% de mindste grupper:


1220 har også lav mobilitet, men har 3449 personer og må siges at være vigtig i sig selv. 


% de største grupper 


%
\begin{table}[H] \centering
\caption{Operationalisering af \texttt{SOCIO}/\texttt{SOCIO02}, 1996-2009. Kilde: DST}
\label{tab_socio}
\resizebox{0.9\textwidth}{!}{
\begin{tabular}{@{}l| llrrrrr@{}} \toprule
Mellemtotaler	&	Status	&	Gennemsnit	&	Gennemsnit (pct.)	&	Standardafv.	&	Min.	&	Maks.	\\	\midrule
1 -2 uger fuld ledig	&	.	&	97,261	&	19.62	&	13,524	&	80,742	&	123,197	\\	
3-4 uger fuld ledig	&	Arbejdsløs	&	53,193	&	10.73	&	5,748	&	44,521	&	62,617	\\	
5-10 uger fuld ledig	&	Arbejdsløs	&	97,279	&	19.62	&	11,556	&	73,137	&	116,719	\\	
11-20 uger fuld ledig	&	Arbejdsløs	&	104,872	&	21.15	&	17,132	&	67,225	&	136,017	\\	
21-30 uger fuld ledig	&	Arbejdsløs	&	62,102	&	12.53	&	13,642	&	36,701	&	88,447	\\	
31-40 uger fuld ledig	&	Arbejdsløs	&	38,430	&	7.75	&	10,677	&	20,143	&	58,587	\\	
41-53 uger fuld ledig	&	Arbejdsløs	&	42,622	&	8.60	&	19,754	&	15,824	&	83,483	\\	
Total	&		&	495,759	&	100.00	&	84,815	&	341,592	&	662,410	\\	\bottomrule
\end{tabular} }
\end{table}
% 

















\chapter{ANALYSE: HOVEDKORT} \label{analyse}

%%%%%%%%%%%%%%%%%%%%%%%%%%%%%%%%%%%%%%%%%%%%%%%%%%%%%%%%%%%%%%%%%%%%%%%%%%%%%%%%%%%%%%%%%%%%%%%%%%%
%%%%%%%%%%%%%%%%%%%%%%%%%%%%%%%%%%%%%%%%%%%%%%%%%%%%%%%%%%%%%%%%%%%%%%%%%%%%%%%%%%%%%%%%%%%%%%%%%%%
%%%%%%%%%%%%%%%%%%%%%%%%%%%%%%%%%%%%%%%%%%%%%%%%%%%%%%%%%%%%%%%%%%%%%%%%%%%%%%%%%%%%%%%%%%%%%%%%%%%


% note til billede på moneca-kortet: en marmorplade med kugler - hvor de lige ender er ikke tilfældigt, men grænserne er flydende, og kuglernes placering afhænger af deres vægt og de andre kuglers vægt som skabes via dybden af hulet, der afgøres af de samlede kuglers vægt. 


Kortet viser et netværk af forskellige arbejdsstillinger indelt i 150 \texttt{DISCO}-kategorier og 32 segmenter. I netværket bevæger individer sig mellem forskellige typer af arbejdsstillinger. Det sker når en person går fra at være beskæftiget i en arbejdsstilling til at være beskæftiget i en anden arbejdsstilling efter en mellemliggende periode med ledighed eller uden beskæftigelse. Arbejdsstillingerne kommer til udtryk som de 150 \texttt{DISCO}-kategorier  tager form som noder i netværket, og personernes bevægelser mellem forskellige arbejdsstillinger er det som frembringer  i netværket. Nedenstående kort giver os mulighed for at overskue beskæftigelsesmønstre for lediges bevægelser ind og ud af arbejdsmarkedet.
% 
\begin{figure}[h]
\begin{centering}
	% \caption{Mobilitetsmønstre hos ledige, 1996-2009}
	\includegraphics[width=\textwidth]{fig/metode/hovedkort.pdf}
	\label{fig_hist_beskaeftigede_allekategorier}
\end{centering}
\end{figure}

Som det fremgår af kortet findes der nogle meget store segmenter som omfatter meget forskelligt type arbejde. Her er det eksempelvis relevant at se på det store faglige segment (5.3), omsorgs-segmentet (3.14), kontor-segmentet (5.2), magister-segmentet (5.1) og salgs-segmentet (5.4), som repræsenterer de fem største segmenter i forhold til antal unikke personer, antal skift ud-og-ind på arbejdsmarkedet og antal noder (\texttt{DISCO-kategorier}). 

Det faglige segmenter er det absolut største og repræsenterer 37 noder (og dermed 37 \texttt{DISCO}-kategorier) samt indeholder 24 \% af alle disco-kategorierne. Eftersom størrelsen på kategorierne varierer ganske betragteligt, som beskrevet i afsnit \ref{fig_hist_beskaeftigede_allekategorier}, er et mere sigende mål hvor mange beskæftigede, der i gennemsnit er tale om over perioden. Her har segmentet en andel på 36 \%,hvilket svarer til 237.411 personer. De to største \texttt{DISCO}-kategorier befinder sig i dette segment, og de står sammen for 12 \% af arbejdsmarkedet for ledige. Taget i betragtning af at det næststørste segment kun har en andel på 16,5 \%, må man sige at segmentet fylder ganske meget på det danske arbejdsmarked. 

Det fremgår også af kortet, at det ikke er alle noder som samler sig med andre til større segmenter. Der er i alt 13 noder, som ligger sig for sig selv. Når en node er for sig selv, betyder det, at der ikke er en signifikant bevægelse til andre noder. For at bruge et eksempel med lægerne. Lægerne er både en node og et segment, hvilket egentlig giver meget god mening, da lægearbejdet er så specialiseret, at ingen andre faggrupper kan varetager en læges job hverken uden eller med en periode uden for beskæftigelse. Det er hvad man inden for arbejdsmarkedssegmenteringsteorien ville kalde for enten funktionel eller institutionel form for beskæftigelsesmønster. På den ene side skal man nemlig have de rette færdigheder for at have mulighed for at indtage en bestemt arbejdsstilling, og for det andet skal man egentlig også have det rette certifikat for at komme i betragtning til en bestemt arbejdsstilling \parencite[3]{TouboelLarsenJensen2013} \parencite[4]{TouboelLarsen2015}. Dette er dog ikke ensbetydende med, at lægerne ikke kunne bevæge sig mod andre type arbejdsstillinger (noder/\texttt{DISCO}kategorier) især efter en ledighedsperiode. Vi kan se, at der har været skift fra lægearbejde mod sygeplejearbejde i 43, mod undervisning paa universiteter og andre hoejere laereanstalter 15 tilfælde, mod ledelse af virksomhed faerre end 10 ansatte i 13, mod militaert arbejde i seks tilfælde, mod alment kontorarbejde i seks tilfælde og mod rengoerings- og koekkenhjaelpsarbejde i fem tilfælde. Dette skal selvfølgelig sammenlignes med de 1917 tilfælde af skift fra lægearbejde til lægearbejde. Der er altså meget få længer som vælge at skifte væk fra lægearbejdet efter en ledighedsperiode. Hovedårsagen er mest sandsynligt, at det har at gøre med at manglen på læger altid er stor og derfor ledighedsperioden stor (henvisning), hvilket betyder at selvom man er ledig i en periode vender man typisk tilbage i arbejde inden for en periode, hvilket fremgår summen af ledighedsperioder gennem hele arbejdslivet, som er særlig lav for læger se afsnit \ref{?}. Hvis det modsatte var tilfældet nemlig, at der er for mange læger i arbejdsstyrken, kunne man forestiller sig, at lægerne ville søge nye jobs og at gruppen som bevæger sig mod regnørings-- og køkkenhjælpsarbejde ville være større. Hvis vi kortlage beskæftigelsesmønstre i for eksempel Cuba, som flest antal læger pr. indbyggere i verden (henvisning), kunne det eksempelvis være tilfældet, hvilket muligvis kunne resultere i at lægerne ikke lå i et segment for sig selv.

For at illustrere kortet vil vi tage fat i en case med akademikernes fordeling på kortet.

% Toubøl, Larsen og Jensen \parencite[3]{TouboelLarsenJensen2013} \parencite[4]{TouboelLarsen2015} kan beskæftigelsesmønstre have en funktionel, en institutionel og en normativ form. Den funktionelle form opstår, når man skal have de rette færdigheder for at have mulighed for at indtage en bestemt arbejdsstilling. Den institutionelle form opstår, når man skal have det rette certifikat for at komme i betragtning til en bestemt arbejdsstilling. Og den normativ form opstår, når der ekskluderes personer med et bestemt køn eller en bestemt race fra visse arbejdsstillinger.


%Local Variables: 
%mode: latex
%TeX-master: "report"
%End: 