% -*- coding: utf-8 -*-
% !TeX encoding = UTF-8
% !TeX root = ../report.tex


%%%%%%%%%%%%%%%%%%%%%%%%%%%%%%%%%%%%%%%%%%%%%%%%%%%%%%%%%%%
\section{\textsc{Faglærte og ufaglræte} \label{cluster5.3}}
%%%%%%%%%%%%%%%%%%%%%%%%%%%%%%%%%%%%%%%%%%%%%%%%%%%%%%%%%%%

% Segment 5.3 er med sine 37 disco-kategorier det absolut største, og indeholder 24 \% af alle disco-kategorierne. Eftersom størrelsen på kategorierne varierer ganske betragteligt, som beskrevet i afsnit \ref{fig_hist_beskaeftigede_allekategorier}, er et mere sigende mål hvor mange beskæftigede, der i gennemsnit er tale om over perioden. Her har segment 5.3 en andel på 36 \%,hvilket svarer til 237.411 personer. De to største disco-kategorier befinder sig i dette segment, og de står sammen for 12 \% af arbejdsmarkedet for ledige. Taget i betragtning af at det næststørste segment kun har en andel på 16,5 \%, må man sige at segmentet fylder ganske meget på det danske arbejdsmarked. 

% Som kort XX viser i oversigten over de 1-cifrede Disco-koder, består størstedelen af segmentet af arbejder indenfor \texttt{Operatør- og monteringsarbejde smat transportarbejde}, samt \emph{Andet manuelt arbejde}. Faktisk indeholder dette segment alle disco-kategorier indenfor \emph{9: Andet manuelt arbejde} undtagen medhjælp indenfor landbrug, gartneri fiskeri og skovbrug, hvilket tæller blandt andet skovhuggere, plantagearbejdere, land-og staldarbejder og frugtplukker. Indenfor gruppen af manuelt arbejde kan vi dermed forstå karakteren af dette arbejde som så distinkt anderledes, at det åbenbart ikke ligger lige for at skifte fra det industrielle manuelle arbejde til det landlige ditto.

% At der er tale om den industrielle, manuelle klynge, bekræftiges når man ser hvilke \texttt{disco(8)}-kategorier der er inkluderet, og hvilke der ikke er. Udaf de 22 


% %husk at nævn at "andet arbejde" bare ikke har så mange udfald som mange af de andre, derfor er den sådan. skriv ind at gennemsnittet ikke er brugbart, men det er medianen tilgengæld

% %Husk at skriv Collins på om "the credential society"

% 6.43+5.51

% Det betyder at lidt over 

% 33+32+22+21+14+9+8+6+5


%noter

% brug segmenter til at finde ud af hvor mange SOC_STIL etc der er i hvert cluster

%%%%%%%%%%%%%%%%%%%%%%%%%%%%%%%%%%%%%%%%%%%%%%%%%%%%%%%%%%%
% Trash
%%%%%%%%%%%%%%%%%%%%%%%%%%%%%%%%%%%%%%%%%%%%%%%%%%%%%%%%%%%


%Local Variables: 
%mode: latex
%TeX-master: "report"
%End: 