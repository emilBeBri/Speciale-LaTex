% -*- coding: utf-8 -*-
% !TeX encoding = UTF-8
% !TeX root = ../report.tex

%%%%%%%%%%%%%%%%%%%%%%%%%%%%%%%%%%%%%%%%%%%%%%%%%%%%%%%%%%%
\newpage \section{\textsc{Arbejdsløshed \label{}}}
%%%%%%%%%%%%%%%%%%%%%%%%%%%%%%%%%%%%%%%%%%%%%%%%%%%%%%%%%%%

Metatekst.

%%%%%%%%%%%%%%%%%%%%%%%%%%%%%%%%%%%%%%%%%%%%%%%%%%%%%%%%%%%
\subsection{\textsc{Arbejdsløshed som problemstilling}}
%%%%%%%%%%%%%%%%%%%%%%%%%%%%%%%%%%%%%%%%%%%%%%%%%%%%%%%%%%%

Arbejdsløshed som socialt problem. %%% \parencite[1-12]{Halvorsen1999}
	Definition af et socialt problem: "No problem can be adequately formulated unless the values involved [...] stated" (mills1959)
	Videnskabelig historik af arbejdsløshed som et socialt problem.
	Dagens syn på arbejdsløshed.

Diskurser om arbejdsløshed. %%% \parencite[12-17]{Halvorsen1999}
	Diskurser om arbejde
	Diskurser om arbejdsløshed

Arbejdsløshedens remtrædelsesformer og sociale problemer %%% \parencite[18-23]{Halvorsen1999}


% Fokus på arbejdsmarkedsparate arbejdsløse, men værd opmærksom på, at arbejdsmarkedsparathed kan have en mening (økonomiske incitamenter) vi ikke ønsker.

% Kernen i vores teoretiske og empiriske arbejde er arbejdsløshed. Arbejdsløshed defineres i \textit{Den Store Danske} som den manglende overensstemmelse på arbejdsmarkedet mellem udbud af arbejdskraft og efterspørgsel efter arbejdskraft \parencite{2015}. I de almindelige statistiske definitioner opdeles den danske befolkning i folk der er inden for og uden for arbejdsstyrken. Dem der er inden for arbejdsstyrken er enten i beskæftigede eller arbejdsløse, mens alle andre per definition betragtes som værende uden for arbejdsstyrken \parencite{2015a}. Problemet med denne opdeling er, at personer der klassificeres som uden for arbejdsstyrken ofte kommer fra beskæftigelse og bevæger sig tilbage i beskæftigelse. De følger måske ikke standarddefinitionen på arbejdsløshed\footnote{\textit{International Labour Organization} definerer arbejdsløse som det antal personer som står uden beskæftigelse samtidig med at være til rådighed for arbejdsmarkedet og aktivt arbejdssøgende \parencite{ILO1982}. Denne definition fremgår både af \textit{Den Store Danske}, Danmarks Statistik, Beskæftigelsesministeriet med flere.}, men kan i et lidt bredere perspektiv godt betragtes som arbejdsløse. Eksempelvis kan personer på revalideringsydelse, kontanthjælp og førtidspension godt vende tilbage i beskæftigelse igen. For at få en bedre forståelse af arbejdsløses sociale mobilitet på arbejdsmarkedet må vi derfor bryde med de almindelige definitioner af arbejdsløshed\footnote{Vi er vel opmærksomme på den symbolske kamp, der ligger i at gøre dette, hvilket fremgår af de politiske diskussioner om arbejdsløse og kontanthjælpsmodtagere, der er ’skjult’ i aktiveringsforløb...}. For at bryde med standarddefinitionerne af arbejdsløshed vil først og fremmest redegøre for sociologisk-videnskabelige og økonomisk-videnskabelige tilgang til arbejdsløse og arbejdsløshed for så til sidst at anvende Bourdieu og marginaliseringsbegrebet til at lave en teoretisk operationalisering af arbejdsløshed.


% \subsubsection{Årsagerne til arbejdsløshed i Danmark ifølge Den Store Danske}

% Når en stor gruppe får sværere ved at konkurrere om de ledige jobs, vil arbejdsløsheden i mindre grad lægge pres på løndannelsesprocessen, og løntilpasningen vil således ikke kunne sikre en tilbagevenden til høj beskæftigelse. Jo længere tid arbejdsmarkedet er præget af høj arbejdsløshed, jo flere af de arbejdsløse vil blive marginaliserede, og jo større bliver den strukturelle ledighed \parencite{2015}.

% Arbejdsløshed forårsages og inddeles typisk i friktionsledighed\footnote{Friktionsledighed (også kaldet skifteledighed) opstår ofte i forbindelse med indtræden på arbejdsmarkedet eller ved jobskift, hvor en mellemliggende ledighedsperiode kan forekomme selv i perioder med høj økonomisk aktivitet \parencite{2015}.}, sæsonledighed\footnote{Sæsonledighed opstår på grund af sæsonmæssige svingninger i produktionen, som for eksempel kan skyldes vejret. En stor del af sæsonarbejdsløsheden forekommer som midlertidig hjemsendelsesledighed, idet den arbejdsløse efter en kortere periode som arbejdsløs kommer tilbage til den samme arbejdsgiver. Denne type ledighed rammer for eksempel bygge- og anlægsaktiviteten og turisterhvervene \parencite{2015}.}, konjunkturledighed\footnote{Konjunkturledighed skyldes, at den samlede efterspørgsel i samfundet efter varer og tjenester ikke er tilstrækkelig til at skabe en beskæftigelse svarende til fuld beskæftigelse \parencite{2015}.} og strukturledighed\footnote{Strukturarbejdsløshed forårsages af manglende faglig eller geografisk fleksibilitet på arbejdsmarkedet, hvorved markedsmekanismen ved hjælp af tilpasning af lønnen ikke er i stand til at sikre fuld beskæftigelse \parencite{2015}.}. I praksis er det ifølge \textit{Den Store Danske} vanskeligt at udskille friktions- og sæsonledighed fra strukturledighed, hvilket betyder, at der oftest i den økonomisk-politiske debat kun sondres mellem konjunkturarbejdsløshed og strukturarbejdsløshed \parencite{2015}.
% I praksis er det ifølge \textit{Den Store Danske} vanskeligt at udskille friktions- og sæsonledighed fra strukturledighed, hvilket betyder, at der oftest i den økonomisk-politiske debat kun sondres mellem konjunkturarbejdsløshed og strukturarbejdsløshed \parencite{2015}.


% \subsection{Arbejdsløse versus ledige}

% Arbejdsløse versus ledige... Arbejdsløshed er en person uden arbejde, mens ledig er en person som står til rådig på arbejdsmarkedet. Spørgsmål om skyld




% \section{Arbejdsløshed i en historisk rammer \label{}}

% \section{Centrale aktører og tal på arbejdsløshedsområdet \label{}}

% \subsection{Arbejdsløshedstal \label{}}
% I perioden 1996 til 2009 er der mellem to og otte procent nettoledige. Ind til da tog det lang tid at nedbringe ledigheden fra de 10-12 procent, som den var vokset til efter de to oliekriser i 1970’erne og de syv magre år fra 1987-1993. Fra 1994 begynder ledigheden at falde. i 1996 ligger ledigheden således på 8 procent. Efter 1996 falder ledigheden forholdsvist stabilt bortset omkring årtusindeskiftet, hvor ledigheden først er forholdsvis stabil, hvorefter den falder stiger meget lidt. I 2008 er ledigheden således faldet til to procent. Fra 2009 begynder den så at stige kraftigt (AE-rådet 2012).

% \subsection{Centrale aktører \label{}}
% Arbejdsløshedskasserne, staten og kommunen er centrale aktører som har spillet centrale roller i  danske arbejdsløshedsforsikringssystem siden etableringen i 1907 og er kendetegnet som at følge den såkaldte Gent-model, hvor staten anerkender og yder tilskud til arbejdsløshedskasser organiseret af forsikringstagere (i praksis fagbevægelsen), og at det for det enkelte individ er frivilligt om denne vil forsikre sig mod arbejdsløshed (Jensen 2007: 33f). Et væsentligt skifte siden 1907 skete med Wechselmann-udvalget, hvor kommunale fik mindre ansvar og staten overtog den marginale risiko ved ledighedsstigninger fra arbejdsløshedskasserne, hvilket vil sige, at staten fuldt ud betalte medudgifter ved stigende ledighed. *Arbejdsformidlingen* blev på samme tid etableret for at formidle ledige jobs og ledig arbejdskraft (Pedersen 2007:83) og eksisterede ind til den blev nedlagt med Kommunalreformen i 2007 og erstattet af de kommunale jobcentre, hvor staten og kommunen i fællesskabet samarbejder om beskæftigelsesindsatsen på lokalniveau.




%%%%%%%%%%%%%%%%%%%%%%%%%%%%%%%%%%%%%%%%%%%%%%%%%%%%%%%%%%%
\subsection{\textsc{Økonomiske forståelser af arbejdsløshedsproblemet}}
%%%%%%%%%%%%%%%%%%%%%%%%%%%%%%%%%%%%%%%%%%%%%%%%%%%%%%%%%%%

Incitatementer. %%% \parencite{Halvorsen1999}, økonomilitteratur

Arbejde vs. fritid. %%% \parencite{Halvorsen1999}, økonomilitteratur

Søgeteori. %%% \parencite{Halvorsen1999}, økonomilitteratur

Kritik. %%% \parencite{Halvorsen1999}, Granovetter, Hedstroem



% \subsubsection{Økonomi}

% Dette fokus vurderer vi hænger sammen med hvad Jørgen Goul Andersen kalder for et skifte inden for en bred strømning af økonomisk teori fra slutningen af 1970'erne, hvor velfærdsstaten blev anskuet som et middel at afbøde “markedsfejl” til i stigende grad at fokusere på “politikfejl” og “forvridninger” på markedet, hvilket kan karakteriseres som et skifte fra efterspørgsel til udbud og et skifte fra makro til mikro \parencite[19]{Andersen2003}. Vi vil i det følgende fremhæve den generelle søgeteori som i forskellige afarter dominerer den økomiske-videnskabelige debat og har stor indflydelse på policy-studier og den danske beskæftigelsespolitik.

% \subsection{Søgeteoriens forskellige søgemodeller\label{}}

% % I lærebøgerne fra økonomisk institut ved KU fremhæves xx, xx og xx's teorier om arbejdsløshed/forsikring/økonomiske incitamenter.

% Forskning inden for søgeteorien har vist, at jobsøgningsintensiteten og den løn som de er villige til at accepterer er centrale for længden af en arbejdsløshedsperiode. Det betyder, at arbejdsløshedsunderstøttelse har indflydelse på arbejdsløses adfærd og længden på af en arbejdsløshedsperiode. Martin Feldstein er en kritiker af arbejdsløshedsforsikring, fordi den får arbejdsløsheden til at stige og understøttelsen er for høj. 

% (Bailey 1978: 379).

% \subsubsection{Bailey-Chetty-betingelsen (teoretisk matchingmodel)\label{}}

% Martin Neil Bailey (1978) definerer det optimale arbejdsløshedsforsikringsystem som en afvejning mellem jobsøgningsincitament og arbejdsløshedsforsikringen.

% Den marginale gevinst af arbejdsløshedsforsikring er lig den marginale omkostning ved øget arbejdsløshed.

% konkluderer, at der er en afvejning mellem incitament til jobsøgning og arbejdsmarkedsforsikring.

% % Intro - Forsikring - Moralfare

% \subsubsection{Søgemodeller \label{}}
% Søgeteorien blev grundlagt i 1960'erne og kendte teoretikere er John J. McCall, George Stigler, Peter Diamon, Dale T. Mortensen og Christopher A. Pissarides og den basale søgemodel handler om arbejdsløse som søger beskæftigelse. Forestil en arbejdsløs som af og til får jobtilbud. Den arbejdsløse kender først lønnen på jobtilbuddet, når det modtages. Efter at have modtaget tilbuddet, skal den arbejdsløse beslutte sig for, om tilbuddet accepteres eller afslås. Dette gør den arbejdsløse på baggrund af en reservationsløn, som er fastsat af forventningerne til lønnen og viden om lønfordelingerne på arbejdsmarkedet. Hvis det modtagne tilbud er større end reservationslønnen, accepteres tilbuddet, og ellers afslås det \parencite[159f]{Rosholm2009}. Marginalisering kan i denne sammenhæng forklares med, at den arbejdsløses kvalifikationer bliver mindre værd efterhånden som ledighedsperioden bliver længere, at den arbejdsløse gradvist mister forbindelsen til gamle kolleger eller at den arbejdsløse dømmes på baggrund af sin langtidsledighed \parencite[160f]{Rosholm2009} \parencite[20]{Andersen2003}.

% \subsubsection{Partielle søgemodeller \label{}}

% \subsubsection{Matchingmodeller \label{}}
% Matching-teorien deler arbejdsmarkedet op i to typer agenter: arbejdsgivere og lønmodtagere og er kendt for teoretikere som Dale T. Mortensen, Christopher A. Pissarides Peter A. Diamond, Alvin E. Roth and Lloyd Shapley. Her leder lønmodtagerne efter job i de perioder, hvor de er arbejdsløse, mens arbejdsgivere slår stillinger op, så længe de vurderer, at det kan betale sig. Ledige lønmodtagere og job mødes på arbejdsmarkedet i en matching-proces. Når der skabes kontakt mellem en arbejdsgiver og lønmodtagere, opstår der en forhandling mellem arbejdsgiver og lønmodtager om fordelingen af overskuddet i en eventuel ansættelse. Arbejdsgiveren kan på den ene side presse lønmodtageren til at acceptere en løn, som ligger under arbejdskraftens marginalprodukt, fordi lønmodtageren ikke uden videre kan finde nyt job på anden vis end ved at vente på det næste jobtilbud. Lønmodtageren kalkulerer på baggrund af forskellen mellem den tilbudte løn og værdien af at være ledig (“outside option”). Matching-teorien kan bruges til at analyse den umiddelbare effekt af at ændre i ledighedsydelser som eksempelvis dagpenge eller kontakthjælp \parencite[162f]{Rosholm2009}

% \subsection{Insider-outsider-teorien \label{}}
% Insider-outsider-terien er udviklet af Assar Lindbeck og Dennis Snower i 1980'erne og fokuserer på omkostningerne forbundet med ansættelser og afskedigelser af arbejdskraft. Omkostningerne kan være i forbindelse med søge- og optræning, afskedigelse, uproduktiv konkurrence mellem to grupper af lønmodtagere. Insidere og outsidere kan blandt andet defineres som beskæftige og arbejdsløse, fagforeningsmedlemmer og ikke-medlemmer, ansatte i gode jobs/dårlige jobs. Insidere vil forsøge at forhandle sig til så høje lønninger som muligt og afholde andre for at underbyde dem på markedet. Insidernes magt består blandt andet i, at arbejdsgiverne har omkostninger vil at afskedige dem og kan skabe omkostninger ved strejke, aktioner og mobning af nyansatte. Langtidsledige, i modsætning til beskæftige og korttidsledige, kan tilbyde sin arbejdskraft til reduceret løn og forsøge at overbevise en arbejdsgiver om at blive ansat så længere arbejdsgiverens omkostninger ved at ansætte en outsider ikke stiger gevinsten ved at gøre det \parencite[164]{Rosholm2009} \parencite[20]{Andersen2003}.



% \chapter{LITTERATUR OM ARBEJDSLÆSE}

% %%%%%%%%%%%%%%%%%%%%%%%%%%%%%%%%%%%%%%%%%%%%%%%%%%%%%%%%%%%%%%%%%%%%%%%%%%%%%%%%%%%%%%%%%%%%%%%%

% “”

% Inspiration:
% - Changes in alternative opportunities - “...if the wage earned by pear pickers suddenly rises, some apple pickers may choose to switch occupations...” \parencite[389]{Mankiw2011}.
% - Når vi taler om arbejdsløshed/arbejdsmarkedet, så begynder økonomerne begynder med udbud, efterspørgsel og løn \parencite{Mankiw2011}.
% - Statistisk definition af arbejdsløse, uden for arbejdsstyrken mv. \parencite[584]{Mankiw2011}


% %%%%%%%%%%%%%%%%%%%%%%%%%%%%%%%%%%%%%%%%%%%%%%%%%%%%%%%%%%%%%%%%%%%%%%%%%%%%%%%%%%%%%%%%%%%%%%%%
% %%%%%%%%%%%%%%%%%%%%%%%%%%%%%%%%%%%%%%%%%%%%%%%%%%%%%%%%%%%%%%%%%%%%%%%%%%%%%%%%%%%%%%%%%%%%%%%%
% %%%%%%%%%%%%%%%%%%%%%%%%%%%%%%%%%%%%%%%%%%%%%%%%%%%%%%%%%%%%%%%%%%%%%%%%%%%%%%%%%%%%%%%%%%%%%%%%

% \subsection{Makiw og Taylor: Economics}

% - 1. semester Makiw og Taylor: Economics - part 6 The economics of Labour Markets s. 381-436 + kapitel 28 Unemployment s. 592-614, kapitel 36 The short-run trade-off between inflation and unemployment 782-814 (der er nok også noget andre steder)

% %%%%%%%%%%%%%%%%%%%%%%%%%%%%%%%%%%%%%%%%%%%%%%%%%%%%%%%%%%%%%%%%%%%%%%%%%%%%%%%%%%%%%%%%%%%%%%%%

% (1) Introduction

% (1.1) Ten Principles of Economics. How people make decisions: (1) trade-offs: efficiency/equity, (2) costs, (3) marginal changes: marginal benefits/marginal costs, (4) incentives ; How people interact: (5) trade, (6) market economy, (7) governments: promoting efficiency/equity, market failure, externality, market power ; How the economy as a whole works: (8) economic growth, standard of living/productivity, (9), inflation, (10), Phillips curve: inflation/unemployment, business cycle (s. 1-20)

% (1.2) Thinking Like an Economist. The Economist as Scientist: theory, observations, assumptions, models,	the circular flow diagram: economy simplified to include two types of decisions makers - households and firm, the production possibilities frontier, microeconomics/macroeconomics ; The Economist as Policy Advisor: positive statements/normative statements ; Why Economists Disagree (s. 21-51)

% (1.3) Interdependence and the Gains from Trade. A Parable for the Modern Economy: production, specialization, trade ; The Principle of Comparative Advantage: absolute advantage - lowest cost of production, opportunity cost, comparative advantage, trade ; Application of Comparative Advantage: import/exports (s. 52-67)


% %%%%%%%%%%%%%%%%%%%%%%%%%%%%%%%%%%%%%%%%%%%%%%%%%%%%%%%%%%%%%%%%%%%%%%%%%%%%%%%%%%%%%%%%%%%%%%%%

% (6) The Economics of Labour Markets

% (6.18) The Markets for the Factors of Production. Factors of production: labour, land and capital (s. 382-402)
% 	(A) The demand for labour: 
% 		- The competitive profit-maximizing firm: two assumptions (1) the firm is competitive and (2) the firm is profit maximizing ;
% 		- The production function and the marginal product of labour: the production function = the relationship between the quantity of inputs used to make a good and the quantity of output of that good, the marginal product of labour = the increase in the amount of output from an additional unit of labour ; 
% 		- The value of the marginal product and the demand for labour: the value of the marginal product = the marginal product of an input x the price of the output
% 		- What cause the labour demand curve to shift: the output price, technological change, the supply of other factors
% 	(B) The supply of labour
% 		- The trade-off between work and leisure
% 		- What causes the labour supply curve to shift: changes in tastes (increase in female labour), changes in alternative opportunities, immigration
% 	(C) Equilibrium in the labour market: (1) the wage adjusts to balance the supply and demand for labour and (2) the wage equals the value of the marginal product of labour
% 		- Shifts in labour supply (fewer or more workers leads to higher or lower wages)
% 		- Shifts in labour demand (more or less demand leads to more or less workers / higher or lower wages)
% 	(D) The other factors of production: land and capital
% 		- Equilibrium in the markets for land and capital: purchase price/rental price 
 
% (6.19) Earnings and Discrimination (s. 403-417)
% 	(A) Some Determinants of Equilibrium Wages
% 		- Compensating differentials = a difference in wages that arises to offset the non-monetary characteristics of different jobs (fx night shift, dangerous job, boring job etc.)
% 		- Human capital = the accumulation of investments in people such as education and on-the-job training
% 		- Ability, effort and chance
% 		- An alternative view of education - signalling: 
% 		- Above-equilibrium wages: minimum wage laws, unions and efficiency wages: 
% 	(B) The Economics of Discrimination (Det er helt fucked det her afsnit)
% 		- Measuring Labour Market Discrimination: women/ethnic groups, explanation (human capital, raising children etc., compensating differentials)
% 		- Discrimination by employers: Gary Becker's employer taste model - conclusion! employers put profit before discrimination with the example of an English employer who hires immigrant workers in stead of the local, therefore discrimination does not exist
% 		- Discrimination by customers and governments: 

% (6.20) Income Inequality and Poverty
% 	(A) The Measurement of Inequality
% 		- European income inequality
% 		- Problems in measuring inequality: the economic life cycle (income change vary over people's lives), transitory versus permanent income, economic mobility (temporarily/persistently poor), the poverty rate (poverty line, absolute/relative poverty)
% 	(B) The Political Philosophy of Redistributing Income
% 		- Utilitarianism (Bentham, Mill) = the government should choose policies to maximize the total utility of everyone in society - diminishing marginal utility
% 		- Liberalism (Rawls) = the government should aim to maximize the well-being of the worst-off person in society - maximim criterion
% 		- Libertarianism (Nozick) = the government should punish crimes and enforce voluntary agreements but not redistribute income
% 		- Libertarian paternalism (Thaler/Sunstein) = the government should nudge in the direction of improving their own and society's welfare
% 		- Policies to reduce poverty
% 		- Minimum wage laws
% 		- Social security
% 		- Negative income tax = high-income families would pay a tax based on their incomes, low-income families would receive a subsidy
% 		- In-kind transfers = transfers to the poor given in the form of goods and services rather than cash
% 		- Anti-poverty policies and work incentives


% %%%%%%%%%%%%%%%%%%%%%%%%%%%%%%%%%%%%%%%%%%%%%%%%%%%%%%%%%%%%%%%%%%%%%%%%%%%%%%%%%%%%%%%%%%%%%%%%

% (9) The Real Economy in the Long Run

% (9.28) Unemployment (s. 592-614): the long-run problem and the short-run problem, natural rate of unemployment, cyclical unemployment 
% 	(A) Identifying Unemployment
% 		- What is unemployment? someone who does not have a job, someone who does not have a job and who is available for work someone who is at the working age who are able and available for work at current wage rates and who do not have a job
% 		- How is unemployment measured?
% 			-the claimant count: novembertal, problem: governmental changes in who are entitled to unemployment benefits
% 			-labour force surveys: ILO definition, employed/unemployed/not in the labour force (or economically active), labour force = the total number of workers including the employed and the unemployed, unemployment rate = the percentage of the labour force that is unemployed, labour force participation rate (or economic active rate) = the percentage of the adult population that is in the labour force
% 		- How long are the unemployed without work? short-term/long-term
% 		- Why are there always som people unemployed? frictional unemployment = unemployment that results because it takes time for workers to search for the jobs that best suit their tastes and skills, structural unemployment = unemployment that results because the number of jobs available in some labour markets is insufficient to provide a job for everyone who wants one.
% 	(B) Job Search = the process by which workers find appropriate jobs given their tastes and skills
% 		- Why some frictional unemployment is inevitable
% 		- Public policy and job search: government policies try to facilitate job search in various ways (jobcenter, arbejdsformidlingen)
% 		- Unemployment insurance: can increase unemployment (lack of incentives) - Tatsiramos the benefits to workers searching for jobs and receiving unemployment insurrance is greater than the cost
% 	(C) Minimum Wage Laws: if the wage is kept above the equilibrium level for any reason, the result is unemployment
% 	(D) Unions and Collective Bargaining
% 		- The economics of unions: collective bargaining = the process by which uions and firms agree on the terms of employment - when a union raises the wage above the quilibrium level, it rases the quantity of labour supplied and reduces the quantitaty of labour demanded, resulting in unemployment (insiders/outsiders)
% 		- Are unions good or bad for the economy? 
% 	(E) The Theory of Efficiency Wages: firms operate more efficently if wages are above the quilibrium level - Works health - Worker turnover (workers wont leave) - Worker effort (incentive to work hard) - Worker quality (icentive to seek jobs - reservations wage)



% %%%%%%%%%%%%%%%%%%%%%%%%%%%%%%%%%%%%%%%%%%%%%%%%%%%%%%%%%%%%%%%%%%%%%%%%%%%%%%%%%%%%%%%%%%%%%%%%

% (12) Short-run Economic Fluctuations

% (12.33) Keynes and IS-LM Analysis (s. 707-725): Keynes attempt to eplain short-run economic fluctuation in general and the Great Depression in particular, and he critizes classical economic theory for only eplaing the long-run effect of policies.
% 	(A) The Keynesian Cross
% 		- Planned spending, saving or investment = the desired or intened actions of households and firms
% 		- Actual spending, saving or investment = the realized or ex port outcome resulting from actions of households and firms
% 		- Deflationary and inflationary gaps = gaps between equilibrium and full employment
% 	(B) The Multiplier Effect = the additional shifts in aggregate demand that result when expansionary fiscal policy increases income and thereby increases consumer spending (the government makes a contract for £10 billion to bud three new nuclea power plants -> raise employment and profits in the construction company -> the construction company buy ressources from other contractors -> raise employment and profits among contracters -> spending in comsumer goods -> each pound spend can raise the aggreate demand for goods and services by more than a pound.
% 	(C) The IS and LM curves: IS-LM describes equilibrium in two markets and together determines general equilibrium in the economy. General equlibrium in the economy occurs at the point where the goods market and money market are both in equlibrium at a particular interst rate and level of income. IS stands for Investment and Saving; LM stands for Liquidity and Money. The thing linking these two markets is the rate of interst (i). The IS vcurve shows the relationsship between the interst rate and the level of income (Y) n the goods market.
% 	(D) General Equilibrium Using the IS-LM Model
% 		- Fiscal policy: increase spending to boost economic activity
% 		- Monetary policy: the central bank decied to expand the money supply
% 	(E) From IS-LM to Aggregate Demand

% (12.36) The Short-run Trade-off Between Inflation and Unemployment (s. 782-815)
% 	(A) The Phillips Curve: negative corralation between the rate unemployment and the rate of inflation - low unemployment tend to have high inflation, and high unemployment tend to have low inflation. A curve that shows the short-run trade-off between inflation and unemployment

%%%%%%%%%%%%%%%%%%%%%%%%%%%%%%%%%%%%%%%%%%%%%%%%%%%%%%%%%%%%%%%%%%%%%%%%%%%%%%%%%%%%%%%%%%%%%%%%

% \subsection{Makiw: Macroeconomics}

% - 2. semester Makiw: Macroeconomics - part 1 introduction, kapitel 6 Unemploytment 159-185, Introduction to Economic Fluctuations 252-277, Aggreagatet supply and the short-run tradeoff between inflations and unemploytment 373-405 (der er nok også noget andre steder)




% \subsection{Birch Sørensen: Introducing Advanced Macroeconomics}

% - 3. semester Birch Sørensen: Introducing Advanced Macroeconomics -  kapitel 1 Macroecnomics and the long run and in the short run,  - part 4 Struktural Unemployment 275-352, kapitel 17 inflations, unemployment and aggregate supply 477-515 (der er nok også noget andre steder)





%%%%%%%%%%%%%%%%%%%%%%%%%%%%%%%%%%%%%%%%%%%%%%%%%%%%%%%%%%%
\subsection{\textsc{Sociologiske forståelser af arbejdsløshedsproblemet}}
%%%%%%%%%%%%%%%%%%%%%%%%%%%%%%%%%%%%%%%%%%%%%%%%%%%%%%%%%%%

Sociale omgivelsers tilnærmelser - funktion og normativ deprivationsteori. %%% \parencite{Halvorsen1999} 

Institutionel teori. %%% \parencite{Halvorsen1999} 

Agent Teori. %%% \parencite{Halvorsen1999} 

Handlingsteori. %%% \parencite{Halvorsen1999} 

Teorier om afhængighed. %%% \parencite{Halvorsen1999} 

Marginalisering. %%% \parencite{Halvorsen1999} 

Kritik. %%% \parencite{Halvorsen1999}, Hedström


% Det Nationale Forskningscenter for Velfærd (SFI) er storproducent af policy-studier på arbejdsmarkeds- og beskæftigelsesområdet med rapporter som typisk er bestilt af Beskæftigelesministeriet eller forskellige ministerier og kommuner. Rapporterne tager typisk udgangspunkt i en grupper af personer. Hvis man kigger på SFIs rapporter de sidste 20 år er målgrupperne hovedsageligt ledige\footnote{Også kaldet udsatte ledige, arbejdsmarkedsparate ledige, ikke-arbejdsmarkedsparate ledige, langtidsledige og forsikrede ledige}, dagpengemodtagere\footnote{Også kaldet sygedagpengemodtagere og aktiverede dagpengemodtagere.}, kontanthjælpsmodtagere\footnote{Også kaldet ikke arbejdsmarkedsparatemodtagere, de svageste kontanthjælpsmodtagere og aktiverede kontanthjælpsmodtagere.}, sygemeldte og arbejdsskadede\footnote{Også kaldet skadeslidte beskæftigede og personer som har nedsat arbejdsevne efter en ulykke i fritiden.} samt pensionister og efterlønsmodtagere\footnote{Der er også lavet en hel del undersøgelse om handicappede, indvandrere, efterkommere, mænd, kvinder, ældre og højtuddannede.}. Fokus handler hovedsageligt om at få dem i beskæftigelse eksempelvis ved at bringe de langtidsledige tættere på arbejdsmarkedet i \textit{Tættere på arbejdsmarkedet} (2011), ved at måle beskæftigelseseffekten af dagpengeophør i \textit{Dagpengemodtagers situation omkring dagpengeophør} (2014), ved at kigge på indsatser over for ikke arbejdsmarkedsparate kontanthjælpsmotagere i \textit{Veje til beskæftigelse} (2010), ved at måle effekten af den beskæftigelesrettede indsats for sygemeldte i \textit{Effekten af den beskæftigelsesrettede indsats for sygemeldte} (2012) eller ved at kigge på pensionisters og efterlønsmodtageres genindtræden op arbejdsmarkedet i \textit{Pensionisters og efterlønsmodtageres arbejdskraftspotentiale} (2012).

% Det som er kendetegnede ved denne typer rapporter er et grundlæggende fokus på at få de pågældende personer tilbage i beskæftigelse hvad man kan gøre og ikke hvad deres situation egentlig betyder for deres liv\footnote{Der skal ikke menes med, at disse rapporter slet ikke forholder sig til de pågældende personers liv. I \textit{Veje til Beskæftigelse} (2010) fortælles der igennem 30 kvalitative interviews med sagsbehandlere, at de oplever de ikke-arbejdsmarkedsparate kontanthjælpsmodtagere som værende en heterogen gruppe som har gavn af forskellige typer indsatser alt efter, hvilke udfordringer de har. Nogle har for eksempel helbredsproblemer, mens andre har brug for hjælp til daglige gøremål.}.


% \subsubsection{Jahoda, Lazarsfeld og Zeizels Marienthal-studie} %%% (Nørup:23)
% Marienthal er et klassisk studie af de sociale konsekvenser af arbejdsløshed i et lille samfund gennemført af Marie Jahoda i samarbejde med Paul Lazarsfeld og Hans Zeizel (1971). Marienthal var et industriby som led af høj arbejdsløshed i 1920'erne, og studiet undersøger hvad der sker med arbejderne i den østrigske by Marienthal, når de oplever arbejdsløshed. Med Marienthal udvikler Jahoda deprivationsperspektivet, som er det mest udbredte perspektiv i de teoretiske diskussioner af arbejdsløshed og eksklusion fra arbejdsmarkedet (Creed og Macintyre:2001). Hovedargumentet er, at arbejdsløshed medfører social eksklusion og isolation, tab af struktur i hverdagen og selvtillid og en betydelig øget risiko for psykiske problemer (Jahoda, 1981, Jahoda m.fl. 1997). Marienthal baserer sig på en grundlæggende antagelse om arbejde deltagelse på arbejdsmarkedet opfylder både et psykologisk behov for individet og et økonomisk behov for indtægt. Jahoda opstiller ikke knækket vilje, resignation, fortvivlelse og apati som fire stadier eller reaktioner den arbejdsløse gennemgår (Jahoda 1979, Jahoda m.fl. 1997). Arbejdsdeltagelsen har ifølge Jahoda fem funktioner: tidsmæssig struktur i dagligdagen, sociale kontakter, deltagelse i kollektive formål, status og identitet og regelmæssig aktivitet (Jahoda, 1981, Jahoda m.fl. 1997).
% Nørup kritiserer brugen af deprivationsperspektivet i dansk regi, fordi det danske samfund i dag er en moderne velfærdstat med relativt højtuddannet arbejdskraft, og Marienthal er en mindre industriby med lavt uddannet arbejdskraft i et 1930'ernes Østrig som ikke er i nærheden af et velfærdssamfund (Nørup2012:34).

% \subsubsection{Eisenberg og Lazarsfeld} %%% (Nørup:24)
% I *The psychological effects of unemployment* (1938) konkluderer Philip Eisenberg og Paul Lazarfeld, at arbejdsløse gennemlever tre stadier: “First there is shock, which is followed by an active hunt for a job, during which the individual is still optimistic and unresigned; he still maintains an unbroken attitude. Second, when all efforts fail, the individual becomes pessimistic, anxious, and suffers active distress; this is the most crucial state of all. And third, the individual becomes fatalistic and adapts himself to his new state but with a narrower scope He now has a broken attitude.” (Eisenberg og Lazarsfeld 1938:378).
% Studiet har vundet udbredelse inden for socialpsykologien (Boyd 2014, Wang og Greenwood 2014, Kahn 2013, Ezzy 1993, Ragland-Sulivan og Barglow 1981, Finley og Lee 1981, Hayes og Nutman 1981, Hill 1978, Briar 1977, Harison 1976). Men studiet er også blevet kritiseret på baggrund af en problematisk og modsætningsfuld metode (Fryer 1985, Ezzy 1993) og på baggrund af det empiriske fundament i særdeleshed vedrørende psykologiske faktorer som eksempelvis mentalt helbred og selværd (Fryer 1985, Hartley 2011, Shamir 1986). (Nørup kalder det for *Stadie Model*)

% \subsubsection{Tiffany, Cowan og Tiffany} %%% (Nørup:25)
% Hovedargumentet i studiet *The unemployed: A social-psychological portrait*  af Donald Tiffany, James Cowan og Phyllis Tiffany (1970) er, at majoriteten af arbejdsløse og ekskluderede fra arbejdsmarkedet står uden for arbejdsmarkedet på grund af psykologiske problemer. Sammenhængen mellem arbejdsløshed og psykologiske problemer går derfor begge veje, hvilket betyder, at psykologiske kan være årsagen til arbejdsløshed på samme tid med, at arbejdsløshed i sig selv også medfører psykologiske problemer: ”They show avoidance behaviour patterns or what has been referred to as ”work inhibition” which implies that they are physically capable of work but prevented from work because of psychological disabilities” (Tiffany, Cowan and Tiffany, 1970). Ifølge Tiffany, Cowan og Tiffany er løsningen, at staten rehabiliterer disse arbejdsløse, så de kan komme tilbage på arbejdsmarkedet gennem træning eller terapi (Tiffany, Cowan and Tiffany: 1970). Douglas Ezzy peger på, at denne tilgang har ligheden mellem den historiske distinktion mellem *deserving poor*, som fysisk var ude af stand til at arbejde og fortjente støtte og *non-deserving poor*, som ikke arbejde selvom de var i fysisk stand til at arbejde (Ezzy 1993). Ezzy påpeger ligeledes på, at tilgangen har været mest toneangivende i perioder med højkonjunktur og relativ lav ledighed til sammenligning med perioder med lavkonjunktur og lav ledig (Ezzy 1993).
% Perspektivet kritiseres for at have lighedstræk med den neoklassiske økonomiske betragtning af arbejdsløshed som frivilligt og derfor ved at skyde skylden på ofret (Miles:1987) (Nørup kalder det for *Rehabiliteringstilgangen*)

% \subsubsection{Warr} %%% (Nørup:25) 
% I *Work, Unemployment and Mental Health* opstiller Peter Warr ni faktorer i omgivelser som har betydning for det mentale helbred i forbindelse med arbejdsløshed: mulighed for kontrol, mulighed for at benytte erhvervede færdigheder, eksternt genererede mål, variation, klarhed i forhold til omgivelserne, penge og  indtjening, fysisk sikkerhed, mulighed for social kontakt og social position (Warr:1987). Individets mentale helbred afspejler det akkumulerede niveau af faktorerne, så det at miste et arbejde eller det at have et dårligt arbejde i omgivelserne afspejler individets mentale heldbred (Ezzy:1993). (Nørup kalder det for *vitaminmodellen*)

% \subsubsection{Halvorsens teori om mestring} %%% (Nørup:28)
% Knut Halvorsen udvikler i sit forfatterskab en modpol til Jahoda, Lazarsfeld og Zeizel (1971), Eisenberg og Lazarsfeld (1938), Tiffany, Cowan og Tiffany (1970) og Warrs (1987) passive og ensartede individperspektiv ved at betragte arbejdsløse som forskelligartede og handlende. De arbejdsløse anskues derfor som aktive aktører, der kan påvirke og forandre deres situation i stedet for at være ofre for omstændighederne (Halvorsen 1994, 1999). Den arbejdsløse vil indgå i forskellige fysiske, psykiske og sociale aktiviteter for at afbøde effekter af arbejdsløshed og minimere stress og mental belastning (Halvorsen: 1999, Fryer og Fagan:1993, Fryer:1986, Fryer og Payne:1984, O’Brien:1985). Halvorsen skelner mellem den problemorienterede mestring, som udgøres af konkrete strategi med formål at fjerne belastningen af den marginaliserede position (for eksempel jobsøgning) og den emotionsorienterede mestring, som handler om hvordan man ser verdenen (Halvorsen 1999)

% Nørup kritiserer Halvorsen for i sin ontologiske individualisme at fokuserer på, hvordan individet mestrer bestemte livssituationer under bestemte rammer (graden af eksklusion forklares som et resultat af individuelle handlinger og ressourcer) frem for på hvordan de samfundsmæssige strukturer og sociale relationer påvirker eksklusionen (Nørup 2012:37).

% \subsubsection{Glaser og Strauss’ teori om sociale passager \label{}}
% Barney Glaser og Anselm Strauss definerer en status passage som et individs ”movement into a different part of a social structure, or loss or gain of privilige, influence, or power, and changed behaviour”. Ezzy beskriver anvendelsen af teorien på arbejdsløshed og exit fra arbejdsmarkedet, som processer frem for enten-eller tilstande. Hermed kan exit fra arbejdsmarkedet sammenlignes med andre status passage som eksempelvis skilsmisse, sygdom eller dødsfald i familien (Ezzy:1993). Van Gennep benytter separation (begravelse), transition (overgangsfase mellem to jobs) og integration (ægteskab) som kategorier for sociale passager (Ezzy:1993, Van Gennep 1977). Ezzy identificerer exit fra arbejdsmarkedet som tab af job som en afhændelespassage i modsætning til exit fra arbejdsmarkedet som indtræden i uddannelsessystemet som noget helt andet (Ezzy 1993). En afhændelsespassager fører ikke nødvendigvis sig selv til mentale problemer, mistrivsel eller eksklusion, da det afhænger af den enkeltes identitet og selvopfattelse i relation til andre og samfundet og en lang række andre faktorer samt om afhændelsespassagen efterfølges af en reintegrativ passage, hvor den enkelte får en ny status eksempelvis et nyt job (Ezzy 1993).

% Kommentarer
% Deprivation (Jahoda, Lazarsfeld og Zeizel 1971; Eisenberg og Lazarsfeld 1938; Tiffany, Cowan og Tiffany 1970; Warrs (987) er delvist brugbart i forståelsen af arbejdsløshed som havende en social og psykologisk påvirkvning på arbejdsløse. Halvorsen er delvis brugbar i forståelsen af at individerne bliver aktører som kan handle og har ressourcer. Social passage (Glaser og Strauss 1971)er relevant i vores definition af arbejdsløshed som midlertidigt.






% \section{Marginalisering \label{}}

% For at forstå de arbejdsløses bevægelser i tid og rum, anvender vi marginaliseringsbegrebet, hvor vi trækker på Jørgen Elms Larsens perspektiver om marginalisering som en midterkategori mellem inklusion og eksklusion. Larsen definerer eksklusion som en ufrivillig ikke-deltagelse gennem forskellige typer af udelukkelsesmekanismer og -processer, som det ligger uden for indvidets og gruppens muligheder at få kontrol over \parencite[137]{Larsen2009}. Larsen er kritisk over for Luhmanns inddeling inklusion og eksklusion, som han i sin binære form ikke mener er særlig hensigtsmæssig i forhold til at beskrive virkeligheden \parencite[130]{Larsen2009}. Derfor argumenter han for, at marginalisering kan anvendes som en midtergruppe mellem de to, hvor individet bevæger sig i en proces mod inklusion eller eksklusion.

% Hvis vi benytter os af de førnævnte statistiske definitioner, som fortæller, at beskæftigede og arbejdsløse er en del af arbejdsstyrken og resten af befolkningen står uden for arbejdsstyrken, ser vi to processer, hvilket fremgår af den model\footnote{Modellen er også inspireret af lignende modeller benyttet af Lars Svedberg \parencite[44]{Svedberg1995} og Catharina Juul Kristensen \parencite[18]{Kristensen1999}.}, vi har udviklet i tabel \ref{tab_marginaliseringsmodel_1}. 

% Den \textbf{første} proces består af individer som går fra at være inkluderet til at indgå i en proces i retning mod marginalisering. Et eksempel her på kunne for eksempel være en person som går fra at være beskæftiget som sygeplejerske til at være arbejdsløs. 

% Den \textbf{anden} proces består af individer som går fra at være marginaliseret til at indgå i en proces i retnng mod inklusion. Et eksempel her på kunne for eksempel være den samme person fra foregående eksempel som går fra at være arbejdsløs til at blive beskæftiget som sygeplejerske igen.
% %
% \begin{table}[H] \centering
% \caption{Model over marginalisering 1}
% \label{tab_marginaliseringsmodel_1}
% \begin{tabular}{@{} m{5,9cm} m{5,9cm} @{}} \toprule
% \textbf{Inkluderet} & \textbf{Marginaliseret} \\ \midrule
%   beskæftiget & arbejdsløs \\  
% \end{tabular} \end{table} %%%
% \begin{table}[H] \centering
% \begin{tabular}{@{} m{12,3cm} @{}} 
%   \textbf{Marginaliseringsproces} \\  
%   <--------------------------------------------------------------------------------------------- \\
% \end{tabular} \end{table} %%%
% \begin{table}[H] \centering
% \begin{tabular}{@{} m{12,3cm} @{}} 
%   \textbf{Inklusionsproces} \\  
%   <--------------------------------------------------------------------------------------------- \\ \bottomrule
% \end{tabular} \end{table}
% %
% Modellen, som fremgår af tabel \ref{tab_marginaliseringsmodel_1}, ser bort fra dem som står uden for arbejdsstyrken, selvom der reelt er udveksling mellem arbejdsstyrken og dem uden for arbejdsstyrken. For eksemepl kan en sygeplejestuderende gå fra at stå uden for arbejdsstyrken til at være en del af arbejdsstyrken enten som arbejdsløs eller i beskæftigelse som sygeplejerske. Et andet eksempel er en forhenværende sygeplejerske som er folkepensionist og derved står uden for arbejdsstyrken, men som genindtræder i beskæftigelse som sygeplejeske. Hermed opstår to nye processer som fremgår af den nye model i tabel \ref{tab_marginaliseringsmodel_2}, som indeholder alle fire processer. Den \textbf{tredje} proces består af individer som går fra at være ekskluderet, i kraft af at stå uden for arbejdsstyrken, til at indgå i en proces i retning mod retning mod marginalisering ved at blive arbejdsløs. Den \textbf{fjerde} proces består ligeledes af individer som går fra at være ekskluderet, i kraft af at stå uden for arbejdsstyrken, men som kommer til at indgå i en proces i retning mod retning mod inklusion ved at komme i beskæftigelse.
% % 
% \begin{table}[H] \centering
% \caption{Model over marginalisering 2}
% \label{tab_marginaliseringsmodel_2}
% \begin{tabular}{@{} m{3,2cm} c m{3,5cm} c m{4cm} @{}} \toprule
% \textbf{Inkluderet} & & \textbf{Marginaliseret} & & \textbf{Ekskluderet} \\ \midrule
%   beskæftiget  & & arbejdsløs & & udenfor arbejdsstyrken \\  
% \end{tabular} \end{table} %%%
% \begin{table}[H] \centering
% \begin{tabular}{@{} m{5,9cm} m{5,9cm} @{}} 
%   \textbf{Marginaliseringsproces} & \textbf{Eksklusionsproces} \\  
%   --------------------------------------------> & --------------------------------------------> \\ 
% \end{tabular} \end{table} %%%
% \begin{table}[H] \centering
% \begin{tabular}{@{} m{12,3cm} @{}} 
%   \textbf{Inklusionsproces} \\  
%   <--------------------------------------------------------------------------------------------- \\ \bottomrule
% \end{tabular} \end{table}
% %

%   % beskæftiget  & & “midlertidigt” uden beskæftigelse & & vender ikke tilbage i beskæftigelse \\  

% Vores model viser inklusion, marginalisering og eksklusion på arbejdsmarkedet i et spektrum mellem inkluderet og ekskluderet. 

% Formålet med modellen er ikke at komme med en ny model for arbejdsmarkedet som alternativ til andre måde at anskue arbejdsmarkedet på, men at anvende en model som åbner op for forskellige måde at anskue arbejdsløse på og have en model som indeholder alle i den danske befolkning. Med inkluderet på arbejdsmarkedet forstår vi ingen eller kort afstand til arbejdsmarkedet og med ekskluderet forstår vi en stor afstand til arbejdsmarkedet. Idealtypen på de inkluderede er fuldtidsansatte i faste og sikre stillinger\footnote{Guy Standing fremhæver i \textit{The Precariat, The New Dangerous Class} prekariatet som atypisk beskæftigelse som er mindre sikkert end fastansættelser \parencite{Standing2011}.}. De ekskluderede som forbliver ekskluderet er dem som aldrig bliver en del af eller vender tilbage på arbejdsmarkedet. I et spektrum mellem det at være inkluderet og ekskluderet ligger forskellige grader af marginalisering. En nyuddannet farmaceut som er i løntilskud i en farmaceutstilling, hvor vedkommende er blevet lovet beskæftigelse efter løntilskudsstillingens ophør er derfor tættere på at være inkluderet end en nyuddannet farmaceut som efter halvandet år på dagpenge endnu ikke er kommet til samtale til nogle af de farmaceutstillinger vedkommende har søgt.

% Den sidstnævnte farmaceut som ikke har held med ansøgningerne, vil efter dagpengereformen i 2010 miste retten til dagpenge efter to år, hvilket betyder, at vedkommende har et halvt års dagpenge igen. I den sammenhæng har vi en antagelse om at længden på arbejdsløshedsperioden (eller med andre ord processen mod inklusion eller eksklusion alt efter hvordan det går) spiller en rolle for, hvilke strategier vedkommende har for at vende tilbage i beskæftigelse\footnote{Denne antagelse bakkes især op økonomiske teorier og policy-studier som vil blive behandlet i dette kapitel}. Vedkommende kan eksempelvis vælge at søge videre i farmaceutstillinger, søge bredere i ud stillinger som ikke er farmaceutstillinger, men ligger tæt på farmaceut eller som kræver akademiske kompetencer eller søge stillinger som intet har at gøre med farmaceutarbejde eller akademisk arbejde. Her bliver det et spørgsmål om vedkommende skal vente til at få det “rette arbejde” (hvad end det vil sige) eller “bare” få et eller andet arbejde. Dette bliver i høj grad et spørgsmål mellem arbejdet som en nødvendighed eller arbejdsmarkedets doxa om at arbejde ikke bare er noget man varetager af nødvendighed, men fordi man synes, at det er meningsfyldt\footnote{Ifølge Lars Svendsen findes der inden for europæisk idéhistorie to grundlæggende forskellige opfattelser af hvad arbejde er for en størrelse. Op til reformationen blev arbejde anset som en “meningsløs forbandelse”, og efter reformationen blev arbejdet anset som et “meningsfuldt kald” parencite[13]{Svendsen2010}, hvilket Bourdieu ville kalde et udtryk for arbejdslivets illusio, hvilket vil sige, at troen på at arbejdet er en vigtig del af livet.}.

% Arbejdets værdi og individets strategier for at komme i beskæftigelse er centralt i de to efterfølgende afsnit om den sociologisk-videnskabelige og økonomisk-videnskabelige tilgang til arbejdsløse og arbejdsløshed. Det at komme i beskæftigelse igen er ikke bare at komme i beskæftigelse igen. For farmaceuten er det noget andet at komme i beskæftigelse som farmaceut, som vedkommende har taget en lang videregående uddannelse for at have kompetencer til end at tage et arbeje som kassemedarbejder. Bourdieu skelner mellem “den store elendighed” og “den lille elendighed” \parencite[4]{Bourdieu1999}. Det at komme i beskæftigelse er et udtryk for at komme ud af  “den store elendighed” som arbejdsløsheden og alle de problemer som følger det at være arbejdsløs, men det at komme i beskæftigelse kan også være et udtryk for at vælge den “den lille elendighed”, som for består af forringede arbejdsvilkår og det at blive tvunget til at ofre sig lidt før man bliver ofret og samtidig være taknemmelig over, at man ikke hører til blandt de allersvageste. Hermed tilpasser farmaceuten sig arbejdsmarkedets umiddelbare behov ved i kraft af at være arbejdsløs accepterer samfundets objektive strukturer og derved får nogle realistiske forventninger i forhold til vedkommendes position i den sociale verden, hvor vedkommende som arbejdsløs i sidste ende må tage det arbejde hvad vedkommende kan få.




%%%%%%%%%%%%%%%%%%%%%%%%%%%%%%%%%%%%%%%%%%%%%%%%%%%%%%%%%%%
\subsection{\textsc{Opsummering}}
%%%%%%%%%%%%%%%%%%%%%%%%%%%%%%%%%%%%%%%%%%%%%%%%%%%%%%%%%%%

Marginalisering. %%% Halvorsen 

Incitamenter og arbejdsvilje. %%% Halvorsen 

Psykiske plager og selvrespekt. %%% Halvorsen 

Økonomiske vanskeligheder og det offentlige og private sikkerhedsnet. %%% Halvorsen 



% Konklusioner:
% %
%  \begin{itemize} [topsep=6pt,itemsep=-1ex]
%    \item Marginaliseringbegrebet åbner op for arbejdsløshed i forhold til at inkluderer dem som står uden for arbejdsstyrken.
%    \item Den økonomiske gennemgang bidrager med at vise den dominerende perspektiv på arbejdsløse. Fokus ligger på at få folk i beskæftigelse (fra marginalisering til inklusion).
%    \item Den sociologiske og socialpsykologiske gennemgang bidrager med at få et indblik i de arbejdsløses vilkår. Fokus på arbejdsløshed (marginalisering og eksklusion).
%    \item Flere måde at anskue de arbejdsløse på som helhed, det vil sige alle eller som dele, hvilket både kan være i grupper (for eksempel ledige, langtidsledige og kontanthjælpsmodtagere) eller som årsag (friktionsledighed, konjunkturledighed, strukturledighed og sæsonledighed).
%    \item Blik for uden for arbejdsstyrken, hvilket vil sige studerende, efterløn, pensionister mv.
%    \item Vores definition af arbejdsløse er så bred som overhovedet muligt. Med arbejdsløs har vi som udgangspunkt den bredeste definition overhovedet, hvilket er det at stå uden arbejde. 
%    \item Den teoretiske pointe er at arbejdsløshed defineres og behandles forskelligt alt efter om det er økonomer, sociologer mv.. Vores fokus ligger i forlængelse af marginaliseringsbegrebet \parencite{Larsen2009} samt Bourdieus perspektiver om at være placeret et specifikt sted i det sociale rum og at være på kanten af arbejdsmarkedet.
%    \item Vi mangler noget litteratur om arbejdsstyrken - måske det skal ind over sociologerne.
%  \end{itemize}
% %



%%%%%%%%%%%%%%%%%%%%%%%%%%%%%%%%%%%%%%%%%%%%%%%%%%%%%%%%%%%
% Trash
%%%%%%%%%%%%%%%%%%%%%%%%%%%%%%%%%%%%%%%%%%%%%%%%%%%%%%%%%%%

% Vi ønsker at bruge Bourdieu i vores primære teoriapparat til at fortælle en historie om hvilke muligheder man har for at ernære sig, når man oplever arbejdsløshed, med særligt fokus på hvilken beskæftigelse man får efter en periode med ledighed. Vender man tilbage til arbejde i samme felt, eller bevæger man sig ind på et nyt? Derved vil vi diskutere, hvilken praksis der hænger sammen med hvilke felter, og hvad det siger om hvilke felter der ligger nær hinanden. Eller måske siger noget om hvor desperat man skal være, for at bevæge sig ud over det felt man er trænet ind i. Vi vil gerne diskutere hvad der strukturerer folk, der oplever arbejdsløsheds opfattelse af handlingsrum, som vi ser det komme til udtryk igennem deres praksis mellem forskellige typer af jobs efter perioder med ledighed.

% Her inddrager vi centrale økonomiske og sociologiske teorier om arbejdsløshed samt hvordan den danske arbejdsløshedsmodel historisk har udviklet sig til det den er i dag, og hvordan den ser ud i dag. Relevante forskere inden for arbejdsmarkedsforskning historisk, sociologisk og økonomisk er eksempelvis Jesper Due, Jørgen Steen Madsen, Bent Jensen, Per H. Jensen, Aage Huulgaard og Hans-Carl Jørgensen. Relevante økonomiske teorier er for eksempel søgeteorien, matching-teorien og insider-outsider-teorien udarbejdet af George Stigler, Peter Diamon, Dalte T. Mortensen, Christopher A. Pissarides, Assar Lindbeck, Dennis Snower, mv. Relevante sociologiske teorier og teoretikere kunne for eksempel være Marie Jahoda  Marienthal-studie, Philip Eisenberg og Paul Lazarfeld, Donald Tiffany, James Cowan og Phyllis Tiffany, Peter Warr, Knut Halvorsen, Barney Glaser og Anselm Strauss, Catharina Juul Kristensen og Jørgen Elm Larsen.

% Vi kan endnu ikke endnu sige i hvilken grad den nævnte teori blive anvendt. Vi vil kæde disse retninger sammen med vores primære inspirationskilde Bourdieu, eller bruge det som en baggrund for at forstå den eksisterende litteratur om arbejdsløshed til så at vise i hvilken tradition vi skriver os ind på.


%%%%%%%%%%%%%%%%%%%%%%%%%%%%%%%%%%%%%%%%%%%%%%%%%%%%%%%%%%%



%Local Variables: 
%mode: latex
%TeX-master: "report"
%End: