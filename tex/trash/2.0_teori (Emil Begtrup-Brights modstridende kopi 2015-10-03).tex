% -*- coding: utf-8 -*-
% !TeX encoding = UTF-8
% !TeX root = ../report.tex

%%%%%%%%%%%%%%%%%%%%%%%%%%%%%%%%%%%%%%%%%%%%%%%%%%%%%%%%%%%
\chapter{\textsc{Teori}} \label{teori}
%%%%%%%%%%%%%%%%%%%%%%%%%%%%%%%%%%%%%%%%%%%%%%%%%%%%%%%%%%%
\textbf{Mobilitet kontra jobsøgning - Forklaringer på bevægelser ind og ud af beskæftigelse}
%%%%%%%%%%%%%%%%%%%%%%%%%%%%%%%%%%%%%%%%%%%%%%%%%%%%%%%%%%%

Formålet med dette kapitel er at skitsere vores teori med henblik på at vise vores teoretiske afsæt, og hvordan det adskiller sig fra resten af forskningen %%%% Jens: Måske var det bedre at skitsere hvordan det placerer sig i forhold til anden forskning. Det handler om hvor I adksiller jer fra anden forskning og hvad I bidrager med til forskningen. 
, som ligger inden for vores område. I kapitlets første hovedafsnit vil vi beskrive arbejdsløshedsområdet i forhold til at kontekstualisere arbejdsløshed som problemstilling og som forskningsfelt. %%%% Jens: Denne sætning gør mig ikke klogere på hvad der skal ske i første afsnit.
Andet hovedafsnit beskriver mobilitet som et forskningsfelt for sig selv og i kombination med den viden vi har om arbejdsløshedsområdet. Tredje hovedafsnit anvender Bourdieu med henblik på at give et kritisk blik på de to første hovedafsnit og vise, at vores primær teoretiske afsæt har afsæt i Bourdieu. %%%% Søren: Bourdieu - kritik af arbejdsløshedsfeltet og mobilitetsfeltet. Kritisk praksis felt. Kritik af ømonomisme og sociologiske kategorier. Præsentere et perspektiv som forsøger realistisk at se hvordan folk er placeret i en social struktur.
Kapitlets fjerde og sidste hovedafsnit samler op på de tre foregående afsnit og viser vores teoretiske operationalisering af de arbejdsløses jobmobilitet på arbejdsmarkedet, og hvordan denne operationalisering adskiller sig fra og trækker på forskningen på området. %%%% Søren: Teoretiske operationalisering og anvendt marginalisering/økonomiske beskæftigelsesmodeller mv. Vise feltet og kridte banen op, så vi kan vise styrker og svagheder på området og i vores operationalisering med henblik på en diskusion af vores resultater i forhold til andre undersøgelser på et senere tidspunkt.

Vores udgangspunkt er overordnet en datadrevet model og ikke så meget en teoretisk model. Vi trækker dog på teoretikere som Bourdieu, Toubøl og Larsen, arbejdsmarkedssegmenteringsteorien. Målet er at anvende en anden metode en økonomerne til at vise, at overgange fra arbejdsløsehed til beskæftigelse ikke bare er økonomiske incitamtenter og reservertionsløn. Vores metode er klar anderledes en den økonomiske, fordi vi er sociologer %%% Jens: Hvem er vores helt? Det handler om er vise, hvor I får redskaberne til at tænke på en anden måde fx Bourdieu, Granovetter, Jonas/Anton, segmenteringsteori. I læner jeg op af en datadrevet model - ikke en teori. Vi kigger på det på en anden metode, fordi vi er sociologer.

\begin{table}[h] 
\begin{tabular}{@{}||l||l||r||@{}} \hline \hline 
 2.	 & \textbf{Teori} 								& 1/1 sider \\ \hline \hline 
 2.1 & \textbf{Arbejdsløshed som problemstilling} 	& 8/5 sider \\ \hline \hline 
 2.2 & \textbf{Jobsøgningsteori} 					& 4/5 sider \\ \hline \hline 
 2.3 & \textbf{Sociologisk teori} 					& 7/5 sider \\ \hline \hline 
 2.4 & \textbf{Arbejdsmarkedssegmenteringsteori} 	& 6/5 sider \\ \hline \hline 
 2.5 & \textbf{Bourdieu} 							& 0/5 sider \\ \hline \hline 
 2.6 & \textbf{Opsamling} 							& 0/2 sider \\ \hline \hline 
 	 &												& 28/28 sider \\ \hline \hline 
\end{tabular} \end{table}


Litteratur som gerne må læses ifm. med dette afsnit: 
%
 \begin{enumerate} [topsep=6pt,itemsep=-1ex]
   \item Sociologi: (Dencker Larsen 2014, s. 17-38), \parencite{Baum2006}
   \item Økonomi: \parencite{Clement2006}, \parencite{Cahuc2004}
   \item Segmentering: 
   \item Ny arbejdsløshedsforskning (ev. Web of Science)
 \end{enumerate}

Cahuc
% 
 \begin{itemize} [topsep=6pt,itemsep=-1ex]
   \item PART 1: SUPPLY AND DEMAND BEHAVIORS
   \item \textit{Chapter 1: Labor supply \parencite[1-58]{Cahuc2004}}
   \item \textit{Chapter 2: Education and human capital \parencite[59-108]{Cahuc2004}}
   \item \textit{Chapter 3: Job search \parencite[109-170]{Cahuc2004}}
   \item \textit{Chapter 4: Labor demand \parencite[171-242]{Cahuc2004}}
   \item PART 2: WAGE FORMATIONS
   \item Chapter 5: Compensating differentials and discrimination \parencite[243-304]{Cahuc2004}
   \item Chapter 6: Contracts, risk-sharing, and incentive \parencite[305-368]{Cahuc2004}
   \item Chapter 7: Collective bargaining \parencite[369-440]{Cahuc2004}
   \item PART 3: UNEMPLOYMENT AND INEQUALITY
   \item \textit{Chapter 8: Unemployment and inflation \parencite[441-502]{Cahuc2004}}
   \item \textbf{Chapter 9: Job reallocation and unemployment \parencite[503-562]{Cahuc2004}}
   \item Chapter 10: Technological progress, globalozation, and inequalities \parencite[563-632]{Cahuc2004}
   \item PART 4: INSTITUTIONS AND ECONOMIC POLICY
   \item Chapter 10: Labor market policies \parencite[633-712]{Cahuc2004}
   \item \textbf{Chapter 11: Institutions and labor market performance \parencite[713-810]{Cahuc2004}}
 \end{itemize}


%Local Variables: 
%mode: latex
%TeX-master: "report"
%End: