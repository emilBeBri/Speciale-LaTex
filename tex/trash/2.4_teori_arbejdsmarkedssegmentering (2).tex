% -*- coding: utf-8 -*-
% !TeX encoding = UTF-8
% !TeX root = ../report.tex


%%%%%%%%%%%%%%%%%%%%%%%%%%%%%%%%%%%%%%%%%%%%%%%%%%%%%%%%%%%
\newpage \section{\textsc{Arbejdsmarkedssegmenteringsteori} \label{}}
%%%%%%%%%%%%%%%%%%%%%%%%%%%%%%%%%%%%%%%%%%%%%%%%%%%%%%%%%%%

Et job er ikke bare et job, så derfor er vi interesseret i at anvende jobmobilitet som en faktor til at skelne mellem forskellige typer af arbejdsstillinger. Vores inspiration til mobilitet ligger hos segmenteringsteorien, som først kom frem i 1960'erne og 1970'erne i en tid, hvor der forekom sociale reformer i forbindelse med kampen mod fattigdom og fuld deltagelse i økonomien for blandt andet minoritetsgrupper og kvinder \parencite[1216]{Cain1976}\footnote{Segmenteringsteorien var dog ses som i forlængelse af tidligere debat og kritik af klassisk og neoklassisk økonomisk teori. John Stuart Mills kritik af Adam Smiths teori om “compensating differentials” og hans egen teori om ikke-konkurrerende grupper på arbejdsmarkedet er et eksempel herpå \parencite[1224]{Cain1976}. Ellers har marxistiske økokonomi, institutionel økonomi, neoinstitutionalistisk økonomi, kyenisianere og strukturalister også deltaget i denne debat \parencite[1226ff]{Cain1976}.}. Teorien er blevet identificeret som en gruppe, som primært består af økononer som senere hen har fået selskab af sociologer. Overordnet set anskuer denne gruppe arbejdsmarkedet som fragmenteret og vægter det sociale og institutioner højt i forhold til at forklare løn og beskæftigelse \parencite[63]{Leontaridi1998}.


%%%%%%%%%%%%%%%%%%%%%%%%%%%%%%%%%%%%%%%%%%%%%%%%%%%%%%%%%%%
\subsection{Kritik af neoklassisk teori}
%%%%%%%%%%%%%%%%%%%%%%%%%%%%%%%%%%%%%%%%%%%%%%%%%%%%%%%%%%%

Som tidligere beskrevet består den neoklassiske arbejdsmarkedsøkonomi overordnet af den marginale produktivitetsteori, som er baseret på arbejdsgivernes profitmaksimerende adfærd, og udbudsteori, som er baseret på nyttemaksimering af arbejdskraft. Tidligere har vi beskrevet, at udbudsteorien består af trade-off'et mellem arbejde og fritid, “compensating differentials”, human kapital og søgeteori. Segmenteringsteorien opstod som et alternativ til disse neoklassiske økonomiske teorier henblik på at forklare lønvilkår, arbejdsforhold og muligheder på arbejdsmarkedet i forhold til eksistensen af segmentering \parencite[69]{Doeringer1971} \parencite[95]{Leontaridi1998} \parencite[1216]{Cain1976} \parencite[359]{Reich1973} \parencite[1180]{Daw2012}. %%% Søren: Hvis der kun skal være en reference, så gå med \parencite[69]{Doeringer1971} og evt. \parencite[95]{Leontaridi1998}
\footnote{På trods af at, klassiske teorier accepterer, at arbejdsmarkedet er segmenteret, men ikke i samme grad som hos segmenteringsteoretikerne. I klassiske modeller forårsager geografiske og biologiske forhold (for eksempel alder) sammen med markedsinstitutioner (for eksempel fagforeninger) og lovgivning (for eksempel minimumslønninger) markedssegmentering. I modsætning til segmenteringsteoretikerne skyldes forskellene lønstivhed \parencite[95]{Leontaridi1998}.}.

Megen af litteraturen beskæftiger tager udgangspunkt i lønforhold og bliver en direkte kritik mod compensating differentials og human kapital. I modsætning til human kapital teorien, der er fortaler for, at human kapital er årsagen til lønforskelle på arbejdsmarkedet, opstår segmenteringsteorien på baggrund af en stigende inddeling af arbejdskraften på baggrund af race, køn, uddannelse, industri med videre. Disse grupper ser ud til at arbejde i forskellige arbejdsmarkeder med forskellige arbejdsforhold, forskellige muligheder, forskellige lønforhold og forskellige markedsinstitutioner \parencite[359]{Reich1973}. På samme måde er arbejdsmarkedssegmentering og “compensating differentials” derfor modsætningsrettede. “Compensating differentials” forudsiger, at jobs med dårlige arbejdsforhold, giver en højere løn end andre jobs med bedre arbejdsforhold, fordi ellers ville arbejdstagere ikke tage de dårlige jobs. I kontrast her til, så forudsiger segmenteringen, at dårlige lønninger klynger sig sammen med dårlige arbejdsforhold \parencite[1180]{Daw2012}.


%%%%%%%%%%%%%%%%%%%%%%%%%%%%%%%%%%%%%%%%%%%%%%%%%%%%%%%%%%%
\subsection{Forskellige segmenteringsteorier}
%%%%%%%%%%%%%%%%%%%%%%%%%%%%%%%%%%%%%%%%%%%%%%%%%%%%%%%%%%%

Kært barn har mange navne. Der er næsten lige så mange versioner af arbejdsmarkedssegmentering, som der er forfattere \parencite[77]{Leontaridi1998}. Derfor går segmenteringsteori også under mange navne. Blandt de mange teorier findes den radikale segmenteringsteori, det todelt arbejdsmarked opdelt i et primært og sekundært arbejdsmarked, det tredelte arbejdsmarked opdelt i kerne, periferi og det uregelmæssige, det stratificeret arbejdsmarked, det hierarkiske arbejdsmarkedet og jobkonkurrenceteorien \parencite[1215]{Cain1976}. I det følgende vil vi beskrive tre af de nævnte teorier, herunder det todelte arbejdsmarked, den radikale segmenteringsteori og jobkonkurrenceteorien.

Det todelt arbejdsmarked (“Dual labor markets”) er udviklet af Piore, Doeringer og Bluestone. Som en af de de mest udbredte segmenteringsteorier forklarer denne lønforskelle med videre som et resultat af tvedeling af arbejdsmarkedet frem for forskelle i færdigheder (human kapital). Denne opdeler arbejdsmarkedet i primære og sekundære markeder, hvor førstnævnte indeholder alle de “gode” jobs og sidstnævnte indeholder alle de “dårlige” jobs. Det primære marked indeholder bedre betalte, stabile og de foretrukne jobs i samfundet. Dem som er i arbejde her har jobsikkerhed og forfremmelsesmulighed og gode arbejdsforhold. Vedkommende vil muligvis tage et mindre attraktivt job midlertidig, men venter på at vende tilbage i en lignende jobposition. Det sekundære marked består af lavtlønnede og ustabile jobs afskedigelser ofte \parencite[70]{Doeringer1971}. Det tredelte arbejdsmarkedet kan ses som en viderebygning af de todelte arbejdsmarked

Den radikale segmenteringsteori definerer arbejdsmarkedsmarkedssegmentering som: “the historical process whereby political-economic forces encourage the division of the labor market into seperate submarkets, or segments, distinguished by different labor market characteristics and behavorial rules.” \parencite[359]{Reich1973}. Hvor tvedelingen af arbejdsmarkedet, som beskrevet af Doeringer, forekom som en konsekvens af en tvedeling af industrien, så ser den radikale segmenteringsteori det som en konsekvens af at få hierarkisk kontrol ved at stratificere arbejderklassen \parencite[63]{Leontaridi1998}.

I Thurow's jobkonkurrenceteori er de foretrukne arbejdstagere dem med færdigheder, som resulterer i de laveste “træningsudgifter” og som dermed har lettest ved at tilpasse sig ved en ansættelse. Dette minder om human kapital, men det som adskiller jobkonkurrenceteorien fra human kapital er, at lønninger stort set bestemmes af sociale og institutionelle forhold \parencite[1221]{Cain1976}. Tildeling af jobs er i denne teori mest afhængig af den teknologiske udvikling og fleksibilitet og succesen ved “jobtræningsprocessen” \parencite[74]{Leontaridi1998}.


%%%%%%%%%%%%%%%%%%%%%%%%%%%%%%%%%%%%%%%%%%%%%%%%%%%%%%%%%%%
\subsection{Fælles for segmenteringsteorier}
%%%%%%%%%%%%%%%%%%%%%%%%%%%%%%%%%%%%%%%%%%%%%%%%%%%%%%%%%%%

Da der næsten er lige så mangler versioner af arbejdsmarkedssegmentering, som der er forfattere, udgør arbejdsmarkedssegmentering ikke et samlet alternativ til den neoklassiske teori \parencite[77]{Leontaridi1998}. Det er dog muligt at afgrænser det til, at der eksisterer få klart identificerbare segmenter findes på arbejdsmarkedet med forskellige beskæftigelses- og lønfastsættelsesmekanismer og på tværs af disse segmenter er der mobilitetsbarrierer og neoklassisk teori om human kapital er ringe eller ikke relevant i den nedre del af arbejdsmarkedet \parencite[78]{Leontaridi1998}. Ligesom der er forskellige teoretiske versioner af arbejdsmarkedssegmenter, er der også forskellige metodiske kriterier til at definere og skabe segmenterne \parencite[78]{Leontaridi1998}. Nogle bruger jobkarakteristikker, industrielle karakteristikker, subjektive målinger/foranstaltninger og erhvervsmæssige ratingskalaer \parencite[79]{Leontaridi1998}. Til statistiske og økonometriske analyser er de fire mest karakteristiske metoder: testning med human kapital modeller givet a priori segment bestemmelse, faktoranalyse, klyngeanalyse og skiftende regressioner (switching regressions) \parencite[80]{Leontaridi1998}.

Ifølge Leontaridi eksisterer der forskellige belønnings
og incitamentsordninger på tværs af arbejdsmarkedssegmenterne \parencite[92]{Leontaridi1998}. I denne forståelse er mobilitet central i arbejdsmarkedssegmenteringsteorien. I en stor mængde af teorierne er de fattige begrænset til det sekundære segment/arbejdsmarked, hvilket er den mest grundlæggende kritik af human kapital, fordi det indebærer, at arbejdsmarkederne er opdelt. I dualistisk undersøgelser i USA og i England har flere forfattere behandlet spørgsmålet om mobilitet mellem de to sektorer. Mere markant, de hævdede, at der er et hierarki af sektorer med adgang til højeste betalende være den sværeste, men flere af resultaterne er meget modstridende \parencite[93]{Leontaridi1998}.

Segmenteringsstudiernes store svaghed er, at de forudsætter segmenterne på arbejdsmarkedet \parencite[96]{Leontaridi1998}. Det vil sige, at de undersøger segmenterne \textit{a priori} ved, at de deduktivt opstiller en eller flere hypoteser og tester den. Dette gøre for det første ved at udpege visse arbejdsstillinger til visse segmenter, som Stier og Grusky eksempelvis gør\footnote{Stier og Grusky anvender beskæftigelseskategorierne “professionals, managers, “sales”, “clerical”, “crafts”, “service”, “operatives”, “laborers”, “farmers” and “farm laboreres”, som de klassificerer i forhold til om de ligger i kernesektoren eller perifisektoren\parencite[738]{Stier1990}.} \parencite[738]{Stier1990}. For det andet gøres det ved at lade løn, færdigheder og arbejdsvilkår bestemme, hvilket segmenter de forskellige arbejdsstillinger hører til, som eksempelvis Daw og Hardie gør\footnote{Daw og Hardie anvender en række jobkarakteristikker såsom løn, frynsegoder, interessen, renhed, trættende, prestige, sikkerhed, frihed, arbejdspres, avancement, anvendelse af færdigheder og antal timer (det vil sige løn, erhvervsmæssig prestige, jobtilfredshed, og socioøkonomiske forhold) til en inddeling i primær, mellemliggende og sekundære segmenter \parencite[]1187]{Daw2012}.} \parencite[]1187]{Daw2012}.

Ifølge Toubøl, Larsen og Jensen er segmenteringsstudiernes største svaghed netop, at segmenterne ikke tager udgangspunkt i de reelle barrierer, som strukturerer jobmobilitet på arbejdsmarkedet, men teoretiske eller empiriske a priori modeller til at lave segmeterne som efterfølgende kan anvendes til at sammenligne karakteristika hos de forskellige segmenter \parencite[3]{Touboel2013}. Cain har tidligere pointeret, at det er problematisk, at dette ikke er blevet gjort, fordi det dermed ikke er muligt at vide om segmenternes grænser er korrekte, hvilket gør det umuligt at drage præcise konklusioner om arbejdsmarkedssegmenternes årsager og virkninger \parencite[1231]{Cain1976}.Hermed bliver den typiske teoridrevne tilgang at opstille en teori om arbejdsmarkedets segmentering såsom det todelte arbejdsmarked og så efterfølgende opdele arbejdsmarkedets ud fra denne teori om segmentering. Så kan man beskrive de forskelle, der er på arbejdskraften i de forskellige dele af arbejdsmarkedet. Derefter følger en analyse om der kan siges at være tale om en segmentering, der bekræfter teorien, hvilket eksempelvis kunne være i forhold til de dele af arbejdsmarkedet der er præget af usikre ansættelsesvilkår og dårlige løn- og arbejdsforhold. Problemet herved er netop, at man ikke direkte observere mobiliteten på arbejdsmarkedet, men i stedet for ser på øjebliksbilleder som ikke er dynamiske. Segmenterne er derfor ikke empirisk overserverede størrelser defineret ud fra tilstedeværelsen af arbejdskraftsstrømme. 

Toubøl, Larsen og Jensen beskriver segmenternes faktiske grænser og egenskaber i forhold til jobmobilitet og mobilitetsgrænser \parencite[3]{Touboel2013}. Dette kan anvendes til at kortlægge mobiliteten på arbejdsmarkedet og dermed identificere segmenterne ud fra strømme af arbejdskraft. Hermed bliver interessen at betragte stillingskift – og ikke individer, hvilket resulterer i en metode til empirisk at identificere de givne segmenter som supplement til den teoridrevne tilgang. I modsætning til Toubøl, Larsen og Jensen er vores målsætning ikke at finde segmenternes faktiske grænser og egenskaber, men i stedet se på, hvordan arbejdsmarkedssegmenteringen ser ud for forskellige grupperinger, herunder primært de arbejdsløse og de beskæftigede.










%%%%%%%%%%%%%%%%%%%%%%%%%%%%%%%%%%%%%%%%%%%%%%%%%%%%%%%%%%%
\newpage \subsection{Mobilitet på det danske arbejdsmarked}
%%%%%%%%%%%%%%%%%%%%%%%%%%%%%%%%%%%%%%%%%%%%%%%%%%%%%%%%%%%


%%%%%%%%%%%%%%%%%%%%%%%%%%%%%%%%%%%%%%%%%%%%%%%%%%%%%%%%%%%
\subsubsection{Introduktion}

Det danske arbejdsmarked er et af de meste fleksible i EU. Den danske flexicurity giver på den ene side arbejdstagere høj social sikkerhed og arbejdsløshedsforsikring, og på den anden side gør det det relativt nemt og billigt for arbejdsgiverne at hyre og fyre. På den måde justere andelen af arbejdskraft i produktionen til efterspørgslen på markedet \textbf{ \parencite{Madsen2006} \parencite{Jensen2011}} %%% \parencite[10]{Touboel2013} (Jensen 2008, Jensen 2012, Jørgensen og Madsen 2007).


%%%%%%%%%%%%%%%%%%%%%%%%%%%%%%%%%%%%%%%%%%%%%%%%%%%%%%%%%%%
\subsubsection{Flexicurity}

Det danske arbejdsmarkeds flexicurity er blevet udviklet gennem stærke arbejdsmarkedsinstitutioner. I 2007 var 67 procent af arbejdsstyrken organiseret i fagforeninger \parencite[11]{Touboel2013}, hvor 55 procent af af arbejdstagerne var ansat i virksomheder, som var organiseret i arbejdsgiverorganisationer \textbf{\parencite[11]{Touboel2013}} %%%% (Jensen 2012).
Kollektive overenskomster dækker omkring 80 procent \textbf{\parencite[11]{Touboel2013}.} %%%% (Due et al 2010:81) 
Staten, fagforeninger og arbejsgiverorganisationer danner til sammen et arbejdsmarkedsforhold (IR relation), som kaldes for “den danske model” \textbf{\parencite[11]{Touboel2013}}. %%% (Due et al. 1994)}
Dette arbejdsmarkedsforhold er særlig kendetegnet ved, at lønreguleringer, arbejdstid, arbejdsvilkår med videre i høj grad er forhandlet mellem fagforeninger og arbejdsgivere og ikke staten. \sout{Når det kommer til statsligt reguleret arbejdsmarkedsforhold såsom uddannelse, så har fagforeninger og arbejdsgiverne stor indflydelse herpå.} \textbf{\parencite[11]{Touboel2013}.}


%%%%%%%%%%%%%%%%%%%%%%%%%%%%%%%%%%%%%%%%%%%%%%%%%%%%%%%%%%%
\subsubsection{Jobmobilitet}

At det danske arbejdsmarked er fleksibelt med høj job mobilitet er også bekræftet af statistiske studier. I 2005 var andelen af arbejdstagere som skiftede job i Danmark det højeste i EU med 11,5 procent, som kan sammenlignes med et EU gennemsnit på 8,8 procent \textbf{\parencite[21]{Andersen2008}.} %%%% \parencite[10]{Touboel2013} 
I 2006 var gennemsnittet for længden på en ansættelsesperiode i en virksomhed den laveste i Europa på ca. 4,8 år sammenlignet med EU gennemsnittet på 8,3 år \textbf{\parencite[27]{Andersen2008}}. %%%% \parencite[10]{Touboel2013}
Disse tal viser, at mobiliteten på det danske arbejdsmarked er høj. Tallene siger dog ikke noget om mobiliteten mellem sektorer eller arbejdsstillinger. Der kan altså både være tale om en høj mobilitet inden for sektorer eller mellem relaterede arbejdsstillinger \parencite[11]{Touboel2013}.


%%%%%%%%%%%%%%%%%%%%%%%%%%%%%%%%%%%%%%%%%%%%%%%%%%%%%%%%%%%
\subsubsection{Arbejdsmarkedssegmenter}

I 1980'erne foretog Boje en stor analyse af danske arbejdsmarkedssegmenter. Resultatet var, at det høje institutionaliseringsniveau, fleksibiliten, den aktive arbejdsmarkedspolitik og det relativt lige og høje uddannelsesniveau gør Danmark til til et særligt arbejdsmarkedstilfælde med en række submarkeder frem for få store segmenter, som den gængse arbejdsmarkedssegmenteringsteori ville forvente \textbf{\parencite{Boje1986}.} %%%% \parencite[11]{Touboel2013} (Boje 1985, Boje 1990, Boje og Toft 1989).


%%%%%%%%%%%%%%%%%%%%%%%%%%%%%%%%%%%%%%%%%%%%%%%%%%%%%%%%%%%
\subsubsection{Andet}

I 2015 foretog arbejderbevægelsens Erhvervsråd. \textbf{\parencite{Kirk2015}.}


%%%%%%%%%%%%%%%%%%%%%%%%%%%%%%%%%%%%%%%%%%%%%%%%%%%%%%%%%%%
\subsection{Opsummering}
%%%%%%%%%%%%%%%%%%%%%%%%%%%%%%%%%%%%%%%%%%%%%%%%%%%%%%%%%%%


%Local Variables: 
%mode: latex
%TeX-master: "report"
%End: