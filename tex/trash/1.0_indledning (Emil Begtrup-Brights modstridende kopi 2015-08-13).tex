% -*- coding: utf-8 -*-
% !TeX encoding = UTF-8
% !TeX root = ../report.tex


%%%%%%%%%%%%%%%%%%%%%%%%%%%%%%%%%%%%%%%%%%%%%%%%%%%%%%%%%%%
\chapter{\textsc{Indledning} \label{indledning}}
%%%%%%%%%%%%%%%%%%%%%%%%%%%%%%%%%%%%%%%%%%%%%%%%%%%%%%%%%%%


%%%%%%%%%%%%%%%%%%%%%%%%%%%%%%%%%%%%%%%%%%%%%%%%%%%%%%%%%%%
% \section{\textsc{Teaser}}
%%%%%%%%%%%%%%%%%%%%%%%%%%%%%%%%%%%%%%%%%%%%%%%%%%%%%%%%%%%

% Teaseren er sidst revideret d. 23. marts i forbindelse med vejledning hos Jens

% Den danske beskæftigelsespolitik beskæftiger sig med at få ledige i job. I 2013 nedsatte regeringen det seneste ekspertudvalg på området med tidligere skatteminister Carsten Koch i spidsen. Udvalgets anbefalinger er den første af to rapporter (2014) var blandt andet en ny, individuel og jobrettet indsats for den enkelte, målrettet brug af opkvalificering i beskæftigelsesindsatsen, styrket fokus på virksomhedernes behov, styrkede økonomiske incitamenter for beskæftigelsessystemet og mindre bureaukrati. Beskæftigelsesanbefalingerne fra Carsten Koch-udvalget ligger sig i forlængelse af en lang række af reguleringer på beskæftigelsesområdet, hvor dansk beskæftigelsespolitik siden Wechselmann-udvalgets i 1960'erne med små skridt af gangen er gået mere og mere væk fra at være tryghedsorienterede velfærdspolitik som sikrer arbejdstagernes økonomisk og social tryghed (Pedersen 2007).

% Den historie vi med vores speciale ønsker at fortælle handler om at forstå arbejdsløshed som et multifacetteret begreb, der tydeliggør distinktioner mellem grupper, alt efter deres position i samfundsstrukturen, og som spiller en afgørende rolle for fordelingen af goder i samfundet. Der er grundlæggende tre ting på spil i konstitueringen af et centralt socialt fænomen som arbejdsløshed. Det handler lige så meget om den symbolske magt til at definere, hvad \emph{problemet med arbejdsløshed} i det hele taget \emph{er}, som det handler om, hvad der er \emph{de bedste midler} til at løse problemerne forbundet med det. Og uadskilleligt forbundet med disse to størrelser er beskrivelsen af forskelle og ligheder mellem dem, der står uden job. Vi mener at kunne yde et unikt bidrag til det sidste, på en måde som også kaster lys over de to første.


%%%%%%%%%%%%%%%%%%%%%%%%%%%%%%%%%%%%%%%%%%%%%%%%%%%%%%%%%%%
\section{\textsc{Problemformulering og undersøgelsesspørgsmål}}
%%%%%%%%%%%%%%%%%%%%%%%%%%%%%%%%%%%%%%%%%%%%%%%%%%%%%%%%%%%

% Formålet med dette speciale er at udvikle nye måder at kortlægge arbejdsløses beskæftigelsesmobilitet på det danske arbejdsmarked.

Vores problemformulering er:
\begin{quote} %\small %\raggedright %(bloktekst on/off)
  \textbf{Hvordan hænger arbejdsløshed sammen med sociale mobilitetsmønstre, og kan vi derigennem se tendenser til differentiering af grupper af arbejdsløse ud fra dette mønster?}%
\end{quote}
%
Dette overordnede spørgsmål besvarer vi med følgende underspørgsmål:
%
 \begin{enumerate} [topsep=6pt,itemsep=-1ex]
   \item Hvilken betydning har (forskellige definitioner af) arbejdsløshed for beskæftigelsesmobiliteten på arbejdsmarkedet?
   \item Kan man se sandsynlige strategier afspejlet i mobilitetsmønstrene, som arbejdsløse benytter sig af for at komme tilbage i beskæftigelse?
    \item Finder vi gennem disse mobilitetsmønstre relativt afgrænsede grupper, der har en række vilkår tilfælles?
 \end{enumerate}
% 
% Her skal der være en uddybelse af problemformulering og underspørgsmål.
% Uddybelse af problemformulering og underspørgsmål er sidst revideret d. 23. marts i forbindelse med vejledning hos Jens

% Vi er særligt interesserede i mobilitet i forbindelse med arbejdsløshed Hvilke jobs får man efter perioder med ledighed? Derved vil vi undersøge hvilken praksis der hænger sammen med hvilke felter, og hvad det siger om hvilke felter der ligger nær hinanden samtidig med hvad der skal til for at bevæge sig ud over det felt man oprindeligt kommer fra. Dette betyder, at vi ønsker at diskutere, hvad der strukturerer folk der oplever arbejdsløsheds opfattelse af handlingsrum, som vi ser det komme til udtryk igennem deres praksis mellem forskellige typer af jobs. Specialets primære metode til at fortælle denne historie er en netværksanalyse af registerdata fra Danmarks Statistiks arbejdsmarkedsdatabaser i perioden 1996 til 2009. Som det centrale redskab til at analysere ledighed bruger vi den den såkaldte MONECA-algoritme, der er en måde at lave en datadrevet skitse af hvilke beskæftigelsestyper der ligger tæt på hinanden frem for en teoretisk og institutionel a priori-inddeling. Det vil sige når en arbejdstager er ledig i en periode, kan MONECA bruges til at se hvilken type job vedkommende kommer fra og går til før og efter den pågældende periode med ledighed.

% Når vores speciale beskæftiger sig med en bestemt gruppe - arbejdstagere - som er i beskæftigelse efter at have oplevet periode med ledighed - og hvordan de i praksis - handler - og kommer ud af ledighed, er vores fokus et sted mellem at se arbejdsløshed som et økonomisk problem og at se på arbejdsløshedens sociale konsekvenser i et historisk perspektiv. Vi læner os op af søgeteoriens økonomiske fokus på hvad der skal til for at en arbejdsløs tager imod et job på baggrund af forventninger om løn (Rosholm 2009:162) uden at give et entydigt mål for hvornår det kan betale sig at arbejde, som mere abstrakt rationelle undersøgelser har for vane. I stedet viser vi, hvilke kanaler der eksisterer mellem forskellige jobtyper og derved hvilken praksis som er mulig for hvilke arbejdsløse. Vi lægger os samtidig op af at den sociologiske tilgang til den enkeltes arbejdsløshed skal ses i lyset af bestemte økonomiske markedskræfter og inden for en bestemt historisk udvikling uden, at fokus bliver på arbejdsløse i en slags passiv offerrolle (Goul Andersen 2003:35). I modsætning til det, anskuer vi arbejdsløse som forskelligartede og handlende i et socialt rum, der så til gengæld er betinget af omstændighederne, med sammenfald mellem mentale og objektive strukturer, der kommer til udtryk i “hvad der er muligt”.
% Dette følger hen til det tredje og sidste undersøgelsesspørgsmål som fokuserer på et videnskabeligt spørgsmål om hvilken måde vores tilgang til sociale mobilitet på arbejdsmarkedet åbner op at anskue ledighedsbegrebet på end anderledes måde end de nævnte metoder.



%%%%%%%%%%%%%%%%%%%%%%%%%%%%%%%%%%%%%%%%%%%%%%%%%%%%%%%%%%%
% \section{\textsc{Arbejdsløshed og arbejdsløse - introduktion og afgrænsning}}
%%%%%%%%%%%%%%%%%%%%%%%%%%%%%%%%%%%%%%%%%%%%%%%%%%%%%%%%%%%

% Fokus på arbejdsmarkedsparate arbejdsløse, men værd opmærksom på, at arbejdsmarkedsparathed kan have en mening (økonomiske incitamenter) vi ikke ønsker.

% Her skal der stå en introduktion af hvad der kommer til at ske i teoretisk og metodisk i forhold til operationalisering af arbejdsløshed (pr. 3. juni er der tale om teoriafsnittet “2. Teori - Bourdieu og arbejdsløshedforskerne” og metodeafsnittet “Metode om arbejdsløshed”).

% Den teoretiske pointe er at arbejdsløshed defineres og behandles forskelligt alt efter om det er økonomer, sociologer mv.. Vores fokus ligger i forlængelse af marginaliseringsbegrebet \parencite{Larsen2009} samt Bourdieus perspektiver om at være placeret et specifikt sted i det sociale rum og at være på kanten af arbejdsmarkedet.

% Den metodiske pointe er, at arbejdsløse opgøres på forskellige måder blandt andet i nettoledige, bruttoledige, AKU-ledige, dagpengemodtagere, kontanthjælpsmodtagere og så videre alt efter hvad vi taler om. Når vi taler om arbejdsløse er der truffet en masse valg. Vores fokus er at samle forskellige definitioner/former for arbejdsløshed for at se hvilken betydning det har for beskæftigelsesmobilitet. Som noget helt nyt ønsker vi bl.a. at inkludere “uden for arbejdsstyrken” og kortere perioder “deltidsledighed” og “fuldtidsledighed”. 

% Hvad er så vores analyseudvalg? Vores analyseudvalg består af personer som går fra at være beskæftiget i en arbejdsstilling til at være beskæftiget i en anden arbejdsstilling efter en mellemliggende periode med ledighed eller uden beskæftigelse. For at være med skal personerne være beskæftiget så det passer overens med en af vores 150 \texttt{DISCO}-kategorier og være ledig så det passer overens med vores binære ledighedsbegreb lavet på baggrund af \texttt{SOCSTIL} og \texttt{SOCIO}. Vi har udvalgt perioden 1996 til 2009, fordi denne periode giver os mulighed for at have et datamateriale som har en forholdsvis god kvalitet uden så mange databrud, som vi ville skulle forholde os til, hvis vi tog perioderne før 1996 og efter 2009. Vi har indskrænket arbejdsmarkedet i forhold til aldersgruppen 16 til 70 år. I flere andre ledighedsstatistikker indskrænkes ledige til det år, hvor man har mulighed for at gå på pension fx \parencite{Bjoersted2012, Bang-Petersen2012, DST2014a}. I perioden 1996 til 2009 ville det derfor være hensigtsmæssigt at anvende aldersgruppen 16-64 år. Vi har dog valgt at udvide aldersgruppen med fem år, fordi vi gerne vil have en så bred gruppe af ledige som kommer ind og ud af arbejdsmarkedet\footnote{Når vi indskrænker det til aldersgruppen 16-70 år mister vi 8508 personer som enten er yngre eller ældre. Det er især inden for kategorien \emph{Arbejde med dyr og skovbrug, primært landbrugsmedarbejder og landmand} (\texttt{6120}), hvor der er mange +70-årige som falder fra. Dette skyldes ifølge DST, at der findes mange landmænd som fortætter langt over pensionsalderen \parencite{DST2012}.}.


%%%%%%%%%%%%%%%%%%%%%%%%%%%%%%%%%%%%%%%%%%%%%%%%%%%%%%%%%%%
% \section{\textsc{Skitse af metode, analyse og diskussion}} 
%%%%%%%%%%%%%%%%%%%%%%%%%%%%%%%%%%%%%%%%%%%%%%%%%%%%%%%%%%%

% Her skal der være et afsnit som skitsere specialets indhold i forhold til metode, analyse og diskussion.
% Skitse af metode, analyse og diskussion er sidst revideret d. 23. marts i forbindelse med vejledning hos Jens. Teksten skal skrives på baggrund af hele specialet.

% \subsection{Metode: netværksanalyse, Moneca og \texttt{DISCO} \label{}} 

% Vores tilgang er inspireret af Pierre Bourdieus måde at anskue den sociale verden objektivt på. Det handler om at synliggøre de principper, der ligger til grund for de valg og fravalg af metode, operationalisering, fremstilling med videre. Undervejs har det ikke været helt klart, hvilken vej den netværksanalytiske metode har ville føre os hen, da den netop er datadrevet og på mange måder induktiv\footnote{Det er klart at datamaterialet i sig selv er struktureret på måder der synliggør visse ting og skjuler andre, samt netværksmetoden selv giver et bestemt blik. Men derfor er det stadig svært at gisne om kortenes udformning på nuværende stadie.}. Derfor hjælper disse principper til at klargøre hvad det er for valg og fravalg vi har taget undervejs i specialeprocessen og ultimativt hvad vi har valgt at fokusere på og hvorledes vi har valgt at fremstille det. Med inspiration i en relationelt orienteret sociologi vil vi benytte Moneca-algoritmen til at afgrænse individer i klynger alt efter deres tilknytning til arbejdsmarkedet.

% Vi betragter de arbejdsløse som sociale agenter, hvis handlinger er udtryk for en praktisk logik, der er indskrevet i deres kroppe gennem de livsbaner, de har haft, det vil sige de felter de er formet af. Den logik som disse handlinger er udtryk for er derfor feltspecifikke, så for at forstå disse handlinger, må vi afdække den logik de er udtryk for.  Netværksanalysen hjælper os med kortlægge strukturen i handlinger omhandlende jobskift, men vores analyse kræver også at gøre forudsætningerne for disse handlinger forståelige. Det er afgørende at vise de generative mekanismer\footnote{Dermed ikke sagt, at der er noget som helst \emph{mekanisk} ved det.}, der skaber denne struktur. Det vil finde sted i forlængelse af netværksanalysen, og har flere elementer.

%%%%%%%%%%%%%%%%%%%%%%%%%%%%%%%%%%%%%%%%%%%%%%%%%%%%%%%%%%%
% \section{\textsc{Analyse og diskussion \texttt{DISCO}}} 
%%%%%%%%%%%%%%%%%%%%%%%%%%%%%%%%%%%%%%%%%%%%%%%%%%%%%%%%%%%

% Da Moneca-metoden er datadrevet og vi endnu ikke har fået de første netværkskort i hus svarer det til, at vi lige er gået i gang med vores etnografiske feltarbejde. Vi ved altså endnu ikke, hvilke kanaler der bliver etableret, og eftersom vi arbejder med 100+ beskæftigelseskategorier, er det svært at komme med andet end meget rudimentære gæt. Basalt set mener vi at forskellige jobtyper ligger indenfor forskellige felter, og de sociale agenters habitus er tilpasset disse forskellige felter, så nogle job ligger tættere på nogle typer mennesker end andre. , hvilket kommer til udtryk ved at nogle jobtyper ligger tættere på og længere væk fra hinanden end andre. Det er tæt på at være en truisme, men ved at tænke denne opdeling gennem Bourdieus forståelse af felter og habitus, mener vi at kunne frame disse opdelinger på måder, som gør det muligt at forstå de generative mekanismer bedre. Dernæst har vi en antagelse om, at længden på ledighedsperioden siger noget om habitus på arbejdsmarkedsfeltet. I den sammenhæng er det særlig interessant at se på om langtidsledige har større disposition til at bevæge sig længere væk fra de jobtyper der normalt ligger tættere på der hvor man kommer fra før end ledighedsperiode sammenlignet med korttidsledige og arbejdstagere som ikke har oplevet ledighed. Vender man tilbage til arbejde i samme felt, eller bevæger man sig ind på et nyt? Derved vil vi diskutere, hvilken praksis der hænger sammen med hvilke felter, og hvad det siger om hvilke felter der ligger nær hinanden. Eller måske siger noget om hvor desperat man skal være, for at bevæge sig ud over det felt man er trænet ind i. Vi vil gerne diskutere hvad der strukturerer folk, der oplever arbejdsløsheds opfattelse af handlingsrum, som vi ser det komme til udtryk igennem deres praksis mellem forskellige typer af jobs efter perioder med ledighed.

% Vi er desuden interesseret i at diskutere om arbejdstagere som vender tilbage til en anden type arbejde end det de havde i forvejen opretholder arbejdsmarkedets doxa om at arbejde ikke bare er noget man varetager af nødvendighed, men fordi man synes at det er meningsfuldt eller om det sker et brud med arbejdslivets illusio. I den sammenhæng kunne det være interessant at passe på med at betegne det med at komme i en beskæftigelse igen som \emph{at komme videre} eller som en udelukkende positiv ting. For eksempel kan beskæftigelsen personen kommer i være sæsonarbejde eller være et plaster på sået, der gør, at personen inden alt for længe bliver ledighed igen. Derfor kunne det alternativ være bedre eksempelvis som akademiker at vente på \emph{et mere passende job} eller som ufaglært at tage en uddannelse (Arnholz 2004:24). Dette kan ud fra Bourdieu være et udtryk for at fokusere på og komme ud af  \emph{den store elendighed} som arbejdsløsheden og alle de problemer man har som arbejdsløshed (både dem som er forårsaget af arbejdsløshed og dem som ikke er forårsaget af arbejdsløsheden) og derved ignorerer “den lille elendighed” som består i at få forringet sine arbejdsvilkår, blive tvunget til at ofre sig lidt før man bliver ofret og samtidig være taknemmelig over, at man ikke hører til blandt de svage. Hermed tilpasser man sig arbejdsmarkedets umiddelbare behov ved at man som ledige accepterer samfundets objektive strukturer  (Arnholz 2004:27) og får derved nogle realistiske forventninger i forhold til deres position i den sociale verden (“De arbejdsløse må tage hvad de kan få”) (Arnholz 2004: 35).


%%%%%%%%%%%%%%%%%%%%%%%%%%%%%%%%%%%%%%%%%%%%%%%%%%%%%%%%%%%
\newpage \section{\textsc{Disposition}}
%%%%%%%%%%%%%%%%%%%%%%%%%%%%%%%%%%%%%%%%%%%%%%%%%%%%%%%%%%%

\begin{table}[h] 
\begin{tabular}{@{}||l||l||@{}} \hline \hline 
  1.  & \textbf{Indledning} \\ \hline \hline 
  2.  & \textbf{Teori} \\ 
  2.1 & - Bourdieu \\ 
  2.2 & - Arbejdsløshed \\ 
  2.3 & - Mobiltet \\ 
  2.4 & - Hypoteser og opreationalisering \\ \hline \hline 
  3.  & \textbf{Metode}  \\ 
  3.1 & - Arbejdsløshed  \\ 
  3.2 & - Netværksanalyse  \\ 
  3.3 & - Disco \\ \hline \hline 
  4.  & \textbf{Analyse}  \\ 
  4.1 & - Hovedkort  \\ 
  4.2 & - Faglærte og ufaglærte  \\ 
  4.3 & - Akademikerne  \\ \hline \hline 
  5.  & \textbf{Diskussion}  \\ \hline \hline 
  6.  & \textbf{Konklusion}  \\ \hline \hline 
\end{tabular} \end{table}



%%%%%%%%%%%%%%%%%%%%%%%%%%%%%%%%%%%%%%%%%%%%%%%%%%%%%%%%%%%
% \section{\textsc{Læsevejledning}}
%%%%%%%%%%%%%%%%%%%%%%%%%%%%%%%%%%%%%%%%%%%%%%%%%%%%%%%%%%%

% 1) Brugen af monotype skrift for at understrege at der er tale om \texttt{variable} og \texttt{udfald} eller generelle \texttt{kategoriseringer} af data. Formålet er at forholde sig til at vi har at gøre med staten og vores egne statistiske kategorier. Det skulle gerne fremkalde en fremmedgørelseseffekt, som minder læseren om hvad det er vi ser verden igennem, når vi ser verden igennem registerdata, og det er struktureret i forhold et forsknings- og administrativt system. Det er for at vi skal se de briller vi ser med, med andre ord. Et mål er også at nå udover denne administrative tilgang, at give et indtryk af praksis, og derfor vil vi også forsøge at tale mere direkte om de job som kategorierne dækker. så ovenstående forsøg på at forholde sig ærligt til tilgangen skulle gerne vekselvirke med at kunne sige noget om den praksis kategorierne på den ene side afslører med muligheden for at vise (det der står det andet sted om de objektive strukturer), samtidig med den tilslører i samme bevægelse. Læseren må dømme om det er lykkedes blah blah.

% 2) En komplet liste over vores disco-kategorisering findes i et excelark, der forefindes her. Vi har valgt dette fordi vi ikke mener det ville skabe større overblik at reducere excelarket til et format, der kunne være på et A4-ark. Vi har i stedet skabt et overskueligt dokument, der kan hentes her: http blah blah. En vejledning til at bruge dokumentet findes i bilag XXX (LAV GUIDE PÅ ET TIDSPUNKT)

% 3) Af hensyn til Danmarks Statistiks politik om datafortrolig er alle tabeller med celler med under 3 observationer sat til 0 for at hindre identifikation af enkeltindivider.


%%%%%%%%%%%%%%%%%%%%%%%%%%%%%%%%%%%%%%%%%%%%%%%%%%%%%%%%%%%
% Trash
%%%%%%%%%%%%%%%%%%%%%%%%%%%%%%%%%%%%%%%%%%%%%%%%%%%%%%%%%%%



%Local Variables: 
%mode: latex
%TeX-master: "report"
%End: 