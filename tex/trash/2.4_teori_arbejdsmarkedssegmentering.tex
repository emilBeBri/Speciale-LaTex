% -*- coding: utf-8 -*-
% !TeX encoding = UTF-8
% !TeX root = ../report.tex


%%%%%%%%%%%%%%%%%%%%%%%%%%%%%%%%%%%%%%%%%%%%%%%%%%%%%%%%%%%
\newpage \section{\textsc{Arbejdsmarkedssegmenteringsteori og mobilitet} \label{}}
%%%%%%%%%%%%%%%%%%%%%%%%%%%%%%%%%%%%%%%%%%%%%%%%%%%%%%%%%%%

Et job er ikke bare et job, så derfor er vi interesseret i at anvende jobmobilitet som en faktor til at skelne mellem forskellige typer af arbejdsstillinger. Vores inspiration til mobilitet ligger hos segmenteringsteorien, som først kom frem i 1960'erne og 1970'erne i en tid, hvor der forekom sociale reformer i forbindelse med kampen mod fattigdom og fuld deltagelse i økonomien for blandt andet minoritetsgrupper og kvinder \parencite[1216]{Cain1976}\footnote{Segmenteringsteorien var dog ses som i forlængelse af tidligere debat og kritik af klassisk og neoklassisk økonomisk teori. John Stuart Mills kritik af Adam Smiths teori om “compensating differentials” og hans egen teori om ikke-konkurrerende grupper på arbejdsmarkedet er et eksempel herpå \parencite[1224]{Cain1976}. Ellers har marxistiske økokonomi, institutionel økonomi, neoinstitutionalistisk økonomi, kyenisianere og strukturalister også deltaget i denne debat \parencite[1226ff]{Cain1976}.}. Teorien er blevet identificeret som en gruppe, som primært består af økononer som senere hen har fået selskab af sociologer og kernen i deres kritik kan opsumeres således: 

\begin{displayquote} “The core of this criticism is first that neoclassical labour-market theories in their analogies to the commodity market do not take into account the heterogeneous character of the labour market and the lack of homogeneity in the labour force. Secondly the theories do not take into account that the behaviour of individuals on the labour market is not governed by economic utility maximization, but, on the contrary, determined by social relations and institutions of which the individual is an integral part.” \parencite[171]{Boje1986}. \end{displayquote}

Kritikken spiller altså på de samme tangenter som Atkinson og andre sociologer som tidligere beskrevet i henholdsvis afsnit ? og ?.


%%%%%%%%%%%%%%%%%%%%%%%%%%%%%%%%%%%%%%%%%%%%%%%%%%%%%%%%%%%
\subsection{Kritik af neoklassisk teori}
%%%%%%%%%%%%%%%%%%%%%%%%%%%%%%%%%%%%%%%%%%%%%%%%%%%%%%%%%%%

Som tidligere beskrevet består den neoklassiske arbejdsmarkedsøkonomi overordnet af den marginale produktivitetsteori, som er baseret på arbejdsgivernes profitmaksimerende adfærd, og udbudsteori, som er baseret på nyttemaksimering af arbejdskraft. Tidligere har vi beskrevet, at udbudsteorien består af trade-off'et mellem arbejde og fritid, “compensating differentials”, human kapital og søgeteori. Segmenteringsteorien opstod som et alternativ til disse neoklassiske økonomiske teorier henblik på at forklare lønvilkår, arbejdsforhold og muligheder på arbejdsmarkedet i forhold til eksistensen af segmentering \parencite[69]{Doeringer1971} \parencite[95]{Leontaridi1998} \parencite[1216]{Cain1976} \parencite[359]{Reich1973} \parencite[1180]{Daw2012}. %%% Søren: Hvis der kun skal være en reference, så gå med \parencite[69]{Doeringer1971} og evt. \parencite[95]{Leontaridi1998}
\footnote{På trods af at, klassiske teorier accepterer, at arbejdsmarkedet er segmenteret, men ikke i samme grad som hos segmenteringsteoretikerne. I klassiske modeller forårsager geografiske og biologiske forhold (for eksempel alder) sammen med markedsinstitutioner (for eksempel fagforeninger) og lovgivning (for eksempel minimumslønninger) markedssegmentering. I modsætning til segmenteringsteoretikerne skyldes forskellene lønstivhed \parencite[95]{Leontaridi1998}.}.

Megen af litteraturen beskæftiger tager udgangspunkt i lønforhold og bliver en direkte kritik mod compensating differentials og human kapital. I modsætning til human kapital teorien, der er fortaler for, at human kapital er årsagen til lønforskelle på arbejdsmarkedet, opstår segmenteringsteorien på baggrund af en stigende inddeling af arbejdskraften på baggrund af race, køn, uddannelse, industri med videre. Disse grupper ser ud til at arbejde i forskellige arbejdsmarkeder med forskellige arbejdsforhold, forskellige muligheder, forskellige lønforhold og forskellige markedsinstitutioner \parencite[359]{Reich1973}. På samme måde er arbejdsmarkedssegmentering og “compensating differentials” derfor modsætningsrettede. “Compensating differentials” forudsiger, at jobs med dårlige arbejdsforhold, giver en højere løn end andre jobs med bedre arbejdsforhold, fordi ellers ville arbejdstagere ikke tage de dårlige jobs. I kontrast her til, så forudsiger segmenteringen, at dårlige lønninger klynger sig sammen med dårlige arbejdsforhold \parencite[1180]{Daw2012}.


%%%%%%%%%%%%%%%%%%%%%%%%%%%%%%%%%%%%%%%%%%%%%%%%%%%%%%%%%%%
\subsection{Tre forskellige segmenteringsteorier}
%%%%%%%%%%%%%%%%%%%%%%%%%%%%%%%%%%%%%%%%%%%%%%%%%%%%%%%%%%%

Kært barn har mange navne. Der er næsten lige så mange versioner af arbejdsmarkedssegmentering, som der er forfattere \parencite[77]{Leontaridi1998}. Derfor går segmenteringsteori også under mange navne. Blandt de mange teorier findes den radikale segmenteringsteori, det todelt arbejdsmarked opdelt i et primært og sekundært arbejdsmarked, det tredelte arbejdsmarked opdelt i kerne, periferi og det uregelmæssige, det stratificeret arbejdsmarked, det hierarkiske arbejdsmarkedet og jobkonkurrenceteorien \parencite[1215]{Cain1976}. I det følgende vil vi beskrive tre af de nævnte teorier, herunder det todelte arbejdsmarked, den radikale segmenteringsteori og jobkonkurrenceteorien.

Det todelt arbejdsmarked (“Dual labor markets”) er udviklet af Piore, Doeringer og Bluestone. Som en af de de mest udbredte segmenteringsteorier forklarer denne lønforskelle med videre som et resultat af tvedeling af arbejdsmarkedet frem for forskelle i færdigheder (human kapital). Denne opdeler arbejdsmarkedet i primære og sekundære markeder, hvor førstnævnte indeholder alle de “gode” jobs og sidstnævnte indeholder alle de “dårlige” jobs. Det primære marked indeholder bedre betalte, stabile og de foretrukne jobs i samfundet. Dem som er i arbejde her har jobsikkerhed og forfremmelsesmulighed og gode arbejdsforhold. Vedkommende vil muligvis tage et mindre attraktivt job midlertidig, men venter på at vende tilbage i en lignende jobposition. Det sekundære marked består af lavtlønnede og ustabile jobs afskedigelser ofte \parencite[70]{Doeringer1971}. Det tredelte arbejdsmarkedet kan ses som en viderebygning af de todelte arbejdsmarked

Den radikale segmenteringsteori definerer arbejdsmarkedsmarkedssegmentering som: “the historical process whereby political-economic forces encourage the division of the labor market into seperate submarkets, or segments, distinguished by different labor market characteristics and behavorial rules.” \parencite[359]{Reich1973}. Hvor tvedelingen af arbejdsmarkedet, som beskrevet af Doeringer, forekom som en konsekvens af en tvedeling af industrien, så ser den radikale segmenteringsteori det som en konsekvens af at få hierarkisk kontrol ved at stratificere arbejderklassen \parencite[63]{Leontaridi1998}.

I Thurow's jobkonkurrenceteori er de foretrukne arbejdstagere dem med færdigheder, som resulterer i de laveste “træningsudgifter” og som dermed har lettest ved at tilpasse sig ved en ansættelse. Dette minder om human kapital, men det som adskiller jobkonkurrenceteorien fra human kapital er, at lønninger stort set bestemmes af sociale og institutionelle forhold \parencite[1221]{Cain1976}. Tildeling af jobs er i denne teori mest afhængig af den teknologiske udvikling og fleksibilitet og succesen ved “jobtræningsprocessen” \parencite[74]{Leontaridi1998}.


%%%%%%%%%%%%%%%%%%%%%%%%%%%%%%%%%%%%%%%%%%%%%%%%%%%%%%%%%%%
\subsection{Fælles for segmenteringsteorierne}
%%%%%%%%%%%%%%%%%%%%%%%%%%%%%%%%%%%%%%%%%%%%%%%%%%%%%%%%%%%

Da der næsten er lige så mangler versioner af arbejdsmarkedssegmentering, som der er forfattere, udgør arbejdsmarkedssegmentering ikke et samlet alternativ til den neoklassiske teori \parencite[77]{Leontaridi1998}. Det er dog muligt at afgrænser det til, at der eksisterer få klart identificerbare segmenter findes på arbejdsmarkedet med forskellige beskæftigelses- og lønfastsættelsesmekanismer og på tværs af disse segmenter er der mobilitetsbarrierer og neoklassisk teori om human kapital er ringe eller ikke relevant i den nedre del af arbejdsmarkedet \parencite[78]{Leontaridi1998}. Ligesom der er forskellige teoretiske versioner af arbejdsmarkedssegmenter, er der også forskellige metodiske kriterier til at definere og skabe segmenterne \parencite[78]{Leontaridi1998}. Nogle bruger jobkarakteristikker, industrielle karakteristikker, subjektive målinger/foranstaltninger og erhvervsmæssige ratingskalaer \parencite[79]{Leontaridi1998}. Til statistiske og økonometriske analyser er de fire mest karakteristiske metoder: testning med human kapital modeller givet a priori segment bestemmelse, faktoranalyse, klyngeanalyse og skiftende regressioner (switching regressions) \parencite[80]{Leontaridi1998}.

Ifølge Leontaridi eksisterer der forskellige belønnings
og incitamentsordninger på tværs af arbejdsmarkedssegmenterne \parencite[92]{Leontaridi1998}. I denne forståelse er mobilitet central i arbejdsmarkedssegmenteringsteorien. I en stor mængde af teorierne er de fattige begrænset til det sekundære segment/arbejdsmarked, hvilket er den mest grundlæggende kritik af human kapital, fordi det indebærer, at arbejdsmarkederne er opdelt. I dualistisk undersøgelser i USA og i England har flere forfattere behandlet spørgsmålet om mobilitet mellem de to sektorer. Mere markant, de hævdede, at der er et hierarki af sektorer med adgang til højeste betalende være den sværeste, men flere af resultaterne er meget modstridende \parencite[93]{Leontaridi1998}.

Segmenteringsstudiernes store svaghed er, at de forudsætter segmenterne på arbejdsmarkedet \parencite[96]{Leontaridi1998}. Det vil sige, at de undersøger segmenterne \textit{a priori} ved, at de deduktivt opstiller en eller flere hypoteser og tester den. Dette gøre for det første ved at udpege visse arbejdsstillinger til visse segmenter, som Stier og Grusky eksempelvis gør\footnote{Stier og Grusky anvender beskæftigelseskategorierne “professionals, managers, “sales”, “clerical”, “crafts”, “service”, “operatives”, “laborers”, “farmers” and “farm laboreres”, som de klassificerer i forhold til om de ligger i kernesektoren eller perifisektoren\parencite[738]{Stier1990}.} \parencite[738]{Stier1990}. For det andet gøres det ved at lade løn, færdigheder og arbejdsvilkår bestemme, hvilket segmenter de forskellige arbejdsstillinger hører til, som eksempelvis Daw og Hardie gør\footnote{Daw og Hardie anvender en række jobkarakteristikker såsom løn, frynsegoder, interessen, renhed, trættende, prestige, sikkerhed, frihed, arbejdspres, avancement, anvendelse af færdigheder og antal timer (det vil sige løn, erhvervsmæssig prestige, jobtilfredshed, og socioøkonomiske forhold) til en inddeling i primær, mellemliggende og sekundære segmenter \parencite[]1187]{Daw2012}.} \parencite[1187]{Daw2012}.

Ifølge Toubøl, Larsen og Jensen er segmenteringsstudiernes største svaghed netop, at segmenterne ikke tager udgangspunkt i de reelle barrierer, som strukturerer jobmobilitet på arbejdsmarkedet, men teoretiske eller empiriske a priori modeller til at lave segmenterne som efterfølgende kan anvendes til at sammenligne karakteristika hos de forskellige segmenter \parencite[3]{Touboel2013}. Cain har tidligere pointeret, at det er problematisk, at dette ikke er blevet gjort, fordi det dermed ikke er muligt at vide om segmenternes grænser er korrekte, hvilket gør det umuligt at drage præcise konklusioner om arbejdsmarkedssegmenternes årsager og virkninger \parencite[1231]{Cain1976}.Hermed bliver den typiske teoridrevne tilgang at opstille en teori om arbejdsmarkedets segmentering såsom det todelte arbejdsmarked og så efterfølgende opdele arbejdsmarkedets ud fra denne teori om segmentering. Så kan man beskrive de forskelle, der er på arbejdskraften i de forskellige dele af arbejdsmarkedet. Derefter følger en analyse om der kan siges at være tale om en segmentering, der bekræfter teorien, hvilket eksempelvis kunne være i forhold til de dele af arbejdsmarkedet der er præget af usikre ansættelsesvilkår og dårlige løn- og arbejdsforhold. Problemet herved er netop, at man ikke direkte observere mobiliteten på arbejdsmarkedet, men i stedet for ser på øjebliksbilleder som ikke er dynamiske. Segmenterne er derfor ikke empirisk overserverede størrelser defineret ud fra tilstedeværelsen af arbejdskraftsstrømme. 

Toubøl, Larsen og Jensen beskriver segmenternes faktiske grænser og egenskaber i forhold til jobmobilitet og mobilitetsgrænser \parencite[3]{Touboel2013}. Dette kan anvendes til at kortlægge mobiliteten på arbejdsmarkedet og dermed identificere segmenterne ud fra strømme af arbejdskraft. Hermed bliver interessen at betragte stillingskift – og ikke individer, hvilket resulterer i en metode til empirisk at identificere de givne segmenter som supplement til den teoridrevne tilgang. I modsætning til Toubøl, Larsen og Jensen er vores målsætning ikke at finde segmenternes faktiske grænser og egenskaber, men i stedet se på, hvordan arbejdsmarkedssegmenteringen ser ud for forskellige grupperinger, herunder primært de arbejdsløse og de beskæftigede.



%%%%%%%%%%%%%%%%%%%%%%%%%%%%%%%%%%%%%%%%%%%%%%%%%%%%%%%%%%%
\subsection{Jobmobilitet}
%%%%%%%%%%%%%%%%%%%%%%%%%%%%%%%%%%%%%%%%%%%%%%%%%%%%%%%%%%%

Jobmobilitet er et komplekst fænomen. Andersen med flere skelner mellem job-til-job mobilitet, erhvervsmæssig mobilitet,  og beskæftigelsesmobilitet. Job-til-job mobilitet referer til en persons skifte fra en arbejdsgiver til en anden. Dette kan defineres og måles som andelen af arbejdstagere, som har skiftet job for nyligt eksempelvis inden for det seneste år (“Current job mobility”), jobmobilitet i et livsperspektiv ved at se på jobskifte og arbejdsløshedsperioder gennem et helt arbejdsløshed (“Job mobility in a life course perspektive”), motivationer for at skifte job i fremtiden (“expected job mobility”) og skelen mellem tvunget og frivillige jobtransitioner (“forced versus voluntary job transitions”) \parencite[16]{Andersen2008}.

Erhvervsmæssig mobilitet hænger ofte sammen med karriere avancement og referer til ens person skifte i beskæftigelsesstatus, hvilket vil sige et skifte af jobprofil- og indhold. Her kan man skelne mellem intern mobilitet, som referer til bevægelser inden for et arbejdssted og ekstern mobilitet, som referer til bevægelser mellem arbejdssteder. Erhvervsmæssig mobilitet kan defineres og måles som skifte i arbejdsgrupper, defineret ved ISCO-klassifikationssystemet, jobtitler eller arbejdsindhold \parencite[17]{Andersen2008}.

Beskæftigelsesmobilitet referer til skifte ind og ud af beskæftigelse, hvilket kan være mellem beskæftigelse, arbejdsløshed  og uden for arbejdsstyrken (ikke aktiv) eller mellem forskellige typer af beskæftigelse såsom deltids- og fuldtidsjob eller midlertidig og permanent beskæftigelse \parencite[19]{Andersen2008}.


%%%%%%%%%%%%%%%%%%%%%%%%%%%%%%%%%%%%%%%%%%%%%%%%%%%%%%%%%%%
\subsection{Perspektiver på arbejdsløshed som overgange og processer} 
%%%%%%%%%%%%%%%%%%%%%%%%%%%%%%%%%%%%%%%%%%%%%%%%%%%%%%%%%%%
%%%% Jens: Gør mere ud af status passage og marginalisering. Måske mere Van Gennep - Bourdieu har skrevet en lille tekst som hedder Rites Of Institutions, som er et modspil til Van Gennep i forhold til hvem som overhovedet har mulighed for at lave disse overgange - eliten er bedre til at være studerende, og man er bedre til kvinde i forhold til....

Status passagemodellen og marginaliseringsperpektivet er to perspektiver som er mere generelle end de struktur- og aktørbaserede teorier, fordi de i højere grad beskæftiger sig med perspektiver som kan overføres på arbejdsmarkedet såvel som på andre området. Vi mener, at perspektiver er særlig anvendelige til at bløde kategorier som arbejdsløshed, beskæftigelse og uden for arbejdsstyrken op, så de mere bliver overgange eller processer frem for bestemte faser. %%%%% Jens: Endelig nærmer vi os noget vi skal bruge til noget...

Ezzy anvender begrebet jobtab i stedet for arbejdsløshed\footnote{Denne problematik adskiller sig på den ene side fra andre veje til arbejdsløshed såsom dimittendledighed eller et vende tilbage til arbejdsstyrken igen og veje ud af beskæftigelse såsom pensionering, orlov eller at tage en uddannelse \parencite[48]{Ezzy1993}.} og sammenligner det at miste sit job med at gå i gennem et skilsmisseforløb, et sygdomsforløb eller opleve et dødsfald i familien. Disse overgange skal forstås i forlængelse af Glaser og Strauss' definition af en \textbf{status passage} som et individs ”movement into a different part of a social structure, or loss or gain of privilige, influence, or power, and changed behaviour” \parencite[48]{Ezzy1993}. En status passage er en del af individets biografi, hvilket involverer samtidige og tidligere erfaringer, som påvirker meningen der tillægges arbejdsløsheden. Ezzy skelner mellem integrative passager og afhændelsespassager\footnote{Denne skelnen er foretaget med udgangspunkt i Van Gennep skelnen mellem separation (for eksempel begravelse), transition (for eksempel jobskifte) og integration (for eksempel ægteskab) som kategorier for sociale passager\parencite[48]{Ezzy1993}.}. Integrative passager er oftest en overgangsperiode efterfulgt af integration i en ny status gennem en ceremonial proces som for eksempel bryllupper. Afhændelsespssager er en separation fra en status som oftest er en længerevarende overgangsfase med en usikker varighed for eksempel skilsmisser, sygdomsforløb, arbejdsløshed eller dødsfald \parencite[49]{Ezzy1993}. En afhændelsespassager fører ikke nødvendigvis i sig selv til mentale problemer, mistrivsel eller eksklusion, da det afhænger af den enkeltes identitet og selvopfattelse i relation til andre og samfundet og en lang række andre faktorer samt om afhændelsespassagen efterfølges af en reintegrativ passage, hvor den enkelte får en ny status eksempelvis et nyt job \parencite[32]{Noerup2014}. %%%%% Jens: Interessant. Der burde stå noget à la, at vi mener overgange og faser er særlig relevante. Sætte det i spil over for fx den økonomiske teori 

\textbf{Marginaliseringsperspektivet} anvendes både på arbejdsmarkedet, men også en række andre områder. Ifølge Elm larsen skal marginaliseringsperspektivet ses som en midterkategori mellem inklusion og eksklusion. Larsen definerer eksklusion som en ufrivillig ikke-deltagelse gennem forskellige typer af udelukkelsesmekanismer og -processer, som det ligger uden for indvidets og gruppens muligheder at få kontrol over \parencite[137]{Larsen2009}. Hermed er Larsen kritisk over for Luhmanns dikotomi inklusion/eksklusion, som Larsen i dens binære form ikke mener er særlig hensigtsmæssig i forhold til at beskrive virkeligheden \parencite[130]{Larsen2009}. Derfor argumenter han for, at marginalisering kan anvendes som en midtergruppe mellem de to, hvor individet bevæger sig i en proces mod inklusion eller eksklusion. På baggrund af dette har vi tegnet følgende model:\footnote{Modellen er også inspireret af lignende modeller benyttet af Lars Svedberg \parencite[44]{Svedberg1995} og Catharina Juul Kristensen \parencite[18]{Kristensen1999}.} som fremgår af tabel \ref{tab_marginaliseringsmodel_1}. 
% 
\begin{table}[H] \centering
\caption{Model over marginalisering}
\label{tab_marginaliseringsmodel_3}
\begin{tabular}{@{} m{3,4cm} c m{3,6cm} c m{3,6cm} @{}} \toprule
\textbf{Inkluderet} & & \textbf{Marginaliseret} & & \textbf{Ekskluderet} \\ \midrule
\end{tabular} \end{table} %%%%%
\begin{table}[H] \centering
\label{tab_marginaliseringsmodel}
\begin{tabular}{@{} m{5,9cm} m{5,9cm} @{}} 
  \textbf{Marginaliseringsproces} & \textbf{Eksklusionsproces} \\  
  --------------------------------------------> & --------------------------------------------> \\ 
\end{tabular} \end{table} %%%%%
\begin{table}[H] \centering
\label{tab_marginaliseringsmodel}
\begin{tabular}{@{} m{12,3cm} @{}} 
  \textbf{Inklusionsproces} \\  
  <--------------------------------------------------------------------------------------------- \\ \bottomrule
\end{tabular} \end{table}
%
Den første proces består af individer som går fra at være inkluderet til at indgå i en proces i retning mod marginalisering. Den anden proces består af individer som går fra at være marginaliseret til at indgå i en proces i retning mod inklusion. Marginaliseringsperspektivet giver hermed mulighed for at se de  bevægelser på arbejdsmarkedet inden for kategorier som ikke rigide.

%%% Søren: Bauman. Bauman skriver også om eksklusion - eksklusion som kan kædes sammen med diskuserne \parencite[128-129]{Bauman2002}

%%% Søren: Lav tegninger som kombinerer mobilitet med marginalisering


%%%%%%%%%%%%%%%%%%%%%%%%%%%%%%%%%%%%%%%%%%%%%%%%%%%%%%%%%%%
\subsection{Flexicurity og mobilitet på det danske arbejdsmarked}
%%%%%%%%%%%%%%%%%%%%%%%%%%%%%%%%%%%%%%%%%%%%%%%%%%%%%%%%%%%

I 1980'erne foretog Boje en stor analyse af danske arbejdsmarkedssegmenter. Resultatet var, at det høje institutionaliseringsniveau, fleksibiliten, den aktive arbejdsmarkedspolitik og det relativt lige og høje uddannelsesniveau gør Danmark til til et særligt arbejdsmarkedstilfælde med en række submarkeder frem for få store segmenter, som den gængse arbejdsmarkedssegmenteringsteori ville forvente \textbf{\parencite{Boje1986}.} %%%% \parencite[11]{Touboel2013} (Boje 1985, Boje 1990, Boje og Toft 1989).



Det danske arbejdsmarked er et af de meste fleksible i Europa \parencite[722]{Jensen2011} \parencite[140]{Madsen2006} \parencite[4]{Andersen2008}. Den udvikling som det danske arbejdsmarked har været igennem har ansporet ideen om eksistensen af et arbejdsmarkedsforhold, som kaldes “den danske model” af flexicurity\footnote{Flexicurity kan ses som en betingelse for arbejdsmarkedet, hvor visse forbindelser mellem fleksibilitet (“flexibility”) og sikkerhed (“security”) kan observeres. I flexicurity-litteraturen diskuteres fleksibilitet oftest vedrørende emner såsom numerisk fleksibilitet (hyringer og fyringer), funktionel fleksibilitet (ansatte udfører varierede arbejdsopgaver), lønfleksibilitet (tilpasse lønningerne til ændringer i markedsforholdene), og arbejdstidsfleksibilitet. Fleksibilitet diskuteres oftest vedrørende emner såsom jobsikkerhed (risiko for at blive fyret), beskæftigelsessikkerhed (mulighed for at få et nyt job), indkomstsikkerhed (arbejdsløshedsunderstøttelse) og  velfærdssikkerhed (adgang til sundhedsydelser, uddannelse med videre under arbejdsløshed). Flexicurity-litteraturen forsøger at transcendere skellet mellem velfærd og beskæftigelsesegnethed ved, at velfærden (eller sikkerheden) gør arbejdstagerne mere fleksible i forhold til deres eksisterende beskæftigelse, hvilket gør det muligt for virksomhederne at være fleksible i deres forretningsstrategier \parencite[723]{Jensen2011}.}. Først er det værd at nævne, at den danske model er karakteriseret af et fleksibelt arbejdsmarked med store strømme ind og ud af beskæftigelse og arbejdsløshed. Et lavt niveau af \textit{jobbeskyttelse}\footnote{Jobbeskyttelse (“Employment protection legislation”) er et udtryk som er udbredt blandt kredse af økonomer, som omfatter alle former for beskyttelsesforanstaltninger for beskæftigelse uanset om det kommer fra lovgivning, domstolsafgørelser eller kollektive overenskomster. En af de mere hyppigst anvendte mål for jobbeskyttelse er den såkaldte “Employment Protection Legislation Index” udarbejdet af OECD \parencite[50-51]{OECD1999}.} giver mulighed for arbejdsgivere at justere antallet af medarbejdere på den enkelte arbejdsplads gennem ansættelse og fyring, således kan arbejdsgiveren frit tilpasse arbejdsstyrken til ændrede økonomiske vilkår. Dernæst er det værd at nævne, at den danske model også er karakteriseret af et \textit{generøs system for økonomisk støtte til de arbejdsløse}. Til sidst er det værd at nævne, at den danske model er karakteriseret ved en \textit{aktiv arbejdsmarkedspolitik} rettet mod at opkvalificere de arbejdsløse, der ikke kan vende tilbage direkte til beskæftigelse \parencite[140]{Madsen2006}. Ifølge Madsen er har den aktive arbejdsmarkedspolitik to vigtige funktioner. For det første skal den aktive arbejdsmarkedspolitik forbedre beskæftigelsesegnetheden for de arbejdsløse, som ikke har været i til at vende direkte tilbage til beskæftigelse efter en kort periode med arbejdsløshed især ved at forbedre kvalifikationerne hos arbejdsløse og dermed deres muligheder for at blive genansat. For det andet skal den aktive arbejdsmarkedspolitik giver incitament til jobsøgning ved at kontrollere for tilgængeligheden af jobs, hvilket fungerer som en trussel (pisk) eller motivationseffekt (gulerod) \parencite[153]{Madsen2006}. Den danske flexicurity giver hermed på den ene side arbejdstagere høj social sikkerhed og arbejdsløshedsforsikring, og på den anden side gør det det relativt nemt og billigt for arbejdsgiverne at hyre og fyre, så det giver mulighed for at justere andelen af arbejdskraft i produktionen til efterspørgslen på markedet \parencite[11]{Touboel2013}.

Flexicurity på det danske arbejdsmarked er blevet udviklet gennem stærke arbejdsmarkedsinstitutioner. Til sammen danner staten, fagforeninger og arbejdsgiverorganisationer et arbejdsmarkedsforhold, som er særlig kendetegnet ved, at lønreguleringer, arbejdstid, arbejdsvilkår med videre i høj grad er forhandlet mellem fagforeninger og arbejdsgivere og ikke staten. Dette har resulteret i, at 67 procent af arbejdsstyrken organiseret i fagforeninger, 55 procent af af arbejdstagerne var ansat i virksomheder organiseret i arbejdsgiverorganisationer og kollektive overenskomster dækker omkring 80 procent\footnote{Tallen er fra 2007 \parencite[11]{Touboel2013}.} \parencite[11]{Touboel2013}. At det danske arbejdsmarked er fleksibelt med høj job mobilitet er også bekræftet af statistiske studier. I 2005 var andelen af arbejdstagere som skiftede job i Danmark det højeste i EU med 11,5 procent, som kan sammenlignes med et EU gennemsnit på 8,8 procent \parencite[21]{Andersen2008}. I 2006 var gennemsnittet for længden på en ansættelsesperiode i en virksomhed den laveste i Europa på ca. 4,8 år sammenlignet med EU gennemsnittet på 8,3 år \parencite[27]{Andersen2008}. Ifølge Toubøl, Larsen og Jensen viser disse, at mobiliteten på det danske arbejdsmarked er høj, men ikke noget om mobiliteten mellem sektorer eller arbejdsstillinger \parencite[11]{Touboel2013}  er et parameter, vi ligesom Toubøl, Larsen og Jensen benytter os af.




%%%%%%%%%%%%%%%%%%%%%%%%%%%%%%%%%%%%%%%%%%%%%%%%%%%%%%%%%%%
\subsubsection{Andet}

I 2015 foretog arbejderbevægelsens Erhvervsråd. \textbf{\parencite{Kirk2015}.}


%%%%%%%%%%%%%%%%%%%%%%%%%%%%%%%%%%%%%%%%%%%%%%%%%%%%%%%%%%%
\subsubsection{Opsummering}
%%%%%%%%%%%%%%%%%%%%%%%%%%%%%%%%%%%%%%%%%%%%%%%%%%%%%%%%%%%





























%Local Variables: 
%mode: latex
%TeX-master: "report"
%End: