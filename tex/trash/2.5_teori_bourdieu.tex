% -*- coding: utf-8 -*-
% !TeX encoding = UTF-8
% !TeX root = ../report.tex


%%%%%%%%%%%%%%%%%%%%%%%%%%%%%%%%%%%%%%%%%%%%%%%%%%%%%%%%%%%
\newpage \section{\textsc{Bourdieu \label{bourdieu}}}
%%%%%%%%%%%%%%%%%%%%%%%%%%%%%%%%%%%%%%%%%%%%%%%%%%%%%%%%%%%

% Vi betragter de arbejdsløse som sociale agenter, hvis handlinger er udtryk for en praktisk logik, der er indskrevet i deres kroppe gennem de livsbaner, de har haft, det vil sige de felter deres habitus er formet af \parencite[83-125]{Bourdieu1996} blah blah Bourdiues forestilling om sociale agenter, handlinger, praksis, habitus, sociale rum, livsbaner, felter, hanging on a thread, på kanten af arbejdsmarkedet blah blah blah.

% Arbejdets værdi og individets strategier for at komme i beskæftigelse er centralt i de to efterfølgende afsnit om den sociologisk-videnskabelige og økonomisk-videnskabelige tilgang til arbejdsløse og arbejdsløshed. Det at komme i beskæftigelse igen er ikke bare at komme i beskæftigelse igen. For farmaceuten er det noget andet at komme i beskæftigelse som farmaceut, som vedkommende har taget en lang videregående uddannelse for at have kompetencer til end at tage et arbejde som kassemedarbejder. Bourdieu skelner mellem “den store elendighed” og “den lille elendighed” \parencite[4]{Bourdieu1999}. Det at komme i beskæftigelse er et udtryk for at komme ud af  “den store elendighed” som arbejdsløsheden og alle de problemer som følger det at være arbejdsløs, men det at komme i beskæftigelse kan også være et udtryk for at vælge den “den lille elendighed”, som for består af forringede arbejdsvilkår og det at blive tvunget til at ofre sig lidt før man bliver ofret og samtidig være taknemmelig over, at man ikke hører til blandt de allersvageste. Hermed tilpasser farmaceuten sig arbejdsmarkedets umiddelbare behov ved i kraft af at være arbejdsløs accepterer samfundets objektive strukturer og derved får nogle realistiske forventninger i forhold til vedkommendes position i den sociale verden, hvor vedkommende som arbejdsløs i sidste ende må tage det arbejde hvad vedkommende kan få.

% Jens: Det vigtigste er ikke at få Bourdieu ind over det - Bourdieu skal ikke lede os af sporet i forhold til at kritisere søgemodellen


%%%%%%%%%%%%%%%%%%%%%%%%%%%%%%%%%%%%%%%%%%%%%%%%%%%%%%%%%%%
\subsection{På kanten af arbejdsmarkedet}
%%%%%%%%%%%%%%%%%%%%%%%%%%%%%%%%%%%%%%%%%%%%%%%%%%%%%%%%%%%


%%%%%%%%%%%%%%%%%%%%%%%%%%%%%%%%%%%%%%%%%%%%%%%%%%%%%%%%%%%
\subsection{Strukturel konstruktivisme}
%%%%%%%%%%%%%%%%%%%%%%%%%%%%%%%%%%%%%%%%%%%%%%%%%%%%%%%%%%%


%%%%%%%%%%%%%%%%%%%%%%%%%%%%%%%%%%%%%%%%%%%%%%%%%%%%%%%%%%%
\subsection{Opsummering}
%%%%%%%%%%%%%%%%%%%%%%%%%%%%%%%%%%%%%%%%%%%%%%%%%%%%%%%%%%%


%Local Variables: 
%mode: latex
%TeX-master: "report"
%End: