% -*- coding: utf-8 -*-
% !TeX encoding = UTF-8
% !TeX root = ../report.tex


%%%%%%%%%%%%%%%%%%%%%%%%%%%%%%%%%%%%%%%%%%%%%%%%%%%%%%%%%%%
\section{\textsc{Akademikerne} \label{}}
%%%%%%%%%%%%%%%%%%%%%%%%%%%%%%%%%%%%%%%%%%%%%%%%%%%%%%%%%%%

% Når man taler om uddannelsesgrupper opdeles de typisk i faglærte, ufaglærte og videregående uddannelser, de videregående uddannelser opdeles i de korte, de mellemlange og de lange, de lange videregående uddannelser i hmuanistiske, naturvidenskabelig, sundhedsvidenskabelige, samfundsvidenskabelige, tekniske, farmaceutiske, teologiske og så videre \parencite{Groes2014}. 

% For at skelne mellem på den ene side de forskellige udddannelsesgrupper og på den anden side opdelingen af de lange videregående uddannelser i faggrupper, kan man sige at førstnævnte tager udgangspunkt i den internationale uddannelsesklassifikation\footnote{International Standard Classification of Education (\texttt{ISCED}) placerer uddannelse på ti niveauer: førskoleniveau (børnehaveklasse), grundskoleniveau I (1.-6. klasse), grundskoleniveau II (7.-10. klasse/årgang), gymnasialt niveau I (10. uddannelsesår), gymnasialt niveau II (11.-12. uddannelsesår), korte videregående uddannelser (13.-14. uddannelsesår), mellemlange videregående uddannelser (15.-16. uddannelsesår), lange videregående uddannelser (17.-18. uddannelsesår) og forskerniveau (19.- uddannelsesår) (henvisning).} og sidstnævnte følger de danske universiteters inddeling på baggrund af hvilket fakultet, man er på. Sidstnævnte er problematisk, fordi psykologi på Aarhus Universitet for eksempel ligger under det sundhedsvidenskabelige fakultet, fordi det har rødder i sundhedsvidenskaben, mens det på Københavns Universitet ligger under det samfundsvidenskabelige fakultet. Vore indelinger af akademikre på arbejdsmarkedet bryder med begge selninger. Som det fremgår af kortet, kan man se alle de mørkeblå noder er akademikerarbejdsstillinger (eller viden på højeste niveau). Her er der 12 segmenter ud af de 32 forskellige som indeholder personer med viden på højeste niveau, som fremadrettet vil blive kaldt for akademisk arbejdskraft. De tre segmenter, vi ønsker at fremhæve er karakteriseret ved, at det er tre clusters med flest akademikere og som vi kalder for magisterclusteret, djøfclusteret og kreaclusteret\footnote{De ni andre segmenter er: 1) \emph{Udvikling og analyse af software og applikationer} (\texttt{2130}) og \emph{Edb teknisk arbejde, primaert programmoer} (\texttt{3120}), 2) \emph{Ingenioerer og arkitekter} (\texttt{2141}) og Fire forskllige typer af teknikerarbejde (\texttt{3112, 3115, 3118, 3181}), 3) \emph{Laege} (\texttt{2221}), 4) \emph{Tandlaege} (\texttt{2222}), 5) \emph{Jordemoder, overordnet sygepleje mv} (\texttt{2230}) og \emph{Sygeplejearbejde} (\texttt{3230}), 6) \emph{Kulturformidling og informationsarbejde, primaert bibliotekar} (\texttt{2430}), 7) \emph{Overordnet revisions og regnskabsarbejde, herunder registeret revisor og statsautoriseret revisor} (\texttt{2411}) og otte andre \texttt{DISCO}-kategorier med kontor-, administrations- og revisionsarbejde mv., 8) \emph{Religioest arbejde} (\texttt{2482}) og 9) \emph{Overordnet socialraadgivningsarbejde} (\texttt{2446}) og \emph{Administrativt arbejde vedr. offentlige ydelser og afgifter} (\emph{3440}).}

% I perioden 1996 til 2009 har de lange videregående uddannelser haft den største  procentvise tilvækst, men de talte alligevel mindre en halvdelen af antallet med en mellemlang mellemlang videregående uddannelse. %%%% Den samlede arbejdsstyrke har været nogenlunde konstant. 
% Nettoledigheden for de lange videregående uddannelser bevæger sig parallelt med de andre faggrupper i perioden 1996 til 2009 er nettoledighedsgennemsnittet 6000 personer. 
% % 
% \begin{table}[H] \centering
% \caption{Arbejdsstyrken fordelt på uddannelsesgrupper. Kilde: DST}
% \label{tab_uddannelse}
% \begin{tabular}{lrrrrrrr} \toprule
% 	& \multicolumn{1}{c}{Grundskole} & \multicolumn{1}{c}{GYM} & \multicolumn{1}{c}{EUD} & \multicolumn{1}{c}{KVU}	& \multicolumn{1}{c}{MVU} & \multicolumn{1}{c}{LVU} & Alle	\\ \midrule
% 1996	&	1.572.425	&	310.367	&	1.294.603	&	118.561	&	419.763	&	162.520	&	\\
% 1997	&	1.556.567	&	317.815	&	1.314.644	&	123.732	&	434.072	&	170.017	&	\\
% 1998	&	1.536.264	&	323.421	&	1.335.701	&	130.000	&	449.752	&	177.948	&	\\
% 1999	&	1.524.624	&	325.966	&	1.348.020	&	135.721	&	465.883	&	185.261	&	\\
% 2000	&	1.510.944	&	325.431	&	1.364.746	&	140.052	&	482.040	&	192.667	&	\\
% 2001	&	1.499.835	&	324.567	&	1.379.370	&	145.075	&	498.671	&	201.119	&	\\
% 2002	&	1.488.688	&	322.904	&	1.391.768	&	150.452	&	515.013	&	210.416	&	\\
% 2003	&	1.480.263	&	320.959	&	1.399.158	&	156.772	&	530.979	&	220.133	&	\\
% 2004	&	1.473.175	&	321.088	&	1.406.980	&	159.774	&	545.612	&	230.323	&	\\
% 2005	&	1.466.360	&	321.403	&	1.411.090	&	162.922	&	559.638	&	239.798	&	\\
% 2006	&	1.455.210	&	322.732	&	1.414.756	&	165.711	&	572.289	&	249.919	&	\\
% 2007	&	1.439.702	&	324.260	&	1.419.023	&	168.987	&	584.270	&	261.475	&	\\
% 2008	&	1.517.184	&	329.613	&	1.428.161	&	173.227	&	598.317	&	273.095	&	\\
% 2009	&	1.460.590	&	329.557	&	1.439.554	&	177.572	&	613.044	&	285.460	&	\\  \bottomrule
% \end{tabular} \end{table}
% %


%%%%%%%%%%%%%%%%%%%%%%%%%%%%%%%%%%%%%%%%%%%%%%%%%%%%%%%%%%%
\subsubsection{Magisterclusteren \label{}}
%%%%%%%%%%%%%%%%%%%%%%%%%%%%%%%%%%%%%%%%%%%%%%%%%%%%%%%%%%%

% 
Magisterclusteren består primært af meget forskelligt arbejde. Clusteren indeholder
 \begin{enumerate} [topsep=6pt,itemsep=-1ex]
   \item \emph{Arbejde med emner inden for fysik, kemi, astronomi, meteorologi, geologi og geofysik} (\texttt{2110})
   \item \emph{Arbejde med emner inden for de biologiske grene af naturvidenskab} (\texttt{2210})
   \item \emph{Dyrlaege} (\texttt{2223})
   \item \emph{Farmaceut} (\texttt{2224})
   \item \emph{Arbejde med emner inden for medicin, odontologi, veterinaervidenskab og farmaci i oevrigt} (\texttt{2229})
   \item \emph{Undervisning paa universiteter og andre hoejere laereanstalter} (\texttt{2311})
   \item \emph{Undervisning paa gymnasier, erhvervsskoler mv} (\texttt{2321})
   \item \emph{Folkeskolelaerer} (\texttt{2331})
   \item \emph{Undervisning af handicappede mennesker} (\texttt{2341})
   \item \emph{Arbejde vedr. undervisning i oevrigt, primaert kursusvirksomhed} (\texttt{2350})
   \item \emph{Samfundsvidenskabeligt arbejde og historie} (\texttt{2442})
   \item \emph{Sprogvidenskabeligt arbejde} (\texttt{2444})
   \item \emph{Psykolog} (\texttt{2445})
   \item \emph{Arbejde med administration af lovgivningen inden for den offentlige sektor} (\texttt{2470})
   \item \emph{Blandet undervisning i folkeskoler, erhvervsskoler, gymnasier og hoejere laereanstalter samt forskningstilrettelaeggelse og kontrol af undervisningsarbejde} (\texttt{2930}) 
 \end{enumerate}
% 


%%%%%%%%%%%%%%%%%%%%%%%%%%%%%%%%%%%%%%%%%%%%%%%%%%%%%%%%%%%
% \subsubsection{Kreaclusteren \label{}}
%%%%%%%%%%%%%%%%%%%%%%%%%%%%%%%%%%%%%%%%%%%%%%%%%%%%%%%%%%%

% % 
% Kreaclusteren består af forskelligt kreativt arbejde. Clusteren indeholder:
%  \begin{enumerate} [topsep=6pt,itemsep=-1ex]
%    \item \emph{Ledelse af virksomhed faerre end 10 ansatte} (\texttt{1300})
%    \item \emph{Alment journalistisk arbejde og skribentarbejde} (\texttt{2451}) 
%    \item \emph{Illustrationsgrafisk arbejde vedr. formidling og kunstnerisk arbejde vedr. billedkunst og formgivning} (\texttt{2452}) 
%    \item \emph{Kunsterisk arbejde indenfor dans, musik, koreografi, skuespil eller film} (\texttt{2481}) 
%    \item \emph{Blandet journalist, kunst og skribentarbejde} (\texttt{2945}) 
%    \item \emph{Arbejde med lyd, lys og billeder ved film og teaterforestillinger mv samt betjening af medicinsk udstyr} (\texttt{3130}) 
%    \item \emph{Arbejde inden for kunst, underholdning og sport} (\texttt{3470})
%  \end{enumerate}
% % 


%%%%%%%%%%%%%%%%%%%%%%%%%%%%%%%%%%%%%%%%%%%%%%%%%%%%%%%%%%%
% \subsubsection{Djøfferclusteren \label{}}
%%%%%%%%%%%%%%%%%%%%%%%%%%%%%%%%%%%%%%%%%%%%%%%%%%%%%%%%%%%

% % 
% Djøfferclusteren består primært af forskelligt økonomi- og juridisk arbejde. Clusteren indeholder
%  \begin{enumerate} [topsep=6pt,itemsep=-1ex]
%    \item \emph{Lovgivningsarbejde samt ledelse i offentlig administration og interesseorganisationer} (\texttt{1100}) 
%    \item \emph{Arbejde med matematik, aktuariske og statistiske metoder} (\texttt{2120}) 
%    \item \emph{Udvikling og planlaegning af personalespoergsmaal} (\texttt{2412}) 
%    \item \emph{Ledelsesraadgivning og andre specialfunktioner indenfor organisation} (\texttt{2419}) 
%    \item \emph{Advokat, dommer og andet juridisk arbejde} (\texttt{2420}) 
%    \item \emph{Oekonomi} (\texttt{2441}) 
%  \end{enumerate}

%%%%%%%%%%%%%%%%%%%%%%%%%%%%%%%%%%%%%%%%%%%%%%%%%%%%%%%%%%%

% % 
% \begin{table}[H] \centering
% \caption{Arbejdsløshed fordelt på uddannelsesgrupper. Kilde: DST}
% \label{tab_uddannelse_arbejdsloeshed}
% \begin{tabular}{lrrrrrr} \toprule
% 	& \multicolumn{1}{c}{Grundskole} & \multicolumn{1}{c}{STX} & \multicolumn{1}{c}{EUD} & \multicolumn{1}{c}{KVU}	& \multicolumn{1}{c}{MVU} & \multicolumn{1}{c}{LVU} & Alle	\\ \midrule
% 1996	&	81.151	&	13.622	&	60.666	&	6.145	&	11.871	&	6.418	&	\\
% 1997	&	68.797	&	11.806	&	58.219	&	5.610	&	11.304	&	6.641	&	\\
% 1998	&	55.107	&	8.872	&	45.082	&	4.204	&	8.799	&	5.221	&	\\
% 1999	&	47.700	&	7.363	&	41.245	&	3.868	&	8.517	&	5.090	&	\\
% 2000	&	47.203	&	7.031	&	42.168	&	4.250	&	8.964	&	5.039	&	\\
% 2001	&	43.330	&	6.396	&	40.099	&	4.143	&	8.315	&	4.906	&	\\
% 2002	&	43.826	&	7.018	&	43.365	&	5.310	&	9.853	&	6.620	&	\\
% 2003	&	52.604	&	8.981	&	53.668	&	6.801	&	13.079	&	8.639	&	\\
% 2004	&	48.517	&	8.307	&	47.637	&	5.850	&	12.629	&	8.075	&	\\
% 2005	&	39.159	&	7.106	&	35.808	&	4.706	&	10.888	&	6.975	&	\\
% 2006	&	30.184	&	5.715	&	24.019	&	3.430	&	8.451	&	5.776	&	\\
% 2007	&	23.436	&	4.133	&	16.806	&	2.459	&	5.888	&	4.813	&	\\
% 2008	&	15.707	&	2.669	&	15.353	&	1.916	&	3.824	&	3.297	&	\\
% 2009	&	29.452	&	5.283	&	39.438	&	4.581	&	8.227	&	6.459	&	\\ \bottomrule
% \end{tabular} \end{table}
% %

%%%%%%%%%%%%%%%%%%%%%%%%%%%%%%%%%%%%%%%%%%%%%%%%%%%%%%%%%%%

% Toubøl, Larsen og Jensen \parencite[3]{TouboelLarsenJensen2013} \parencite[4]{TouboelLarsen2015} kan beskæftigelsesmønstre have en funktionel, en institutionel og en normativ form. Den funktionelle form opstår, når man skal have de rette færdigheder for at have mulighed for at indtage en bestemt arbejdsstilling. Den institutionelle form opstår, når man skal have det rette certifikat for at komme i betragtning til en bestemt arbejdsstilling. Og den normativ form opstår, når der ekskluderes personer med et bestemt køn eller en bestemt race fra visse arbejdsstillinger.

%%%%%%%%%%%%%%%%%%%%%%%%%%%%%%%%%%%%%%%%%%%%%%%%%%%%%%%%%%%
% Trash
%%%%%%%%%%%%%%%%%%%%%%%%%%%%%%%%%%%%%%%%%%%%%%%%%%%%%%%%%%%


%Local Variables: 
%mode: latex
%TeX-master: "report"
%End: 