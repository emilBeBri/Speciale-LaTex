% -*- coding: utf-8 -*-
% !TeX encoding = UTF-8
% !TeX root = ../report.tex

\chapter{metode} \label{metode}



\section{Netværksanalyse \label{}}




\subsection{MONECA}

Social netværksanalyse er en relationel analysemetode. Vi benytter en udgave af denne, som Toubøl \& Larsen har udviklet til at inddele sociale grupper i større overgrupper, for at sige det kort. Givet at sociale grupper og deres relationer til andre grupper er komplekse, samt at mange grupper kan have relationer til mange andre grupper, hvordan kan man så inddele disse grupper i meningsfulde overgrupper? Formuleringens simplicitet bør ikke tages for enfoldighed. Der er tale om et af sociologiens grundlæggende spørgsmål: Hvad konstituerer en social gruppe, og efter hvilke principper finder en sådan konstituering sted? Samt den særlig opgave det er for sociologien at finde ud af hvilke principper, der bør ligge til grund for en inddeling, som ingen i sagens natur kan have det fulde overblik over. 




Baseret på disse gruppers relationer, er Moneca-algoritmen udviklet 





\subsection{Netværksanalytisk videnskabsteori}


\subsection{Borudieus videnskabsteori}








#### ideer til afsnit

- Disco
- opbygning af relativ risiko gennem total antal beskæftigede, teoretisk diskussion (se Emils logbog)














%Local Variables: 
%mode: latex
%TeX-master: "report"
%End: 