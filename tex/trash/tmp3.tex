
%%%%%%%%%%%%%%%%%%%%%%%%%%%%%%%%%%%%%%%%%%%%%%
\section{Arbejdsmarkedssegmenteringsteori. Eller hvordan jeg lærte at holde op med at bekymre mig og elske Bomben.}
%%%%%%%%%%%%%%%%%%%%%%%%%%%%%%%%%%%%%%%%%%%%%%


Arbejdsmarkedssegmenteringsteorien opstod historisk som en kritik af og et alternativ til den nyklassicistiske økonomiske skole, hvori arbejdsmarkedet behandles stort set som et hvilket som helst andet varemarked. Kernen i denne kritik er en forståelse af individerne på arbejdsmarkedet som drevet ikke af nyttemaksimerende økonomiske interesser, men bestemt gennem de sociale relationer og institutioner, det er indspundet i \parencite[171]{Boje1986}. Jeg vil i det følgende kalde det segmenteringsteori\emph{erne}, da der er tale om teorier med nogle fælles antagelser, men vidt forskellige ideer om hvilke sociale dynamikker, der er vigtige indenfor disse antagelser.

Denne afhandling vil tage udgangspunkt i segmenteringsteorierns grundlæggende forståelse af  arbejdsmarkedet og individerne på det, og kun sjældent forholde sig til den neoklassicistiske økonomiske skole. For en kritisk gennemgang af denne skole, med særligt fokus på dennes forståelse af arbejdsløshed, kan man læse Gravholt-Nielsens (2016) afhandling. 

Arbejdsmarkedets funktionsmåde og beskæftigelsespostioner på arbejdsmarkedet er i disse teorier set som "\emph{(...) den vigtigste institution, hvorigennem social ulighed og sociale konflikter om fordelingen af samfundets materielle og immaterielle goder opstår og afgøres}" \parencite[10]{Boje1985}. Selvom denne forståelse indenfor sociologien har været kritiseret de sidste femoogtyve år \parencite[23]{Scott2000}, er det udgangspunktet for denne afhandling, at segmenteringsteorien og klasseteorier har ret i denne grundantagelse, om end det anerkendes at betydningen af position på arbejdsmarkedet, særligt i forbrugsvaner og individernes selvforståelse, har ændret sig markant i løbet af det tyvende århundrede. 

Arbejdsmarkedet opfattes i segmenteringsteorierne som bestemt ud fra samfundets struktur og organisering, det vil sige forholdene i produktionen af varer, de institutioner og deres normer, der er opstået som følge af historiske konflikter, samt de forskellige sociale grupper, som arbejdsmarkedet udgøres af \parencite[9]{Boje1985}. Det  vil sige, at disse barierer, som forhindrer fri mobilitet på arbejdsmarkedet, ikke ses som barrierer for den fri allokering af arbejdskraft ud fra udbud/efterspørgsmekanismer i et perfekt marked. I stedet ses disse barrierer for mobilitet som udtryk for grundlæggende sociale mekanismer, som arbejdsmarkedet består af. Arbejdsmarkedet er, som Boje siger, derfor mere kendetegnet af opdeling end af fri konkurrence \parencite[8]{Boje1985}. Segmenteringsteoretikerne trækker på den institutionalistiske skole, der ser økonomien som et åbent system, og en del af en samfundsmæssig helhed. Heri ligger, at individernes adfærd på arbejdsmarkedet, eksempelvis allokering af arbejdskraft i bestemte virksomhedstyper, arbejdsstillinger og erhvervsgrupper er bestemt ud fra to forhold: Dels af samfundsmæssige institutioner og organisationer, dels af de normer og præferencer, som individerne har med sig fra de sociale grupper, de kommer fra og bevæger sig i \parencite[9]{Boje1985}. 





lønarbejdernes uddannelsesniveau

beskæftigelsesstruktur






 de muligheder og begrænsninger, som udgør 










Sociologien har altid beskæftiget sig med



- arbejdsmarkedets institutionelle forhold barrierer 





- neoklassisk økonomisk teori behandler arbejdsmarkedet analogt til varemarkedet (find refencer fra Søren og henvis til hans speciale)





######## noter til segmenteringsteori 

1. Boje 1986 - kort artikel, god opsummering af hans teori

2. Leon Tarridi 1998
meget grundig gennemgang af hvad det handler om det her, med bl.a Adam Smith, John Stuart Mill. gennemgår de centrale teoriretninger i segmenteringsteori.

3. Cain 1976
Samme som ovenstående, men ældre, fra 1976. Den er oldschool. Humlen er at det er som om de ikke kommer videre med det, ifht 1998 teksten, dvs de kører i det samme.

4. Door 2012
4 sider teori, derefter hans egen teori. Det skal være *kort*.

Start med at læse Door. Humlen i 2) og 3) er at de gør det meget langt, og det gammeldags. De andre tekster.

Henvisning til Grusky tekst i Sørens speciale. Her snakker han 





































