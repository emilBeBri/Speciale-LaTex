% -*- coding: utf-8 -*-
% !TeX encoding = UTF-8
% !TeX root = ../report.tex

\chapter{metode} \label{metode}


\section{Hvad er så vores endelige population \label{}}


\subsection{Segmenteringer på arbejdsmarkedet}

Vi har segmenteret et arbejdsmarked på baggrund af DISCO-koder

Samt lavet skifte på baggrund af SOCIO, SOCSTIL



\subsection{Et segmenteret arbejdsmarked af ledige}

Vi har udvalgt ledige på baggrund af SOCIO, SOCSTIL



\subsection{Alder}

Vi har indskrænket arbejdsmarkedet i forhold til aldersgruppen 16-70 år.

I flere andre ledighedsstatistikker indskrænkes ledige til det år, hvor man har mulighed for at gå på pension. I perioden 1996 til 2009 ville det derfor være hensigtsmæssigt at bruge aldersgruppen 16-65 år. Vi har dog valgt at udvide aldersgruppen med fem år, fordi vi gerne vil have en så bred gruppe af ledige som kommer ind og ud af arbejdsmarkedet.


Når vi indskrænker arbejdsmarkedet mister vi omkring -8508 personer som enten er yngre eller ældre. Det er især inden for kategorien *6120: Arbejde med dyr og skovbrug, primaert landbrugsmedarbejder og landmand", hvor der er mange som falder fra. Dette kan skyldes, at der findes mange landmænd som er over 70 år gamle som fortsætter selvstændig erhvervsdrivende til langt over de 70 år samtidig med, at de i periode er uden for arbejdsmarkedet.







%Local Variables: 
%mode: latex
%TeX-master: "report"
%End: 