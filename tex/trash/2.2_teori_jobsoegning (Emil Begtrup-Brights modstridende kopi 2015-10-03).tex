% -*- coding: utf-8 -*-
% !TeX encoding = UTF-8
% !TeX root = ../report.tex


%%%%%%%%%%%%%%%%%%%%%%%%%%%%%%%%%%%%%%%%%%%%%%%%%%%%%%%%%%%
\newpage \section{\textsc{Jobsøgningsteori} \label{}}
%%%%%%%%%%%%%%%%%%%%%%%%%%%%%%%%%%%%%%%%%%%%%%%%%%%%%%%%%%%
%%%% Jens: 	- Det er vigtigt, at I demonstere, at I har gang i noget vigtigt. Mortensen vandt Nobelprosen for søgemodellen, så det her er big business. Sæt søgeteorien op på en pedistal, og gør det til en værdi modstander, da man hurtigt kan lave en stråmand og banke ham ned.

Fra 1970'erne og fremefter har økonomisk teori haft en betydelig gennemslagskraft i arbejdsløshedsforskning samtidig med at være det ideologiske grundlag for de lovændringer på arbejdsløshedsområdet, som har været med til at sænke arbejdsløshedsydelserne og været med til at kontrollere de arbejdsløses bestræbelser på at få et arbejde i Danmark jævnfør \ref{teori_arbejdsloeshed} \parencite[19]{Andersen2003} \parencite[1679]{Atkinson1991}. De økonomiske modeller er orienteret mod at sikre en effektiv økonomi og et effektivt fungerende arbejdsmarked. Disse modeltyper som økonomerne benytter sig af svarer til \textit{covering law-modellen}\footnote{Carl Hempel udviklede covering law-modellen for at forstå naturvidenskabelige forklaringer \parencite[15]{Hedstroem2005}.}, hvor et foreliggende faktum forklares ud fra andre udsagn, herunder minimum en almen lov, som det pågældende faktum derfor er underordnet. Det vil sige, at for at forklare et faktum, bygger økonomerne oftest deres modeller på en eller flere antagelser.

Dette teoretiske afsnit om økonomiske forståelser af arbejdsløshed indeholder først en definition af arbejdsløshed. Herefter følger arbejdsløshed på lang sigt og modellerne trade-off mellem arbejde og fritid, basal jobsøgningsteori og principal-agent-modellen Til sidst følger arbejdsløshed på kort sigt samt modellen hysterese.

Modellerne trade-off mellem arbejde og fritid, den basale jobsøgningsteori og principal-agent-modellen, som vil blive gennemgået i forbindelse med arbejdsløshed på lang sigt beskæftiger sig primært med arbejdsløshed som friktionsledighed.


%%%%%%%%%%%%%%%%%%%%%%%%%%%%%%%%%%%%%%%%%%%%%%%%%%%%%%%%%%%
\subsection{Udbudsteori}
%%%%%%%%%%%%%%%%%%%%%%%%%%%%%%%%%%%%%%%%%%%%%%%%%%%%%%%%%%%

(1) Som tidligere beskrevet består den neoklassiske arbejdsmarkedsøkonomi overordnet af den marginale produktivitetsteori, som er baseret på arbejdsgivernes profitmaksimerende adfærd, og udbudsteori, som er baseret på nyttemaksimering af arbejdskraft. Udbudsteori kan tage formen af human kapital og trade-off'et mellem arbejde og fritid \parencite[1216]{Cain1976}.

(2) \textbf{Trade-off'et mellem arbejde og fritid} bygger på en antagelse fra den neoklassiske økonomiske teori om, at for at have et arbejde skal man have valgt at tage et. Individet har begrænset tid, og står derfor over for et trade-off mellem arbejde og fritid\footnote{Implicit er dette også et trade-off mellem at forbruge goder og forbruge fritid \parencite[5]{Cahuc2004}.}. Det betyder, at når en handling vælges frem for andre mulige handlinger er der omkostninger herved \parencite[389]{Mankiw2011}. Eksempelvis, hvis hvis man vælger at holde en times fri frem for at arbejde den samme time til en timeløn på 100 kroner, så er omkostningerne herved altså 100 kroner\footnote{Trade-off'et mellem arbejde og fritid er senere hen blevet mere kompleks. Eksempelvis er den tid man ikke arbejder ikke nødvendigvis blot fritid, fordi denne tid kan bruges på produktion i husholdningen, som jo også er et supplement for ens lønindtægter \parencite[14]{Cahuc2004}.}. Ifølge Halvorsen hænger trade-offet mellem arbejde og fritid sammen med diskursen om at lønarbejdet er lig med et nødvendigt onde. Et onde, fordi mennesket hellere vil have fritid, og nødvendigt, fordi mennesket er nyttemaksimerende \parencite[26]{Halvorsen1999}.

(3) Gary Beckers human kapital bygger på teorien “compensating differentials” udviklet af Adam Smith. Teorien foreslår, at arbejdstagere står over for trade-offs, når de vælger et arbejde, afvejer de løn over for bekvemmeligheder, herunder arbejdsbetingelser, jobkvalifikationer; stabilitet, prestige og  vanskeligheden for succes. \parencite[1180]{Daw2012}.

(4) Gary Becker udviklede i 1960'erne den mest udbredte forståelse af human kapital. Her skal human kapital forstås ved, at uddannelse, certificeret med et diplom, ofte ses som en nødvendighed for at have en velbetalt arbejde \parencite[60]{Cahuc2004}. Teorien antager, at enkeltindividet vælger at tage en uddannelse, som en investering i viden med en forventning om, at det vil bidrage til at øge vedkommendes produktivitet og derved øge vedkommendes fremtidige indkomst \parencite[69]{Cahuc2004}. Øget produktion fører ikke nødvendigvis til en højere løn, fordi arbejdstageren skal kunne være profitabel i mere end et arbejde for at arbejdsgivere skal kunne have interesse i at konkurrere om vedkommendes arbejdskraft\footnote{Derfor skelner Becker mellem mellem specifik og generel træning. Specifik kapital relaterer sig mod træning som øger produktiviteten i et specifikt job, mens generel kapital relaterer sig mod træning, som øger produktionen i alle jobtyper \parencite[70]{Cahuc2004}.} \parencite[70]{Cahuc2004}. I den forstand bliver de arbejdsløse til dem, hvis formåen, færdigheder og produktionskapicitet er utilstrækkelig til at gøre det værd for arbejdsgivere at hyre dem til lønninger på de gældende markedsvilkår \parencite[70]{Doeringer1971}. %%%% Søren: Alternativt kunne man bruge (Becker 1993) som reference i stedet for (Cahuc og Zylberberg 2004).


%%%%%%%%%%%%%%%%%%%%%%%%%%%%%%%%%%%%%%%%%%%%%%%%%%%%%%%%%%%
\subsection{Den basale jobsøgningsmodel}
%%%%%%%%%%%%%%%%%%%%%%%%%%%%%%%%%%%%%%%%%%%%%%%%%%%%%%%%%%%

\textbf{Den basale jobsøgningsmodel} kaldes også for den partielle model og blev udviklet i 1970'erne af blandt andet McCall og Mortensen \parencite[109]{Cahuc2004}. Modellen forklarer, hvorfor friktionsledighed forekommer og bryder hermed med to væsentlige neoklassiske perspektiver. Den første perspektiv er, at der er fuld beskæftigelse, når arbejdsmarkedet er i equilibrium. Her vil en fyret arbejdstager automatisk vil finde et nyt job til markedsløn, fordi alle arbejdstagere kan varetage alle jobs \parencite[163]{Mankiw2007}. Det anden perspektiv er, at individer har \textit{fuldkommen information} om arbejdsmarkedet, og derfor ikke har brug for tid til at søge arbejde. Jobsøgningsmodellen tager udgangspunkt i jobsøgning som en proces, hvor individer under ufuldkommen information har brug for tid til at finde de rette jobs i forhold til deres præferencer og færdigheder med henblik på at få den højeste løn som betaling for sin ydelse \parencite[108]{Cahuc2004}. Den optimale søgestrategi for en person som leder efter arbejde består i at vælge en \textit{reservationsløn}. En reservationsløn er den laveste løn en person er villig til at acceptere, hvilket betyder, at alle jobtilbud under dette beløb afvises \parencite[114]{McCall1970}. Og jo højere reservationslønnen er sat til, jo længere vil den gennemsnitlige arbejdsløshedsperiode være \parencite[848]{Mortensen1970}. %%%% Jens markeret. Har økonomerne empirisk belæg for at sige det her. Her kunne man passende skrive, at dette er en antagelse, som vi vil kigge på, og som vi skal se senere viser vores data at denne antagelse har implikationer. Er det muligt faktisk at anfægte dem eller smider vi dem bare ud. Det kunne være særlig relevant at kigge på søgeteori og måske droppe en eller to af de andre økonomiske teorier. Hvad er det for modeller I vil diskutere med senere.
Valget af reservationsløn kan betragtes som en bestræbelse på at maksimere egennytte. Det vil sige, at fordelene man kan opnå ved at acceptere et job dags dato må opvejes med fordelene i form af bedre job på længere sigt. Et bedre job på længere sigt kan både give en bedre løn, men kan også medfører indkomsttab mens søgningen foregår \parencite[1698]{Atkinson1991}.

% De skriver også om mobilitet og job flows i kombination med søgeteori \parencite[2567-2627]{Mortensen1999}

%%%%%%%%%%%%%%%%%%%%%%%%%%%%%%%%%%%%%%%%%%%%%%%%%%%%%%%%%%%
\subsection{Udvidelser af jobsøgningsmodellen...}
%%%%%%%%%%%%%%%%%%%%%%%%%%%%%%%%%%%%%%%%%%%%%%%%%%%%%%%%%%%

\textbf{Principal-agent-modellen} udvikles fra slutningen af 1970'ernes af blandt andet Baily og Flemming på baggrund af den basale jobsøgningsmodel. Her tages der udgangspunkt i en kontrakt mellem en principal og en agent, som gør, at agenten kan tage flere risici, fordi principalen bærer byrden af disse risici. Adfærden hos principalen og agenten er afhængig af om jobsøgningsindsatsen er kontrollerbar eller ej, hvilket vil sige, om der er bevis for, at agenten virkelig har foretaget en indsats for at søge job \parencite[134]{Cahuc2004}. Når indsatsen \textit{er} kontrollerbar, kan principalen gøre udbetalingen af arbejdsløshedsydelsen afhængig af agenternes jobsøgningsindsats, hvilket betyder, at den optimale kontrakt mellem principal og agent giver agenten fuld kompensationsgrad for at søge arbejde, når agenten er arbejdsløs \parencite[138]{Cahuc2004}. Når indsatsen \textit{ikke} er kontrollerbar kan der være mulighed for, at agenten modtager en arbejdsløshedsydelse uden at gøre en indsats for at søge arbejde. Derfor er det nødvendigt for principalen, at den optimale kontakt \textit{ikke} giver fuld kompensationsgrad for arbejdsløshed med henblik på at give agenten større incitament for at søge arbejde\footnote{Baily definerer det optimale arbejdsløshedsforsikringsystem som en afvejning mellem jobsøgningsincitament og arbejdsløshedsforsikringen. Her er den marginale gevinst af arbejdsløshedsforsikring lig med den marginale omkostning ved øget arbejdsløshed. Med andre ord påviser Baily som en af de første sammenhængen mellem incitament for at søge jobs og størrelsen på arbejdsløshedsforsikring \parencite[379]{Baily1978}.} \parencite[379]{Baily1978}. %%%% Jens: Hvis man kontrollere Baily for uddannelse, så er det jo egentlig ikke dumt. Vores anke går mere på forståelsen af hvad et arbejdsmarked egentlig er for noget.


%%%%%%%%%%%%%%%%%%%%%%%%%%%%%%%%%%%%%%%%%%%%%%%%%%%%%%%%%%%
\subsection{Kritik}
%%%%%%%%%%%%%%%%%%%%%%%%%%%%%%%%%%%%%%%%%%%%%%%%%%%%%%%%%%%

Ifølge Atkinson bygger politiske valg om at sænke ydelsesniveauerne for arbejdsløse på en oversimplificering af arbejdsmarkedet, hvor økonomer typisk ser arbejdsløshedsydelser som havende en negativ effekt på arbejdsmarkedet med høje ydelser som forårsager, at arbejdsløse er mindre villige til at tage et arbejde \parencite[1680]{Atkinson1991}. Atkinson selv kritiserer modellerne for typisk at antage, at arbejdsløshedens effekt kan reduceres til beløbet på arbejdsløshedsydelsen uden at forholde sig til de institutionelle forhold i arbejdsløshedssystemet\footnote{I OECD-lande er betingelserne for modtagelse af arbejdsløshedsydelser eksempelvis typisk, at personen ikke må være frivillig arbejdsløs, der skal gøres en reel jobsøgningsindsats, man må ikke blive ved med at afvise jobtilbud og der er ydelsen dækker over en begrænset periode \parencite[1689]{Atkinson1991}.} \parencite[1688]{Atkinson1991}. De økonomiske modeller er altså ikke gode nok til at forklare institutionelle forhold som for eksempel kommunikation med jobcentret og a-kassen og deres påvirkning på den arbejdsløse på trods af forsøg på at tage højde for hvor længe der kan modtages ydelser, forskelle på arbejdsløshedsydelser og sociale ydelser og dem som ikke er berettiget til arbejdsløshedsydelser \parencite[1692]{Atkinson1991} \parencite[33-34]{Halvorsen1999} \parencite[114]{Cahuc2004}.

De økonomiske modeller forudsætter typisk en positiv sammenhæng mellem søgeintensitet og antallet af jobtilbud man modtager. Modellerne har dog typisk ikke fokus på, at jobsøgningen ikke er enstationær tilstand og jo længere arbejdsløsheden varer, jo færre jobtilbud vil der for det meste komme \parencite[119]{Cahuc2004}. %%%%% Jens markeret 
Ifølge Halvorsen er et væsentligt problem med jobsøgningsmodellerne, at arbejdsløse næsten altid vil tage imod det første tilbud de modtager, hvilket betyder at variationer i arbejdsløshedslængden primært opstår ud fra sandsynligheden for at modtage et jobtilbud \parencite[28]{Halvorsen1999}.

Omvendt er en høj arbejdsløshedsydelsen også blevet sat i et mere positivt lys, fordi det kan bidrage til at have en positiv effeekt på søgeaktiviteten. Ifølge Tatsiramos er fordelene ved at modtage arbejdsløshedsforsikring større end omkostningerne. Selvom en høj arbejdsløshedsforsikring kan medføre en længere arbejdsløshedsperiode, viser et europæisk studie, at de arbejdsløse som modtager arbejdsløshedsydelse forbliver 2-4 måneder længere i det efterfølgende jobs end arbejdsløse som ikke modtager en arbejdsløshedsydelse \parencite[602]{Mankiw2011}. Dette perspektiv åbner op for, at et job ikke bare er et job, hvilket bliver vigtigt for vores analyse. %%%% Jens: Hvad betyder det for jeres analyse. Jens slet: hvilket kommer af at beskæftigelse som tidligere nævnt ligesom arbejdsløshed og personer uden for arbejdsstyrken er en heterogen kategori. %%% Emil: Det ligger måske endnu mere op til Bourdieu

Til sidst skal det nævnes, at de økonomiske modeller typisk ser også bort fra ikke-økonomiske incitamenter, hvor lønarbejde typisk foretrækkes frem for arbejdsløshed, fordi det giver selvrespekt, anderkendelse og man lever op til de sociale forventninger. Disse sociale og psykologiske forhold dominerer den sociologiske arbejdsløshedsforskning som vi som før nævnt vil gå i dybden med senere \parencite{Jahoda1971, Eisenberg1938, Ezzy1993, Halvorsen1999, Baum2001, Noerup2014}. %%% Emil: Senere - udnyt latex

Efter denne gennemgang af forskellige årsager til arbejdsløshed på lang sigt samt modellerne trade-off'et mellem arbejde og fritid, den basale jobsøgningsmodel og principal-agent-modellen, vil vi gå over til arbejdsløshed på kort sigt.

Arbejdssegmenteringskritik af økonomisk teori og søgeteori \parencite[1237-1241]{Cain1976}.


%%%%%%%%%%%%%%%%%%%%%%%%%%%%%%%%%%%%%%%%%%%%%%%%%%%%%%%%%%%
\subsection{Opsummering}
%%%%%%%%%%%%%%%%%%%%%%%%%%%%%%%%%%%%%%%%%%%%%%%%%%%%%%%%%%%


%Local Variables: 
%mode: latex
%TeX-master: "report"
%End: