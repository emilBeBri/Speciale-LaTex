% -*- coding: utf-8 -*-
% !TeX encoding = UTF-8
% !TeX root = ../report.tex

\chapter{INTRODUKTIONS TIL JENS} \label{intro}



Hej Jens,

Her sender vi dig teksten, som vores 3. vejledermøde tager udgangspunkt i.

Vi synes, at det vigtigste på mandag er at få at respons på den måde, vi er gået til metodeafsnittet, samt tale om hvad analysen kan og skal indeholde. Vi er lige nu i lidt af et vadested, fordi vi har brugt så pokkers meget tid på at få det tekniske op at stå. Det har handlet om at behandle datamateriale fra Danmarks Statistik i STATA så det kunne læses af moneca, samt lære at benytte moneca i R.

Det har vi så nu, men det betyder at vi har tænkt meget få teoretiske tanker om hvor vi skal bevæge os hen af, og det bliver den store udfordring her i juni. 

Siden sidst har vi ikke ændret vores problemstilling. Vi har slet ikke kigget på den. Så indtil videre ser den (stadigvæk) sådan her ud:

\textbf{Hvad betyder kortere og længere perioder med ledighed for arbejdstageres sociale mobilitet på det danske arbejdsmarked?}. Dette overordnede spørgsmål besvarer vi med følgende underspørgsmål:
 \begin{enumerate} [topsep=6pt,itemsep=-1ex]
   \item Hvilke former for beskæftigelse kommer arbejdstagere fra og går til før og efter perioder med ledighed?
   \item Hvilke sammenhænge er der sammenhængen mellem hvilke dispositioner disse arbejdstagere har til rådighed og hvilke strategier de benytter sig af for at komme ud af ledighed og tilbage i beskæftigelse.
   \item Kan man med vores tilgang til ledighed se noget andet end andre tilgange, og hvordan kan vores empiri og andres empiri forstås som en del af en konstant symbolsk kamp om konstituering af arbejdsløshed?
 \end{enumerate}

Teksten er knap 20-25 sider lang (eksklusiv appendiks og tabeller). Hensigten er at give dig et indblik i vores datamateriale og derved mulighederne i analysen. Det er hensigten, at kapitlet \textbf{2. Kortlægning af de ledige på arbejdmarkedet} skal fungere som pendulerende mellem videnskabsteori, metode og databeskrivelse. Hvor vi beskriver vores metodiske tilgang samt lave de første analytiske spadestik. Afsnit \textbf{2.1: At kortlægge arbejdsmarkedet med netværksanalyse} har vi ikke nået at skrive endnu, men i dette afsnit ønsker vi at beskrive nogle grundlæggende elementer i netværksanalyse. Afsnit \textbf{2.2: At kategorisere arbejdsmarkedet i segmenter} og afsnit \textbf{2.3: At skabe en kritisk masse af ledige} er to lange afsnit, hvor henholdsvis begreberne arbejdsstillinger og ledighed operationalisereres. Vi har ladet os inspirere en del af dit speciale og din måde at lade forskellige elementer fungere organisk vekselvirkende på, og vil gerne hvordan du synes formen virker. Afsnit \textbf{2.4 Et videnskabsteoretisk perspektiv på vores kortlægning} har vi heller ikke nået at beskrive, men hensigten er at det skal bruges til at bruge videnskabsteori til vurdere og diskutere vores metode. Det er egentlig meningen at afsnittet \textbf{2.5 Kort over de lediges bevægelser på arbejdsmarkedet} skal starte specialet, men da vi ikke har nået at gennemnarbejde det så meget som vi kunne ønske, har vi valgt at lægge det sidst i dette kapitlet for nu.

Så vi vi gerne have mødet handler om to ting: 1) feedback på hvordan formen virker i dette metodeafsnit 2) brainstorm på hvad analysen skal fokusere på, nu vi har kortene i hus.

efter metoden så tænker vi at fokuserer på to felter akademikerne og den store faglige cluster.

Mvh.
Emil og Søren


%Local Variables: 
%mode: latex
%TeX-master: "report"
%End: 