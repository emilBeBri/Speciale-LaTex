

%%%%%%%%%%%%%%%%%%%%%%%%%%%%%%%%%%%%%%%%%%%%%%
\section{Standoff \label{2_klassestandoff}}
%%%%%%%%%%%%%%%%%%%%%%%%%%%%%%%%%%%%%%%%%%%%%%

En række kritikker af Grufsky er kommet, og jeg vil i det følgende koncentrere mig om, hvad jeg anser som de to vigtigste emner. Det første omhandler den teoretiske definition af klasse, og den anden omhandler målbarheden af disse.


%%%%%%%%%%%%%%%%%%%%%%%%%%%%%%%%%%%%%%%%%%%%%%
\subsection{teoridiskussion \label{2_kritikafgruskyteori}}
%%%%%%%%%%%%%%%%%%%%%%%%%%%%%%%%%%%%%%%%%%%%%%

Goldthorpe påpeger, at Grufskys privilegering af realistiske klasser frem for nominelle, og fordelene ved deres reelle fremtrædelsesform i den sociale praksis, har en række problemer, Grusky enten overser eller ikke adresserer. 

Ved at forsage analytiske kategorier i klasseinddelingen, er der risiko for at miste det teoretiske grundlag for at forstå hvorfor folk gør som de gør - der ikke altid kan forstås umiddelbart af dem, der gør dets egne kategorier. For Goldthorpe er klasse derfor forskelligt fra erhvervsgrupper, da der er tale om en strukturation af livsbetingelser på et niveau \emph{over} den præcise beskæftigelse. En teoretisk konceptualisering af hvad klasse \emph{er}, udover klasse-fur-sich, er nødvendig, for at forstå denne latent sociale logik. For Goldthorpe er det centrale i en klassekontekst ikke, hvorfor lægesønnen bliver læge, men hvorfor lægesønnen udover læge har langt større chancer for at få en management-stilling eller blive akademiker, fremfor kulminearbejder \parencite[214]{Goldthorpe2002}. 



















Klasse, som Therborn påpeger i sin kritik af Grusky, er en sociologisk konceptualisering af aktører i samfundet, defineret ud fra deres positioner og interesser \parencite[222]{Therborn2002}. Hvordan grænserne skal drages er genstand for diskussion, men at guldstandarden bør være menneskers egen selvforståelse, er problematisk. Som eksempel bruger Therborn begrebet udbytning, der står som central i en klasseanalyse. Således også hos Grusky, der ser erhvervsgrupper som “\emph{fundamental units of exploration}” \parencite[212]{Grusky2001}. At de skulle være det, forekommer Therborn absurd, da fokus dermed bliver, hvordan sygeplejerske udnytter ikke-sygeplejersker, tømrerer udnytter ikke-tømrerer, etcera. Fremfor de økonomiske strukturer, som gør at man i bestemte klassepositioner, som fremført i afsnittet om Wright på side \pref{2_wright}. At det er den eneste centrale mekanisme i klasserelationer er Therborn ikke overbevist om. Men det er forfladigende at reducere klasseudbytning til enkelte erhvervsgruppers lukningsstrategier overfor hinanden \parencite[222]{Therborn2002}. At benytte begrebet udbytning og klasse på et aggregeret niveau, er ikke det samme som at abonnere på en kollektiv handlen baseret på “\emph{The Storming of the Winter Palace}”-modellen, som Goldthorpe kalder det \parencite[215]{Goldthorpe2002}. Ved det mener han, at kollektiv klassebaseret handlen kan antage mange former, og rationel klassehandlen kan meget vel være korporativistisk fremfor antagonistisk. 

Skriv om det som sociale processer, der fungerer på aggregeret niveau, og netop er det, der gør at man kan karakterisere det som segmenter - det rammer på samme måde indenfor jobs af samme type 


%%%%%%%%%%%%%%%%%%%%%%%%%%%%%%%%%%%%%%%%%%%%%%
\subsection{metodediskussion \label{2_kritikafgruskymetode}}
%%%%%%%%%%%%%%%%%%%%%%%%%%%%%%%%%%%%%%%%%%%%%%

Tre kritikker, der kan anskues fra en metodevinkel, er Therborn og Birkelund, der kritiserer Grusky for at fokusere på et detaljeniveau, hvor det store billede af samfundsformation, som for dem at se er en af klasseteoriens styrker, skulle forsvinde. 

For Therborn har en teoretisk funderet klasseinddeling en basal funktion: At klasse, blandt andre ting, også er en måde at håndtere information på, et tradeoff mellem den observerede virkeligheds uoverskuelighed og overblikkets simplificering. At benytte nominel klasseinddeling letter evnen til at se tendenser i den sociale struktur, ved at operere med kategorier store nok til at kunne overskue de overordnede klasserelationer og effekter. Han er bekymret for om et fokus på erhvervsgrupper ikke vil miste dette overblik (Therborn 2002:223)  

Som eksempel på risikoen ved at fokusere på erhvervsgrupper, kan Birkelunds kritik af Grusky benyttes. Birkelund giver Grusky ret i at klase som ulighedsskabende stratifikation overskygges af andre skel i befolningen, og at klasser er fragmenterede i erhverv. Sideløbende, og sikkert samenhængende med dette, er tilhørsforhold til erhvervsgruppe vigtigere for social identitet end klasse er det. Det finder han på baggrund af en applicering af Wrights klassemodel på Norges befolkning i 1987 (Birkelund 2002:218). 

Alligevel er Birkelund bekymret for det manglende makroperspektiv, som et skift i fokus til mikroklasser kan betyde. Han fremhæver globalisering af varehandel og vareproduktion, og multinationale selskabers flugt fra nationalstaten, som strukturelle ændringer, der har afgørende betydning for klasserelationerne også internt i lande. Han er bekymret for at et skift til klasser set som erhvervsgrupper, vil have svært ved at fange disse klasseeffekter på det makroniveau, de hører til på (Birkelund 2002:220). 

I en dansk kontekst har \textcite{Andrade2014} forsket i forskellige klassedefinitioners forklaringskraft for indkomst på lang sigt, baseret på registerdata. % måske argument om hvorfor det er vigtigt at se på indkomst i det hele taget. 
Andrade har undersøget søskendekorrelationer i langtidsindkomst, og set på hvilken effekt, forældres klasseposition har på denne indkomst. Her har han benyttet 6 forskellige klassedefinitioner: \textbf{1)}, lavet ud fra samme DISCO-erhvervsklassifikationsskema, som jeg benytter i denne afhandling. De 6 klasseskemaer han benytter er: Poulantzas tidligere 2-klassemodel, med skellet mellem produktivt og uproduktivt arbejde, Goldthorpe-Erikssons klasseskema i alle dets af dets oprindelige varianter med henholdsvis 3 og 9 klasser, samt to udvidede versioner. En lettere udvidet model med 13 klasser, samt en udvidet version med 15 klasser. De sidste to er udvidet for at tage hensyn til kritik af skemaet for at mangle henholdsvis nyopståede klasser af sociale og kulturelle specialister, og en der yderligere kapitalstærke større entreprenører og landejere. Den sjette klassemodel er (en version af) Gruskys mikroklassetypologi med 74 klasser \parencite[93f]{Andrade2014}.

Andrade finder, at den udvidede 15 klassemodel for forældre bedst beskriver langtidsindkomst for personer født mellem 1968 og 1974, også når forældres indkomst og uddannelse er med som kontrolvariable \parencite[108]{Andrade2014}. Gruskys mikroklasser, selvom forklaringskraften er en anelse højere, bruger så mange ekstra frihedsgrader i regressionsanalysen, grundet de yderligere 59 klassekategorier, at denne forbedring i forklaringskraft ikke er signifikant. Hans konklusion er at den udvidede EGP model med 15 klasser bedst beskriver de observerede indkomstforskelle, og man derfor ikke kan karakterisere Danmark som et land, hvori klassestruktureren er fragmenteret langs erhvervsgrupper \parencite[109]{Andrade2014}, som dem mikroklasseteorien foreskriver, hvor erhvervsgruppers partikulære sociale lukningsmekanismer er vigtigere end de aggregerede klasser, de indgår i, grundet enshed i specificiteten af deres kompetencer samt enshed i kontrollen over arbejdsprocessen. Både Grusky og Goldthorpe har fremhævet Sverige som eksempel på et land, hvori aggregerede klasser, grundet traditionen for bredt favnende og succesfulde fagforeninger, både teoretisk såvel som empirisk er fundet som havende bredere konstituerede klasser end tilfældet er i eksempelvis USA, Japan og ??? (find reference) 

Mens 10 \% af indkomstfordelingen kan forklares ved hjælp af forældrenes indkomst, giver tilføjelsen af forældrenes position i 15-klassemodellen yderligere 12 \% forklaringskraft. Der er dermed ca. 4/5 af indkomstforskellen, der ikke kan forklares ved hjælp af forældres indkomst, uddannelse og klasse (PEERREVIEwED Andrade 2016 s.20 ikke publiseret endnu!!).


%%%%%%%%%%%%%%%%%%%%%%%%%%%%%%%%%%%%%%%%%%%%%%
\subsection{mit take \label{2_gruskycomeback}}
%%%%%%%%%%%%%%%%%%%%%%%%%%%%%%%%%%%%%%%%%%%%%%

Konflikten mellem de to positioner mener jeg godt kan medieres, og det vil jeg i det følgende prøve.

På den ene side forekommer det mig tvivlsomt, om et udgangspunkt i menneskers oplevede arbejdsfælleskaber virkelig er det centrale kriterie for en klassedefinition. For mig at se er klasse bedst forstået som et teoretisk begreb, der hænger sammen med teorier om samfundets indretning, hvad enten det er dominansforhold, prestige, symbolsk magt eller udbytningsrelationer. Hvis man udelukker sig fra disse forklaringer på samfundets stratifikation i klasser, bliver afgrænsningen anderledes empirisk beskrivende, hvilket lader til at være netop hvad Grusky mener at klasseteorien bør efterstræbe. Han kalder det \emph{kulturel funktionalisme}, og det er netop på dette plan, Grusky mener at klasseteori bør bevæge sig på, for at fange de grundlæggende stratifikationsmekanismer, som de finder sted for dem, der udfører disse stratifikationer i praksis \parencite[229]{Grusky2002}. Denne form for redegørelse for klassers funktion, som vi skal se, tilbyde stærke og fremfor alt lægmands-genkendelige redegørelser, men kan, eller vil ikke, binde det sammen med en social logik, der stikker særlig langt udover dette niveau. Men hvordan forklarer man så den logik, der gør mobiliteten mellem bestemte erhvervsgrupper meget mere sandsynlig, uden at disse erhvervsgrupper har nogen tilsyneladende funktionel enshed, eller har den form for social nærhed i arbejdslivet, Grusky mener bør være basis for klasseinddelinger? 

Som tilfældet er hos Parkin sker der en form for forfladigelse af forskellige analytiske koncepter, for at ramme en forståelse, der ligger tæt op af individernes sociale identiteter. Det betyder at social status og social klasse, en grundlæggende skelnen ikke bare i den nyweberianske tradition som hos Goldthorpe, men også \textcite{Scott2002}. Og ikke mindst Bourdieu, hvis bestræbelser, i Goldthorpes formulering, på at overkomme modsætningen mellem klasse og status i den weberianske tradition, behandler status som den symbolske effekt af klasse, der ikke kan reduceres til dens økonomiske relationer \parencite[513]{Goldthorpe2007b}. Pointen er, at denne form for status i form af symbolsk kapital, der måske med en tilsnigelse kan siges at være de symbolske effekter af klasse, som Goldthorpe beskriver Bourdieus videreudvikling af social status begrebet \parencite[Goldthorpe2007b]{513}

Man kan måske sige, at Grusky forsøger at løse forholdet mellem social status, der nødvendigvis anerkendes af aktørerne selv, med klasseinteresse. Dermed kommer man naturligvis udover problemerne med at definere hvad en klasseinteresse er, og de genvordigheder, som marxisterne har tumlet i over 100 år. Ikke desto mindre mister man måske noget vigtigt, som er netop de rationelle klassehandlinger, personer i bestemte klassehandlinger vil have affinitet til at handle efter, medmindre andre kræfter trækker i den anden retning. Faren er derfor en forfladigelse af analysen. Men det fornuftige er for mig at se netop kritikken i stive kategorier,

(slutter lidt brat sorry)