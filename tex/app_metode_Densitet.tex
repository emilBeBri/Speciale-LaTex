%!TEX root = ../report.tex

%%%%%%%%%%%%%%%%%%%%%%%%%%%%%%%%%%%%%%%%%%%%%%%%%%%%%%%%%%%
\newpage \chapter{Netværksmål \label{app_netvaerksmaal}}
%%%%%%%%%%%%%%%%%%%%%%%%%%%%%%%%%%%%%%%%%%%%%%%%%%%%%%%%%%%


Densiten i et netværk er et mål for “forbundetheden” i et netværk. Det vil sige antallet af forbindelser i et netværk i forhold til antallet af mulige forbindelser. Se Scott (\citeyear{Scott2000}), kapitel 4 for den matematiske definition. 

I Moneca benyttes densitet på en specifik måde: Ikke til at vurdere netværket som helhed, men til at vurdere hver enkelt klynges densitet. Her behandles hver enkelt klynge således som sit eget netværk. Densiteten er derfor udtryk for hvor godt hver enkelt klynge hænger sammen. Man skal dog holde for øje, at fordi densitet beregnes ud fra det totale antal mulige forbindelser, er det et \emph{relativt mål}. Det kan ikke sammenlignes på tværs af netværk af forskellig størrelse. Det betyder i Monecas tilfælde, at densitetsværdien kun kan sammenlignes direkte mellem to to givne klynger, hvis antallet af noder er ens. Hvilket de sjældent er. Større klynger vil have sværere ved at opnå en høj densitet. Densitet er derfor et mål for et netværks forbundethed, men bliver nødt til at vurderes ud fra den konkrete sammenhæng for at være meningsfuld. %(indsæt reference til Scott, kapitel 4, \#todo)

Nedenfor 


% Table generated by Excel2LaTeX from sheet 'app_netvaerksmaal_densitet'
\begin{table}[htbp]
  \centering
  \caption{Add caption}
    \begin{tabular}{lccc}
          &       & \multicolumn{1}{l}{Antal} & \multicolumn{1}{l}{Maksimal } \\
    \rowcolor[rgb]{ .753,  .753,  .753} Klynge & Densitet & noder & stilængde \\
    3.14  & 0,86  & 7     & 2 \\
    3.36  & 0,85  & 5     & 2 \\
    3.15  & 0,83  & 3     & 2 \\
    3.2   & 0,83  & 4     & 2 \\
    3.9   & 0,80  & 5     & 2 \\
    3.35  & 0,77  & 6     & 2 \\
    3.24  & 0,75  & 4     & 2 \\
    3.37  & 0,75  & 4     & 2 \\
    3.12  & 0,75  & 4     & 2 \\
    3.30  & 0,73  & 6     & 2 \\
    4.1   & 0,67  & 14    & 3 \\
    3.25  & 0,67  & 3     & 2 \\
    3.18  & 0,65  & 5     & 2 \\
    3.21  & 0,64  & 7     & 3 \\
    4.9   & 0,64  & 13    & 3 \\
    4.4   & 0,63  & 6     & 2 \\
    4.11  & 0,61  & 9     & 2 \\
    3.26  & 0,60  & 5     & 3 \\
    3.20  & 0,60  & 7     & 3 \\
    4.8   & 0,57  & 6     & 2 \\
    3.8   & 0,55  & 8     & 2 \\
    4.12  & 0,51  & 10    & 3 \\
    5.2   & 0,50  & 13    & 2 \\
    4.2   & 0,50  & 6     & 3 \\
    5.1   & 0,49  & 28    & 3 \\
    4.6   & 0,47  & 22    & 3 \\
    4.10  & 0,46  & 8     & 3 \\
    4.13  & 0,46  & 16    & 3 \\
    \end{tabular}%
  \label{tab:addlabel}%
\end{table}%




Der er dog et problem med densitet, og det er at det er et \emph{relativt mål}, der afhænger



To mål centrale Moneca: Densitet der viser...

% Densitet
%  Densitets relativitetsproblem: densitet afhænger af grafens størrelse, hvilket forhindrer at densitet
% kan sammenlignes over netværk af forskellige størrelser. Desuden er der grund til at tro, at det
% maksimale antal mulige forbindelser er mindre end i det mættede netværk (som i vores tilfælde, da
% hele adelen næppe kan være i familie eller gift med hinanden). Hvis der er en øvre grænse for, hvor
% mange forbindelser en aktør kan have, vil større netværk alt andet lige have lavere densiteter end små
% netværk. Ligeledes må det forventes, at aktører kan have færre kærlighedsrelationer end andre typer
% relationer, hvorfor denne type netværk nødvendigvis må have en relativ lav densitet.

% ikke et godt mål til vurderinger på tværs af størrelser.


Moneca gør det dermed muligt at opdele netværk med mange og komplekse forbindelser mellem noder en på sådan vis, at klynger kan identificeres, hvor man med en vis rimelighed kan sige at noderne hører sammen. Dette...

