%!TEX root = ../report.tex

%%%%%%%%%%%%%%%%%%%%%%%%%%%%%%%%%%%%%%%%%%%%%%%%%%%%%%%%%%%
\newpage \chapter{Netværksmål \label{app_netvaerksmaal}}
%%%%%%%%%%%%%%%%%%%%%%%%%%%%%%%%%%%%%%%%%%%%%%%%%%%%%%%%%%%


Densiten i et netværk er et mål for “forbundetheden” i et netværk. Det vil sige antallet af forbindelser i et netværk i forhold til antallet af mulige forbindelser. Se Scott (\citeyear{Scott2000}), kapitel 4 for den matematiske definition. 

I Moneca benyttes densitet på en specifik måde: Ikke til at vurdere netværket som helhed, men til at vurdere hver enkelt klynges densitet. Her behandles hver enkelt klynge således som sit eget netværk. Densiteten er derfor udtryk for hvor godt hver enkelt klynge hænger sammen. Man skal dog holde for øje, at fordi densitet beregnes ud fra det totale antal mulige forbindelser, er det et \emph{relativt mål}. Det kan ikke sammenlignes på tværs af netværk af forskellig størrelse. Det betyder i Monecas tilfælde, at densitetsværdien kun kan sammenlignes direkte mellem to to givne klynger, hvis antallet af noder er ens. Hvilket de sjældent er. Større klynger vil have sværere ved at opnå en høj densitet. Densitet er derfor et mål for et netværks forbundethed, men bliver nødt til at vurderes ud fra den konkrete sammenhæng for at være meningsfuld. %(indsæt reference til Scott, kapitel 4, \#todo)

I tabel \ref{tab_app_densitet} ses densitetsmål for alle klynger, der består af over 2 noder. Det skyldes at for klynger med kun to noder er densiteten altid 1, da kun én forbindelse er mulig, og denne per definition er opfyldt. for enkeltstående noder er målet naturligvis ikke meningsfuldt. 

% Table generated by Excel2LaTeX from sheet 'app_netvaerksmaal_densitet'
\begin{table}[H]
  \centering
  \caption{Densitet i oprindelig form}
    \begin{tabular}{lcccr}
          &       & \multicolumn{1}{l}{Antal} & \multicolumn{1}{l}{Maksimal } &  \\
    Klynge & \multicolumn{1}{l}{Densitet} & \multicolumn{1}{l}{noder} & \multicolumn{1}{l}{stilængde} &  \\
\cmidrule{1-4}    3.14  & 0,86  & 7     & 2     &  \\
    3.36  & 0,85  & 5     & 2     &  \\
    3.15  & 0,83  & 3     & 2     &  \\
    3.2   & 0,83  & 4     & 2     &  \\
    3.9   & 0,80  & 5     & 2     &  \\
    3.35  & 0,77  & 6     & 2     &  \\
    3.24  & 0,75  & 4     & 2     &  \\
    3.37  & 0,75  & 4     & 2     &  \\
    3.12  & 0,75  & 4     & 2     &  \\
    3.30  & 0,73  & 6     & 2     &  \\
    4.1   & 0,67  & 14    & 3     &  \\
    3.25  & 0,67  & 3     & 2     &  \\
    3.18  & 0,65  & 5     & 2     &  \\
    3.21  & 0,64  & 7     & 3     &  \\
    4.9   & 0,64  & 13    & 3     &  \\
    4.4   & 0,63  & 6     & 2     &  \\
    4.11  & 0,61  & 9     & 2     &  \\
    3.26  & 0,60  & 5     & 3     &  \\
    3.20  & 0,60  & 7     & 3     &  \\
    4.8   & 0,57  & 6     & 2     &  \\
    3.8   & 0,55  & 8     & 2     &  \\
    4.12  & 0,51  & 10    & 3     &  \\
    5.2   & 0,50  & 13    & 2     &  \\
    4.2   & 0,50  & 6     & 3     &  \\
    5.1   & 0,49  & 28    & 3     &  \\
    4.6   & 0,47  & 22    & 3     &  \\
    4.10  & 0,46  & 8     & 3     &  \\
    4.13  & 0,46  & 16    & 3     &  \\
\cmidrule{1-4}    \end{tabular}%
  \label{tab_app_densitet}%
\end{table}%

Det ses af tabel \ref{tab_app_densitet}, De færreste klynger har en densitet på under 0,5. De fleste af de store klynger, med over 10 noder, ligger som forventet lavt på listen. Enkelte klynger så som 4.2 og 4.10 har ikke mange noder, og ligger alligevel lavt på listen. Disse bliver splittet op, da det viser sig at der er gode grunde til deres lave densitet: Der er sociologisk meningsfulde årsager til den lave densitet.

I klynge 4.10s tilfælde er der tale om at \emak{d3224} bliver lagt sammen med stillinger udelukkende fra overgruppe 1. Ved nærmere eftersyn ses det at jobbet optiker har forbindelse til \emak{d1224} & \emak{1239}, men ikke til de andre jobs i klyngen. Det giver mening at optikere kan få ledelsstillinger indenfor visse virksomheder, men det er misvisende når det kun gælder meget få af de andre i klyngen. Den interne mobilitet for optikere er i sig selv er på hele 93 \%,  hvorimod alle andre i klyngen har internt mobilitet i omegnen af gennemsnittet på 68 \%. På baggrund af disse to parametre, såvel som på den common sense indikator der ligge i, at alt andet i klyngen er ledelsesarbejde,  vælger jeg derfor at splitte klyngen op, som den var på niveau 3: Det vil sige, \emak{d3224} bliver ikke integreret i den såkaldte ledelsesklynge. 


 






