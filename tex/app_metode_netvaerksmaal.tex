%!TEX root = ../report.tex

%%%%%%%%%%%%%%%%%%%%%%%%%%%%%%%%%%%%%%%%%%%%%%%%%%%%%%%%%%%
\newpage \chapter{Netværksmål \label{app_netvaerksmaal}}
%%%%%%%%%%%%%%%%%%%%%%%%%%%%%%%%%%%%%%%%%%%%%%%%%%%%%%%%%%%


Densiten i et netværk er et mål for “forbundetheden” i et netværk. Det vil sige antallet af forbindelser i et netværk i forhold til antallet af mulige forbindelser. Se Scott (\citeyear{Scott2000}), kapitel 4 for den matematiske definition. 

I Moneca benyttes densitet på en specifik måde: Ikke til at vurdere netværket som helhed, men til at vurdere hver enkelt klynges densitet. Her behandles hver enkelt klynge således som sit eget netværk. Densiteten er derfor udtryk for hvor godt hver enkelt klynge hænger sammen. Man skal dog holde for øje, at fordi densitet beregnes ud fra det totale antal mulige forbindelser, er det et \emph{relativt mål}. Det kan ikke sammenlignes på tværs af netværk af forskellig størrelse. Det betyder i Monecas tilfælde, at densitetsværdien kun kan sammenlignes direkte mellem to to givne klynger, hvis antallet af noder er ens. Hvilket de sjældent er. Større klynger vil have sværere ved at opnå en høj densitet. Densitet er derfor et mål for et netværks forbundethed, men bliver nødt til at vurderes ud fra den konkrete sammenhæng for at være meningsfuld. %(indsæt reference til Scott, kapitel 4, \#todo)

%%%%%%%%%%%%%%%%%%%%%%%%%%%%%%%%%%%%%%%%%%%%%%
\section{Oprindelig Moneca densitetsmål}
%%%%%%%%%%%%%%%%%%%%%%%%%%%%%%%%%%%%%%%%%%%%%%


I tabel \ref{tab_app_densitet_original} ses densitetsmål for alle klynger, der består af over 2 noder. Det skyldes at for klynger med kun to noder er densiteten altid 1, da kun én forbindelse er mulig, og denne per definition er opfyldt. for enkeltstående noder er målet naturligvis ikke meningsfuldt. 


\begin{table}[H]
  \centering
  \resizebox{5cm}{!}{%
% Table generated by Excel2LaTeX from sheet 'app_netvaerksmaal_densitet_orig'
\begin{tabular}{lccc}
      &       & \multicolumn{1}{l}{Antal} & \multicolumn{1}{l}{Maksimal } \\
Klynge & \multicolumn{1}{l}{Densitet} & \multicolumn{1}{l}{noder} & \multicolumn{1}{l}{stilængde} \\
\midrule
4.14  & 0,46  & 16    &  3 \\
4.11  & 0,46  &  8    &  3 \\
4.6   & 0,47  & 22    &  3 \\
4.8   & 0,47  &  9    &  4 \\
5.1   & 0,49  & 28    &  3 \\
5.2   & 0,50  & 13    &  2 \\
4.2   & 0,50  &  6    &  3 \\
4.13  & 0,51  & 10    &  3 \\
3.8   & 0,55  &  8    &  2 \\
4.9   & 0,57  &  6    &  2 \\
3.20  & 0,60  &  7    &  3 \\
4.12  & 0,61  &  9    &  2 \\
4.4   & 0,63  &  6    &  2 \\
4.10  & 0,64  & 13    &  3 \\
3.21  & 0,64  &  7    &  3 \\
3.18  & 0,65  &  5    &  2 \\
3.25  & 0,67  &  3    &  2 \\
4.1   & 0,67  & 14    &  3 \\
3.30  & 0,73  &  6    &  2 \\
3.24  & 0,75  &  4    &  2 \\
3.37  & 0,75  &  4    &  2 \\
3.35  & 0,77  &  6    &  2 \\
3.9   & 0,80  &  5    &  2 \\
3.15  & 0,83  &  3    &  2 \\
3.2   & 0,83  &  4    &  2 \\
3.36  & 0,85  &  5    &  2 \\
3.14  & 0,86  &  7    &  2 \\
\bottomrule
\end{tabular}%
 
}
  \caption{Densitet i oprindelig Moneca}
  \label{tab_app_densitet_original}%
\end{table}%


% %!TEX root = ../report.tex

%%%%%%%%%%%%%%%%%%%%%%%%%%%%%%%%%%%%%%%%%%%%%%%%%%%%%%%%%%%
\newpage \chapter{Netværksmål \label{app_netvaerksmaal}}
%%%%%%%%%%%%%%%%%%%%%%%%%%%%%%%%%%%%%%%%%%%%%%%%%%%%%%%%%%%


Densiten i et netværk er et mål for “forbundetheden” i et netværk. Det vil sige antallet af forbindelser i et netværk i forhold til antallet af mulige forbindelser. Se Scott (\citeyear{Scott2000}), kapitel 4 for den matematiske definition. 

I Moneca benyttes densitet på en specifik måde: Ikke til at vurdere netværket som helhed, men til at vurdere hver enkelt klynges densitet. Her behandles hver enkelt klynge således som sit eget netværk. Densiteten er derfor udtryk for hvor godt hver enkelt klynge hænger sammen. Man skal dog holde for øje, at fordi densitet beregnes ud fra det totale antal mulige forbindelser, er det et \emph{relativt mål}. Det kan ikke sammenlignes på tværs af netværk af forskellig størrelse. Det betyder i Monecas tilfælde, at densitetsværdien kun kan sammenlignes direkte mellem to to givne klynger, hvis antallet af noder er ens. Hvilket de sjældent er. Større klynger vil have sværere ved at opnå en høj densitet. Densitet er derfor et mål for et netværks forbundethed, men bliver nødt til at vurderes ud fra den konkrete sammenhæng for at være meningsfuld. %(indsæt reference til Scott, kapitel 4, \#todo)

%%%%%%%%%%%%%%%%%%%%%%%%%%%%%%%%%%%%%%%%%%%%%%
\section{Oprindelig Moneca densitetsmål}
%%%%%%%%%%%%%%%%%%%%%%%%%%%%%%%%%%%%%%%%%%%%%%


I tabel \ref{tab_app_densitet_original} ses densitetsmål for alle klynger, der består af over 2 noder. Det skyldes at for klynger med kun to noder er densiteten altid 1, da kun én forbindelse er mulig, og denne per definition er opfyldt. for enkeltstående noder er målet naturligvis ikke meningsfuldt. 


\begin{table}[H]
  \centering
  \resizebox{5cm}{!}{%
% Table generated by Excel2LaTeX from sheet 'app_netvaerksmaal_densitet_orig'
\begin{tabular}{lccc}
      &       & \multicolumn{1}{l}{Antal} & \multicolumn{1}{l}{Maksimal } \\
Klynge & \multicolumn{1}{l}{Densitet} & \multicolumn{1}{l}{noder} & \multicolumn{1}{l}{stilængde} \\
\midrule
4.14  & 0,46  & 16    &  3 \\
4.11  & 0,46  &  8    &  3 \\
4.6   & 0,47  & 22    &  3 \\
4.8   & 0,47  &  9    &  4 \\
5.1   & 0,49  & 28    &  3 \\
5.2   & 0,50  & 13    &  2 \\
4.2   & 0,50  &  6    &  3 \\
4.13  & 0,51  & 10    &  3 \\
3.8   & 0,55  &  8    &  2 \\
4.9   & 0,57  &  6    &  2 \\
3.20  & 0,60  &  7    &  3 \\
4.12  & 0,61  &  9    &  2 \\
4.4   & 0,63  &  6    &  2 \\
4.10  & 0,64  & 13    &  3 \\
3.21  & 0,64  &  7    &  3 \\
3.18  & 0,65  &  5    &  2 \\
3.25  & 0,67  &  3    &  2 \\
4.1   & 0,67  & 14    &  3 \\
3.30  & 0,73  &  6    &  2 \\
3.24  & 0,75  &  4    &  2 \\
3.37  & 0,75  &  4    &  2 \\
3.35  & 0,77  &  6    &  2 \\
3.9   & 0,80  &  5    &  2 \\
3.15  & 0,83  &  3    &  2 \\
3.2   & 0,83  &  4    &  2 \\
3.36  & 0,85  &  5    &  2 \\
3.14  & 0,86  &  7    &  2 \\
\bottomrule
\end{tabular}%
 
}
  \caption{Densitet i oprindelig Moneca}
  \label{tab_app_densitet_original}%
\end{table}%


% %!TEX root = ../report.tex

%%%%%%%%%%%%%%%%%%%%%%%%%%%%%%%%%%%%%%%%%%%%%%%%%%%%%%%%%%%
\newpage \chapter{Netværksmål \label{app_netvaerksmaal}}
%%%%%%%%%%%%%%%%%%%%%%%%%%%%%%%%%%%%%%%%%%%%%%%%%%%%%%%%%%%


Densiten i et netværk er et mål for “forbundetheden” i et netværk. Det vil sige antallet af forbindelser i et netværk i forhold til antallet af mulige forbindelser. Se Scott (\citeyear{Scott2000}), kapitel 4 for den matematiske definition. 

I Moneca benyttes densitet på en specifik måde: Ikke til at vurdere netværket som helhed, men til at vurdere hver enkelt klynges densitet. Her behandles hver enkelt klynge således som sit eget netværk. Densiteten er derfor udtryk for hvor godt hver enkelt klynge hænger sammen. Man skal dog holde for øje, at fordi densitet beregnes ud fra det totale antal mulige forbindelser, er det et \emph{relativt mål}. Det kan ikke sammenlignes på tværs af netværk af forskellig størrelse. Det betyder i Monecas tilfælde, at densitetsværdien kun kan sammenlignes direkte mellem to to givne klynger, hvis antallet af noder er ens. Hvilket de sjældent er. Større klynger vil have sværere ved at opnå en høj densitet. Densitet er derfor et mål for et netværks forbundethed, men bliver nødt til at vurderes ud fra den konkrete sammenhæng for at være meningsfuld. %(indsæt reference til Scott, kapitel 4, \#todo)

%%%%%%%%%%%%%%%%%%%%%%%%%%%%%%%%%%%%%%%%%%%%%%
\section{Oprindelig Moneca densitetsmål}
%%%%%%%%%%%%%%%%%%%%%%%%%%%%%%%%%%%%%%%%%%%%%%


I tabel \ref{tab_app_densitet_original} ses densitetsmål for alle klynger, der består af over 2 noder. Det skyldes at for klynger med kun to noder er densiteten altid 1, da kun én forbindelse er mulig, og denne per definition er opfyldt. for enkeltstående noder er målet naturligvis ikke meningsfuldt. 


\begin{table}[H]
  \centering
  \resizebox{5cm}{!}{%
% Table generated by Excel2LaTeX from sheet 'app_netvaerksmaal_densitet_orig'
\begin{tabular}{lccc}
      &       & \multicolumn{1}{l}{Antal} & \multicolumn{1}{l}{Maksimal } \\
Klynge & \multicolumn{1}{l}{Densitet} & \multicolumn{1}{l}{noder} & \multicolumn{1}{l}{stilængde} \\
\midrule
4.14  & 0,46  & 16    &  3 \\
4.11  & 0,46  &  8    &  3 \\
4.6   & 0,47  & 22    &  3 \\
4.8   & 0,47  &  9    &  4 \\
5.1   & 0,49  & 28    &  3 \\
5.2   & 0,50  & 13    &  2 \\
4.2   & 0,50  &  6    &  3 \\
4.13  & 0,51  & 10    &  3 \\
3.8   & 0,55  &  8    &  2 \\
4.9   & 0,57  &  6    &  2 \\
3.20  & 0,60  &  7    &  3 \\
4.12  & 0,61  &  9    &  2 \\
4.4   & 0,63  &  6    &  2 \\
4.10  & 0,64  & 13    &  3 \\
3.21  & 0,64  &  7    &  3 \\
3.18  & 0,65  &  5    &  2 \\
3.25  & 0,67  &  3    &  2 \\
4.1   & 0,67  & 14    &  3 \\
3.30  & 0,73  &  6    &  2 \\
3.24  & 0,75  &  4    &  2 \\
3.37  & 0,75  &  4    &  2 \\
3.35  & 0,77  &  6    &  2 \\
3.9   & 0,80  &  5    &  2 \\
3.15  & 0,83  &  3    &  2 \\
3.2   & 0,83  &  4    &  2 \\
3.36  & 0,85  &  5    &  2 \\
3.14  & 0,86  &  7    &  2 \\
\bottomrule
\end{tabular}%
 
}
  \caption{Densitet i oprindelig Moneca}
  \label{tab_app_densitet_original}%
\end{table}%


% %!TEX root = ../report.tex

%%%%%%%%%%%%%%%%%%%%%%%%%%%%%%%%%%%%%%%%%%%%%%%%%%%%%%%%%%%
\newpage \chapter{Netværksmål \label{app_netvaerksmaal}}
%%%%%%%%%%%%%%%%%%%%%%%%%%%%%%%%%%%%%%%%%%%%%%%%%%%%%%%%%%%


Densiten i et netværk er et mål for “forbundetheden” i et netværk. Det vil sige antallet af forbindelser i et netværk i forhold til antallet af mulige forbindelser. Se Scott (\citeyear{Scott2000}), kapitel 4 for den matematiske definition. 

I Moneca benyttes densitet på en specifik måde: Ikke til at vurdere netværket som helhed, men til at vurdere hver enkelt klynges densitet. Her behandles hver enkelt klynge således som sit eget netværk. Densiteten er derfor udtryk for hvor godt hver enkelt klynge hænger sammen. Man skal dog holde for øje, at fordi densitet beregnes ud fra det totale antal mulige forbindelser, er det et \emph{relativt mål}. Det kan ikke sammenlignes på tværs af netværk af forskellig størrelse. Det betyder i Monecas tilfælde, at densitetsværdien kun kan sammenlignes direkte mellem to to givne klynger, hvis antallet af noder er ens. Hvilket de sjældent er. Større klynger vil have sværere ved at opnå en høj densitet. Densitet er derfor et mål for et netværks forbundethed, men bliver nødt til at vurderes ud fra den konkrete sammenhæng for at være meningsfuld. %(indsæt reference til Scott, kapitel 4, \#todo)

%%%%%%%%%%%%%%%%%%%%%%%%%%%%%%%%%%%%%%%%%%%%%%
\section{Oprindelig Moneca densitetsmål}
%%%%%%%%%%%%%%%%%%%%%%%%%%%%%%%%%%%%%%%%%%%%%%


I tabel \ref{tab_app_densitet_original} ses densitetsmål for alle klynger, der består af over 2 noder. Det skyldes at for klynger med kun to noder er densiteten altid 1, da kun én forbindelse er mulig, og denne per definition er opfyldt. for enkeltstående noder er målet naturligvis ikke meningsfuldt. 


\begin{table}[H]
  \centering
  \resizebox{5cm}{!}{%
\input{./tabel/app_netvaerksmaal_densitet_orig} 
}
  \caption{Densitet i oprindelig Moneca}
  \label{tab_app_densitet_original}%
\end{table}%


% \input{tex/app_metode_netvaerksmaal}

Det ses af tabel \ref{tab_app_densitet_original}, De færreste klynger har en densitet på under 0,5. De fleste af de store klynger, med over 10 noder, ligger som forventet lavt på listen. Enkelte klynger så som 4.2 og 4.10 har ikke mange noder, og ligger alligevel lavt på listen. Følgende er en gennemgang af klyngerne med en densitet på under 0,5, da dette er tegn på, at klyngesammenlægningen muligvis er problematisk. 
%
\section{Klynger med lav densitet: Beskrivelse}
%

%
\begin{description}
\itemsep-0.3em

    \item[\underline{Klynge 3.8}] Det virker umiddelbart suspekt, når to beskæftigelser fra overgruppe 1, ledelsesarbejde, lægges sammen med arbejde der ligger hierarkisk lavere i statushierarkiet. Når densitetet samtidig er på ,55, giver det anledning til mistanke. Det viser sig dog, at de 7 ud af 8 noder ligger sammen allerede på nivea 2, og derfor har en densitet på 1. Det er først da \emak{d5121} kommer med på niveau 3, at densiteten falder til 0,55. Det skyldes at dette job ikke har nogen forbindelse til de to ledelsesjobs i klyngen, men har forbindelse til alle de andre jobs. Det er interessant, og bliver diskuteret nærmere i ?? \#todo. Det giver dog ikke anledning til at splitte klyngen op. Denne klynge beholdes som den er.

    \item[\underline{Klynge 4.2}] er en sammenlægning af en niveau 2-klynge og en niveau 3-klynge. Niveau 3 klyngen består af skibsrelateret arbejde, der enten handler om styring, administration eller teknisk vedligholdelse af skibe. Niveau 2 klyngen består af \emak{d6181} og \emak{d9213}. Ved nærmere eftersyn er den relative risiko faktisk ganske udemærket fordelt. Den lave densitet skyldes, at \emak{d9213} ikke har nogen forbindelse til niveau 3 klyngen. Men \emak{d6181} har ganske gode forbindelser til både medhjælp i fiskere samt niveau 3-klyngen. Jeg havde mistanke om at enten \emak{d6181} eller \emak{d9213} havde en enkelt forbindelse til en af beskæftigelserne i niveau 3-klyngen. Det er ikke tilfældet. Jeg vælger derfor at beholde denne klynge. 

    \item[\underline{Klynge 4.6}] er en sammenlægning af to store niveau 3 klynger. Disse har i sig selv ganske fine densiteter på over 0,6. I sammenlægningen på niveau 4, falder den dog drastisk, til 0,47. Ved at kigge på typerne af beskæftigelse i de to klynger, er det tydeligt, at der er tale om vidt forskelligt slags arbejde, omend det foregår på samme uddannelsesniveau. Den ene klynge drejer sig om samfundsvidenskab og humanora på universitetsniveau, inklusiv jura. Den anden klynge er også universitetsuddannede jobtyper, men her er der tale om naturvidenskab. Det er et godt spørgsmål om en sådan sammenlægning bør accepteres. Jeg vælger at splitte dem op, da en lav densitet netop går imod hvad Monecas formål er: At lave segmenter, hvor der er let og hyppig mobilitet mellem kategorierne. Denne klynge splittes op.

    \item[\underline{klynge 4.8}] Denne klynge har en maksimal stilængde på 4. Denne længde mellem noderne virker ikke troværdig, hvilket understøttes af en densitet på 0,47. Den interne mobilitet er ganske høj, 85 \%. Trods dette, virker det ganske enkelt utroværdigt, at beholde en klynge, hvor den maksimale stilængde er på 4. Det ses desuden, at sammenlægningen ligger socialt set meget forskellige grupper sammen. Det giver i sig selv anledning til omtanke, og når dette kombineres med de dårlige kvalitetsmål for aggregeringen, bliver det til mistanke. Denne klynge splittes op. 

    \item[\underline{Klynge 4.11}] Der er tale om at \emak{d3224} bliver lagt sammen med stillinger udelukkende fra overgruppe 1. Ved nærmere eftersyn ses det at jobbet optiker har forbindelse til \emak{d1224} \& \emak{d1239}, men ikke til de andre jobs i klyngen. Det giver mening at optikere kan få ledelsstillinger indenfor visse virksomheder, men det er misvisende når det kun gælder meget få af de andre i klyngen. Den interne mobilitet for optikere er i sig selv er på hele 93 \%,  hvorimod alle andre i klyngen har internt mobilitet i omegnen af gennemsnittet på 68 \%. På baggrund af disse to parametre, såvel som på den common sense indikator der ligge i, at alt andet i klyngen er ledelsesarbejde,  vælger jeg derfor at splitte klyngen op, som den var på niveau 3: Det vil sige, \emak{d3224} bliver ikke integreret i den såkaldte ledelsesklynge. 

    \item[\underline{Klynge 4.13}] Umiddelbart virker sammenlægningen af \emak{d3413} og en række kreative erhverv såsom \emak{d2455} betænkelig. Ved nærmere gennemgang af aggregeringsprocessen, ses det dog at denne sammenlægning sker tidligt, og har en ganske fin densitet på disse forstadier. Der kommer arkitekt-relateret arbejde på i niveau 4, hvilket giver ganske fin mening. Klyngen bliver ikke splittet op. 

    \item[\underline{Klynge 4.14}] har den laveste densitet overhovedet, på 0,46, men har tilgengæld en meget høj intern mobilitet. Dette skyldes \emph{ikke} at noderne i sig selv har en “naturlig” høj intern jobmobilitet: I så fald ville den høje interne mobilitet på nodeniveau, i kombination med lav densitet, jo være udtryk for, at der  var tale om noder der intet har med de andre at gøre, men er blevet sat kunstigt sammen, så at sige. Det er ikke tilfældet her. Den høje jobmobilitet er netop fremkommet som resultat af sammenlægningen. De to sammenlagte niveau 3-klynger ser ud til at have en vis grad af forskel i typen af arbejde. Densiteten stiger væsentligt ved en opdeling. Denne klynge splittes op. 

    \item[\underline{Klynge 5.1 \& 5.2}] består fortrinsvis af manuelt arbejde. Her er problematikken, at densiteten er ret lav på det 5. niveau. Men på de lavere niveauer er den interne mobilitet tilgengæld væsentligt under enkeltnode-niveauet på 68 \%. Dette siger til omtrent 70 \% for begge på dette højere niveau. Jeg vælger at beholde dem som de er, fordi tabet i densitet er acceptabelt i forhold til tabet i intern mobilitet. Grunden er, at den interne mobilitet er mit primære grundlag for netværket, og jeg vælger derfor at prioritere dette højere. Der skal højes for øje i analysen, at disse to klynger ikke nødvendigvis har de tætteste forbindelse alle sammen “sammen”, men tilgengæld er mobilitet god langs de kanaler, der nu engang findes i klyngen. Ydermere er den interne mobilitet en stærkere indikator, da den siger noget om styrken af forbindelse mellem noderne, hvor densiteten bare beregner hvor mange forbindelser, der er mulige, uden at tage hensyn til \emph{styrken} af disse forskellige forbindelser. 

\end{description}
% 
Efter denne beskrivelse af opsplitninger, går vi videre til den endelige, modificerede klyngedannelse, og densitetsmålene for disse. 

%
\section{Endelige densitetsmål efter opsplitning}
%

Densitetsmålene for de endelige klynger kan aflæses i tabel \ref{tab_app_densitet_final}. Der er ikke anledning til mange kommentarer, da de muligt problematiske klyngedannelser er blevet vurderet, og der er redegjort til deres fortsatte eksistens. Det ses at langt de fleste klynger har en tilfredsstillende densitet samt stilængde. I forhold til netværksmål er der ikke anledning til yderligere kommentarer end i foregående afsnit. 
%
\begin{table}[H]
  \centering
    \resizebox{5cm}{!}{%
\input{./tabel/app_netvaerksmaal_densitet_fina}
}
  \caption{Densitet i endelig Moneca version}
  \label{tab_app_densitet_final}%
\end{table}
%


Det ses af tabel \ref{tab_app_densitet_original}, De færreste klynger har en densitet på under 0,5. De fleste af de store klynger, med over 10 noder, ligger som forventet lavt på listen. Enkelte klynger så som 4.2 og 4.10 har ikke mange noder, og ligger alligevel lavt på listen. Følgende er en gennemgang af klyngerne med en densitet på under 0,5, da dette er tegn på, at klyngesammenlægningen muligvis er problematisk. 
%
\section{Klynger med lav densitet: Beskrivelse}
%

%
\begin{description}
\itemsep-0.3em

    \item[\underline{Klynge 3.8}] Det virker umiddelbart suspekt, når to beskæftigelser fra overgruppe 1, ledelsesarbejde, lægges sammen med arbejde der ligger hierarkisk lavere i statushierarkiet. Når densitetet samtidig er på ,55, giver det anledning til mistanke. Det viser sig dog, at de 7 ud af 8 noder ligger sammen allerede på nivea 2, og derfor har en densitet på 1. Det er først da \emak{d5121} kommer med på niveau 3, at densiteten falder til 0,55. Det skyldes at dette job ikke har nogen forbindelse til de to ledelsesjobs i klyngen, men har forbindelse til alle de andre jobs. Det er interessant, og bliver diskuteret nærmere i ?? \#todo. Det giver dog ikke anledning til at splitte klyngen op. Denne klynge beholdes som den er.

    \item[\underline{Klynge 4.2}] er en sammenlægning af en niveau 2-klynge og en niveau 3-klynge. Niveau 3 klyngen består af skibsrelateret arbejde, der enten handler om styring, administration eller teknisk vedligholdelse af skibe. Niveau 2 klyngen består af \emak{d6181} og \emak{d9213}. Ved nærmere eftersyn er den relative risiko faktisk ganske udemærket fordelt. Den lave densitet skyldes, at \emak{d9213} ikke har nogen forbindelse til niveau 3 klyngen. Men \emak{d6181} har ganske gode forbindelser til både medhjælp i fiskere samt niveau 3-klyngen. Jeg havde mistanke om at enten \emak{d6181} eller \emak{d9213} havde en enkelt forbindelse til en af beskæftigelserne i niveau 3-klyngen. Det er ikke tilfældet. Jeg vælger derfor at beholde denne klynge. 

    \item[\underline{Klynge 4.6}] er en sammenlægning af to store niveau 3 klynger. Disse har i sig selv ganske fine densiteter på over 0,6. I sammenlægningen på niveau 4, falder den dog drastisk, til 0,47. Ved at kigge på typerne af beskæftigelse i de to klynger, er det tydeligt, at der er tale om vidt forskelligt slags arbejde, omend det foregår på samme uddannelsesniveau. Den ene klynge drejer sig om samfundsvidenskab og humanora på universitetsniveau, inklusiv jura. Den anden klynge er også universitetsuddannede jobtyper, men her er der tale om naturvidenskab. Det er et godt spørgsmål om en sådan sammenlægning bør accepteres. Jeg vælger at splitte dem op, da en lav densitet netop går imod hvad Monecas formål er: At lave segmenter, hvor der er let og hyppig mobilitet mellem kategorierne. Denne klynge splittes op.

    \item[\underline{klynge 4.8}] Denne klynge har en maksimal stilængde på 4. Denne længde mellem noderne virker ikke troværdig, hvilket understøttes af en densitet på 0,47. Den interne mobilitet er ganske høj, 85 \%. Trods dette, virker det ganske enkelt utroværdigt, at beholde en klynge, hvor den maksimale stilængde er på 4. Det ses desuden, at sammenlægningen ligger socialt set meget forskellige grupper sammen. Det giver i sig selv anledning til omtanke, og når dette kombineres med de dårlige kvalitetsmål for aggregeringen, bliver det til mistanke. Denne klynge splittes op. 

    \item[\underline{Klynge 4.11}] Der er tale om at \emak{d3224} bliver lagt sammen med stillinger udelukkende fra overgruppe 1. Ved nærmere eftersyn ses det at jobbet optiker har forbindelse til \emak{d1224} \& \emak{d1239}, men ikke til de andre jobs i klyngen. Det giver mening at optikere kan få ledelsstillinger indenfor visse virksomheder, men det er misvisende når det kun gælder meget få af de andre i klyngen. Den interne mobilitet for optikere er i sig selv er på hele 93 \%,  hvorimod alle andre i klyngen har internt mobilitet i omegnen af gennemsnittet på 68 \%. På baggrund af disse to parametre, såvel som på den common sense indikator der ligge i, at alt andet i klyngen er ledelsesarbejde,  vælger jeg derfor at splitte klyngen op, som den var på niveau 3: Det vil sige, \emak{d3224} bliver ikke integreret i den såkaldte ledelsesklynge. 

    \item[\underline{Klynge 4.13}] Umiddelbart virker sammenlægningen af \emak{d3413} og en række kreative erhverv såsom \emak{d2455} betænkelig. Ved nærmere gennemgang af aggregeringsprocessen, ses det dog at denne sammenlægning sker tidligt, og har en ganske fin densitet på disse forstadier. Der kommer arkitekt-relateret arbejde på i niveau 4, hvilket giver ganske fin mening. Klyngen bliver ikke splittet op. 

    \item[\underline{Klynge 4.14}] har den laveste densitet overhovedet, på 0,46, men har tilgengæld en meget høj intern mobilitet. Dette skyldes \emph{ikke} at noderne i sig selv har en “naturlig” høj intern jobmobilitet: I så fald ville den høje interne mobilitet på nodeniveau, i kombination med lav densitet, jo være udtryk for, at der  var tale om noder der intet har med de andre at gøre, men er blevet sat kunstigt sammen, så at sige. Det er ikke tilfældet her. Den høje jobmobilitet er netop fremkommet som resultat af sammenlægningen. De to sammenlagte niveau 3-klynger ser ud til at have en vis grad af forskel i typen af arbejde. Densiteten stiger væsentligt ved en opdeling. Denne klynge splittes op. 

    \item[\underline{Klynge 5.1 \& 5.2}] består fortrinsvis af manuelt arbejde. Her er problematikken, at densiteten er ret lav på det 5. niveau. Men på de lavere niveauer er den interne mobilitet tilgengæld væsentligt under enkeltnode-niveauet på 68 \%. Dette siger til omtrent 70 \% for begge på dette højere niveau. Jeg vælger at beholde dem som de er, fordi tabet i densitet er acceptabelt i forhold til tabet i intern mobilitet. Grunden er, at den interne mobilitet er mit primære grundlag for netværket, og jeg vælger derfor at prioritere dette højere. Der skal højes for øje i analysen, at disse to klynger ikke nødvendigvis har de tætteste forbindelse alle sammen “sammen”, men tilgengæld er mobilitet god langs de kanaler, der nu engang findes i klyngen. Ydermere er den interne mobilitet en stærkere indikator, da den siger noget om styrken af forbindelse mellem noderne, hvor densiteten bare beregner hvor mange forbindelser, der er mulige, uden at tage hensyn til \emph{styrken} af disse forskellige forbindelser. 

\end{description}
% 
Efter denne beskrivelse af opsplitninger, går vi videre til den endelige, modificerede klyngedannelse, og densitetsmålene for disse. 

%
\section{Endelige densitetsmål efter opsplitning}
%

Densitetsmålene for de endelige klynger kan aflæses i tabel \ref{tab_app_densitet_final}. Der er ikke anledning til mange kommentarer, da de muligt problematiske klyngedannelser er blevet vurderet, og der er redegjort til deres fortsatte eksistens. Det ses at langt de fleste klynger har en tilfredsstillende densitet samt stilængde. I forhold til netværksmål er der ikke anledning til yderligere kommentarer end i foregående afsnit. 
%
\begin{table}[H]
  \centering
    \resizebox{5cm}{!}{%
% Table generated by Excel2LaTeX from sheet 'app_netvaerksmaal_densitet_fina'
\begin{tabular}{lrrr}
      &       & \multicolumn{1}{l}{Antal} & \multicolumn{1}{l}{Maksimal } \\
Klynge & \multicolumn{1}{l}{Densitet} & \multicolumn{1}{l}{noder} & \multicolumn{1}{l}{stilængde} \\
\midrule
5.1   & 0,49  & 28    & 3 \\
5.2   & 0,50  & 13    & 2 \\
4.2   & 0,50  & 6     & 3 \\
4.10  & 0,51  & 10    & 3 \\
3.8   & 0,55  & 8     & 2 \\
4.7   & 0,57  & 6     & 2 \\
3.20  & 0,60  & 7     & 3 \\
3.26  & 0,60  & 5     & 3 \\
4.9   & 0,61  & 9     & 2 \\
3.3   & 0,61  & 10    & 3 \\
4.4   & 0,63  & 6     & 2 \\
4.8   & 0,64  & 13    & 3 \\
3.21  & 0,64  & 7     & 3 \\
3.18  & 0,65  & 5     & 2 \\
3.25  & 0,67  & 3     & 2 \\
4.1   & 0,67  & 14    & 3 \\
3.4   & 0,67  & 12    & 3 \\
3.30  & 0,73  & 6     & 2 \\
3.34  & 0,74  & 9     & 2 \\
3.24  & 0,75  & 4     & 2 \\
3.33  & 0,75  & 4     & 2 \\
3.37  & 0,75  & 4     & 2 \\
3.29  & 0,75  & 4     & 2 \\
3.12  & 0,75  & 4     & 2 \\
3.35  & 0,77  & 6     & 2 \\
3.9   & 0,80  & 5     & 2 \\
3.15  & 0,83  & 3     & 2 \\
3.2   & 0,83  & 4     & 2 \\
3.36  & 0,85  & 5     & 2 \\
3.14  & 0,86  & 7     & 2 \\
3.7   & 0,90  & 7     & 2 \\
\bottomrule
\end{tabular}%
}
  \caption{Densitet i endelig Moneca version}
  \label{tab_app_densitet_final}%
\end{table}
%


Det ses af tabel \ref{tab_app_densitet_original}, De færreste klynger har en densitet på under 0,5. De fleste af de store klynger, med over 10 noder, ligger som forventet lavt på listen. Enkelte klynger så som 4.2 og 4.10 har ikke mange noder, og ligger alligevel lavt på listen. Følgende er en gennemgang af klyngerne med en densitet på under 0,5, da dette er tegn på, at klyngesammenlægningen muligvis er problematisk. 
%
\section{Klynger med lav densitet: Beskrivelse}
%

%
\begin{description}
\itemsep-0.3em

    \item[\underline{Klynge 3.8}] Det virker umiddelbart suspekt, når to beskæftigelser fra overgruppe 1, ledelsesarbejde, lægges sammen med arbejde der ligger hierarkisk lavere i statushierarkiet. Når densitetet samtidig er på ,55, giver det anledning til mistanke. Det viser sig dog, at de 7 ud af 8 noder ligger sammen allerede på nivea 2, og derfor har en densitet på 1. Det er først da \emak{d5121} kommer med på niveau 3, at densiteten falder til 0,55. Det skyldes at dette job ikke har nogen forbindelse til de to ledelsesjobs i klyngen, men har forbindelse til alle de andre jobs. Det er interessant, og bliver diskuteret nærmere i ?? \#todo. Det giver dog ikke anledning til at splitte klyngen op. Denne klynge beholdes som den er.

    \item[\underline{Klynge 4.2}] er en sammenlægning af en niveau 2-klynge og en niveau 3-klynge. Niveau 3 klyngen består af skibsrelateret arbejde, der enten handler om styring, administration eller teknisk vedligholdelse af skibe. Niveau 2 klyngen består af \emak{d6181} og \emak{d9213}. Ved nærmere eftersyn er den relative risiko faktisk ganske udemærket fordelt. Den lave densitet skyldes, at \emak{d9213} ikke har nogen forbindelse til niveau 3 klyngen. Men \emak{d6181} har ganske gode forbindelser til både medhjælp i fiskere samt niveau 3-klyngen. Jeg havde mistanke om at enten \emak{d6181} eller \emak{d9213} havde en enkelt forbindelse til en af beskæftigelserne i niveau 3-klyngen. Det er ikke tilfældet. Jeg vælger derfor at beholde denne klynge. 

    \item[\underline{Klynge 4.6}] er en sammenlægning af to store niveau 3 klynger. Disse har i sig selv ganske fine densiteter på over 0,6. I sammenlægningen på niveau 4, falder den dog drastisk, til 0,47. Ved at kigge på typerne af beskæftigelse i de to klynger, er det tydeligt, at der er tale om vidt forskelligt slags arbejde, omend det foregår på samme uddannelsesniveau. Den ene klynge drejer sig om samfundsvidenskab og humanora på universitetsniveau, inklusiv jura. Den anden klynge er også universitetsuddannede jobtyper, men her er der tale om naturvidenskab. Det er et godt spørgsmål om en sådan sammenlægning bør accepteres. Jeg vælger at splitte dem op, da en lav densitet netop går imod hvad Monecas formål er: At lave segmenter, hvor der er let og hyppig mobilitet mellem kategorierne. Denne klynge splittes op.

    \item[\underline{klynge 4.8}] Denne klynge har en maksimal stilængde på 4. Denne længde mellem noderne virker ikke troværdig, hvilket understøttes af en densitet på 0,47. Den interne mobilitet er ganske høj, 85 \%. Trods dette, virker det ganske enkelt utroværdigt, at beholde en klynge, hvor den maksimale stilængde er på 4. Det ses desuden, at sammenlægningen ligger socialt set meget forskellige grupper sammen. Det giver i sig selv anledning til omtanke, og når dette kombineres med de dårlige kvalitetsmål for aggregeringen, bliver det til mistanke. Denne klynge splittes op. 

    \item[\underline{Klynge 4.11}] Der er tale om at \emak{d3224} bliver lagt sammen med stillinger udelukkende fra overgruppe 1. Ved nærmere eftersyn ses det at jobbet optiker har forbindelse til \emak{d1224} \& \emak{d1239}, men ikke til de andre jobs i klyngen. Det giver mening at optikere kan få ledelsstillinger indenfor visse virksomheder, men det er misvisende når det kun gælder meget få af de andre i klyngen. Den interne mobilitet for optikere er i sig selv er på hele 93 \%,  hvorimod alle andre i klyngen har internt mobilitet i omegnen af gennemsnittet på 68 \%. På baggrund af disse to parametre, såvel som på den common sense indikator der ligge i, at alt andet i klyngen er ledelsesarbejde,  vælger jeg derfor at splitte klyngen op, som den var på niveau 3: Det vil sige, \emak{d3224} bliver ikke integreret i den såkaldte ledelsesklynge. 

    \item[\underline{Klynge 4.13}] Umiddelbart virker sammenlægningen af \emak{d3413} og en række kreative erhverv såsom \emak{d2455} betænkelig. Ved nærmere gennemgang af aggregeringsprocessen, ses det dog at denne sammenlægning sker tidligt, og har en ganske fin densitet på disse forstadier. Der kommer arkitekt-relateret arbejde på i niveau 4, hvilket giver ganske fin mening. Klyngen bliver ikke splittet op. 

    \item[\underline{Klynge 4.14}] har den laveste densitet overhovedet, på 0,46, men har tilgengæld en meget høj intern mobilitet. Dette skyldes \emph{ikke} at noderne i sig selv har en “naturlig” høj intern jobmobilitet: I så fald ville den høje interne mobilitet på nodeniveau, i kombination med lav densitet, jo være udtryk for, at der  var tale om noder der intet har med de andre at gøre, men er blevet sat kunstigt sammen, så at sige. Det er ikke tilfældet her. Den høje jobmobilitet er netop fremkommet som resultat af sammenlægningen. De to sammenlagte niveau 3-klynger ser ud til at have en vis grad af forskel i typen af arbejde. Densiteten stiger væsentligt ved en opdeling. Denne klynge splittes op. 

    \item[\underline{Klynge 5.1 \& 5.2}] består fortrinsvis af manuelt arbejde. Her er problematikken, at densiteten er ret lav på det 5. niveau. Men på de lavere niveauer er den interne mobilitet tilgengæld væsentligt under enkeltnode-niveauet på 68 \%. Dette siger til omtrent 70 \% for begge på dette højere niveau. Jeg vælger at beholde dem som de er, fordi tabet i densitet er acceptabelt i forhold til tabet i intern mobilitet. Grunden er, at den interne mobilitet er mit primære grundlag for netværket, og jeg vælger derfor at prioritere dette højere. Der skal højes for øje i analysen, at disse to klynger ikke nødvendigvis har de tætteste forbindelse alle sammen “sammen”, men tilgengæld er mobilitet god langs de kanaler, der nu engang findes i klyngen. Ydermere er den interne mobilitet en stærkere indikator, da den siger noget om styrken af forbindelse mellem noderne, hvor densiteten bare beregner hvor mange forbindelser, der er mulige, uden at tage hensyn til \emph{styrken} af disse forskellige forbindelser. 

\end{description}
% 
Efter denne beskrivelse af opsplitninger, går vi videre til den endelige, modificerede klyngedannelse, og densitetsmålene for disse. 

%
\section{Endelige densitetsmål efter opsplitning}
%

Densitetsmålene for de endelige klynger kan aflæses i tabel \ref{tab_app_densitet_final}. Der er ikke anledning til mange kommentarer, da de muligt problematiske klyngedannelser er blevet vurderet, og der er redegjort til deres fortsatte eksistens. Det ses at langt de fleste klynger har en tilfredsstillende densitet samt stilængde. I forhold til netværksmål er der ikke anledning til yderligere kommentarer end i foregående afsnit. 
%
\begin{table}[H]
  \centering
    \resizebox{5cm}{!}{%
% Table generated by Excel2LaTeX from sheet 'app_netvaerksmaal_densitet_fina'
\begin{tabular}{lrrr}
      &       & \multicolumn{1}{l}{Antal} & \multicolumn{1}{l}{Maksimal } \\
Klynge & \multicolumn{1}{l}{Densitet} & \multicolumn{1}{l}{noder} & \multicolumn{1}{l}{stilængde} \\
\midrule
5.1   & 0,49  & 28    & 3 \\
5.2   & 0,50  & 13    & 2 \\
4.2   & 0,50  & 6     & 3 \\
4.10  & 0,51  & 10    & 3 \\
3.8   & 0,55  & 8     & 2 \\
4.7   & 0,57  & 6     & 2 \\
3.20  & 0,60  & 7     & 3 \\
3.26  & 0,60  & 5     & 3 \\
4.9   & 0,61  & 9     & 2 \\
3.3   & 0,61  & 10    & 3 \\
4.4   & 0,63  & 6     & 2 \\
4.8   & 0,64  & 13    & 3 \\
3.21  & 0,64  & 7     & 3 \\
3.18  & 0,65  & 5     & 2 \\
3.25  & 0,67  & 3     & 2 \\
4.1   & 0,67  & 14    & 3 \\
3.4   & 0,67  & 12    & 3 \\
3.30  & 0,73  & 6     & 2 \\
3.34  & 0,74  & 9     & 2 \\
3.24  & 0,75  & 4     & 2 \\
3.33  & 0,75  & 4     & 2 \\
3.37  & 0,75  & 4     & 2 \\
3.29  & 0,75  & 4     & 2 \\
3.12  & 0,75  & 4     & 2 \\
3.35  & 0,77  & 6     & 2 \\
3.9   & 0,80  & 5     & 2 \\
3.15  & 0,83  & 3     & 2 \\
3.2   & 0,83  & 4     & 2 \\
3.36  & 0,85  & 5     & 2 \\
3.14  & 0,86  & 7     & 2 \\
3.7   & 0,90  & 7     & 2 \\
\bottomrule
\end{tabular}%
}
  \caption{Densitet i endelig Moneca version}
  \label{tab_app_densitet_final}%
\end{table}
%


Det ses af tabel \ref{tab_app_densitet_original}, De færreste klynger har en densitet på under 0,5. De fleste af de store klynger, med over 10 noder, ligger som forventet lavt på listen. Enkelte klynger så som 4.2 og 4.10 har ikke mange noder, og ligger alligevel lavt på listen. Følgende er en gennemgang af klyngerne med en densitet på under 0,5, da dette er tegn på, at klyngesammenlægningen muligvis er problematisk. 
%
\section{Klynger med lav densitet: Beskrivelse}
%

%
\begin{description}
\itemsep-0.3em

    \item[\underline{Klynge 3.8}] Det virker umiddelbart suspekt, når to beskæftigelser fra overgruppe 1, ledelsesarbejde, lægges sammen med arbejde der ligger hierarkisk lavere i statushierarkiet. Når densitetet samtidig er på ,55, giver det anledning til mistanke. Det viser sig dog, at de 7 ud af 8 noder ligger sammen allerede på nivea 2, og derfor har en densitet på 1. Det er først da \emak{d5121} kommer med på niveau 3, at densiteten falder til 0,55. Det skyldes at dette job ikke har nogen forbindelse til de to ledelsesjobs i klyngen, men har forbindelse til alle de andre jobs. Det er interessant, og bliver diskuteret nærmere i ?? \#todo. Det giver dog ikke anledning til at splitte klyngen op. Denne klynge beholdes som den er.

    \item[\underline{Klynge 4.2}] er en sammenlægning af en niveau 2-klynge og en niveau 3-klynge. Niveau 3 klyngen består af skibsrelateret arbejde, der enten handler om styring, administration eller teknisk vedligholdelse af skibe. Niveau 2 klyngen består af \emak{d6181} og \emak{d9213}. Ved nærmere eftersyn er den relative risiko faktisk ganske udemærket fordelt. Den lave densitet skyldes, at \emak{d9213} ikke har nogen forbindelse til niveau 3 klyngen. Men \emak{d6181} har ganske gode forbindelser til både medhjælp i fiskere samt niveau 3-klyngen. Jeg havde mistanke om at enten \emak{d6181} eller \emak{d9213} havde en enkelt forbindelse til en af beskæftigelserne i niveau 3-klyngen. Det er ikke tilfældet. Jeg vælger derfor at beholde denne klynge. 

    \item[\underline{Klynge 4.6}] er en sammenlægning af to store niveau 3 klynger. Disse har i sig selv ganske fine densiteter på over 0,6. I sammenlægningen på niveau 4, falder den dog drastisk, til 0,47. Ved at kigge på typerne af beskæftigelse i de to klynger, er det tydeligt, at der er tale om vidt forskelligt slags arbejde, omend det foregår på samme uddannelsesniveau. Den ene klynge drejer sig om samfundsvidenskab og humanora på universitetsniveau, inklusiv jura. Den anden klynge er også universitetsuddannede jobtyper, men her er der tale om naturvidenskab. Det er et godt spørgsmål om en sådan sammenlægning bør accepteres. Jeg vælger at splitte dem op, da en lav densitet netop går imod hvad Monecas formål er: At lave segmenter, hvor der er let og hyppig mobilitet mellem kategorierne. Denne klynge splittes op.

    \item[\underline{klynge 4.8}] Denne klynge har en maksimal stilængde på 4. Denne længde mellem noderne virker ikke troværdig, hvilket understøttes af en densitet på 0,47. Den interne mobilitet er ganske høj, 85 \%. Trods dette, virker det ganske enkelt utroværdigt, at beholde en klynge, hvor den maksimale stilængde er på 4. Det ses desuden, at sammenlægningen ligger socialt set meget forskellige grupper sammen. Det giver i sig selv anledning til omtanke, og når dette kombineres med de dårlige kvalitetsmål for aggregeringen, bliver det til mistanke. Denne klynge splittes op. 

    \item[\underline{Klynge 4.11}] Der er tale om at \emak{d3224} bliver lagt sammen med stillinger udelukkende fra overgruppe 1. Ved nærmere eftersyn ses det at jobbet optiker har forbindelse til \emak{d1224} \& \emak{d1239}, men ikke til de andre jobs i klyngen. Det giver mening at optikere kan få ledelsstillinger indenfor visse virksomheder, men det er misvisende når det kun gælder meget få af de andre i klyngen. Den interne mobilitet for optikere er i sig selv er på hele 93 \%,  hvorimod alle andre i klyngen har internt mobilitet i omegnen af gennemsnittet på 68 \%. På baggrund af disse to parametre, såvel som på den common sense indikator der ligge i, at alt andet i klyngen er ledelsesarbejde,  vælger jeg derfor at splitte klyngen op, som den var på niveau 3: Det vil sige, \emak{d3224} bliver ikke integreret i den såkaldte ledelsesklynge. 

    \item[\underline{Klynge 4.13}] Umiddelbart virker sammenlægningen af \emak{d3413} og en række kreative erhverv såsom \emak{d2455} betænkelig. Ved nærmere gennemgang af aggregeringsprocessen, ses det dog at denne sammenlægning sker tidligt, og har en ganske fin densitet på disse forstadier. Der kommer arkitekt-relateret arbejde på i niveau 4, hvilket giver ganske fin mening. Klyngen bliver ikke splittet op. 

    \item[\underline{Klynge 4.14}] har den laveste densitet overhovedet, på 0,46, men har tilgengæld en meget høj intern mobilitet. Dette skyldes \emph{ikke} at noderne i sig selv har en “naturlig” høj intern jobmobilitet: I så fald ville den høje interne mobilitet på nodeniveau, i kombination med lav densitet, jo være udtryk for, at der  var tale om noder der intet har med de andre at gøre, men er blevet sat kunstigt sammen, så at sige. Det er ikke tilfældet her. Den høje jobmobilitet er netop fremkommet som resultat af sammenlægningen. De to sammenlagte niveau 3-klynger ser ud til at have en vis grad af forskel i typen af arbejde. Densiteten stiger væsentligt ved en opdeling. Denne klynge splittes op. 

    \item[\underline{Klynge 5.1 \& 5.2}] består fortrinsvis af manuelt arbejde. Her er problematikken, at densiteten er ret lav på det 5. niveau. Men på de lavere niveauer er den interne mobilitet tilgengæld væsentligt under enkeltnode-niveauet på 68 \%. Dette siger til omtrent 70 \% for begge på dette højere niveau. Jeg vælger at beholde dem som de er, fordi tabet i densitet er acceptabelt i forhold til tabet i intern mobilitet. Grunden er, at den interne mobilitet er mit primære grundlag for netværket, og jeg vælger derfor at prioritere dette højere. Der skal højes for øje i analysen, at disse to klynger ikke nødvendigvis har de tætteste forbindelse alle sammen “sammen”, men tilgengæld er mobilitet god langs de kanaler, der nu engang findes i klyngen. Ydermere er den interne mobilitet en stærkere indikator, da den siger noget om styrken af forbindelse mellem noderne, hvor densiteten bare beregner hvor mange forbindelser, der er mulige, uden at tage hensyn til \emph{styrken} af disse forskellige forbindelser. 

\end{description}
% 
Efter denne beskrivelse af opsplitninger, går vi videre til den endelige, modificerede klyngedannelse, og densitetsmålene for disse. 


%
\section{Hermeneutiske sammenlægninger}
%

kriterie: skal gå begge veje. må ikke nedsætte den gns interne mobilitet betydeligt, densitet og maks stilængde vigtigt. dvs hermeneutik med begrænsninger. 


\emak{d5162} kunne ikke sættes sammen med klynge 4.7, da densiteten faldt til under 50 \%.  Alle de enkeltstående noder, der er tilbage, er ikke blevet sammenlagt med de mest oplagte klynger af netop denne grund. 

\emak{d8110} sat sammen med klynge 5.2. 


læge og arbejde med religion, intern mobilitet på 95 %. de steder de går hen er de blevet sat sammen med, men burde de stå alene? Måske. i stedet har jeg markeret i parentes hvor den lille mobilitet, de har, går hen. 


\emak{d7224} kunne være enten i 4.8 eller 5.1. intern mob stiger lidt. men i 5.1 falder densiteten en del, og i 4.8 kun lidt. derfor vælges 4.8. altså et klart valg fra min side, men man kan argumentere for at det ikke er mindre et valg end hvad moneca selv laver, og her er det informeret af menneskelig vurdering oveni. 


ligeledes er niveau 2 klyngerne forsøgt sammenlagt ud fra samme principper. 
  Dette er klynge 3.39 et eksempel på. Densiteten i klyngen falder 1, grundet at det er to niveau 2-klynger, der per definition har direkte forbindelse til alle andre. Men tilgengæld stiger den interne gns mobilitet, og klyngens interne mobilitet er skyhøj på 94 \%. Dette er den eneste niveau-sammenlægning der er foretaget manuelt, de resterende 9 klynger har ikke haft tilfredsstillende aggregeringsresultater på såvel hermeneutisk og teknisk niveau.  
  



50 78,79 % Monecas forslag

54 79,09 % Teknisk brydning af klynger

47 81,06 % hermeneutiske forslag der bliver godtaget





%
\section{Endelige densitetsmål efter opsplitning}
%

Densitetsmålene for de endelige klynger kan aflæses i tabel \ref{tab_app_densitet_final}. Der er ikke anledning til mange kommentarer, da de muligt problematiske klyngedannelser er blevet vurderet, og der er redegjort til deres fortsatte eksistens. Det ses at langt de fleste klynger har en tilfredsstillende densitet samt stilængde. I forhold til netværksmål er der ikke anledning til yderligere kommentarer end i foregående afsnit. 
%
\begin{table}[H]
  \centering
    \resizebox{5cm}{!}{%
% Table generated by Excel2LaTeX from sheet 'app_netvaerksmaal_densitet_fina'
\begin{tabular}{lrrr}
      &       & \multicolumn{1}{l}{Antal} & \multicolumn{1}{l}{Maksimal } \\
Klynge & \multicolumn{1}{l}{Densitet} & \multicolumn{1}{l}{noder} & \multicolumn{1}{l}{stilængde} \\
\midrule
5.1   & 0,49  & 28    & 3 \\
5.2   & 0,50  & 13    & 2 \\
4.2   & 0,50  & 6     & 3 \\
4.10  & 0,51  & 10    & 3 \\
3.8   & 0,55  & 8     & 2 \\
4.7   & 0,57  & 6     & 2 \\
3.20  & 0,60  & 7     & 3 \\
3.26  & 0,60  & 5     & 3 \\
4.9   & 0,61  & 9     & 2 \\
3.3   & 0,61  & 10    & 3 \\
4.4   & 0,63  & 6     & 2 \\
4.8   & 0,64  & 13    & 3 \\
3.21  & 0,64  & 7     & 3 \\
3.18  & 0,65  & 5     & 2 \\
3.25  & 0,67  & 3     & 2 \\
4.1   & 0,67  & 14    & 3 \\
3.4   & 0,67  & 12    & 3 \\
3.30  & 0,73  & 6     & 2 \\
3.34  & 0,74  & 9     & 2 \\
3.24  & 0,75  & 4     & 2 \\
3.33  & 0,75  & 4     & 2 \\
3.37  & 0,75  & 4     & 2 \\
3.29  & 0,75  & 4     & 2 \\
3.12  & 0,75  & 4     & 2 \\
3.35  & 0,77  & 6     & 2 \\
3.9   & 0,80  & 5     & 2 \\
3.15  & 0,83  & 3     & 2 \\
3.2   & 0,83  & 4     & 2 \\
3.36  & 0,85  & 5     & 2 \\
3.14  & 0,86  & 7     & 2 \\
3.7   & 0,90  & 7     & 2 \\
\bottomrule
\end{tabular}%
}
  \caption{Densitet i endelig Moneca version}
  \label{tab_app_densitet_final}%
\end{table}
%
