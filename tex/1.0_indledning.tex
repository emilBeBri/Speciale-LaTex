%!TEX root = ../report.tex

%%%%%%%%%%%%%%%%%%%%%%%%%%%%%%%%%%%%%%%%%%%%%%%%%%%%%%%%%%%
\chapter{Indledning \label{indledning}}
%%%%%%%%%%%%%%%%%%%%%%%%%%%%%%%%%%%%%%%%%%%%%%%%%%%%%%%%%%%

Dette speciale er en undersøgelse af hvorvidt arbejdsmarkedet er delt op i flere forskellige segmenter, og i så fald, hvordan en sociologisk analyse kan belyse disse segmenters mulige forskellige funktionsmåder.

Hvorfor er det interessant? Der meget lidt sociologisk teori, der beskæftiger sig med arbejdsmarkedets funktionsmåder på et mesoniveau, og samtidig er åben overfor en empirisk kortlægning. For det meste beskæftiger den arbejdsmarkedssociologiske teori (mig bekendt) enten med et kvalitativt orienteret mikroniveau, eller de helt brede penselstrøg på makroniveau.

Hvorfor er det vigtigt med undersøgelse af arbejdsmarkedet på et mesoniveau, når der laves undersøgelser på et mikro- og makroniveau? Viden om arbejdsmarkedet på et mikroniveau giver et billedet af et bestemt og snævert sted på arbejdsmarkedet og altså ikke viden hele arbejdsmarkedet. Viden om arbejdsmarkedet på et makroniveau giver et billede af abstrakte lovmæssigheder, der skal forklare en eller anden tendens i moderne samfund generelt. Dette er ikke en ny kritik (se GER og Merton find henvisninger \#todo), men noget, der tilsyneladende alligevel ofte glemmes. Med en undersøgelse på mesoniveau fås et billede af hvordan arbejdsmarkdet i et givet land fungerer lige nu.

Den teori, der bedst egner sig til en sådan undersøgelse, er (for mig at se) arbejdsmarkedssegmenteringsteorierne, med elementer fra moderne empirisk klasseforskning. Arbejdsmarkedssegmenteringsteorierne blev udviklet i 1970'erne, men er kun i begrænset omfang er blevet benyttet de sidste 20 år. Denne afhandling vil benytte denne teoriretning, med udgangspunkt i registerdata fra Danmarks Statistik i perioden fra 1996 til 2009.


%%%%%%%%%%%%%%%%%%%%%%%%%%%%%%%%%%%%%%%%%%%%%%
\section{Problemformulering}
%%%%%%%%%%%%%%%%%%%%%%%%%%%%%%%%%%%%%%%%%%%%%%
%
Med udgangspunkt i arbejdsmarkedssegmenteringsteorien, moderne klasseteori og empiri fra registerdata i Danmarks Statistik, har dette speciale følgende problemformulering:
%
\vspace{\baselineskip}
%
\begin{tcolorbox}[title=\textbf{Problemformulering}]
Findes der segmenter på det danske arbejdsmarked, og hvordan kan forskelle i sociale processer være med til at forklare sådanne forskelle i segmentstrukturen?
\end{tcolorbox}
%
\vspace{\baselineskip}
Det vil jeg undersøge med følgende forskningsspørgsmålspørgsmål:
\vspace{\baselineskip}
\begin{tcolorbox}[title=Forskningspørgsmål,
subtitle style={boxrule=0.4pt} ]
	\tcbsubtitle{1.} Er der en opdeling af arbejdsmarkedet for arbejdstagere i delmarkeder, hvor mobilitet indenfor delmarkederne er hyppig, og mellem delmarkederne sjælden?
	\tcbsubtitle{2.} Kan forskelle i de sociale processer vise, at der er tale om segmenter, og ikke blot delmarkeder?
	\tcbsubtitle{3.} Kan klasseteori belyse denne segmentering?
\end{tcolorbox}

Teorier med blik for segmenteringsprocesser på arbejdsmarkedet har en række forskellige indfaldsvinkler: det, der primært binder dem sammen, er ideen om at arbejdsmarkedet mere er karakteriseret ved \emph{opdeling} end af \emph{enshed}, hvilket ellers ville være det basale præmis i neoklassisk økonomisk teori. Dette speciale læner sig op af Thomas Bojes udformning af dette præmis \parencite[174]{Boje1986}.


%%%%%%%%%%%%%%%%%%%%%%%%%%%%%%%%%%%%%%%%%%%%%%
\subsection{Forskningsspørgsmål 1: Opdeling af arbejdsmarkedet i delmarkeder}
%%%%%%%%%%%%%%%%%%%%%%%%%%%%%%%%%%%%%%%%%%%%%%

For Thomas Boje er det første kriterie for arbejdsmarkedssegmentering, at arbejdsmarkedet er delt op i delmarkeder, med begrænset mobilitet mellem de enkelte delmarkeder. Det betyder, at der i mellem visse typer jobs forekommer hyppige skift, og andre jobtyper, hvor der sjældent, eller aldrig, observeres skifte fra det ene til det andet.

Delmarkeder i denne tradition har ofte tilknyttet en forståelse af samfundet som opdelt i forskellige sociale klasser, med forskellige strukturelle livsbetingelser for individerne i dem. Det er denne afhandlings formål at benytte en sådan tilgang, for at vurdere om den kan belyse arbejdsmarkedets struktur.

En klassebaseret forståelse har været genstand for sociologiske analyser siden sociologiens første store teoretikere, Weber, Marx og Durkheim%
%
\footnote{Durkheim er ikke bredt kendt som klasseteoretikere, men elementer i hans sociologi, samt især videreudviklinger af denne, har disse elementer i sig (citer Harrits \#todo)}%
%
. Weber var en af de af de første til at se på forholdet mellem klasse og mobilitet, og til at beskrive klasser ud fra de sociale mobilitetsmønstre, individerne var en del af. Han definerer klasse således: “\emph{A »social class« makes up the totality of those class situations within which individual and generational mobility is easy and typical.}” \parencite[302]{Weber1978}. Weberiansk orienterede sociologer såsom Goldthorpe (henvisning \#todo) og Oesch ( henvisning \#todo) har bibeholdt dette fokus på social mobilitet i nyere forskning. Spørgsmålet om både intra- og intergenerationel mobilitet på arbejdsmarkedet, som det observeres empirisk, har en væsentlig rolle at spille i klasseforskningen både indenfor den marxistiske, weberianske og durkheimianske tradition (henvisning: Wright, Goldthorpe, Grusky).

Den marxistiske tradition er optaget af klasser som et spørgsmål om udbytning, baseret på et socialt forhold til produktionsmidlerne. Nymarxistiske sociologer som Olin-Wright benytter også en stratifikation af lønmodtagerne, hvor positioner på arbejdmarkedet, forstået som arbejderklassens interne sammensætning, også har stor betydning for den faktiske fordeling af goder, og forskellige individers mulighed for opnåelse af disse goder. Arbejdsmarkedet er med andre ord også her delt op ud fra andre kriterier end blot arbejdsgiver-arbejder dikotomien. Den faktiske klassestruktur, udover den grundlæggende modsætning mellem arbejde og kapital, skal undersøges empirisk (henvisning til Wright \#todo). Den weberianske og durkheimianske tradition har ikke samme fokus på blah blah (skriv måske ud: tvivlsomt om afsnittet skal være sådan her? \#todo)

Af nyere sociologisk teori kan desuden nævnes Pierre Bourdieus sammensmeltning og nytænkning af marxistisk og weberiansk funderet teori, i hans forståelse af samfundet som et socialt rum, opdelt i felter, der opererer ud fra forskellige sociale logikker (\#henvisning). Anthony Giddens understreger nødvendigheden af empirisk funderede analyser, for at forstå klassestrukturen i et bestemt samfund, og de muligheder individet har, alt efter "hvor det kommer fra" \parencite[48,110]{Giddens1973}. (inkluder eventuelt Gruskys "mikroklasser")

Indenfor arbejdsmarkedssegmenteringsteorierne har man i en amerikansk optik beskæftiget sig med det såkaldte "dual labour market", det vil arbejdsmarkedet opdelt i to overordnede delmarkeder, hvor det primære indeholder faste stillinger med tryghed i ansættelse og favorable lønninger og gode arbejdsvilkår, hvorimod det sekundære delmarked består af midlertidige ansættelser til lavere lønninger og dårligere arbejdsvilkår \parencite{Piore1980}. Boje karakteriserer denne opdeling et amerikansk fænomen, da en række institutionelle forhold samt en produktionsstruktur, med få, store virksomheder i landet, har skabt denne struktur, og kan ikke generaliseres til lande med andre instutionelle forhold. Dette understreger behovet for en analyse af hvilke danske delmarkeder, der findes. Her finder Boje, at det danske samfund består af langt flere delmarkeder, blandt andet på grund af kollektive overenskomster, sikring af arbejdsløshedsunderstøttelse, samt den langt større rolle, små og mellemstore firmaer spiller, hvilket skaber en anden social dynamik \parencite[36]{Boje1985}

I Danmark er en nyere kortlægning af arbejdsmarkedets delmarkeder allerede påbegyndt, ligeledes med udgangspunkt i arbejdsmarkedssegmenteringsteorien, af Toubøl og Grau Larsen \parencite{Touboel2013}, ved brug af social netværksanalyse. Denne afhandling benytter sig af den af Toubøl og Grau Larsen nyligt udviklede metode til at finde kliker i et socialt netværk, til at finde segmenter i det danske arbejdsmarked, baseret på mobiliet.  

%%%%%%%%%%%%%%%%%%%%%%%%%%%%%%%%%%%%%%%%%%%%%%
\subsection{Forskningsspørgsmål 2: Forskellen på et delmarked og et segment er påvisningen af særegne (specifikke? partikulære) sociale processer indenfor delmarkedet}
%%%%%%%%%%%%%%%%%%%%%%%%%%%%%%%%%%%%%%%%%%%%%%

Det er imidlertidig ikke nok at påvise høj intern mobilitet for at kunne påstå at der er tale om et selvstændigt segment. Det skal være muligt at påvise forskelle i de sociale processer, som findes i delmarkedet i forhold til andre delmarkeder. Et delmarked, hvor forskellen i mobilitet primært skyldes faglige eller geografiske forskelle, men andre væsentlige sociale processer ellers er ens, kan ikke karakteriseres som et segment \parencite[41]{Boje1985}. Det er bare et delmarked, da allokeringen af arbejdskraft i væsentligt grad sker uden (determinerende) sociale stratifikationsmekanismer. Hvis eksempelvis løndannelse og kønsforskelle viser sig tydeligt mellem to ellers sammenlignelige delmarkeder, kan man begynde at tale om segmenter i Bojes forstand. Sammenlignelige skal her forstås som, at uddannelseslængde og 

Sociale processer er et noget abstrakt begreb for den praksis, hvori livet på arbejdsmarkedet udspiller sig for den enkelte lønmodtager. Det teoretiske indhold, samt naturligvis empiriske målbarhed, vil blive udpenslet efterfølgende, her skal blot nævnes to kortfattede perspektiver på sådanne processer.

Frank Parkin definerede i slut 60'erne begrebet \emph{social lukning} som en måde at forstå opdeling mellem sociale grupper, baseret på forhåndenværende distinktionskriterier i et samfund. Begrebet bruges her til at definere de processer, hvorpå forskellige faggrupper sørger for at beskytte egne privilegier, på en sådan vis at det ekskluderer andre, og hæmmer mobiliteten ud fra andre hensyn end (åbenlyst) faglig eller uddannelsesmæssige \parencite{Parkin1994}. 
% Mark  Granovetters benytter social netværksteoris metodik(?? \#todo) om svage og stærke sociale forbindelser for at forstå social lukning. Hans hovedargument er individernes mulighed mobilitet af nye kanaler, gennem deres såkaldte "svage forbindelser" - det er igennem de mennesker, man ikke kender så godt, at der er adgang til nye muligheder indenfor for eksempel arbejdslivet \parencite{Granovetter1973}. (\emph{de her to eksempler skal strammes op, I know - E }) Giver ikke rigtig så god mening her. 



%%%%%%%%%%%%%%%%%%%%%%%%%%%%%%%%%%%%%%%%%%%%%%
\subsection{Forskningsspørgsmål 3:  Et klasseteoretisk perspektiv på segmentering}
%%%%%%%%%%%%%%%%%%%%%%%%%%%%%%%%%%%%%%%%%%%%%%

Her adskiller min afhandling sig væsentligt fra Toubøl \& Grau Larsen, Nielsen-Gravholt og Boje, da dette speciale udover en beskrivelse af delmarkederne, har et klasseteoretisk perspektiv på segmentering og de sociale processer der definerer dem, der derefter undersøges empirisk. Formålet er at bevæge sig over i et klasseteoretisk perspektiv på segmentering, for at undersøge klassebegrebets anvendelig i at forstå disse sociale processer, som jeg ser dem komme til udtryk i min empiri. 

Forbindelsen til segmenteringsteori er ikke kontroversiel, og teoriretningen er bestemt ikke fremmed overfor teorier om sociale klasser, omend det ofte frames anderledes og i mere arbejdsmarkedsfokuserede termer. 
Boje er optaget af, at forskelle i sociale processer skaber ulighed på arbejdsmarkedet, og ses i ulige vilkår for forskellige (segmenterede) delmarkeder, hvorved forskelle i livsvilkår fører til øget seggregering. Dette er tæt beslægtet med tankegangen i klasseteori. Arbejdsmarkedet kan, som den marxistiske arbejdsmarkedssegmentteoretiker Richard Edwards bemærker, ses som et helt særgent marked, der tydeliggør styrkeholdende i produktionen og i den arbejdende befolkning som helhed {\parencite[177]{Edwards1979}. 

Moderne klasseteori, som den kommer til udtryk hos Daniel Oesch og John Goldthorpe, mener jeg har frugtbare teoretiske såvel som empiriske indsigter i, hvad man kan kalde den over-tid segmenterede arbejdsmarkedsstruktur. Deres klasseinddeling er næsten udelukkende baseret på position på arbejdsmarkedet. En årsag til Oesch og Goldthorpes anvendelig i min kortlægning af arbejdsmarkedet, er deres stringente - næsten ydmyge - fokus på \emph{økonomiske} klasse, fremfor det mere vidtløftige begreb \emph{social} klasse. Det vil jeg komme nærmere ind på senere, foreløbigt skal det bare konstateres, at det er yderst anvendeligt, når man som mig har fokus på arbejdsmarkedets struktur. Det betyder, at deres teori og empiri om differentieringer på arbejdsmarkedet er yderst anvendelige for mig.






%%%%%%%%%%%%%%%%%%%%%%%%%%%%%%%%%%%%%%%%%%%%%%
\section{Fremgangsmåde}
%%%%%%%%%%%%%%%%%%%%%%%%%%%%%%%%%%%%%%%%%%%%%%

I de to næste kapitler gennemgår jeg teori. Andet kapitel handler således om arbejdsmarkedssegmenteringsteori og tredje kapitel handler om klasseteori.

I de to efterfølgende kapitler gennemgår jeg metode. Fjerde kapitel handler således om social netværksanalyse og femte kapitel handler om registerdatamaterialet fra Danmarks Statistik.

De tre efterfølgende kapitler er analysen opdelt tre delanalyser, som passer overens med de tre forskningsspørgsmål. Sjette kapitel er en delanalyse af opdeling af arbejdsmarkedet i delmarkeder. Syvende kapitel er en delanalyse af sociale processer i delmarkeder og segmenter. Ottende kapitel er en delanalyse af det klasseteoretiske perspektiv på segmentering.

Niende kapitel er en diskussion af analysen på baggrund af teori og metode.

De tiende og afsluttende kapitel er konklusion.


Denne afhandling vil fokusere på, hvorledes sociale processer kan ses afvige fra hinanden i delmarkederne, på sådan vis at vi kan tale om segmenter. Jeg vil benytte social stratifikationsteori, til at forklare forskellen i sociale processer, som de kommer til udtryk på et empirisk niveau, gennem intern mobilitet i delmarkederne, indkomst, uddannelse og køn. 