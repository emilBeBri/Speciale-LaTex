%!TEX root = ./report.tex
%-*- coding: utf-8 -*-
%!TeX encoding = UTF-8


%%%%%%%%%%%%%%%%%%%%%%%%%%%%%%%%%%%%%
% ----------- Preamble ------------ %
%%%%%%%%%%%%%%%%%%%%%%%%%%%%%%%%%%%%%



% ----------- dokumentclass ------------ %

\documentclass[11pt,a4paper,report,danish,oneside]{memoir} % for a short document
%brug article til kortere opgaver, hvor en af effekterne er at chapters og sections har samme status


% ----------- Font og sprog ------------ %

\usepackage[utf8]{inputenc} % set input encoding to utf8
\usepackage[T1]{fontenc}
\usepackage{kpfonts}
\usepackage[normalem]{ulem} % muliggør underline der brydes fremfor at fortsætte udover marginen. i stedet \underline{} er det \uline{}.
\usepackage[danish]{babel} %dansk sprog
\renewcommand{\danishhyphenmins}{10} % gør ordopdeling bedre men den er lort i forvejen så hvor slemt kan det være hvis det her er det bedste?! Prøv at eksperimentér med det der tal, 22, se hvad der sker.




%%% hyphenation




% \hyphenation{formålsløst} % eksempel på ord der under ingen omstændigheder må deles op


% \usepackage[avantgarde]{quotchap}   % Pænere kapitler





\usepackage[scaled]{helvet}
\renewcommand\familydefault{\sfdefault} 
\usepackage[T1]{fontenc}   




% ----------- Sidelayout------------ %
% Set up the paper to be as close as possible to both A4 & letter:
\setulmarginsandblock{3cm}{3cm}{*} % 50pt upper margins
\setlrmarginsandblock{3cm}{3cm}{*} % golden ratio again for left/right margins
\checkandfixthelayout 
% Ovenstående er fra the memoir manual

\pagestyle{ruled} % try also: empty , plain , headings , ruled , Ruled , companion

%\SingleSpacing
\OnehalfSpacing
%\DoubleSpacing
%\setstretch{1.1}
\setcounter{secnumdepth}{2}
\setcounter{tocdepth}{2}

%indholdsfortegnelse
\renewcommand\cftappendixname{\appendixname~} %gør at der står "bilag A" i stedet for bare "A"

% fjerner sidenr fra \part-sider
\makeatletter
\renewcommand\part{%
  \if@openright
    \cleardoublepage
  \else
    \clearpage
  \fi
  \thispagestyle{empty}%
  \if@twocolumn
    \onecolumn
    \@tempswatrue
  \else
    \@tempswafalse
  \fi
  \null\vfil
  \secdef\@part\@spart}
\makeatother




% Ændre skrifttypen:
% \renewcommand{\familydefault}{\ttdefault} % typewriter font i hele documentet, men fucker med linjebrydningen/hypernation, ved ikke hvorfor.


% ----------- Diverse pakker - rækkefølgen er IKKE ligegyldig ------------ %

%\usepackage{longtable}  For at kunne lave tabeller fx i Stata der rækker over flere sider

\usepackage[table,xcdraw]{xcolor} % farver i tabeller
\usepackage{enumitem} %en ekstra slags liste 
\usepackage{float} % for at kunne placere en floater PRÆCIS her med \begin{figure}[H]
\usepackage{pdfpages} % understøtter inkluderingen af pdf-filer
\usepackage[plainpages=false,pdfpagelabels,pageanchor=false,hidelinks]{hyperref}   % aktive links
\usepackage{memhfixc}       % rettelser til hyperref - ved ikke hvad den gør
\usepackage{booktabs} %fancy tabeller
\usepackage{tabularx} % til statas tabout
\usepackage{multirow} % flere rækker i tabeller
\usepackage{graphicx} % indsæt billeder
\usepackage{pdflscape} % lav visse sider om til landscape mode
\usepackage{rotating} %tillader figurer der står sidelæns
\usepackage{csquotes}
\usepackage{enumitem} %mindre mellemrum mellem items
\usepackage{dashrule} %stiblede linier i tabeller
%\usepackage{comment} % kunne sætte sectioner er ikke helt overbevidst om den her er brugbar, måske bare ifelse-løsningen
\usepackage{nameref} %giver mulighed for at refere til chapters etc med navn i stedet for bare en counter eller sidetal. 
% \usepackage{bytefield} % til at skabe datafigurer, se http://tex.stackexchange.com/questions/108257/implementing-a-way-to-show-data-structures-in-latex
\usepackage[normalem]{ulem} %striketrough
\usepackage{array} %centering 
\usepackage{wallpaper}
\usepackage{wrapfig} % wrap figure
% \usepackage[font=small]{caption}
% \captionnamefont{\scriptsize}

 % or footnotesize

% Test pakker
\usepackage{appendix} %[toc,page]
\usepackage{capt-of} % til at sættte figurer og tabeller side-by-side
\usepackage[table,xcdraw]{xcolor} % farver til tabeller
\usepackage{courier} %sætter \texttt{} command til courier i stedet for den anden. courier er angiveligt bedre fordi den visuelt er mere forskellig fra standardfonten.
\usepackage{tcolorbox}
% \usepackage{siunitx}
% \sisetup{locale = DA}
\usepackage{multirow} 
\usepackage{geometry}
\usepackage{nicefrac}
\usepackage{flowchart}
\usetikzlibrary{arrows}
% \usepackage{pgfplotstable}


% ----------- Bibliografi ------------ %



%%bibliografi med biblatex
\usepackage[citestyle=authoryear-ibid, sorting=nty,backend=biber,bibencoding=UTF8,maxcitenames=2,defernumbers=true]{biblatex}
%% options til ovenstaaende:
%% maxcitenames=x styrer antal af forfattere før der står et.al/m.fl.
%% uniquelist=false/true styrer om den alligevel skal skrive flere navne end specifieret i maxcitenames i tvivlspørgsmål, dvs. hvis en forfatter optræder i flere referencer med andre forfattere.


%\usepackage[style=authoryear,backend=biber]{biblatex}
%bibtex-fil

%lave filtre så bibliografien kan deles op
% \defbibfilter{notonline}{
%   not type=Electronic, 
%   % and not type=manual
%   not keyword={DSTmanual}
% }

% \defbibfilter{DSTmanual}{%
% keyword=DSTmanual %keyword er et felt i en bib-entry
% }

% \defbibfilter{onlinekilder}{%
% type={Electronic}, 
% notkeyword={DSTmanual}
% }



\addbibresource{biblio/bibliografi.bib}



% ----------- Floaters ------------ %


%%% Ikke sikker på effekten af følgende. Måske slet, eksperimenter dig frem når du står i en situation hvor du kan se floaters opfører sig underligt.
%Alter some LaTeX defaults for better treatment of figures:
    % See p.105 of "TeX Unbound" for suggested values.
    % See pp. 199-200 of Lamport's "LaTeX" book for details.
    %   General parameters, for ALL pages:
    \renewcommand{\topfraction}{0.9}	% max fraction of floats at top
    \renewcommand{\bottomfraction}{0.8}	% max fraction of floats at bottom
    %   Parameters for TEXT pages (not float pages):
    \setcounter{topnumber}{2}
    \setcounter{bottomnumber}{2}
    \setcounter{totalnumber}{4}     % 2 may work better
    \setcounter{dbltopnumber}{2}    % for 2-column pages
    \renewcommand{\dbltopfraction}{0.9}	% fit big float above 2-col. text
    \renewcommand{\textfraction}{0.07}	% allow minimal text w. figs
    %   Parameters for FLOAT pages (not text pages):
    \renewcommand{\floatpagefraction}{0.7}	% require fuller float pages
	% N.B.: floatpagefraction MUST be less than topfraction !!
    \renewcommand{\dblfloatpagefraction}{0.7}	% require fuller float pages
	% remember to use [htp] or [htpb] for placement


% ----------- Egne kommandoer ------------ %

% \newcommand{\5.1}{5.1} (bruges ikke længere men for eksemplets skyld)

% \newcommand{\antalkat}{150} (bruges ikke længere men for eksemplets skyld)

% \fref og venner er defineret saaledes i memoir

% \renewcommand{\fref}[1]{\figurerefname~ef{#1}} 
% \renewcommand{\tref}[1]{\tablerefname~ef{#1}} 
% \renewcommand{\pref}[1]{\pagerefname~\pageref{#1}} 
% \renewcommand{\Pref}[1]{\partrefnameef{#1}} 
% \renewcommand{\Cref}[1]{\chapterrefnameef{#1}} 
% \renewcommand{\Sref}[1]{\sectionrefnameef{#1}} 

% og de navne som er anvendt er defineret som det her, dvs her kan man ændre om det er på dansk eller engelsk:

% \renewcommand*{\figurerefname}{Figur} 
% \renewcommand*{\tablerefname}{Tabel} 
\renewcommand*{\pagerefname}{side} 
% \renewcommand*{\partrefname}{Part~} 
% \renewcommand*{\chapterrefname}{Chapter~} 
% \renewcommand*{\sectionrefname}{\S} 

% \renewcommand*{\pref}{side} 

\usepackage{xcolor}

%Local Variables: 
%mode: latex
%TeX-master: "report"
%End: 





